
\addchap{Eltern}



\addsec{Joachim Otto Scheuermann  \& Ingrid Juliana Galm }


\begin{person}[
    surname = {Scheuermann},
    givenname = {Joachim Otto},
    suffix = {1961},
    label = {@I2@},
    filename = {Jochen Scheuermann (1961)}
    ]

\begin{tabular}{cl}
\geboren & 25. März 1961 in Mudau\\
\taufe & 02. April 1961 in Mudau\\
\geheiratet & 05. September 1991 in Mudau mit Ingrid Juliana Galm \\
\end{tabular}\\
\medbreak
\textsc{vater}: \hyperref[@I11@]{Erwin Otto Scheuermann} [18.05.1923--14.11.1982 (4 Kinder)]\\
\textsc{mutter}: \hyperref[@I12@]{Maria Rita Röckel} [06.01.1927--19.07.2012 (4 Kinder)]
\medbreak
\textsc{{geschwister}}
\begin{itemize}
\item \hyperref[@I17@]{Walter Scheuermann} [11.12.1952 (1 Kind)]
\item \hyperref[@I18@]{Hubert Alfons Scheuermann} [08.11.1954 (3 Kinder)]
\item \hyperref[@I19@]{Edeltraud Elisabeth Scheuermann} [19.01.1958 (2 Kinder)]
\end{itemize}
\bigbreak
\textsc{{kinder}}
\begin{itemize}
\item \hyperref[@I1@]{Fabian Scheuermann} [05.10.1994]
\item \hyperref[@I6@]{Alisia Scheuermann} [31.10.1996]
\end{itemize}
\medbreak
\textsc{{quellen}}
\begin{enumerate}[label={[\arabic*]}]
\item Mudau Geburtenbuch 1958–1982, Geburtenregister 1961, Nr. 4
\item Mudau Heiratsbuch 1991–1995, Heiratsregister 1991, Nr. 21
\item Trau-Reg. Nr. 11/91
\end{enumerate}

\end{person}

\begin{person}[
    surname = {Galm},
    givenname = {Ingrid Juliana},
    suffix = {1965},
    label = {@I3@},
    filename = {Ingrid Scheuermann (1965)}
    ]

\begin{tabular}{cl}
\geboren & 23. September 1965 in Langenelz\\
\taufe & 03. Oktober 1965 in Mudau\\
\geheiratet & 05. September 1991 in Mudau mit Joachim Otto Scheuermann \\
\end{tabular}\\
\medbreak
\textsc{vater}: \hyperref[@I4@]{Karl Galm} [27.07.1924--03.01.2015 (6 Kinder)]\\
\textsc{mutter}: \hyperref[@I5@]{Aloisia Lina Schölch} [12.10.1925--07.09.1993 (6 Kinder)]
\medbreak
\textsc{{geschwister}}
\begin{itemize}
\item \hyperref[@I20@]{Philipp Galm} [22.04.1952 (2 Kinder)]
\item \hyperref[@I21@]{Herbert Galm} [04.10.1953 (2 Kinder)]
\item \hyperref[@I22@]{Hugo Galm} [11.01.1957 (4 Kinder)]
\item \hyperref[@I23@]{Konrad Galm} [19.02.1958 (2 Kinder)]
\item \hyperref[@I24@]{Helga Rita Galm} [07.01.1964 (3 Kinder)]
\end{itemize}
\bigbreak
\textsc{{kinder}}
\begin{itemize}
\item \hyperref[@I1@]{Fabian Scheuermann} [05.10.1994]
\item \hyperref[@I6@]{Alisia Scheuermann} [31.10.1996]
\end{itemize}
\medbreak
\textsc{{quellen}}
\begin{enumerate}[label={[\arabic*]}]
\item Langenelz Geburtenbuch 1946–1973, Geburtenregister 1965, Nr. 3
\item Mudau Heiratsbuch 1991–1995, Heiratsregister 1991, Nr. 21
\item Trau-Reg. Nr. 11/91
\end{enumerate}

\end{person}

\begin{person}[
    surname = {Scheuermann},
    givenname = {Fabian},
    suffix = {1994},
    label = {@I1@},
    filename = {Fabian Scheuermann (1994)}
    ]

\begin{tabular}{cl}
\geboren & 05. Oktober 1994 in Buchen\\
\taufe & 13. November 1994 in Mudau\\
\end{tabular}\\
\medbreak
\textsc{vater}: \hyperref[@I2@]{Joachim Otto Scheuermann} [25.03.1961 (2 Kinder)]\\
\textsc{mutter}: \hyperref[@I3@]{Ingrid Juliana Galm} [23.09.1965 (2 Kinder)]
\medbreak
\textsc{{geschwister}}
\begin{itemize}
\item \hyperref[@I6@]{Alisia Scheuermann} [31.10.1996]
\end{itemize}
\bigbreak
\textsc{{quellen}}
\begin{enumerate}[label={[\arabic*]}]
\item Geburtenregister Buchen 1994, Nr. 452
\item Tauf-Reg. Nr. 26/94
\end{enumerate}

\end{person}

\begin{person}[
    surname = {Scheuermann},
    givenname = {Alisia},
    suffix = {1996},
    label = {@I6@},
    filename = {Alisia Scheuermann (1996)}
    ]

\begin{tabular}{cl}
\geboren & 31. Oktober 1996 in Buchen\\
\taufe & 12. Januar 1997 in Mudau\\
\end{tabular}\\
\medbreak
\textsc{vater}: \hyperref[@I2@]{Joachim Otto Scheuermann} [25.03.1961 (2 Kinder)]\\
\textsc{mutter}: \hyperref[@I3@]{Ingrid Juliana Galm} [23.09.1965 (2 Kinder)]
\medbreak
\textsc{{geschwister}}
\begin{itemize}
\item \hyperref[@I1@]{Fabian Scheuermann} [05.10.1994]
\end{itemize}
\bigbreak
\textsc{{quellen}}
\begin{enumerate}[label={[\arabic*]}]
\item Geburtenregister Buchen 1996, Nr. 651
\item Tauf-Reg. Nr. 1/97
\end{enumerate}

\end{person}


\addchap{Gro"seltern}



\addsec{Erwin Otto Scheuermann  \& Maria Rita Röckel }


\begin{person}[
    surname = {Scheuermann},
    givenname = {Erwin Otto},
    suffix = {1923--1982},
    label = {@I11@},
    filename = {Erwin Scheuermann (1923)}
    ]

\begin{tabular}{cl}
\geboren & 18. Mai 1923 in Mudau\\
\geheiratet &  in Stift Neuburg mit Maria Rita Röckel \\
\gestorben & 14. November 1982 in Buchen\\
\bestattet &  in Mudau\\
\end{tabular}\\
\medbreak
\textsc{vater}: \hyperref[@I13@]{Heinrich Scheuermann} [13.05.1886--18.04.1929 (6 Kinder)]\\
\textsc{mutter}: \hyperref[@I14@]{Anna Zimmermann} [07.12.1887--18.06.1969 (6 Kinder)]
\medbreak
\textsc{{geschwister}}
\begin{itemize}
\item \hyperref[@I72@]{Anna Scheuermann} [02.07.1913--28.09.2000 (4 Kinder)]
\item \hyperref[@I73@]{Heinrich Scheuermann} [30.08.1914--16.12.1944]
\item \hyperref[@I74@]{Josef Scheuermann} [12.11.1919--27.05.1997 (2 Kinder)]
\item \hyperref[@I1208@]{Maria Scheuermann} [03.12.1924--03.12.1924]
\item \hyperref[@I75@]{Alfons Scheuermann} [18.06.1926--29.09.2005 (2 Kinder)]
\end{itemize}
\bigbreak
\textsc{{kinder}}
\begin{itemize}
\item \hyperref[@I17@]{Walter Scheuermann} [11.12.1952 (1 Kind)]
\item \hyperref[@I18@]{Hubert Alfons Scheuermann} [08.11.1954 (3 Kinder)]
\item \hyperref[@I19@]{Edeltraud Elisabeth Scheuermann} [19.01.1958 (2 Kinder)]
\item \hyperref[@I2@]{Joachim Otto Scheuermann} [25.03.1961 (2 Kinder)]
\end{itemize}
\medbreak
\textsc{{quellen}}
\begin{enumerate}[label={[\arabic*]}]
\item Mudau Heiratsbuch 1952–1957, Heiratsregister 1952, Nr. 6
\item \href{http://grabsteine.genealogy.net/tomb.php?cem=3902&tomb=406&b=&lang=de}{genealogy.net Grabstein Projekt, Friedhof Mudau}
\item \href{https://www.familysearch.org/tree/person/details/L1TQ-C1D}{FamilySearch, ID: L1TQ-C1D}
\end{enumerate}

\end{person}

\begin{person}[
    surname = {Röckel},
    givenname = {Maria Rita},
    suffix = {1927--2012},
    label = {@I12@},
    filename = {Rita Scheuermann (1927)}
    ]

\begin{tabular}{cl}
\geboren & 06. Januar 1927 in Langenelz\\
\geheiratet &  in Stift Neuburg mit Erwin Otto Scheuermann \\
\gestorben & 19. Juli 2012 in Buchen\\
\bestattet &  in Mudau\\
\end{tabular}\\
\medbreak
\textsc{vater}: \hyperref[@I15@]{Otto Röckel} [20.11.1882--14.06.1965 (6 Kinder)]\\
\textsc{mutter}: \hyperref[@I16@]{Maria Anna Mechler} [15.08.1889--25.10.1966 (6 Kinder)]
\medbreak
\textsc{{geschwister}}
\begin{itemize}
\item \hyperref[@I70@]{Karl Röckel} [08.07.1920--10.08.2003 (5 Kinder)]
\item \hyperref[@I68@]{Otto Röckel} [09.07.1922--21.01.1944]
\item \hyperref[@I69@]{Walter Röckel} [15.02.1924--24.05.2010]
\item \hyperref[@I71@]{Alfred Röckel} [21.04.1925--16.06.1944]
\item \hyperref[@I67@]{Elisabeth Röckel} [14.04.1929--12.10.2015 (2 Kinder)]
\end{itemize}
\bigbreak
\textsc{{kinder}}
\begin{itemize}
\item \hyperref[@I17@]{Walter Scheuermann} [11.12.1952 (1 Kind)]
\item \hyperref[@I18@]{Hubert Alfons Scheuermann} [08.11.1954 (3 Kinder)]
\item \hyperref[@I19@]{Edeltraud Elisabeth Scheuermann} [19.01.1958 (2 Kinder)]
\item \hyperref[@I2@]{Joachim Otto Scheuermann} [25.03.1961 (2 Kinder)]
\end{itemize}
\medbreak
\textsc{{quellen}}
\begin{enumerate}[label={[\arabic*]}]
\item Langenelz Geburtenbuch 1911–1927, Geburtenregister 1927, Nr. 1
\item Mudau Heiratsbuch 1952–1957, Heiratsregister 1952, Nr. 6
\item Sterberegister Buchen 2012, Nr. 190
\item \href{http://grabsteine.genealogy.net/tomb.php?cem=3902&tomb=406&b=&lang=de}{genealogy.net Grabstein Projekt, Friedhof Mudau}
\item \href{https://www.familysearch.org/tree/person/details/L1TQ-C11}{FamilySearch, ID: L1TQ-C11}
\end{enumerate}

\end{person}

\begin{person}[
    surname = {Scheuermann},
    givenname = {Walter},
    suffix = {1952},
    label = {@I17@},
    filename = {Walter Scheuermann (1952)}
    ]

\begin{tabular}{cl}
\geboren & 11. Dezember 1952 in Mudau\\
\geheiratet & 22. Mai 1981 mit Gisela Elisabeth Kühner \\
\end{tabular}\\
\medbreak
\textsc{vater}: \hyperref[@I11@]{Erwin Otto Scheuermann} [18.05.1923--14.11.1982 (4 Kinder)]\\
\textsc{mutter}: \hyperref[@I12@]{Maria Rita Röckel} [06.01.1927--19.07.2012 (4 Kinder)]
\medbreak
\textsc{{geschwister}}
\begin{itemize}
\item \hyperref[@I18@]{Hubert Alfons Scheuermann} [08.11.1954 (3 Kinder)]
\item \hyperref[@I19@]{Edeltraud Elisabeth Scheuermann} [19.01.1958 (2 Kinder)]
\item \hyperref[@I2@]{Joachim Otto Scheuermann} [25.03.1961 (2 Kinder)]
\end{itemize}
\bigbreak
\textsc{{kinder}}
\begin{itemize}
\item Philipp Christian Scheuermann [26.12.1987]
\end{itemize}
\medbreak
\end{person}

\begin{person}[
    surname = {Scheuermann},
    givenname = {Hubert Alfons},
    suffix = {1954},
    label = {@I18@},
    filename = {Hubert Scheuermann (1954)}
    ]

\begin{tabular}{cl}
\geboren & 08. November 1954 in Mudau\\
\geheiratet & 20. November 1980 mit Katharina Karras \\
\end{tabular}\\
\medbreak
\textsc{vater}: \hyperref[@I11@]{Erwin Otto Scheuermann} [18.05.1923--14.11.1982 (4 Kinder)]\\
\textsc{mutter}: \hyperref[@I12@]{Maria Rita Röckel} [06.01.1927--19.07.2012 (4 Kinder)]
\medbreak
\textsc{{geschwister}}
\begin{itemize}
\item \hyperref[@I17@]{Walter Scheuermann} [11.12.1952 (1 Kind)]
\item \hyperref[@I19@]{Edeltraud Elisabeth Scheuermann} [19.01.1958 (2 Kinder)]
\item \hyperref[@I2@]{Joachim Otto Scheuermann} [25.03.1961 (2 Kinder)]
\end{itemize}
\bigbreak
\textsc{{kinder}}
\begin{itemize}
\item Simon Scheuermann [21.09.1980 (2 Kinder)]
\item Eva Scheuermann [24.06.1985 (2 Kinder)]
\item Ann-Kathrin Scheuermann [29.05.1990]
\end{itemize}
\medbreak
\end{person}

\begin{person}[
    surname = {Scheuermann},
    givenname = {Edeltraud Elisabeth},
    suffix = {1958},
    label = {@I19@},
    filename = {Edel Seiferheld (1958)}
    ]

\begin{tabular}{cl}
\geboren & 19. Januar 1958 in Mudau\\
\geheiratet &  mit Stefen Seiferheld \\
\end{tabular}\\
\medbreak
\textsc{vater}: \hyperref[@I11@]{Erwin Otto Scheuermann} [18.05.1923--14.11.1982 (4 Kinder)]\\
\textsc{mutter}: \hyperref[@I12@]{Maria Rita Röckel} [06.01.1927--19.07.2012 (4 Kinder)]
\medbreak
\textsc{{geschwister}}
\begin{itemize}
\item \hyperref[@I17@]{Walter Scheuermann} [11.12.1952 (1 Kind)]
\item \hyperref[@I18@]{Hubert Alfons Scheuermann} [08.11.1954 (3 Kinder)]
\item \hyperref[@I2@]{Joachim Otto Scheuermann} [25.03.1961 (2 Kinder)]
\end{itemize}
\bigbreak
\textsc{{kinder}}
\begin{itemize}
\item Valerie Seiferheld [06.01.1988]
\item Annelie Seiferheld [10.01.1991]
\end{itemize}
\medbreak
\end{person}


\addsec{Karl Galm  \& Aloisia Lina Schölch }


\begin{person}[
    surname = {Galm},
    givenname = {Karl},
    suffix = {1924--2015},
    label = {@I4@},
    filename = {Karl Galm (1924)}
    ]

\begin{tabular}{cl}
\geboren & 27. Juli 1924 in Langenelz\\
\geheiratet & 01. August 1951 in Langenelz mit Aloisia Lina Schölch \\
\gestorben & 03. Januar 2015 in Buchen\\
\bestattet & 08. Januar 2015 in Langenelz\\
\end{tabular}\\
\medbreak
\textsc{vater}: \hyperref[@I7@]{Julius Galm} [28.02.1883--23.05.1929 (7 Kinder)]\\
\textsc{mutter}: \hyperref[@I8@]{Margaretha Schüßler} [11.04.1894--26.07.1978 (7 Kinder)]
\medbreak
\textsc{{geschwister}}
\begin{itemize}
\item \hyperref[@I56@]{Philipp Galm} [19.02.1920--vor 1945]
\item \hyperref[@I52@]{Juliane Galm} [26.03.1921--05.04.2019 (5 Kinder)]
\item \hyperref[@I53@]{Maria Galm} [14.07.1922--15.02.1995 (7 Kinder)]
\item \hyperref[@I57@]{Arthur Galm} [02.10.1925--1944]
\item \hyperref[@I54@]{Leo Galm} [07.09.1927--24.05.2003 (4 Kinder)]
\item \hyperref[@I55@]{Georg Galm} [16.08.1929 (3 Kinder)]
\end{itemize}
\bigbreak
\textsc{{kinder}}
\begin{itemize}
\item \hyperref[@I20@]{Philipp Galm} [22.04.1952 (2 Kinder)]
\item \hyperref[@I21@]{Herbert Galm} [04.10.1953 (2 Kinder)]
\item \hyperref[@I22@]{Hugo Galm} [11.01.1957 (4 Kinder)]
\item \hyperref[@I23@]{Konrad Galm} [19.02.1958 (2 Kinder)]
\item \hyperref[@I24@]{Helga Rita Galm} [07.01.1964 (3 Kinder)]
\item \hyperref[@I3@]{Ingrid Juliana Galm} [23.09.1965 (2 Kinder)]
\end{itemize}
\medbreak
\textsc{{quellen}}
\begin{enumerate}[label={[\arabic*]}]
\item Langenelz Heiratsbuch 1946–1973, Heiratsregister 1951, Nr. 3
\item Ahnentafel Galm
\item \href{https://www.familysearch.org/tree/person/details/L51R-BH8}{FamilySearch, ID: L51R-BH8}
\item \href{http://grabsteine.genealogy.net/tomb.php?cem=3810&tomb=33&b=&lang=de}{genealogy.net Grabstein Projekt, Friedhof Langenelz}
\end{enumerate}

\end{person}

\begin{person}[
    surname = {Schölch},
    givenname = {Aloisia Lina},
    suffix = {1925--1993},
    label = {@I5@},
    filename = {Alise Galm (1925)}
    ]

\begin{tabular}{cl}
\geboren & 12. Oktober 1925 in Langenelz\\
\geheiratet & 01. August 1951 in Langenelz mit Karl Galm \\
\gestorben & 07. September 1993 in Langenelz\\
\bestattet & 11. September 1993 in Langenelz\\
\end{tabular}\\
\medbreak
\textsc{vater}: \hyperref[@I9@]{Alois Adolf Schölch} [12.03.1895--07.08.1963 (10 Kinder)]\\
\textsc{mutter}: \hyperref[@I10@]{Adelheid Anna Schäfer} [17.08.1897--28.07.1959 (10 Kinder)]
\medbreak
\textsc{{geschwister}}
\begin{itemize}
\item \hyperref[@I58@]{Alfons Schölch} [03.08.1927--25.09.1927]
\item \hyperref[@I59@]{Gertrud Schölch} [01.03.1929--02.03.1996 (2 Kinder)]
\item \hyperref[@I61@]{Alfred Schölch} [04.05.1930--30.01.1998 (3 Kinder)]
\item \hyperref[@I60@]{Walter Josef Schölch} [04.08.1931--03.08.1992 (2 Kinder)]
\item \hyperref[@I64@]{Adolf Alois Schölch} [18.01.1934--07.05.1977]
\item \hyperref[@I63@]{Rita Karolina Schölch} [18.01.1934--15.02.2015 (3 Kinder)]
\item \hyperref[@I62@]{Bernhard Schölch} [19.08.1935--04.10.2005 (1 Kind)]
\item \hyperref[@I65@]{Hubert Schölch} [07.02.1938--24.12.1993 (2 Kinder)]
\item \hyperref[@I66@]{Arthur Richard Schölch} [06.02.1940--16.12.2011]
\end{itemize}
\bigbreak
\textsc{{kinder}}
\begin{itemize}
\item \hyperref[@I20@]{Philipp Galm} [22.04.1952 (2 Kinder)]
\item \hyperref[@I21@]{Herbert Galm} [04.10.1953 (2 Kinder)]
\item \hyperref[@I22@]{Hugo Galm} [11.01.1957 (4 Kinder)]
\item \hyperref[@I23@]{Konrad Galm} [19.02.1958 (2 Kinder)]
\item \hyperref[@I24@]{Helga Rita Galm} [07.01.1964 (3 Kinder)]
\item \hyperref[@I3@]{Ingrid Juliana Galm} [23.09.1965 (2 Kinder)]
\end{itemize}
\medbreak
\textsc{{quellen}}
\begin{enumerate}[label={[\arabic*]}]
\item Langenelz Heiratsbuch 1946–1973, Heiratsregister 1951, Nr. 3
\item Ahnentafel Galm
\item \href{https://www.familysearch.org/tree/person/details/L51H-3TC}{FamilySearch, ID: L51H-3TC}
\item \href{http://grabsteine.genealogy.net/tomb.php?cem=3810&tomb=33&b=&lang=de}{genealogy.net Grabstein Projekt, Friedhof Langenelz}
\end{enumerate}

\end{person}

\begin{person}[
    surname = {Galm},
    givenname = {Philipp},
    suffix = {1952},
    label = {@I20@},
    filename = {Philipp Galm (1952)}
    ]

\begin{tabular}{cl}
\geboren & 22. April 1952 in Langenelz\\
\geheiratet & 06. Juni 1975 in Balsbach mit Irmgard Heck \\
\end{tabular}\\
\medbreak
\textsc{vater}: \hyperref[@I4@]{Karl Galm} [27.07.1924--03.01.2015 (6 Kinder)]\\
\textsc{mutter}: \hyperref[@I5@]{Aloisia Lina Schölch} [12.10.1925--07.09.1993 (6 Kinder)]
\medbreak
\textsc{{geschwister}}
\begin{itemize}
\item \hyperref[@I21@]{Herbert Galm} [04.10.1953 (2 Kinder)]
\item \hyperref[@I22@]{Hugo Galm} [11.01.1957 (4 Kinder)]
\item \hyperref[@I23@]{Konrad Galm} [19.02.1958 (2 Kinder)]
\item \hyperref[@I24@]{Helga Rita Galm} [07.01.1964 (3 Kinder)]
\item \hyperref[@I3@]{Ingrid Juliana Galm} [23.09.1965 (2 Kinder)]
\end{itemize}
\bigbreak
\textsc{{kinder}}
\begin{itemize}
\item Mathias Galm [04.04.1978 (4 Kinder)]
\item Stephanie Galm [26.06.1979 (2 Kinder)]
\end{itemize}
\medbreak
\end{person}

\begin{person}[
    surname = {Galm},
    givenname = {Herbert},
    suffix = {1953},
    label = {@I21@},
    filename = {Herbert Galm (1953)}
    ]

\begin{tabular}{cl}
\geboren & 04. Oktober 1953 in Langenelz\\
\geheiratet & 12. Juni 1976 in Balsbach mit Annelise Seifert \\
\end{tabular}\\
\medbreak
\textsc{vater}: \hyperref[@I4@]{Karl Galm} [27.07.1924--03.01.2015 (6 Kinder)]\\
\textsc{mutter}: \hyperref[@I5@]{Aloisia Lina Schölch} [12.10.1925--07.09.1993 (6 Kinder)]
\medbreak
\textsc{{geschwister}}
\begin{itemize}
\item \hyperref[@I20@]{Philipp Galm} [22.04.1952 (2 Kinder)]
\item \hyperref[@I22@]{Hugo Galm} [11.01.1957 (4 Kinder)]
\item \hyperref[@I23@]{Konrad Galm} [19.02.1958 (2 Kinder)]
\item \hyperref[@I24@]{Helga Rita Galm} [07.01.1964 (3 Kinder)]
\item \hyperref[@I3@]{Ingrid Juliana Galm} [23.09.1965 (2 Kinder)]
\end{itemize}
\bigbreak
\textsc{{kinder}}
\begin{itemize}
\item Michael Galm [05.05.1978]
\item Thomas Galm [23.06.1981 (3 Kinder)]
\end{itemize}
\medbreak
\end{person}

\begin{person}[
    surname = {Galm},
    givenname = {Hugo},
    suffix = {1957},
    label = {@I22@},
    filename = {Hugo Galm (1957)}
    ]

\begin{tabular}{cl}
\geboren & 11. Januar 1957 in Langenelz\\
\geheiratet & 11. Juli 1987 mit Marita Späth \\
\end{tabular}\\
\medbreak
\textsc{vater}: \hyperref[@I4@]{Karl Galm} [27.07.1924--03.01.2015 (6 Kinder)]\\
\textsc{mutter}: \hyperref[@I5@]{Aloisia Lina Schölch} [12.10.1925--07.09.1993 (6 Kinder)]
\medbreak
\textsc{{geschwister}}
\begin{itemize}
\item \hyperref[@I20@]{Philipp Galm} [22.04.1952 (2 Kinder)]
\item \hyperref[@I21@]{Herbert Galm} [04.10.1953 (2 Kinder)]
\item \hyperref[@I23@]{Konrad Galm} [19.02.1958 (2 Kinder)]
\item \hyperref[@I24@]{Helga Rita Galm} [07.01.1964 (3 Kinder)]
\item \hyperref[@I3@]{Ingrid Juliana Galm} [23.09.1965 (2 Kinder)]
\end{itemize}
\bigbreak
\textsc{{kinder}}
\begin{itemize}
\item Tobias Galm [18.02.1989]
\item Christian Galm [01.08.1990]
\item Alexander Galm [12.04.1994]
\item Sebastian Galm [29.05.1995]
\end{itemize}
\medbreak
\end{person}

\begin{person}[
    surname = {Galm},
    givenname = {Konrad},
    suffix = {1958},
    label = {@I23@},
    filename = {Konrad Galm (1958)}
    ]

\begin{tabular}{cl}
\geboren & 19. Februar 1958 in Langenelz\\
\geheiratet & 07. September 1985 in Bödigheim mit Helga Heckmann \\
\end{tabular}\\
\medbreak
\textsc{vater}: \hyperref[@I4@]{Karl Galm} [27.07.1924--03.01.2015 (6 Kinder)]\\
\textsc{mutter}: \hyperref[@I5@]{Aloisia Lina Schölch} [12.10.1925--07.09.1993 (6 Kinder)]
\medbreak
\textsc{{geschwister}}
\begin{itemize}
\item \hyperref[@I20@]{Philipp Galm} [22.04.1952 (2 Kinder)]
\item \hyperref[@I21@]{Herbert Galm} [04.10.1953 (2 Kinder)]
\item \hyperref[@I22@]{Hugo Galm} [11.01.1957 (4 Kinder)]
\item \hyperref[@I24@]{Helga Rita Galm} [07.01.1964 (3 Kinder)]
\item \hyperref[@I3@]{Ingrid Juliana Galm} [23.09.1965 (2 Kinder)]
\end{itemize}
\bigbreak
\textsc{{kinder}}
\begin{itemize}
\item Kathrin Galm [31.07.1990]
\item Nicole Galm [05.01.1994]
\end{itemize}
\medbreak
\end{person}

\begin{person}[
    surname = {Galm},
    givenname = {Helga Rita},
    suffix = {1964},
    label = {@I24@},
    filename = {Helga Schoelch (1964)}
    ]

\begin{tabular}{cl}
\geboren & 07. Januar 1964 in Langenelz\\
\geheiratet & 17. September 1986 in Mudau mit Ottmar Schölch \\
\end{tabular}\\
\medbreak
\textsc{vater}: \hyperref[@I4@]{Karl Galm} [27.07.1924--03.01.2015 (6 Kinder)]\\
\textsc{mutter}: \hyperref[@I5@]{Aloisia Lina Schölch} [12.10.1925--07.09.1993 (6 Kinder)]
\medbreak
\textsc{{geschwister}}
\begin{itemize}
\item \hyperref[@I20@]{Philipp Galm} [22.04.1952 (2 Kinder)]
\item \hyperref[@I21@]{Herbert Galm} [04.10.1953 (2 Kinder)]
\item \hyperref[@I22@]{Hugo Galm} [11.01.1957 (4 Kinder)]
\item \hyperref[@I23@]{Konrad Galm} [19.02.1958 (2 Kinder)]
\item \hyperref[@I3@]{Ingrid Juliana Galm} [23.09.1965 (2 Kinder)]
\end{itemize}
\bigbreak
\textsc{{kinder}}
\begin{itemize}
\item Simon Schölch [28.11.1990]
\item Christoph Schölch [31.10.1993]
\item Miriam Alise Schölch [30.08.1995]
\end{itemize}
\medbreak
\end{person}


\addchap{Ur-Gro"seltern}



\addsec{Heinrich Scheuermann  \& Anna Zimmermann }


\begin{person}[
    surname = {Scheuermann},
    givenname = {Heinrich},
    suffix = {1886--1929},
    label = {@I13@}
    ]

\begin{tabular}{cl}
\geboren & 13. Mai 1886 in Mudau\\
\geheiratet & 03. Juni 1912 in Mudau mit Anna Zimmermann \\
\gestorben & 18. April 1929 in Buchen\\
\end{tabular}\\
\medbreak
\textsc{vater}: \hyperref[@I389@]{Johann Valentin Scheuermann} [07.12.1842--12.12.1910 (6 Kinder)]\\
\textsc{mutter}: \hyperref[@I390@]{Margareta Schäfer} [05.10.1852--11.11.1929 (6 Kinder)]
\medbreak
\textsc{{geschwister}}
\begin{itemize}
\item \hyperref[@I1270@]{Johann Valentin Scheuermann} [12.10.1876--11.12.1942 (1 Kind)]
\item \hyperref[@I1213@]{Helena Scheuermann} [09.09.1878]
\item \hyperref[@I1272@]{Josefa/Bertha Scheuermann} [13.03.1881]
\item \hyperref[@I965@]{Otto Scheuermann} [10.11.1883--19.11.1953 (3 Kinder)]
\item \hyperref[@I964@]{Anna Scheuermann} [29.12.1888--09.03.1952]
\end{itemize}
\bigbreak
\textsc{{kinder}}
\begin{itemize}
\item \hyperref[@I72@]{Anna Scheuermann} [02.07.1913--28.09.2000 (4 Kinder)]
\item \hyperref[@I73@]{Heinrich Scheuermann} [30.08.1914--16.12.1944]
\item \hyperref[@I74@]{Josef Scheuermann} [12.11.1919--27.05.1997 (2 Kinder)]
\item \hyperref[@I11@]{Erwin Otto Scheuermann} [18.05.1923--14.11.1982 (4 Kinder)]
\item \hyperref[@I1208@]{Maria Scheuermann} [03.12.1924--03.12.1924]
\item \hyperref[@I75@]{Alfons Scheuermann} [18.06.1926--29.09.2005 (2 Kinder)]
\end{itemize}
\medbreak
\textsc{{quellen}}
\begin{enumerate}[label={[\arabic*]}]
\item Mudau Heiratsbuch 1911–1919, Heiratsregister 1912, Nr. 6
\item Sterberegister Buchen 1929, Nr. 11
\item Ahnentafel USA
\item \href{https://www.familysearch.org/tree/person/details/L1TC-BHN}{FamilySearch, ID: L1TC-BHN}
\end{enumerate}

\end{person}

\begin{person}[
    surname = {Zimmermann},
    givenname = {Anna},
    suffix = {1887--1969},
    label = {@I14@},
    filename = {Anna Zimmermann (1887)}
    ]

\begin{tabular}{cl}
\geboren & 07. Dezember 1887 in Laudenberg\\
\geheiratet & 03. Juni 1912 in Mudau mit Heinrich Scheuermann \\
\gestorben & 18. Juni 1969 in Mudau\\
\end{tabular}\\
\medbreak
\textsc{vater}: \hyperref[@I392@]{Valentin Zimmermann} [24.02.1856--16.03.1923 (10 Kinder)]\\
\textsc{mutter}: \hyperref[@I391@]{Thekla Maria Albert} [23.09.1863--24.11.1917 (6 Kinder)]
\medbreak
\textsc{{geschwister}}
\begin{itemize}
\item \hyperref[@I975@]{Maria Zimmermann} [14.11.1879]
\item \hyperref[@I974@]{Rosa Zimmermann} [06.03.1882 (3 Kinder)]
\item \hyperref[@I1358@]{Valentin Zimmermann} [08.10.1883--02.01.1886]
\item \hyperref[@I973@]{Pius Zimmermann} [12.10.1885--15.04.1968 (4 Kinder)]
\item \hyperref[@I360@]{Thekla Zimmermann} [14.10.1891--24.01.1927 (4 Kinder)]
\item \hyperref[@I968@]{Adolf Zimmermann} [05.03.1894--15.02.1971 (3 Kinder)]
\item \hyperref[@I967@]{Emilie Zimmermann} [02.10.1897--15.09.1983 (4 Kinder)]
\item \hyperref[@I966@]{Elisabeth Zimmermann} [01.01.1904--21.04.1970 (3 Kinder)]
\item \hyperref[@I969@]{Karl Zimmermann} [... (2 Kinder)]
\end{itemize}
\bigbreak
\textsc{{kinder}}
\begin{itemize}
\item \hyperref[@I72@]{Anna Scheuermann} [02.07.1913--28.09.2000 (4 Kinder)]
\item \hyperref[@I73@]{Heinrich Scheuermann} [30.08.1914--16.12.1944]
\item \hyperref[@I74@]{Josef Scheuermann} [12.11.1919--27.05.1997 (2 Kinder)]
\item \hyperref[@I11@]{Erwin Otto Scheuermann} [18.05.1923--14.11.1982 (4 Kinder)]
\item \hyperref[@I1208@]{Maria Scheuermann} [03.12.1924--03.12.1924]
\item \hyperref[@I75@]{Alfons Scheuermann} [18.06.1926--29.09.2005 (2 Kinder)]
\end{itemize}
\medbreak
\textsc{{quellen}}
\begin{enumerate}[label={[\arabic*]}]
\item Geburtenregister Laudenberg 1887, Nr. 14
\item Mudau Heiratsbuch 1911–1919, Heiratsregister 1912, Nr. 6
\item Mudau Sterbebuch 1968–1975, Sterberegister 1969, Nr. 10
\item Ahnentafel USA
\item \href{https://www.familysearch.org/tree/person/details/L1TC-BHG}{FamilySearch, ID: L1TC-BHG}
\end{enumerate}

\end{person}

\begin{person}[
    surname = {Scheuermann},
    givenname = {Anna},
    suffix = {1913--2000},
    label = {@I72@},
    filename = {Anna Speth (1913)}
    ]

\begin{tabular}{cl}
\geboren & 02. Juli 1913 in Mudau\\
\geheiratet & 18. Februar 1939 in Mudau mit Vinzenz Speth \\
\gestorben & 28. September 2000 in Mudau\\
\bestattet &  in Mudau\\
\end{tabular}\\
\medbreak
\textsc{vater}: \hyperref[@I13@]{Heinrich Scheuermann} [13.05.1886--18.04.1929 (6 Kinder)]\\
\textsc{mutter}: \hyperref[@I14@]{Anna Zimmermann} [07.12.1887--18.06.1969 (6 Kinder)]
\medbreak
\textsc{{geschwister}}
\begin{itemize}
\item \hyperref[@I73@]{Heinrich Scheuermann} [30.08.1914--16.12.1944]
\item \hyperref[@I74@]{Josef Scheuermann} [12.11.1919--27.05.1997 (2 Kinder)]
\item \hyperref[@I11@]{Erwin Otto Scheuermann} [18.05.1923--14.11.1982 (4 Kinder)]
\item \hyperref[@I1208@]{Maria Scheuermann} [03.12.1924--03.12.1924]
\item \hyperref[@I75@]{Alfons Scheuermann} [18.06.1926--29.09.2005 (2 Kinder)]
\end{itemize}
\bigbreak
\textsc{{kinder}}
\begin{itemize}
\item Karlheinz Speth [06.02.1940]
\item Friedbert Speth [26.05.1947]
\item Annerose Speth [26.05.1947--13.07.1947]
\item Kuniberth Speth [23.05.1951]
\end{itemize}
\medbreak
\textsc{{quellen}}
\begin{enumerate}[label={[\arabic*]}]
\item Mudau Geburtenbuch 1911–1919, Geburtenregister 1913, Nr. 24
\item Mudau Heiratsbuch 1938–1947, Heiratsregister 1939, Nr. 3 
\item Mudau Sterbebuch 1996–2002, Sterberegister 2000, Nr. 12
\item \href{http://grabsteine.genealogy.net/tomb.php?cem=3902&tomb=15&b=&lang=de}{genealogy.net Grabstein Projekt, Friedhof Mudau}
\end{enumerate}

\end{person}

\begin{person}[
    surname = {Scheuermann},
    givenname = {Heinrich},
    suffix = {1914--1944},
    label = {@I73@},
    filename = {Heinrich Scheuermann (1914)}
    ]

\begin{tabular}{cl}
\geboren & 30. August 1914 in Mudau\\
\gestorben & 16. Dezember 1944 in Ungarn\\
\end{tabular}\\
\medbreak
\textsc{vater}: \hyperref[@I13@]{Heinrich Scheuermann} [13.05.1886--18.04.1929 (6 Kinder)]\\
\textsc{mutter}: \hyperref[@I14@]{Anna Zimmermann} [07.12.1887--18.06.1969 (6 Kinder)]
\medbreak
\textsc{{geschwister}}
\begin{itemize}
\item \hyperref[@I72@]{Anna Scheuermann} [02.07.1913--28.09.2000 (4 Kinder)]
\item \hyperref[@I74@]{Josef Scheuermann} [12.11.1919--27.05.1997 (2 Kinder)]
\item \hyperref[@I11@]{Erwin Otto Scheuermann} [18.05.1923--14.11.1982 (4 Kinder)]
\item \hyperref[@I1208@]{Maria Scheuermann} [03.12.1924--03.12.1924]
\item \hyperref[@I75@]{Alfons Scheuermann} [18.06.1926--29.09.2005 (2 Kinder)]
\end{itemize}
\bigbreak
\textsc{anmerkung}\\
gestorben in Natroszentimke?
\medbreak
\textsc{{quellen}}
\begin{enumerate}[label={[\arabic*]}]
\item Mudau Geburtenbuch 1911–1919, Geburtenregister 1914, Nr. 22
\item Mudau Sterbebuch 1945–1955, Sterberegister 1946, Nr. 33
\end{enumerate}

\end{person}

\begin{person}[
    surname = {Scheuermann},
    givenname = {Josef},
    suffix = {1919--1997},
    label = {@I74@},
    filename = {Josef Scheuermann (1919)}
    ]

\begin{tabular}{cl}
\geboren & 12. November 1919 in Mudau\\
\geheiratet & 13. Mai 1949 in Mudau mit Emma Knapp \\
 & 25. Juni 1965 in Mudau mit Josefa Klutz \\
\gestorben & 27. Mai 1997 in Heidelberg\\
\bestattet &  in Mudau\\
\end{tabular}\\
\medbreak
\textsc{vater}: \hyperref[@I13@]{Heinrich Scheuermann} [13.05.1886--18.04.1929 (6 Kinder)]\\
\textsc{mutter}: \hyperref[@I14@]{Anna Zimmermann} [07.12.1887--18.06.1969 (6 Kinder)]
\medbreak
\textsc{{geschwister}}
\begin{itemize}
\item \hyperref[@I72@]{Anna Scheuermann} [02.07.1913--28.09.2000 (4 Kinder)]
\item \hyperref[@I73@]{Heinrich Scheuermann} [30.08.1914--16.12.1944]
\item \hyperref[@I11@]{Erwin Otto Scheuermann} [18.05.1923--14.11.1982 (4 Kinder)]
\item \hyperref[@I1208@]{Maria Scheuermann} [03.12.1924--03.12.1924]
\item \hyperref[@I75@]{Alfons Scheuermann} [18.06.1926--29.09.2005 (2 Kinder)]
\end{itemize}
\bigbreak
\textsc{{kinder}}
\begin{itemize}
\item Ekkehard Scheuermann [1953--1973]
\item Margot Scheuermann [...]
\end{itemize}
\medbreak
\textsc{{quellen}}
\begin{enumerate}[label={[\arabic*]}]
\item Mudau Geburtenbuch 1911–1919, Geburtenregister Mudau 1919, Nr. 27
\item Mudau Heiratsbuch 1948–1951, Heiratsregister 1949, Nr. 4
\item Mudau Heiratsbuch 1963–1967, Heiratsregister 1965, Nr. 8
\item Sterberegister Heidelberg 1997, Nr. 1444
\item \href{http://grabsteine.genealogy.net/tomb.php?cem=3902&tomb=295&b=&lang=de}{genealogy.net Grabstein Projekt, Friedhof Mudau}
\end{enumerate}

\end{person}

\begin{person}[
    surname = {Scheuermann},
    givenname = {Maria},
    suffix = {1924--1924},
    label = {@I1208@}
    ]

\begin{tabular}{cl}
\geboren & 03. Dezember 1924 in Mudau\\
\gestorben & 03. Dezember 1924 in Mudau\\
\end{tabular}\\
\medbreak
\textsc{vater}: \hyperref[@I13@]{Heinrich Scheuermann} [13.05.1886--18.04.1929 (6 Kinder)]\\
\textsc{mutter}: \hyperref[@I14@]{Anna Zimmermann} [07.12.1887--18.06.1969 (6 Kinder)]
\medbreak
\textsc{{geschwister}}
\begin{itemize}
\item \hyperref[@I72@]{Anna Scheuermann} [02.07.1913--28.09.2000 (4 Kinder)]
\item \hyperref[@I73@]{Heinrich Scheuermann} [30.08.1914--16.12.1944]
\item \hyperref[@I74@]{Josef Scheuermann} [12.11.1919--27.05.1997 (2 Kinder)]
\item \hyperref[@I11@]{Erwin Otto Scheuermann} [18.05.1923--14.11.1982 (4 Kinder)]
\item \hyperref[@I75@]{Alfons Scheuermann} [18.06.1926--29.09.2005 (2 Kinder)]
\end{itemize}
\bigbreak
\textsc{{quellen}}
\begin{enumerate}[label={[\arabic*]}]
\item Mudau Geburtenbuch 1920–1927, Geburtenregister 1924, Nr. 35
\item Mudau Sterbebuch 1920–1928, Sterberegister 1924, Nr. 16
\item Ahnenblatt USA
\end{enumerate}

\end{person}

\begin{person}[
    surname = {Scheuermann},
    givenname = {Alfons},
    suffix = {1926--2005},
    label = {@I75@}
    ]

\begin{tabular}{cl}
\geboren & 18. Juni 1926 in Mudau\\
\geheiratet & 20. Mai 1950 in Mudau mit Maria Elisabeth Berberich \\
\gestorben & 29. September 2005 in Mudau\\
\bestattet &  in Mudau\\
\end{tabular}\\
\medbreak
\textsc{vater}: \hyperref[@I13@]{Heinrich Scheuermann} [13.05.1886--18.04.1929 (6 Kinder)]\\
\textsc{mutter}: \hyperref[@I14@]{Anna Zimmermann} [07.12.1887--18.06.1969 (6 Kinder)]
\medbreak
\textsc{{geschwister}}
\begin{itemize}
\item \hyperref[@I72@]{Anna Scheuermann} [02.07.1913--28.09.2000 (4 Kinder)]
\item \hyperref[@I73@]{Heinrich Scheuermann} [30.08.1914--16.12.1944]
\item \hyperref[@I74@]{Josef Scheuermann} [12.11.1919--27.05.1997 (2 Kinder)]
\item \hyperref[@I11@]{Erwin Otto Scheuermann} [18.05.1923--14.11.1982 (4 Kinder)]
\item \hyperref[@I1208@]{Maria Scheuermann} [03.12.1924--03.12.1924]
\end{itemize}
\bigbreak
\textsc{{kinder}}
\begin{itemize}
\item Günther Scheuermann [...]
\item Elfriede Scheuermann [...]
\end{itemize}
\medbreak
\textsc{{quellen}}
\begin{enumerate}[label={[\arabic*]}]
\item Mudau Heiratsbuch 1948–1951, Heiratsregister 1950, Nr. 6
\item Ahnentafel USA
\item \href{http://grabsteine.genealogy.net/tomb.php?cem=3902&tomb=416&b=&lang=de}{genealogy.net Grabstein Projekt, Friedhof Mudau}
\end{enumerate}

\end{person}


\addsec{Otto Röckel  \& Maria Anna Mechler }


\begin{person}[
    surname = {Röckel},
    givenname = {Otto},
    suffix = {1882--1965},
    label = {@I15@},
    filename = {Otto Röckel (1882)}
    ]

\begin{tabular}{cl}
\geboren & 20. November 1882 in Langenelz\\
\taufe & 20. November 1882 in Mudau\\
\geheiratet & 06. August 1919 in Langenelz mit Maria Anna Mechler \\
\gestorben & 14. Juni 1965 in Langenelz\\
\bestattet &  in Mudau\\
\end{tabular}\\
\medbreak
\textsc{vater}: \hyperref[@I386@]{Joseph Michael Röckel} [19.03.1839--03.10.1888 (9 Kinder)]\\
\textsc{mutter}: \hyperref[@I387@]{Rosa Noe} [15.06.1857--28.08.1920 (10 Kinder)]
\medbreak
\textsc{{geschwister}}
\begin{itemize}
\item \hyperref[@I1268@]{Theodor Röckel} [03.06.1875--24.02.1876]
\item \hyperref[@I1269@]{Emma Röckel} [28.06.1876]
\item \hyperref[@I489@]{Michael Röckel} [21.07.1877]
\item \hyperref[@I1154@]{Ida Röckel} [12.12.1879--09.12.1955 (5 Kinder)]
\item \hyperref[@I954@]{Rosa Röckel} [28.02.1879--29.1880]
\item \hyperref[@I955@]{Anna Röckel} [05.05.1881--12.05.1882]
\item \hyperref[@I956@]{Josef Röckel} [07.09.1884--21.10.1884]
\item \hyperref[@I472@]{Wilhelm Röckel} [07.01.1887--25.11.1968 (7 Kinder)]
\item \hyperref[@I960@]{Franz Karl Müller} [21.10.1891--08.1918]
\item \hyperref[@I961@]{Sebastian Müller} [19.01.1893--12.11.1915]
\item \hyperref[@I481@]{Maria Müller} [07.09.1895--27.11.1972 (4 Kinder)]
\item \hyperref[@I962@]{Rosa Müller} [02.04.1897--12.07.1979]
\item \hyperref[@I963@]{Pius Müller} [28.01.1899]
\end{itemize}
\bigbreak
\textsc{{kinder}}
\begin{itemize}
\item \hyperref[@I70@]{Karl Röckel} [08.07.1920--10.08.2003 (5 Kinder)]
\item \hyperref[@I68@]{Otto Röckel} [09.07.1922--21.01.1944]
\item \hyperref[@I69@]{Walter Röckel} [15.02.1924--24.05.2010]
\item \hyperref[@I71@]{Alfred Röckel} [21.04.1925--16.06.1944]
\item \hyperref[@I12@]{Maria Rita Röckel} [06.01.1927--19.07.2012 (4 Kinder)]
\item \hyperref[@I67@]{Elisabeth Röckel} [14.04.1929--12.10.2015 (2 Kinder)]
\end{itemize}
\medbreak
\textsc{{quellen}}
\begin{enumerate}[label={[\arabic*]}]
\item Langenelz Standesbuch 1880–1884, Geburtenregister 1882, Nr. 9
\item Langenelz Heiratsbuch 1911–1927, Heiratsregister 1919, Nr. 4
\item Langenelz Sterbebuch 1946–1973, Sterberegister 1965, Nr. 4
\item \href{https://www.familysearch.org/tree/person/details/L1TQ-ND6}{FamilySearch, ID: L1TQ-ND6}
\end{enumerate}

\end{person}

\begin{person}[
    surname = {Mechler},
    givenname = {Maria Anna},
    suffix = {1889--1966},
    label = {@I16@},
    filename = {Maria Mechler (1889)}
    ]

\begin{tabular}{cl}
\geboren & 15. August 1889 in Mudau\\
\geheiratet & 06. August 1919 in Langenelz mit Otto Röckel \\
\gestorben & 25. Oktober 1966 in Langenelz\\
\bestattet &  in Mudau\\
\end{tabular}\\
\medbreak
\textsc{vater}: \hyperref[@I426@]{Valentin Mechler} [26.05.1855--04.01.1928 (4 Kinder)]\\
\textsc{mutter}: \hyperref[@I388@]{Eva Katharina Schäfer} [06.10.1855--21.04.1942 (4 Kinder)]
\medbreak
\textsc{{geschwister}}
\begin{itemize}
\item \hyperref[@I1261@]{Wilhelm Mechler} [17.02.1887--06.04.1956 (1 Kind)]
\item \hyperref[@I480@]{Karl Mechler} [15.08.1889--30.12.1968 (4 Kinder)]
\item \hyperref[@I1267@]{Rosa Theresia Mechler} [19.08.1894--06.02.1982]
\end{itemize}
\bigbreak
\textsc{{kinder}}
\begin{itemize}
\item \hyperref[@I70@]{Karl Röckel} [08.07.1920--10.08.2003 (5 Kinder)]
\item \hyperref[@I68@]{Otto Röckel} [09.07.1922--21.01.1944]
\item \hyperref[@I69@]{Walter Röckel} [15.02.1924--24.05.2010]
\item \hyperref[@I71@]{Alfred Röckel} [21.04.1925--16.06.1944]
\item \hyperref[@I12@]{Maria Rita Röckel} [06.01.1927--19.07.2012 (4 Kinder)]
\item \hyperref[@I67@]{Elisabeth Röckel} [14.04.1929--12.10.2015 (2 Kinder)]
\end{itemize}
\medbreak
\textsc{{quellen}}
\begin{enumerate}[label={[\arabic*]}]
\item Mudau Standesbuch 1888–1890, Geburtenregister 1889, Nr. 26
\item Langenelz Heiratsbuch 1911–1927, Heiratsregister 1919, Nr. 4
\end{enumerate}

\end{person}

\begin{person}[
    surname = {Röckel},
    givenname = {Karl},
    suffix = {1920--2003},
    label = {@I70@},
    filename = {Karl Röckel (1920)}
    ]

\begin{tabular}{cl}
\geboren & 08. Juli 1920 in Langenelz\\
\geheiratet & 14. Mai 1952 in Langenelz mit Gertrud Schäfer \\
\gestorben & 10. August 2003 in Langenelz\\
\bestattet &  in Langenelz\\
\end{tabular}\\
\medbreak
\textsc{vater}: \hyperref[@I15@]{Otto Röckel} [20.11.1882--14.06.1965 (6 Kinder)]\\
\textsc{mutter}: \hyperref[@I16@]{Maria Anna Mechler} [15.08.1889--25.10.1966 (6 Kinder)]
\medbreak
\textsc{{geschwister}}
\begin{itemize}
\item \hyperref[@I68@]{Otto Röckel} [09.07.1922--21.01.1944]
\item \hyperref[@I69@]{Walter Röckel} [15.02.1924--24.05.2010]
\item \hyperref[@I71@]{Alfred Röckel} [21.04.1925--16.06.1944]
\item \hyperref[@I12@]{Maria Rita Röckel} [06.01.1927--19.07.2012 (4 Kinder)]
\item \hyperref[@I67@]{Elisabeth Röckel} [14.04.1929--12.10.2015 (2 Kinder)]
\end{itemize}
\bigbreak
\textsc{{kinder}}
\begin{itemize}
\item Annemarie Röckel [07.03.1953]
\item Irmgard Röckel [25.01.1955 (2 Kinder)]
\item Hannelore Röckel [25.01.1955 (2 Kinder)]
\item Gudrun Röckel [13.02.1958 (4 Kinder)]
\item Margot Röckel [11.02.1960]
\end{itemize}
\medbreak
\textsc{anmerkung}\\
Baschle (von Sebastian Müller)
\medbreak
\textsc{{quellen}}
\begin{enumerate}[label={[\arabic*]}]
\item Langenelz Geburtenbuch 1911–1927, Geburtenregister 1920, Nr. 3
\item Langenelz Heiratsbuch 1946–1973, Heiratsregister 1952, Nr. 1
\item \href{http://grabsteine.genealogy.net/tomb.php?cem=3810&tomb=5}{genealogy.net Grabstein Projekt, Friedhof Langenelz}
\end{enumerate}

\end{person}

\begin{person}[
    surname = {Röckel},
    givenname = {Otto},
    suffix = {1922--1944},
    label = {@I68@},
    filename = {Otto Röckel (1922)}
    ]

\begin{tabular}{cl}
\geboren & 09. Juli 1922 in Langenelz\\
\gestorben & 21. Januar 1944 in Russland\\
\end{tabular}\\
\medbreak
\textsc{vater}: \hyperref[@I15@]{Otto Röckel} [20.11.1882--14.06.1965 (6 Kinder)]\\
\textsc{mutter}: \hyperref[@I16@]{Maria Anna Mechler} [15.08.1889--25.10.1966 (6 Kinder)]
\medbreak
\textsc{{geschwister}}
\begin{itemize}
\item \hyperref[@I70@]{Karl Röckel} [08.07.1920--10.08.2003 (5 Kinder)]
\item \hyperref[@I69@]{Walter Röckel} [15.02.1924--24.05.2010]
\item \hyperref[@I71@]{Alfred Röckel} [21.04.1925--16.06.1944]
\item \hyperref[@I12@]{Maria Rita Röckel} [06.01.1927--19.07.2012 (4 Kinder)]
\item \hyperref[@I67@]{Elisabeth Röckel} [14.04.1929--12.10.2015 (2 Kinder)]
\end{itemize}
\bigbreak
\textsc{{quellen}}
\begin{enumerate}[label={[\arabic*]}]
\item Langenelz Geburtenbuch 1911–1927, Geburtenregister 1922, Nr. 7
\end{enumerate}

\end{person}

\begin{person}[
    surname = {Röckel},
    givenname = {Walter},
    suffix = {1924--2010},
    label = {@I69@},
    filename = {Walter Roeckel (1924)}
    ]

\begin{tabular}{cl}
\geboren & 15. Februar 1924 in Langenelz\\
\gestorben & 24. Mai 2010 in Langenelz\\
\bestattet &  in Langenelz\\
\end{tabular}\\
\medbreak
\textsc{vater}: \hyperref[@I15@]{Otto Röckel} [20.11.1882--14.06.1965 (6 Kinder)]\\
\textsc{mutter}: \hyperref[@I16@]{Maria Anna Mechler} [15.08.1889--25.10.1966 (6 Kinder)]
\medbreak
\textsc{{geschwister}}
\begin{itemize}
\item \hyperref[@I70@]{Karl Röckel} [08.07.1920--10.08.2003 (5 Kinder)]
\item \hyperref[@I68@]{Otto Röckel} [09.07.1922--21.01.1944]
\item \hyperref[@I71@]{Alfred Röckel} [21.04.1925--16.06.1944]
\item \hyperref[@I12@]{Maria Rita Röckel} [06.01.1927--19.07.2012 (4 Kinder)]
\item \hyperref[@I67@]{Elisabeth Röckel} [14.04.1929--12.10.2015 (2 Kinder)]
\end{itemize}
\bigbreak
\textsc{{quellen}}
\begin{enumerate}[label={[\arabic*]}]
\item \href{http://grabsteine.genealogy.net/tomb.php?cem=3810&tomb=3&b=&lang=de}{genealogy.net Grabstein Projekt, Friedhof Langenelz}
\item Sterbebild
\end{enumerate}

\end{person}

\begin{person}[
    surname = {Röckel},
    givenname = {Alfred},
    suffix = {1925--1944},
    label = {@I71@},
    filename = {Alfred Röckel (1925)}
    ]

\begin{tabular}{cl}
\geboren & 21. April 1925 in Langenelz\\
\gestorben & 16. Juni 1944 in Frankreich\\
\end{tabular}\\
\medbreak
\textsc{vater}: \hyperref[@I15@]{Otto Röckel} [20.11.1882--14.06.1965 (6 Kinder)]\\
\textsc{mutter}: \hyperref[@I16@]{Maria Anna Mechler} [15.08.1889--25.10.1966 (6 Kinder)]
\medbreak
\textsc{{geschwister}}
\begin{itemize}
\item \hyperref[@I70@]{Karl Röckel} [08.07.1920--10.08.2003 (5 Kinder)]
\item \hyperref[@I68@]{Otto Röckel} [09.07.1922--21.01.1944]
\item \hyperref[@I69@]{Walter Röckel} [15.02.1924--24.05.2010]
\item \hyperref[@I12@]{Maria Rita Röckel} [06.01.1927--19.07.2012 (4 Kinder)]
\item \hyperref[@I67@]{Elisabeth Röckel} [14.04.1929--12.10.2015 (2 Kinder)]
\end{itemize}
\bigbreak
\textsc{anmerkung}\\
gestorben in Département Calvados, vlt. Mondeville 
\medbreak
\textsc{{quellen}}
\begin{enumerate}[label={[\arabic*]}]
\item Langenelz Geburtenbuch 1911–1927, Geburtenregister 1925, Nr. 1
\item Mudau Sterbebuch 1945–1955, Sterberegister 1945, Nr. 3
\end{enumerate}

\end{person}

\begin{person}[
    surname = {Röckel},
    givenname = {Elisabeth},
    suffix = {1929--2015},
    label = {@I67@},
    filename = {Elisabeth Henn (1929)}
    ]

\begin{tabular}{cl}
\geboren & 14. April 1929 in Langenelz\\
\geheiratet &  mit Otto Henn \\
\gestorben & 12. Oktober 2015 in Laudenberg\\
\bestattet &  in Laudenberg\\
\end{tabular}\\
\medbreak
\textsc{vater}: \hyperref[@I15@]{Otto Röckel} [20.11.1882--14.06.1965 (6 Kinder)]\\
\textsc{mutter}: \hyperref[@I16@]{Maria Anna Mechler} [15.08.1889--25.10.1966 (6 Kinder)]
\medbreak
\textsc{{geschwister}}
\begin{itemize}
\item \hyperref[@I70@]{Karl Röckel} [08.07.1920--10.08.2003 (5 Kinder)]
\item \hyperref[@I68@]{Otto Röckel} [09.07.1922--21.01.1944]
\item \hyperref[@I69@]{Walter Röckel} [15.02.1924--24.05.2010]
\item \hyperref[@I71@]{Alfred Röckel} [21.04.1925--16.06.1944]
\item \hyperref[@I12@]{Maria Rita Röckel} [06.01.1927--19.07.2012 (4 Kinder)]
\end{itemize}
\bigbreak
\textsc{{kinder}}
\begin{itemize}
\item Gerda Henn [... (1 Kind)]
\item Christa Henn [um 1959 (2 Kinder)]
\end{itemize}
\medbreak
\textsc{{quellen}}
\begin{enumerate}[label={[\arabic*]}]
\item \href{http://grabsteine.genealogy.net/tomb.php?cem=3609&tomb=72&b=&lang=de}{genealogy.net Grabstein Projekt, Friedhof Laudenberg}
\item Starbebild
\end{enumerate}

\end{person}


\addsec{Julius Galm  \& Margaretha Schüßler }


\begin{person}[
    surname = {Galm},
    givenname = {Julius},
    suffix = {1883--1929},
    label = {@I7@},
    filename = {Julius Galm (1883)}
    ]

\begin{tabular}{cl}
\geboren & 28. Februar 1883 in Langenelz\\
\geheiratet & 15. Mai 1919 in Langenelz mit Margaretha Schüßler \\
\gestorben & 23. Mai 1929 in Unterscheidental\\
\bestattet &  in Mudau\\
\end{tabular}\\
\medbreak
\textsc{vater}: \hyperref[@I144@]{Franz Karl Galm} [05.09.1854--06.12.1934 (10 Kinder)]\\
\textsc{mutter}: \hyperref[@I145@]{Karolina Schwanninger} [14.09.1858--02.07.1926 (10 Kinder)]
\medbreak
\textsc{{geschwister}}
\begin{itemize}
\item \hyperref[@I163@]{Anna Galm} [04.05.1884--10.08.1955 (10 Kinder)]
\item \hyperref[@I164@]{Karl Galm} [06.10.1886--13.10.1963]
\item \hyperref[@I165@]{Karoline Galm} [06.06.1888--13.07.1951]
\item \hyperref[@I2031@]{Joseph Galm} [23.12.1890--16.01.1891]
\item \hyperref[@I166@]{Philipp Galm} [28.12.1891--25.02.1953 (3 Kinder)]
\item \hyperref[@I167@]{Anton Galm} [02.12.1893--08.05.1915]
\item \hyperref[@I2032@]{Wilhelm Galm} [20.08.1895--24.09.1895]
\item \hyperref[@I168@]{Ida Galm} [31.07.1898--19.10.1988 (3 Kinder)]
\item \hyperref[@I169@]{Maria Galm} [04.09.1899--06.12.1962 (3 Kinder)]
\end{itemize}
\bigbreak
\textsc{{kinder}}
\begin{itemize}
\item \hyperref[@I56@]{Philipp Galm} [19.02.1920--vor 1945]
\item \hyperref[@I52@]{Juliane Galm} [26.03.1921--05.04.2019 (5 Kinder)]
\item \hyperref[@I53@]{Maria Galm} [14.07.1922--15.02.1995 (7 Kinder)]
\item \hyperref[@I4@]{Karl Galm} [27.07.1924--03.01.2015 (6 Kinder)]
\item \hyperref[@I57@]{Arthur Galm} [02.10.1925--1944]
\item \hyperref[@I54@]{Leo Galm} [07.09.1927--24.05.2003 (4 Kinder)]
\item \hyperref[@I55@]{Georg Galm} [16.08.1929 (3 Kinder)]
\end{itemize}
\medbreak
\textsc{anmerkung}\\
trat im Juni 1915 in den Weltkrieg. Er war zuerst Wachman. Zuerst musste er nach Russland in die Gegend der Düna und ?jaman. Er war Wachmann in der ... Gefangenenlager in dem [...]

Er hatte große Liebe zur Landwirtschaft, besonders zum Holzmachen im Wald

Erzählung von Helga/Maria:
Hochzeit vor dem Krieg arrangiert. Kam traumatisiert aus dem Krieg und wollte nicht mehr heiraten. Eltern drängten zu Hochzeit. Hatte sich beim Arbeiten extrem beeilt und Angst nicht fertig zu werden.
\medbreak
\textsc{{quellen}}
\begin{enumerate}[label={[\arabic*]}]
\item Langenelz Heiratsbuch 1911–1927, Heiratsregister 1919, Nr. 1
\item Unterscheidental Sterberegister 1870–1935, Sterberegister Unterscheidental 1929, Nr. 5
\item Ahnentafel Alfons Bauer
\item \href{https://www.familysearch.org/tree/person/details/L51R-B53}{FamilySearch, ID: L51R-B53}
\end{enumerate}

\end{person}

\begin{person}[
    surname = {Schüßler},
    givenname = {Margaretha},
    suffix = {1894--1978},
    label = {@I8@},
    filename = {Margaretha Schuessler (1894)}
    ]

\begin{tabular}{cl}
\geboren & 11. April 1894 in Mörschenhardt\\
\geheiratet & 15. Mai 1919 in Langenelz mit Julius Galm \\
\gestorben & 26. Juli 1978 in Langenelz\\
\bestattet &  in Langenelz\\
\end{tabular}\\
\medbreak
\textsc{vater}: \hyperref[@I150@]{Johann Georg Schüßler} [26.08.1858--06.12.1937 (10 Kinder)]\\
\textsc{mutter}: \hyperref[@I151@]{Helena Gramlich} [28.01.1864--21.11.1943 (10 Kinder)]
\medbreak
\textsc{{geschwister}}
\begin{itemize}
\item \hyperref[@I170@]{Franz Schüssler} [24.11.1892--28.06.1915]
\item \hyperref[@I171@]{Konstantin Schüßler} [08.10.1895--27.07.1979 (5 Kinder)]
\item \hyperref[@I176@]{Helena Schüßler} [28.02.1897--07.04.1987 (6 Kinder)]
\item \hyperref[@I172@]{Johann Georg Schüßler} [25.08.1898--29.05.1984 (7 Kinder)]
\item \hyperref[@I174@]{Wilhelm Schüßler} [29.08.1900--16.08.1918]
\item \hyperref[@I1776@]{Maria Schüßler} [02.03.1902--02.03.1902]
\item \hyperref[@I177@]{Emma Wilhelmina Schüßler} [26.03.1903--25.08.1990 (4 Kinder)]
\item \hyperref[@I175@]{Anton Schüßler} [13.12.1904--04.06.1988 (3 Kinder)]
\item \hyperref[@I179@]{Juliana Regina Schüßler} [13.02.1907--25.05.1955 (3 Kinder)]
\end{itemize}
\bigbreak
\textsc{{kinder}}
\begin{itemize}
\item \hyperref[@I56@]{Philipp Galm} [19.02.1920--vor 1945]
\item \hyperref[@I52@]{Juliane Galm} [26.03.1921--05.04.2019 (5 Kinder)]
\item \hyperref[@I53@]{Maria Galm} [14.07.1922--15.02.1995 (7 Kinder)]
\item \hyperref[@I4@]{Karl Galm} [27.07.1924--03.01.2015 (6 Kinder)]
\item \hyperref[@I57@]{Arthur Galm} [02.10.1925--1944]
\item \hyperref[@I54@]{Leo Galm} [07.09.1927--24.05.2003 (4 Kinder)]
\item \hyperref[@I55@]{Georg Galm} [16.08.1929 (3 Kinder)]
\end{itemize}
\medbreak
\textsc{{quellen}}
\begin{enumerate}[label={[\arabic*]}]
\item Mörschenhardt Geburts-, Heirats- und Sterberegister 1890–1899, Geburtenregister 1894, Nr. 1
\item Langenelz Heiratsbuch 1911–1927, Heiratsregister 1919, Nr. 1
\item Mudau Sterbebuch 1976–1980, Sterberegister 1978, Nr. 10
\item \href{https://www.familysearch.org/tree/person/details/L51T-9XP}{FamilySearch, ID: L51T-9XP}
\end{enumerate}

\end{person}

\begin{person}[
    surname = {Galm},
    givenname = {Philipp},
    suffix = {1920--vor 1945},
    label = {@I56@},
    filename = {Philipp Galm (1920)}
    ]

\begin{tabular}{cl}
\geboren & 19. Februar 1920 in Langenelz\\
\taufe & 21. Februar 1920 in Mudau\\
\gestorben & vor 1945\\
\end{tabular}\\
\medbreak
\textsc{vater}: \hyperref[@I7@]{Julius Galm} [28.02.1883--23.05.1929 (7 Kinder)]\\
\textsc{mutter}: \hyperref[@I8@]{Margaretha Schüßler} [11.04.1894--26.07.1978 (7 Kinder)]
\medbreak
\textsc{{geschwister}}
\begin{itemize}
\item \hyperref[@I52@]{Juliane Galm} [26.03.1921--05.04.2019 (5 Kinder)]
\item \hyperref[@I53@]{Maria Galm} [14.07.1922--15.02.1995 (7 Kinder)]
\item \hyperref[@I4@]{Karl Galm} [27.07.1924--03.01.2015 (6 Kinder)]
\item \hyperref[@I57@]{Arthur Galm} [02.10.1925--1944]
\item \hyperref[@I54@]{Leo Galm} [07.09.1927--24.05.2003 (4 Kinder)]
\item \hyperref[@I55@]{Georg Galm} [16.08.1929 (3 Kinder)]
\end{itemize}
\bigbreak
\textsc{anmerkung}\\
gestorben zwischen 1939 und 1945
\medbreak
\textsc{{quellen}}
\begin{enumerate}[label={[\arabic*]}]
\item Heimatbuch Philipp Galm
\item \href{https://www.familysearch.org/tree/person/details/LYVN-3V7}{FamilySearch, ID: LYVN-3V7}
\end{enumerate}

\end{person}

\begin{person}[
    surname = {Galm},
    givenname = {Juliane},
    suffix = {1921--2019},
    label = {@I52@},
    filename = {Juliana Galm (1921)}
    ]

\begin{tabular}{cl}
\geboren & 26. März 1921 in Langenelz\\
\geheiratet & 18. Mai 1949 in Laudenberg mit Franz Müller \\
\gestorben & 05. April 2019 in Waldhausen\\
\bestattet & 17. April 2019 in Laudenberg\\
\end{tabular}\\
\medbreak
\textsc{vater}: \hyperref[@I7@]{Julius Galm} [28.02.1883--23.05.1929 (7 Kinder)]\\
\textsc{mutter}: \hyperref[@I8@]{Margaretha Schüßler} [11.04.1894--26.07.1978 (7 Kinder)]
\medbreak
\textsc{{geschwister}}
\begin{itemize}
\item \hyperref[@I56@]{Philipp Galm} [19.02.1920--vor 1945]
\item \hyperref[@I53@]{Maria Galm} [14.07.1922--15.02.1995 (7 Kinder)]
\item \hyperref[@I4@]{Karl Galm} [27.07.1924--03.01.2015 (6 Kinder)]
\item \hyperref[@I57@]{Arthur Galm} [02.10.1925--1944]
\item \hyperref[@I54@]{Leo Galm} [07.09.1927--24.05.2003 (4 Kinder)]
\item \hyperref[@I55@]{Georg Galm} [16.08.1929 (3 Kinder)]
\end{itemize}
\bigbreak
\textsc{{kinder}}
\begin{itemize}
\item Mechthild Müller [26.02.1950]
\item Konrad Müller [24.02.1951--08.10.2013]
\item Gerhard Müller [16.05.1953]
\item Ruppert Müller [16.12.1956--16.01.1994]
\item Bernarded Müller [25.11.1958]
\end{itemize}
\medbreak
\textsc{{quellen}}
\begin{enumerate}[label={[\arabic*]}]
\item \href{https://www.familysearch.org/tree/person/details/G9L8-66X}{FamilySearch, ID: G9L8-66X}
\end{enumerate}

\end{person}

\begin{person}[
    surname = {Galm},
    givenname = {Maria},
    suffix = {1922--1995},
    label = {@I53@},
    filename = {Maria Meixner (1922)}
    ]

\begin{tabular}{cl}
\geboren & 14. Juli 1922 in Langenelz\\
\geheiratet & 23. Mai 1950 in Hettigenbeuern mit Alois Meixner \\
\gestorben & 15. Februar 1995 in Hettigenbeuern\\
\bestattet & 17. Februar 1995 in Hettigenbeuern\\
\end{tabular}\\
\medbreak
\textsc{vater}: \hyperref[@I7@]{Julius Galm} [28.02.1883--23.05.1929 (7 Kinder)]\\
\textsc{mutter}: \hyperref[@I8@]{Margaretha Schüßler} [11.04.1894--26.07.1978 (7 Kinder)]
\medbreak
\textsc{{geschwister}}
\begin{itemize}
\item \hyperref[@I56@]{Philipp Galm} [19.02.1920--vor 1945]
\item \hyperref[@I52@]{Juliane Galm} [26.03.1921--05.04.2019 (5 Kinder)]
\item \hyperref[@I4@]{Karl Galm} [27.07.1924--03.01.2015 (6 Kinder)]
\item \hyperref[@I57@]{Arthur Galm} [02.10.1925--1944]
\item \hyperref[@I54@]{Leo Galm} [07.09.1927--24.05.2003 (4 Kinder)]
\item \hyperref[@I55@]{Georg Galm} [16.08.1929 (3 Kinder)]
\end{itemize}
\bigbreak
\textsc{{kinder}}
\begin{itemize}
\item Sigfried Meixner [11.07.1951]
\item Eugen Meixner [19.10.1952]
\item Mechthild Meixner [27.03.1954]
\item Erich Meixner [1956]
\item Arnold Meixner [1958]
\item Klemens Meixner [1959]
\item Klaus Meixner [1961]
\end{itemize}
\medbreak
\textsc{{quellen}}
\begin{enumerate}[label={[\arabic*]}]
\item \href{https://www.familysearch.org/tree/person/details/LYCV-4PJ}{FamilySearch, ID: LYCV-4PJ}
\end{enumerate}

\end{person}

\begin{person}[
    surname = {Galm},
    givenname = {Arthur},
    suffix = {1925--1944},
    label = {@I57@},
    filename = {Arthur Galm (1925)}
    ]

\begin{tabular}{cl}
\geboren & 02. Oktober 1925 in Langenelz\\
\gestorben & 1944\\
\end{tabular}\\
\medbreak
\textsc{vater}: \hyperref[@I7@]{Julius Galm} [28.02.1883--23.05.1929 (7 Kinder)]\\
\textsc{mutter}: \hyperref[@I8@]{Margaretha Schüßler} [11.04.1894--26.07.1978 (7 Kinder)]
\medbreak
\textsc{{geschwister}}
\begin{itemize}
\item \hyperref[@I56@]{Philipp Galm} [19.02.1920--vor 1945]
\item \hyperref[@I52@]{Juliane Galm} [26.03.1921--05.04.2019 (5 Kinder)]
\item \hyperref[@I53@]{Maria Galm} [14.07.1922--15.02.1995 (7 Kinder)]
\item \hyperref[@I4@]{Karl Galm} [27.07.1924--03.01.2015 (6 Kinder)]
\item \hyperref[@I54@]{Leo Galm} [07.09.1927--24.05.2003 (4 Kinder)]
\item \hyperref[@I55@]{Georg Galm} [16.08.1929 (3 Kinder)]
\end{itemize}
\bigbreak
\textsc{{quellen}}
\begin{enumerate}[label={[\arabic*]}]
\item \href{https://www.familysearch.org/tree/person/details/LYVN-QL7}{FamilySearch, ID: LYVN-QL7}
\end{enumerate}

\end{person}

\begin{person}[
    surname = {Galm},
    givenname = {Leo},
    suffix = {1927--2003},
    label = {@I54@},
    filename = {Leo Galm (1927)}
    ]

\begin{tabular}{cl}
\geboren & 07. September 1927 in Langenelz\\
\geheiratet & 29. April 1956 in Mudau mit Maria Noe \\
\gestorben & 24. Mai 2003\\
\end{tabular}\\
\medbreak
\textsc{vater}: \hyperref[@I7@]{Julius Galm} [28.02.1883--23.05.1929 (7 Kinder)]\\
\textsc{mutter}: \hyperref[@I8@]{Margaretha Schüßler} [11.04.1894--26.07.1978 (7 Kinder)]
\medbreak
\textsc{{geschwister}}
\begin{itemize}
\item \hyperref[@I56@]{Philipp Galm} [19.02.1920--vor 1945]
\item \hyperref[@I52@]{Juliane Galm} [26.03.1921--05.04.2019 (5 Kinder)]
\item \hyperref[@I53@]{Maria Galm} [14.07.1922--15.02.1995 (7 Kinder)]
\item \hyperref[@I4@]{Karl Galm} [27.07.1924--03.01.2015 (6 Kinder)]
\item \hyperref[@I57@]{Arthur Galm} [02.10.1925--1944]
\item \hyperref[@I55@]{Georg Galm} [16.08.1929 (3 Kinder)]
\end{itemize}
\bigbreak
\textsc{{kinder}}
\begin{itemize}
\item Sigfried Galm [...]
\item Andrea Galm [...]
\item Thomas Galm [...]
\item Volker Galm [...]
\end{itemize}
\medbreak
\textsc{{quellen}}
\begin{enumerate}[label={[\arabic*]}]
\item Mudau Heiratsbuch 1952–1957, Heiratsregister 1956, Nr. 2
\end{enumerate}

\end{person}

\begin{person}[
    surname = {Galm},
    givenname = {Georg},
    suffix = {1929},
    label = {@I55@},
    filename = {Georg Galm (1929)}
    ]

\begin{tabular}{cl}
\geboren & 16. August 1929 in Langenelz\\
\geheiratet &  mit Monika Wörner \\
\end{tabular}\\
\medbreak
\textsc{vater}: \hyperref[@I7@]{Julius Galm} [28.02.1883--23.05.1929 (7 Kinder)]\\
\textsc{mutter}: \hyperref[@I8@]{Margaretha Schüßler} [11.04.1894--26.07.1978 (7 Kinder)]
\medbreak
\textsc{{geschwister}}
\begin{itemize}
\item \hyperref[@I56@]{Philipp Galm} [19.02.1920--vor 1945]
\item \hyperref[@I52@]{Juliane Galm} [26.03.1921--05.04.2019 (5 Kinder)]
\item \hyperref[@I53@]{Maria Galm} [14.07.1922--15.02.1995 (7 Kinder)]
\item \hyperref[@I4@]{Karl Galm} [27.07.1924--03.01.2015 (6 Kinder)]
\item \hyperref[@I57@]{Arthur Galm} [02.10.1925--1944]
\item \hyperref[@I54@]{Leo Galm} [07.09.1927--24.05.2003 (4 Kinder)]
\end{itemize}
\bigbreak
\textsc{{kinder}}
\begin{itemize}
\item Winfried Galm [17.07.1960--03.01.2006]
\item Roswitha Galm [...]
\item Agnes Galm [...]
\end{itemize}
\medbreak
\end{person}


\addsec{Alois Adolf Schölch  \& Adelheid Anna Schäfer }


\begin{person}[
    surname = {Schölch},
    givenname = {Alois Adolf},
    suffix = {1895--1963},
    label = {@I9@},
    filename = {Alois Schoelch (1895)}
    ]

\begin{tabular}{cl}
\geboren & 12. März 1895 in Langenelz\\
\geheiratet & 06. Februar 1925 in Langenelz mit Adelheid Anna Schäfer \\
\gestorben & 07. August 1963 in Langenelz\\
\end{tabular}\\
\medbreak
\textsc{vater}: \hyperref[@I156@]{Johann Josef Schölch} [29.08.1865--04.11.1939 (4 Kinder)]\\
\textsc{mutter}: \hyperref[@I157@]{Karoline Mechler} [31.03.1870--04.12.1933 (4 Kinder)]
\medbreak
\textsc{{geschwister}}
\begin{itemize}
\item \hyperref[@I429@]{Anna Schölch} [30.10.1893--22.03.1946 (4 Kinder)]
\item \hyperref[@I366@]{Maria Schölch} [30.07.1896--30.08.1969 (2 Kinder)]
\item \hyperref[@I430@]{Karolina Schölch} [07.02.1900--23.09.1971]
\end{itemize}
\bigbreak
\textsc{{kinder}}
\begin{itemize}
\item \hyperref[@I5@]{Aloisia Lina Schölch} [12.10.1925--07.09.1993 (6 Kinder)]
\item \hyperref[@I58@]{Alfons Schölch} [03.08.1927--25.09.1927]
\item \hyperref[@I59@]{Gertrud Schölch} [01.03.1929--02.03.1996 (2 Kinder)]
\item \hyperref[@I61@]{Alfred Schölch} [04.05.1930--30.01.1998 (3 Kinder)]
\item \hyperref[@I60@]{Walter Josef Schölch} [04.08.1931--03.08.1992 (2 Kinder)]
\item \hyperref[@I64@]{Adolf Alois Schölch} [18.01.1934--07.05.1977]
\item \hyperref[@I63@]{Rita Karolina Schölch} [18.01.1934--15.02.2015 (3 Kinder)]
\item \hyperref[@I62@]{Bernhard Schölch} [19.08.1935--04.10.2005 (1 Kind)]
\item \hyperref[@I65@]{Hubert Schölch} [07.02.1938--24.12.1993 (2 Kinder)]
\item \hyperref[@I66@]{Arthur Richard Schölch} [06.02.1940--16.12.2011]
\end{itemize}
\medbreak
\textsc{{quellen}}
\begin{enumerate}[label={[\arabic*]}]
\item Langenelz Heiratsbuch 1911–1927, Heiratsregister 1925, Nr. 1
\item Langenelz Sterbebuch 1946–1973, Sterberegister 1963, Nr. 2
\item Sterbebild
\item Ahnentafel Helga Schölch
\item Ahnentafel Galm
\item \href{https://www.familysearch.org/tree/person/details/L5GK-S4W}{FamilySearch, ID: L5GK-S4W}
\end{enumerate}

\end{person}

\begin{person}[
    surname = {Schäfer},
    givenname = {Adelheid Anna},
    suffix = {1897--1959},
    label = {@I10@},
    filename = {Adelheid Anna Schaefer (1897)}
    ]

\begin{tabular}{cl}
\geboren & 17. August 1897 in Mudau\\
\geheiratet & 06. Februar 1925 in Langenelz mit Alois Adolf Schölch \\
\gestorben & 28. Juli 1959 in Langenelz\\
\end{tabular}\\
\medbreak
\textsc{vater}: \hyperref[@I161@]{Josef Eduard Schäfer} [13.10.1861--26.09.1923 (7 Kinder)]\\
\textsc{mutter}: \hyperref[@I162@]{Josefa Karolina Zeller} [25.01.1866--28.11.1931 (7 Kinder)]
\medbreak
\textsc{{geschwister}}
\begin{itemize}
\item \hyperref[@I431@]{Josef Georg Schäfer} [22.06.1889--21.02.1961 (7 Kinder)]
\item \hyperref[@I432@]{August Schäfer} [16.06.1892]
\item \hyperref[@I433@]{Ludwig Eduard Schäfer} [25.02.1894--18.01.1963]
\item \hyperref[@I434@]{Karl Ernst Schäfer} [14.12.1895--09.06.1918]
\item \hyperref[@I436@]{Maria Karolina Schäfer} [03.05.1899--18.04.1911]
\item \hyperref[@I435@]{Wilhelm Edmund Schäfer} [08.11.1900]
\end{itemize}
\bigbreak
\textsc{{kinder}}
\begin{itemize}
\item \hyperref[@I5@]{Aloisia Lina Schölch} [12.10.1925--07.09.1993 (6 Kinder)]
\item \hyperref[@I58@]{Alfons Schölch} [03.08.1927--25.09.1927]
\item \hyperref[@I59@]{Gertrud Schölch} [01.03.1929--02.03.1996 (2 Kinder)]
\item \hyperref[@I61@]{Alfred Schölch} [04.05.1930--30.01.1998 (3 Kinder)]
\item \hyperref[@I60@]{Walter Josef Schölch} [04.08.1931--03.08.1992 (2 Kinder)]
\item \hyperref[@I64@]{Adolf Alois Schölch} [18.01.1934--07.05.1977]
\item \hyperref[@I63@]{Rita Karolina Schölch} [18.01.1934--15.02.2015 (3 Kinder)]
\item \hyperref[@I62@]{Bernhard Schölch} [19.08.1935--04.10.2005 (1 Kind)]
\item \hyperref[@I65@]{Hubert Schölch} [07.02.1938--24.12.1993 (2 Kinder)]
\item \hyperref[@I66@]{Arthur Richard Schölch} [06.02.1940--16.12.2011]
\end{itemize}
\medbreak
\textsc{{quellen}}
\begin{enumerate}[label={[\arabic*]}]
\item Mudau Standesbuch 1896–1899, Geburtenregister 1897, Nr. 35
\item Langenelz Heiratsbuch 1911–1927, Heiratsregister 1925, Nr. 1
\item Langenelz Sterbebuch 1946–1973, Sterberegister 1959, Nr. 1
\item Sterbebild
\item Ahnentafel Helga Schölch
\item Ahnentafel Galm
\item \href{https://www.familysearch.org/tree/person/details/L5GV-55S}{FamilySearch, ID: L5GV-55S}
\end{enumerate}

\end{person}

\begin{person}[
    surname = {Schölch},
    givenname = {Alfons},
    suffix = {1927--1927},
    label = {@I58@}
    ]

\begin{tabular}{cl}
\geboren & 03. August 1927 in Langenelz\\
\gestorben & 25. September 1927 in Langenelz\\
\end{tabular}\\
\medbreak
\textsc{vater}: \hyperref[@I9@]{Alois Adolf Schölch} [12.03.1895--07.08.1963 (10 Kinder)]\\
\textsc{mutter}: \hyperref[@I10@]{Adelheid Anna Schäfer} [17.08.1897--28.07.1959 (10 Kinder)]
\medbreak
\textsc{{geschwister}}
\begin{itemize}
\item \hyperref[@I5@]{Aloisia Lina Schölch} [12.10.1925--07.09.1993 (6 Kinder)]
\item \hyperref[@I59@]{Gertrud Schölch} [01.03.1929--02.03.1996 (2 Kinder)]
\item \hyperref[@I61@]{Alfred Schölch} [04.05.1930--30.01.1998 (3 Kinder)]
\item \hyperref[@I60@]{Walter Josef Schölch} [04.08.1931--03.08.1992 (2 Kinder)]
\item \hyperref[@I64@]{Adolf Alois Schölch} [18.01.1934--07.05.1977]
\item \hyperref[@I63@]{Rita Karolina Schölch} [18.01.1934--15.02.2015 (3 Kinder)]
\item \hyperref[@I62@]{Bernhard Schölch} [19.08.1935--04.10.2005 (1 Kind)]
\item \hyperref[@I65@]{Hubert Schölch} [07.02.1938--24.12.1993 (2 Kinder)]
\item \hyperref[@I66@]{Arthur Richard Schölch} [06.02.1940--16.12.2011]
\end{itemize}
\bigbreak
\textsc{{quellen}}
\begin{enumerate}[label={[\arabic*]}]
\item Langenelz Geburtenbuch 1911–1927, Geburtenregister 1927, Nr. 6
\item Langenelz Sterbebuch 1911–1927, Sterberegister 1927, Nr. 5
\item \href{https://www.familysearch.org/tree/person/details/}{FamilySearch, ID: L1TQ-7XD}
\end{enumerate}

\end{person}

\begin{person}[
    surname = {Schölch},
    givenname = {Gertrud},
    suffix = {1929--1996},
    label = {@I59@},
    filename = {Gertrud Herhuth (1929)}
    ]

\begin{tabular}{cl}
\geboren & 01. März 1929 in Langenelz\\
\geheiratet &  mit Franz Herhuth \\
\gestorben & 02. März 1996 in Mannheim\\
\end{tabular}\\
\medbreak
\textsc{vater}: \hyperref[@I9@]{Alois Adolf Schölch} [12.03.1895--07.08.1963 (10 Kinder)]\\
\textsc{mutter}: \hyperref[@I10@]{Adelheid Anna Schäfer} [17.08.1897--28.07.1959 (10 Kinder)]
\medbreak
\textsc{{geschwister}}
\begin{itemize}
\item \hyperref[@I5@]{Aloisia Lina Schölch} [12.10.1925--07.09.1993 (6 Kinder)]
\item \hyperref[@I58@]{Alfons Schölch} [03.08.1927--25.09.1927]
\item \hyperref[@I61@]{Alfred Schölch} [04.05.1930--30.01.1998 (3 Kinder)]
\item \hyperref[@I60@]{Walter Josef Schölch} [04.08.1931--03.08.1992 (2 Kinder)]
\item \hyperref[@I64@]{Adolf Alois Schölch} [18.01.1934--07.05.1977]
\item \hyperref[@I63@]{Rita Karolina Schölch} [18.01.1934--15.02.2015 (3 Kinder)]
\item \hyperref[@I62@]{Bernhard Schölch} [19.08.1935--04.10.2005 (1 Kind)]
\item \hyperref[@I65@]{Hubert Schölch} [07.02.1938--24.12.1993 (2 Kinder)]
\item \hyperref[@I66@]{Arthur Richard Schölch} [06.02.1940--16.12.2011]
\end{itemize}
\bigbreak
\textsc{{kinder}}
\begin{itemize}
\item Werner Herhuth [... (2 Kinder)]
\item Edith Herhuth [... (2 Kinder)]
\end{itemize}
\medbreak
\textsc{{quellen}}
\begin{enumerate}[label={[\arabic*]}]
\item \href{https://www.familysearch.org/tree/person/details/L51H-42L}{FamilySearch, ID: L51H-42L}
\end{enumerate}

\end{person}

\begin{person}[
    surname = {Schölch},
    givenname = {Alfred},
    suffix = {1930--1998},
    label = {@I61@},
    filename = {Alfred Schoelch (1930)}
    ]

\begin{tabular}{cl}
\geboren & 04. Mai 1930 in Langenelz\\
\geheiratet & 20. Februar 1960 in Langenelz mit Gertrud Anna Brenneis \\
\gestorben & 30. Januar 1998 in Langenelz\\
\bestattet & 03. Februar 1998 in Langenelz\\
\end{tabular}\\
\medbreak
\textsc{vater}: \hyperref[@I9@]{Alois Adolf Schölch} [12.03.1895--07.08.1963 (10 Kinder)]\\
\textsc{mutter}: \hyperref[@I10@]{Adelheid Anna Schäfer} [17.08.1897--28.07.1959 (10 Kinder)]
\medbreak
\textsc{{geschwister}}
\begin{itemize}
\item \hyperref[@I5@]{Aloisia Lina Schölch} [12.10.1925--07.09.1993 (6 Kinder)]
\item \hyperref[@I58@]{Alfons Schölch} [03.08.1927--25.09.1927]
\item \hyperref[@I59@]{Gertrud Schölch} [01.03.1929--02.03.1996 (2 Kinder)]
\item \hyperref[@I60@]{Walter Josef Schölch} [04.08.1931--03.08.1992 (2 Kinder)]
\item \hyperref[@I64@]{Adolf Alois Schölch} [18.01.1934--07.05.1977]
\item \hyperref[@I63@]{Rita Karolina Schölch} [18.01.1934--15.02.2015 (3 Kinder)]
\item \hyperref[@I62@]{Bernhard Schölch} [19.08.1935--04.10.2005 (1 Kind)]
\item \hyperref[@I65@]{Hubert Schölch} [07.02.1938--24.12.1993 (2 Kinder)]
\item \hyperref[@I66@]{Arthur Richard Schölch} [06.02.1940--16.12.2011]
\end{itemize}
\bigbreak
\textsc{{kinder}}
\begin{itemize}
\item Brigitte Schölch [... (2 Kinder)]
\item Renate Schölch [... (2 Kinder)]
\item Ursula Schölch [... (2 Kinder)]
\end{itemize}
\medbreak
\textsc{{quellen}}
\begin{enumerate}[label={[\arabic*]}]
\item Langenelz Heiratsbuch 1946–1973, Heiratsregister 1960, Nr. 1
\item Langenelz Standesbuch 1928–1936, Geburtenregister 1930, Nr. 3
\item \href{https://www.familysearch.org/tree/person/details/L5GV-RHD}{FamilySearch, ID: L5GV-RHD}
\end{enumerate}

\end{person}

\begin{person}[
    surname = {Schölch},
    givenname = {Walter Josef},
    suffix = {1931--1992},
    label = {@I60@},
    filename = {Walter Schoelch (1931)}
    ]

\begin{tabular}{cl}
\geboren & 04. August 1931 in Langenelz\\
\geheiratet &  mit Elfriede Lux \\
\gestorben & 03. August 1992 in Schlierstadt\\
\bestattet & 06. August 1992 in Schlierstadt\\
\end{tabular}\\
\medbreak
\textsc{vater}: \hyperref[@I9@]{Alois Adolf Schölch} [12.03.1895--07.08.1963 (10 Kinder)]\\
\textsc{mutter}: \hyperref[@I10@]{Adelheid Anna Schäfer} [17.08.1897--28.07.1959 (10 Kinder)]
\medbreak
\textsc{{geschwister}}
\begin{itemize}
\item \hyperref[@I5@]{Aloisia Lina Schölch} [12.10.1925--07.09.1993 (6 Kinder)]
\item \hyperref[@I58@]{Alfons Schölch} [03.08.1927--25.09.1927]
\item \hyperref[@I59@]{Gertrud Schölch} [01.03.1929--02.03.1996 (2 Kinder)]
\item \hyperref[@I61@]{Alfred Schölch} [04.05.1930--30.01.1998 (3 Kinder)]
\item \hyperref[@I64@]{Adolf Alois Schölch} [18.01.1934--07.05.1977]
\item \hyperref[@I63@]{Rita Karolina Schölch} [18.01.1934--15.02.2015 (3 Kinder)]
\item \hyperref[@I62@]{Bernhard Schölch} [19.08.1935--04.10.2005 (1 Kind)]
\item \hyperref[@I65@]{Hubert Schölch} [07.02.1938--24.12.1993 (2 Kinder)]
\item \hyperref[@I66@]{Arthur Richard Schölch} [06.02.1940--16.12.2011]
\end{itemize}
\bigbreak
\textsc{{kinder}}
\begin{itemize}
\item Gerlinde Schölch [... (1 Kind)]
\item Arthur Schölch [... (1 Kind)]
\end{itemize}
\medbreak
\textsc{{quellen}}
\begin{enumerate}[label={[\arabic*]}]
\item \href{https://www.familysearch.org/tree/person/details/L51H-HY6}{FamilySearch, ID: L51H-HY6}
\end{enumerate}

\end{person}

\begin{person}[
    surname = {Schölch},
    givenname = {Adolf Alois},
    suffix = {1934--1977},
    label = {@I64@},
    filename = {Adolf Schoelch (1934)}
    ]

\begin{tabular}{cl}
\geboren & 18. Januar 1934 in Langenelz\\
\gestorben & 07. Mai 1977 in Buchen\\
\bestattet & 1977 in Langenelz\\
\end{tabular}\\
\medbreak
\textsc{vater}: \hyperref[@I9@]{Alois Adolf Schölch} [12.03.1895--07.08.1963 (10 Kinder)]\\
\textsc{mutter}: \hyperref[@I10@]{Adelheid Anna Schäfer} [17.08.1897--28.07.1959 (10 Kinder)]
\medbreak
\textsc{{geschwister}}
\begin{itemize}
\item \hyperref[@I5@]{Aloisia Lina Schölch} [12.10.1925--07.09.1993 (6 Kinder)]
\item \hyperref[@I58@]{Alfons Schölch} [03.08.1927--25.09.1927]
\item \hyperref[@I59@]{Gertrud Schölch} [01.03.1929--02.03.1996 (2 Kinder)]
\item \hyperref[@I61@]{Alfred Schölch} [04.05.1930--30.01.1998 (3 Kinder)]
\item \hyperref[@I60@]{Walter Josef Schölch} [04.08.1931--03.08.1992 (2 Kinder)]
\item \hyperref[@I63@]{Rita Karolina Schölch} [18.01.1934--15.02.2015 (3 Kinder)]
\item \hyperref[@I62@]{Bernhard Schölch} [19.08.1935--04.10.2005 (1 Kind)]
\item \hyperref[@I65@]{Hubert Schölch} [07.02.1938--24.12.1993 (2 Kinder)]
\item \hyperref[@I66@]{Arthur Richard Schölch} [06.02.1940--16.12.2011]
\end{itemize}
\bigbreak
\textsc{anmerkung}\\
Spitzname wegen blonden Haare
Arbeit in Mannheim
\medbreak
\textsc{{quellen}}
\begin{enumerate}[label={[\arabic*]}]
\item \href{https://www.familysearch.org/tree/person/details/L51H-4WQ}{FamilySearch, ID: L51H-4WQ}
\end{enumerate}

\end{person}

\begin{person}[
    surname = {Schölch},
    givenname = {Rita Karolina},
    suffix = {1934--2015},
    label = {@I63@},
    filename = {Rita Melkus (1934)}
    ]

\begin{tabular}{cl}
\geboren & 18. Januar 1934 in Langenelz\\
\geheiratet & 08. August 1959 in Langenelz mit Rudolf Franz Melkus \\
\gestorben & 15. Februar 2015 in Mudau\\
\bestattet & 20. Februar 2015 in Mudau\\
\end{tabular}\\
\medbreak
\textsc{vater}: \hyperref[@I9@]{Alois Adolf Schölch} [12.03.1895--07.08.1963 (10 Kinder)]\\
\textsc{mutter}: \hyperref[@I10@]{Adelheid Anna Schäfer} [17.08.1897--28.07.1959 (10 Kinder)]
\medbreak
\textsc{{geschwister}}
\begin{itemize}
\item \hyperref[@I5@]{Aloisia Lina Schölch} [12.10.1925--07.09.1993 (6 Kinder)]
\item \hyperref[@I58@]{Alfons Schölch} [03.08.1927--25.09.1927]
\item \hyperref[@I59@]{Gertrud Schölch} [01.03.1929--02.03.1996 (2 Kinder)]
\item \hyperref[@I61@]{Alfred Schölch} [04.05.1930--30.01.1998 (3 Kinder)]
\item \hyperref[@I60@]{Walter Josef Schölch} [04.08.1931--03.08.1992 (2 Kinder)]
\item \hyperref[@I64@]{Adolf Alois Schölch} [18.01.1934--07.05.1977]
\item \hyperref[@I62@]{Bernhard Schölch} [19.08.1935--04.10.2005 (1 Kind)]
\item \hyperref[@I65@]{Hubert Schölch} [07.02.1938--24.12.1993 (2 Kinder)]
\item \hyperref[@I66@]{Arthur Richard Schölch} [06.02.1940--16.12.2011]
\end{itemize}
\bigbreak
\textsc{{kinder}}
\begin{itemize}
\item Dieter Melkus [...]
\item Irmgard Melkus [... (2 Kinder)]
\item Manfred Melkus [...]
\end{itemize}
\medbreak
\textsc{{quellen}}
\begin{enumerate}[label={[\arabic*]}]
\item Langenelz Standesbuch 1928–1936, Geburtenregister 1934, Nr. 2
\item \href{https://www.familysearch.org/tree/person/details/L51H-CXK}{FamilySearch, ID: L51H-CXK}
\end{enumerate}

\end{person}

\begin{person}[
    surname = {Schölch},
    givenname = {Bernhard},
    suffix = {1935--2005},
    label = {@I62@},
    filename = {Bernhard Schoelch (1935)}
    ]

\begin{tabular}{cl}
\geboren & 19. August 1935 in Langenelz\\
\geheiratet &  mit Katharina ... \\
\gestorben & 04. Oktober 2005 in Sulzbach\\
\bestattet & 08. Oktober 2005 in Sulzbach\\
\end{tabular}\\
\medbreak
\textsc{vater}: \hyperref[@I9@]{Alois Adolf Schölch} [12.03.1895--07.08.1963 (10 Kinder)]\\
\textsc{mutter}: \hyperref[@I10@]{Adelheid Anna Schäfer} [17.08.1897--28.07.1959 (10 Kinder)]
\medbreak
\textsc{{geschwister}}
\begin{itemize}
\item \hyperref[@I5@]{Aloisia Lina Schölch} [12.10.1925--07.09.1993 (6 Kinder)]
\item \hyperref[@I58@]{Alfons Schölch} [03.08.1927--25.09.1927]
\item \hyperref[@I59@]{Gertrud Schölch} [01.03.1929--02.03.1996 (2 Kinder)]
\item \hyperref[@I61@]{Alfred Schölch} [04.05.1930--30.01.1998 (3 Kinder)]
\item \hyperref[@I60@]{Walter Josef Schölch} [04.08.1931--03.08.1992 (2 Kinder)]
\item \hyperref[@I64@]{Adolf Alois Schölch} [18.01.1934--07.05.1977]
\item \hyperref[@I63@]{Rita Karolina Schölch} [18.01.1934--15.02.2015 (3 Kinder)]
\item \hyperref[@I65@]{Hubert Schölch} [07.02.1938--24.12.1993 (2 Kinder)]
\item \hyperref[@I66@]{Arthur Richard Schölch} [06.02.1940--16.12.2011]
\end{itemize}
\bigbreak
\textsc{{kinder}}
\begin{itemize}
\item Gerd Schölch [... (1 Kind)]
\end{itemize}
\medbreak
\textsc{{quellen}}
\begin{enumerate}[label={[\arabic*]}]
\item \href{https://www.familysearch.org/tree/person/details/L51H-CXM}{FamilySearch, ID: L51H-CXM}
\end{enumerate}

\end{person}

\begin{person}[
    surname = {Schölch},
    givenname = {Hubert},
    suffix = {1938--1993},
    label = {@I65@},
    filename = {Hubert Schoelch (1938)}
    ]

\begin{tabular}{cl}
\geboren & 07. Februar 1938 in Langenelz\\
\geheiratet & 08. Juni 1962 in Mudau mit Anna Elisabeth Noe \\
\gestorben & 24. Dezember 1993 in Mannheim\\
\bestattet & 30. Dezember 1993 in Neckarau, Mannheim\\
\end{tabular}\\
\medbreak
\textsc{vater}: \hyperref[@I9@]{Alois Adolf Schölch} [12.03.1895--07.08.1963 (10 Kinder)]\\
\textsc{mutter}: \hyperref[@I10@]{Adelheid Anna Schäfer} [17.08.1897--28.07.1959 (10 Kinder)]
\medbreak
\textsc{{geschwister}}
\begin{itemize}
\item \hyperref[@I5@]{Aloisia Lina Schölch} [12.10.1925--07.09.1993 (6 Kinder)]
\item \hyperref[@I58@]{Alfons Schölch} [03.08.1927--25.09.1927]
\item \hyperref[@I59@]{Gertrud Schölch} [01.03.1929--02.03.1996 (2 Kinder)]
\item \hyperref[@I61@]{Alfred Schölch} [04.05.1930--30.01.1998 (3 Kinder)]
\item \hyperref[@I60@]{Walter Josef Schölch} [04.08.1931--03.08.1992 (2 Kinder)]
\item \hyperref[@I64@]{Adolf Alois Schölch} [18.01.1934--07.05.1977]
\item \hyperref[@I63@]{Rita Karolina Schölch} [18.01.1934--15.02.2015 (3 Kinder)]
\item \hyperref[@I62@]{Bernhard Schölch} [19.08.1935--04.10.2005 (1 Kind)]
\item \hyperref[@I66@]{Arthur Richard Schölch} [06.02.1940--16.12.2011]
\end{itemize}
\bigbreak
\textsc{{kinder}}
\begin{itemize}
\item Bernd Schölch [... (2 Kinder)]
\item Elke Schölch [...]
\end{itemize}
\medbreak
\textsc{{quellen}}
\begin{enumerate}[label={[\arabic*]}]
\item \href{https://www.familysearch.org/tree/person/details/L51H-ZJ3}{FamilySearch, ID: L51H-ZJ3}
\end{enumerate}

\end{person}

\begin{person}[
    surname = {Schölch},
    givenname = {Arthur Richard},
    suffix = {1940--2011},
    label = {@I66@},
    filename = {Artur Schoelch (1940)}
    ]

\begin{tabular}{cl}
\geboren & 06. Februar 1940 in Langenelz\\
\gestorben & 16. Dezember 2011 in Eberbach\\
\bestattet & 29. Dezember 2011 in Langenelz\\
\end{tabular}\\
\medbreak
\textsc{vater}: \hyperref[@I9@]{Alois Adolf Schölch} [12.03.1895--07.08.1963 (10 Kinder)]\\
\textsc{mutter}: \hyperref[@I10@]{Adelheid Anna Schäfer} [17.08.1897--28.07.1959 (10 Kinder)]
\medbreak
\textsc{{geschwister}}
\begin{itemize}
\item \hyperref[@I5@]{Aloisia Lina Schölch} [12.10.1925--07.09.1993 (6 Kinder)]
\item \hyperref[@I58@]{Alfons Schölch} [03.08.1927--25.09.1927]
\item \hyperref[@I59@]{Gertrud Schölch} [01.03.1929--02.03.1996 (2 Kinder)]
\item \hyperref[@I61@]{Alfred Schölch} [04.05.1930--30.01.1998 (3 Kinder)]
\item \hyperref[@I60@]{Walter Josef Schölch} [04.08.1931--03.08.1992 (2 Kinder)]
\item \hyperref[@I64@]{Adolf Alois Schölch} [18.01.1934--07.05.1977]
\item \hyperref[@I63@]{Rita Karolina Schölch} [18.01.1934--15.02.2015 (3 Kinder)]
\item \hyperref[@I62@]{Bernhard Schölch} [19.08.1935--04.10.2005 (1 Kind)]
\item \hyperref[@I65@]{Hubert Schölch} [07.02.1938--24.12.1993 (2 Kinder)]
\end{itemize}
\bigbreak
\textsc{{quellen}}
\begin{enumerate}[label={[\arabic*]}]
\item \href{https://www.familysearch.org/tree/person/details/L51H-CXB}{FamilySearch, ID: L51H-CXB}
\end{enumerate}

\end{person}


\addchap{Ur-Ur-Gro"seltern}



\addsec{Johann Valentin Scheuermann  \& Margareta Schäfer }


\begin{person}[
    surname = {Scheuermann},
    givenname = {Johann Valentin},
    suffix = {1842--1910},
    label = {@I389@}
    ]

\begin{tabular}{cl}
\geboren & 07. Dezember 1842 in Ottorfszell\\
\geheiratet & 21. Oktober 1875 in Mudau mit Margareta Schäfer \\
\gestorben & 12. Dezember 1910 in Mudau\\
\bestattet &  in Mudau\\
\end{tabular}\\
\medbreak
\textsc{vater}: \hyperref[@I950@]{Franz Anton Scheuermann} [29.05.1799--13.12.1869 (7 Kinder)]\\
\textsc{mutter}: \hyperref[@I951@]{Eva Barbara Zeller} [03.12.1811--30.01.1886 (2 Kinder)]
\medbreak
\textsc{{geschwister}}
\begin{itemize}
\item \hyperref[@I1287@]{Anna Maria Scheuermann} [04.10.1826]
\item \hyperref[@I1288@]{Maria Josepha Scheuermann} [10.03.1829]
\item \hyperref[@I1289@]{Eva Barbara Scheuermann} [02.12.1831]
\item \hyperref[@I1290@]{Johann Michael Scheuermann} [20.12.1833--05.06.1888]
\item \hyperref[@I1291@]{Franz Anton Scheuermann} [17.06.1836--04.01.1904]
\item \hyperref[@I1292@]{Johann Josef Scheuermann} [1849]
\end{itemize}
\bigbreak
\textsc{{kinder}}
\begin{itemize}
\item \hyperref[@I1270@]{Johann Valentin Scheuermann} [12.10.1876--11.12.1942 (1 Kind)]
\item \hyperref[@I1213@]{Helena Scheuermann} [09.09.1878]
\item \hyperref[@I1272@]{Josefa/Bertha Scheuermann} [13.03.1881]
\item \hyperref[@I965@]{Otto Scheuermann} [10.11.1883--19.11.1953 (3 Kinder)]
\item \hyperref[@I13@]{Heinrich Scheuermann} [13.05.1886--18.04.1929 (6 Kinder)]
\item \hyperref[@I964@]{Anna Scheuermann} [29.12.1888--09.03.1952]
\end{itemize}
\medbreak
\textsc{{quellen}}
\begin{enumerate}[label={[\arabic*]}]
\item Mudau Geburts-, Heirats- und Sterberegister 1870–1875, Heiratsregister 1875, Nr. 6 
\item Mudau Geburts-, Heirats- und Sterbebuch 1906–1910, Sterberegister 1910, Nr. 18
\item Eheurkunde Heinrich Scheuermann und Anna Zimmermann
\item Friedhof Mudau Grabstein (Falsches Sterbedatum)
\item \href{https://www.familysearch.org/tree/person/details/L1TC-1P1}{FamilySearch, ID: L1TC-1P1}
\end{enumerate}

\end{person}

\begin{person}[
    surname = {Schäfer},
    givenname = {Margareta},
    suffix = {1852--1929},
    label = {@I390@}
    ]

\begin{tabular}{cl}
\geboren & 05. Oktober 1852 in Mudau\\
\taufe & 05. Oktober 1852 in Mudau\\
\geheiratet & 21. Oktober 1875 in Mudau mit Johann Valentin Scheuermann \\
\gestorben & 11. November 1929 in Mudau\\
\bestattet &  in Mudau\\
\end{tabular}\\
\medbreak
\textsc{vater}: \hyperref[@I952@]{Johann Josef Schäfer} [07.10.1822--04.12.1866 (6 Kinder)]\\
\textsc{mutter}: \hyperref[@I953@]{Margaretha Baumann} [28.05.1819--25.06.1878 (6 Kinder)]
\medbreak
\textsc{{geschwister}}
\begin{itemize}
\item \hyperref[@I1204@]{Michael Schäfer} [29.09.1847--26.12.1924 (4 Kinder)]
\item \hyperref[@I1853@]{Franz Karl Schäfer} [1848]
\item \hyperref[@I1205@]{Josefa Schäfer} [03.07.1850]
\item \hyperref[@I1206@]{Joseph John Schäfer} [19.03.1856--12.03.1914 (8 Kinder)]
\item \hyperref[@I1207@]{Johann Schäfer} [23.06.1858]
\end{itemize}
\bigbreak
\textsc{{kinder}}
\begin{itemize}
\item \hyperref[@I1270@]{Johann Valentin Scheuermann} [12.10.1876--11.12.1942 (1 Kind)]
\item \hyperref[@I1213@]{Helena Scheuermann} [09.09.1878]
\item \hyperref[@I1272@]{Josefa/Bertha Scheuermann} [13.03.1881]
\item \hyperref[@I965@]{Otto Scheuermann} [10.11.1883--19.11.1953 (3 Kinder)]
\item \hyperref[@I13@]{Heinrich Scheuermann} [13.05.1886--18.04.1929 (6 Kinder)]
\item \hyperref[@I964@]{Anna Scheuermann} [29.12.1888--09.03.1952]
\end{itemize}
\medbreak
\textsc{anmerkung}\\
Geburtstag ueberpruefen (7. Oktober nach anderer Quelle)
\medbreak
\textsc{{quellen}}
\begin{enumerate}[label={[\arabic*]}]
\item \href{http://www.landesarchiv-bw.de/plink/?f=4-1119480-140}{GLA Karlsruhe, Mudau, katholische Gemeinde: Standesbuch 1845–1855, Geburtenregister 1852, Nr. 28 (Bild 140)}
\item Mudau Geburts-, Heirats- und Sterberegister 1870–1875, Heiratsregister 1875, Nr. 6 
\item Mudau Sterbebuch 1929–1937, Sterberegister 1929, Nr. 17
\item Friedhof Mudau Grabstein
\item Eheurkunde Heinrich Scheuermann und Anna Zimmermann
\item \href{https://www.familysearch.org/tree/person/details/L1TZ-MM2}{FamilySearch, ID: L1TZ-MM2}
\end{enumerate}

\end{person}

\begin{person}[
    surname = {Scheuermann},
    givenname = {Johann Valentin},
    suffix = {1876--1942},
    label = {@I1270@}
    ]

\begin{tabular}{cl}
\geboren & 12. Oktober 1876 in Mudau\\
\taufe & 12. Oktober 1876 in Mudau\\
\geheiratet & 08. September 1909 in Montgomery, Ohio, USA mit Clara Deger \\
\gestorben & 11. Dezember 1942 in Harrison, Montgomery, Ohio\\
\bestattet & 14. Dezember 1942 in Dayton, Ohio, USA\\
\end{tabular}\\
\medbreak
\textsc{vater}: \hyperref[@I389@]{Johann Valentin Scheuermann} [07.12.1842--12.12.1910 (6 Kinder)]\\
\textsc{mutter}: \hyperref[@I390@]{Margareta Schäfer} [05.10.1852--11.11.1929 (6 Kinder)]
\medbreak
\textsc{{geschwister}}
\begin{itemize}
\item \hyperref[@I1213@]{Helena Scheuermann} [09.09.1878]
\item \hyperref[@I1272@]{Josefa/Bertha Scheuermann} [13.03.1881]
\item \hyperref[@I965@]{Otto Scheuermann} [10.11.1883--19.11.1953 (3 Kinder)]
\item \hyperref[@I13@]{Heinrich Scheuermann} [13.05.1886--18.04.1929 (6 Kinder)]
\item \hyperref[@I964@]{Anna Scheuermann} [29.12.1888--09.03.1952]
\end{itemize}
\bigbreak
\textsc{{kinder}}
\begin{itemize}
\item Margaret A. Scheuermann [18.07.1910--09.01.1986]
\end{itemize}
\medbreak
\textsc{{quellen}}
\begin{enumerate}[label={[\arabic*]}]
\item Mudau Geburts-, Heirats- und Sterberegister 1876–1879, Geburtenregister 1876, Nr. 52
\item \href{https://www.familysearch.org/tree/person/details/LD64-37K}{FamilySearch, ID: LD64-37K}
\end{enumerate}

\end{person}

\begin{person}[
    surname = {Scheuermann},
    givenname = {Helena},
    suffix = {1878},
    label = {@I1213@}
    ]

\begin{tabular}{cl}
\geboren & 09. September 1878 in Mudau\\
\end{tabular}\\
\medbreak
\textsc{vater}: \hyperref[@I389@]{Johann Valentin Scheuermann} [07.12.1842--12.12.1910 (6 Kinder)]\\
\textsc{mutter}: \hyperref[@I390@]{Margareta Schäfer} [05.10.1852--11.11.1929 (6 Kinder)]
\medbreak
\textsc{{geschwister}}
\begin{itemize}
\item \hyperref[@I1270@]{Johann Valentin Scheuermann} [12.10.1876--11.12.1942 (1 Kind)]
\item \hyperref[@I1272@]{Josefa/Bertha Scheuermann} [13.03.1881]
\item \hyperref[@I965@]{Otto Scheuermann} [10.11.1883--19.11.1953 (3 Kinder)]
\item \hyperref[@I13@]{Heinrich Scheuermann} [13.05.1886--18.04.1929 (6 Kinder)]
\item \hyperref[@I964@]{Anna Scheuermann} [29.12.1888--09.03.1952]
\end{itemize}
\bigbreak
\textsc{{quellen}}
\begin{enumerate}[label={[\arabic*]}]
\item Mudau Geburts-, Heirats- und Sterberegister 1876–1879, Geburtenregister 1878, Nr.34
\item Ahnentafel USA
\item \href{https://www.familysearch.org/tree/person/details/GMQK-FP3}{FamilySearch, ID: GMQK-FP3}
\end{enumerate}

\end{person}

\begin{person}[
    surname = {Scheuermann},
    givenname = {Josefa/Bertha},
    suffix = {1881},
    label = {@I1272@}
    ]

\begin{tabular}{cl}
\geboren & 13. März 1881 in Mudau\\
\end{tabular}\\
\medbreak
\textsc{vater}: \hyperref[@I389@]{Johann Valentin Scheuermann} [07.12.1842--12.12.1910 (6 Kinder)]\\
\textsc{mutter}: \hyperref[@I390@]{Margareta Schäfer} [05.10.1852--11.11.1929 (6 Kinder)]
\medbreak
\textsc{{geschwister}}
\begin{itemize}
\item \hyperref[@I1270@]{Johann Valentin Scheuermann} [12.10.1876--11.12.1942 (1 Kind)]
\item \hyperref[@I1213@]{Helena Scheuermann} [09.09.1878]
\item \hyperref[@I965@]{Otto Scheuermann} [10.11.1883--19.11.1953 (3 Kinder)]
\item \hyperref[@I13@]{Heinrich Scheuermann} [13.05.1886--18.04.1929 (6 Kinder)]
\item \hyperref[@I964@]{Anna Scheuermann} [29.12.1888--09.03.1952]
\end{itemize}
\bigbreak
\textsc{anmerkung}\\
Name unleserlich. In Seitenrand eher Bertha. Im Text Josefa
\medbreak
\textsc{{quellen}}
\begin{enumerate}[label={[\arabic*]}]
\item Geburtenregister Mudau 1881, Nr. 9
\end{enumerate}

\end{person}

\begin{person}[
    surname = {Scheuermann},
    givenname = {Otto},
    suffix = {1883--1953},
    label = {@I965@}
    ]

\begin{tabular}{cl}
\geboren & 10. November 1883 in Mudau\\
\geheiratet & 19. Mai 1910 in Mudau mit Laura Bingler \\
\gestorben & 19. November 1953 in Mudau\\
\bestattet &  in Mudau\\
\end{tabular}\\
\medbreak
\textsc{vater}: \hyperref[@I389@]{Johann Valentin Scheuermann} [07.12.1842--12.12.1910 (6 Kinder)]\\
\textsc{mutter}: \hyperref[@I390@]{Margareta Schäfer} [05.10.1852--11.11.1929 (6 Kinder)]
\medbreak
\textsc{{geschwister}}
\begin{itemize}
\item \hyperref[@I1270@]{Johann Valentin Scheuermann} [12.10.1876--11.12.1942 (1 Kind)]
\item \hyperref[@I1213@]{Helena Scheuermann} [09.09.1878]
\item \hyperref[@I1272@]{Josefa/Bertha Scheuermann} [13.03.1881]
\item \hyperref[@I13@]{Heinrich Scheuermann} [13.05.1886--18.04.1929 (6 Kinder)]
\item \hyperref[@I964@]{Anna Scheuermann} [29.12.1888--09.03.1952]
\end{itemize}
\bigbreak
\textsc{{kinder}}
\begin{itemize}
\item Margareta Scheuermann [02.04.1911--18.08.2002]
\item Josefa Scheuermann [03.05.1912--08.09.1971]
\item Emil Scheuermann [15.05.1913--16.01.1945]
\end{itemize}
\medbreak
\textsc{anmerkung}\\
Bäckerei (heute VoBa Mudau)
gestorben in Klinik in Heidelberg
\medbreak
\textsc{{quellen}}
\begin{enumerate}[label={[\arabic*]}]
\item Mudau Geburts-, Heirats- und Sterberegister 1882–1884, Geburtenregister 1883, Nr. 39
\item Mudau Geburts-, Heirats- und Sterbebuch 1906–1910, Heiratsregister 1910, Nr. 7
\item Ahnentafel USA
\item \href{https://www.familysearch.org/tree/person/details/GMQK-VXW}{FamilySearch, ID: GMQK-VXW}
\end{enumerate}

\end{person}

\begin{person}[
    surname = {Scheuermann},
    givenname = {Anna},
    suffix = {1888--1952},
    label = {@I964@}
    ]

\begin{tabular}{cl}
\geboren & 29. Dezember 1888 in Mudau\\
\gestorben & 09. März 1952 in Mudau\\
\bestattet &  in Mudau\\
\end{tabular}\\
\medbreak
\textsc{vater}: \hyperref[@I389@]{Johann Valentin Scheuermann} [07.12.1842--12.12.1910 (6 Kinder)]\\
\textsc{mutter}: \hyperref[@I390@]{Margareta Schäfer} [05.10.1852--11.11.1929 (6 Kinder)]
\medbreak
\textsc{{geschwister}}
\begin{itemize}
\item \hyperref[@I1270@]{Johann Valentin Scheuermann} [12.10.1876--11.12.1942 (1 Kind)]
\item \hyperref[@I1213@]{Helena Scheuermann} [09.09.1878]
\item \hyperref[@I1272@]{Josefa/Bertha Scheuermann} [13.03.1881]
\item \hyperref[@I965@]{Otto Scheuermann} [10.11.1883--19.11.1953 (3 Kinder)]
\item \hyperref[@I13@]{Heinrich Scheuermann} [13.05.1886--18.04.1929 (6 Kinder)]
\end{itemize}
\bigbreak
\textsc{anmerkung}\\
ledig, wohnhaft in Mudauer Hauptstrasse 119
\medbreak
\textsc{{quellen}}
\begin{enumerate}[label={[\arabic*]}]
\item Mudau Standesbuch 1888–1890, Geburtenregister 1888, Nr. 41
\item Mudau Sterbebuch 1945–1955, Sterberegister 1952, Nr. 7
\item \href{https://www.familysearch.org/tree/person/details/GMQK-JYV}{FamilySearch, ID: GMQK-JYV}
\end{enumerate}

\end{person}


\addsec{Valentin Zimmermann  \& Thekla Maria Albert }


\begin{person}[
    surname = {Zimmermann},
    givenname = {Valentin},
    suffix = {1856--1923},
    label = {@I392@}
    ]

\begin{tabular}{cl}
\geboren & 24. Februar 1856 in Laudenberg\\
\taufe & 24. Februar 1856 in Limbach\\
\geheiratet & 1878 mit Emilie Brenneis \\
 & 01. Februar 1887 in Laudenberg mit Thekla Maria Albert \\
\gestorben & 16. März 1923 in Laudenberg\\
\end{tabular}\\
\medbreak
\textsc{vater}: \hyperref[@I396@]{Johann Valentin Zimmermann} [31.07.1830--11.12.1861 (4 Kinder)]\\
\textsc{mutter}: \hyperref[@I393@]{Josepha Zimmermann} [19.03.1831--20.11.1898 (7 Kinder)]
\medbreak
\textsc{{geschwister}}
\begin{itemize}
\item \hyperref[@I1348@]{Josepha Zimmermann} [05.05.1852--08.05.1852]
\item \hyperref[@I1349@]{Katharina Zimmermann} [07.01.1854]
\item \hyperref[@I1350@]{Rosa Zimmermann} [02.04.1859--21.03.1862]
\item \hyperref[@I1373@]{Wilhelm Albert} [02.05.1864--18.08.1919 (5 Kinder)]
\item \hyperref[@I1374@]{Johann Adam Albert} [22.06.1866--30.11.1937]
\item \hyperref[@I1375@]{Rosa Albert} [14.04.1869]
\end{itemize}
\bigbreak
\textsc{{kinder}}
\begin{itemize}
\item \hyperref[@I975@]{Maria Zimmermann} [14.11.1879]
\item \hyperref[@I974@]{Rosa Zimmermann} [06.03.1882 (3 Kinder)]
\item \hyperref[@I1358@]{Valentin Zimmermann} [08.10.1883--02.01.1886]
\item \hyperref[@I973@]{Pius Zimmermann} [12.10.1885--15.04.1968 (4 Kinder)]
\item \hyperref[@I14@]{Anna Zimmermann} [07.12.1887--18.06.1969 (6 Kinder)]
\item \hyperref[@I360@]{Thekla Zimmermann} [14.10.1891--24.01.1927 (4 Kinder)]
\item \hyperref[@I968@]{Adolf Zimmermann} [05.03.1894--15.02.1971 (3 Kinder)]
\item \hyperref[@I967@]{Emilie Zimmermann} [02.10.1897--15.09.1983 (4 Kinder)]
\item \hyperref[@I966@]{Elisabeth Zimmermann} [01.01.1904--21.04.1970 (3 Kinder)]
\item \hyperref[@I969@]{Karl Zimmermann} [... (2 Kinder)]
\end{itemize}
\medbreak
\textsc{anmerkung}\\
1. Ehe mit Emilie Brenneis aus Robern
Sterbeort nicht belegt
\medbreak
\textsc{{quellen}}
\begin{enumerate}[label={[\arabic*]}]
\item \href{http://www.landesarchiv-bw.de/plink/?f=4-1119439-261}{GLA Karlsruhe, Laudenberg, katholische Gemeinde: Standesbuch 1810–1870, Geburtenregister 1856, Nr. 4 (Bild 261)}
\item \href{https://www.familysearch.org/tree/person/details/LVKC-ZDP}{FamilySearch, ID: LVKC-ZDP}
\item \href{http://gedbas.genealogy.net/person/show/1172964972}{genealogy.net}
\end{enumerate}

\end{person}

\begin{person}[
    surname = {Albert},
    givenname = {Thekla Maria},
    suffix = {1863--1917},
    label = {@I391@}
    ]

\begin{tabular}{cl}
\geboren & 23. September 1863 in Hambrunn\\
\geheiratet & 01. Februar 1887 in Laudenberg mit Valentin Zimmermann \\
\gestorben & 24. November 1917 in Laudenberg\\
\end{tabular}\\
\medbreak
\textsc{vater}: \hyperref[@I394@]{Johann Martin Albert} [25.12.1825--01.05.1899 (4 Kinder)]\\
\textsc{mutter}: \hyperref[@I395@]{Katharina Hess} [03.07.1832--24.10.1875 (4 Kinder)]
\medbreak
\textsc{{geschwister}}
\begin{itemize}
\item \hyperref[@I1359@]{Johann Joseph Albert} [13.09.1857]
\item \hyperref[@I1360@]{Katharina Hildegard Albert} [25.04.1860--26.04.1860]
\item \hyperref[@I1361@]{Franz Valentin Albert} [01.12.1866--03.07.1930 (5 Kinder)]
\end{itemize}
\bigbreak
\textsc{{kinder}}
\begin{itemize}
\item \hyperref[@I14@]{Anna Zimmermann} [07.12.1887--18.06.1969 (6 Kinder)]
\item \hyperref[@I360@]{Thekla Zimmermann} [14.10.1891--24.01.1927 (4 Kinder)]
\item \hyperref[@I968@]{Adolf Zimmermann} [05.03.1894--15.02.1971 (3 Kinder)]
\item \hyperref[@I967@]{Emilie Zimmermann} [02.10.1897--15.09.1983 (4 Kinder)]
\item \hyperref[@I966@]{Elisabeth Zimmermann} [01.01.1904--21.04.1970 (3 Kinder)]
\item \hyperref[@I969@]{Karl Zimmermann} [... (2 Kinder)]
\end{itemize}
\medbreak
\textsc{anmerkung}\\
7. Kinder
\medbreak
\textsc{{quellen}}
\begin{enumerate}[label={[\arabic*]}]
\item Eheurkunde Heinrich Scheuermann und Anna Zimmermann
\item \href{https://www.familysearch.org/tree/person/details/LV2T-5DY}{FamilySearch, ID: LV2T-5DY}
\item \href{http://gedbas.genealogy.net/person/show/1172975941}{genealogy.net}
\end{enumerate}

\end{person}

\begin{person}[
    surname = {Zimmermann},
    givenname = {Maria},
    suffix = {1879},
    label = {@I975@}
    ]

\begin{tabular}{cl}
\geboren & 14. November 1879\\
\geheiratet &  mit ... Grünwald \\
\end{tabular}\\
\medbreak
\textsc{vater}: \hyperref[@I392@]{Valentin Zimmermann} [24.02.1856--16.03.1923 (10 Kinder)]\\
\textsc{mutter}: \hyperref[@I972@]{Emilie Brenneis} [26.08.1854--24.10.1885 (4 Kinder)]
\medbreak
\textsc{{geschwister}}
\begin{itemize}
\item \hyperref[@I974@]{Rosa Zimmermann} [06.03.1882 (3 Kinder)]
\item \hyperref[@I1358@]{Valentin Zimmermann} [08.10.1883--02.01.1886]
\item \hyperref[@I973@]{Pius Zimmermann} [12.10.1885--15.04.1968 (4 Kinder)]
\item \hyperref[@I14@]{Anna Zimmermann} [07.12.1887--18.06.1969 (6 Kinder)]
\item \hyperref[@I360@]{Thekla Zimmermann} [14.10.1891--24.01.1927 (4 Kinder)]
\item \hyperref[@I968@]{Adolf Zimmermann} [05.03.1894--15.02.1971 (3 Kinder)]
\item \hyperref[@I967@]{Emilie Zimmermann} [02.10.1897--15.09.1983 (4 Kinder)]
\item \hyperref[@I966@]{Elisabeth Zimmermann} [01.01.1904--21.04.1970 (3 Kinder)]
\item \hyperref[@I969@]{Karl Zimmermann} [... (2 Kinder)]
\end{itemize}
\bigbreak
\textsc{{quellen}}
\begin{enumerate}[label={[\arabic*]}]
\item \href{https://www.familysearch.org/tree/person/details/L1TZ-MSY}{FamilySearch, ID: L1TZ-MSY}
\end{enumerate}

\end{person}

\begin{person}[
    surname = {Zimmermann},
    givenname = {Rosa},
    suffix = {1882},
    label = {@I974@}
    ]

\begin{tabular}{cl}
\geboren & 06. März 1882\\
\geheiratet &  mit Karl Friedel \\
\end{tabular}\\
\medbreak
\textsc{vater}: \hyperref[@I392@]{Valentin Zimmermann} [24.02.1856--16.03.1923 (10 Kinder)]\\
\textsc{mutter}: \hyperref[@I972@]{Emilie Brenneis} [26.08.1854--24.10.1885 (4 Kinder)]
\medbreak
\textsc{{geschwister}}
\begin{itemize}
\item \hyperref[@I975@]{Maria Zimmermann} [14.11.1879]
\item \hyperref[@I1358@]{Valentin Zimmermann} [08.10.1883--02.01.1886]
\item \hyperref[@I973@]{Pius Zimmermann} [12.10.1885--15.04.1968 (4 Kinder)]
\item \hyperref[@I14@]{Anna Zimmermann} [07.12.1887--18.06.1969 (6 Kinder)]
\item \hyperref[@I360@]{Thekla Zimmermann} [14.10.1891--24.01.1927 (4 Kinder)]
\item \hyperref[@I968@]{Adolf Zimmermann} [05.03.1894--15.02.1971 (3 Kinder)]
\item \hyperref[@I967@]{Emilie Zimmermann} [02.10.1897--15.09.1983 (4 Kinder)]
\item \hyperref[@I966@]{Elisabeth Zimmermann} [01.01.1904--21.04.1970 (3 Kinder)]
\item \hyperref[@I969@]{Karl Zimmermann} [... (2 Kinder)]
\end{itemize}
\bigbreak
\textsc{{kinder}}
\begin{itemize}
\item Karl Friedel [...]
\item Emilie Friedel [...]
\item Lydia Friedel [...]
\end{itemize}
\medbreak
\textsc{{quellen}}
\begin{enumerate}[label={[\arabic*]}]
\item \href{https://www.familysearch.org/tree/person/details/L1TC-187}{FamilySearch, ID: L1TC-187}
\end{enumerate}

\end{person}

\begin{person}[
    surname = {Zimmermann},
    givenname = {Valentin},
    suffix = {1883--1886},
    label = {@I1358@}
    ]

\begin{tabular}{cl}
\geboren & 08. Oktober 1883 in Laudenberg\\
\taufe &  in Laudenberg\\
\gestorben & 02. Januar 1886\\
\bestattet &  in Laudenberg\\
\end{tabular}\\
\medbreak
\textsc{vater}: \hyperref[@I392@]{Valentin Zimmermann} [24.02.1856--16.03.1923 (10 Kinder)]\\
\textsc{mutter}: \hyperref[@I972@]{Emilie Brenneis} [26.08.1854--24.10.1885 (4 Kinder)]
\medbreak
\textsc{{geschwister}}
\begin{itemize}
\item \hyperref[@I975@]{Maria Zimmermann} [14.11.1879]
\item \hyperref[@I974@]{Rosa Zimmermann} [06.03.1882 (3 Kinder)]
\item \hyperref[@I973@]{Pius Zimmermann} [12.10.1885--15.04.1968 (4 Kinder)]
\item \hyperref[@I14@]{Anna Zimmermann} [07.12.1887--18.06.1969 (6 Kinder)]
\item \hyperref[@I360@]{Thekla Zimmermann} [14.10.1891--24.01.1927 (4 Kinder)]
\item \hyperref[@I968@]{Adolf Zimmermann} [05.03.1894--15.02.1971 (3 Kinder)]
\item \hyperref[@I967@]{Emilie Zimmermann} [02.10.1897--15.09.1983 (4 Kinder)]
\item \hyperref[@I966@]{Elisabeth Zimmermann} [01.01.1904--21.04.1970 (3 Kinder)]
\item \hyperref[@I969@]{Karl Zimmermann} [... (2 Kinder)]
\end{itemize}
\bigbreak
\textsc{{quellen}}
\begin{enumerate}[label={[\arabic*]}]
\item \href{https://www.familysearch.org/tree/person/details/GMSL-RSW}{FamilySearch, ID: GMSL-RSW}
\end{enumerate}

\end{person}

\begin{person}[
    surname = {Zimmermann},
    givenname = {Pius},
    suffix = {1885--1968},
    label = {@I973@},
    filename = {Pius Zimmermann (1885)}
    ]

\begin{tabular}{cl}
\geboren & 12. Oktober 1885 in Laudenberg\\
\geheiratet &  mit Frida ... \\
\gestorben & 15. April 1968 in Laudenberg\\
\end{tabular}\\
\medbreak
\textsc{vater}: \hyperref[@I392@]{Valentin Zimmermann} [24.02.1856--16.03.1923 (10 Kinder)]\\
\textsc{mutter}: \hyperref[@I972@]{Emilie Brenneis} [26.08.1854--24.10.1885 (4 Kinder)]
\medbreak
\textsc{{geschwister}}
\begin{itemize}
\item \hyperref[@I975@]{Maria Zimmermann} [14.11.1879]
\item \hyperref[@I974@]{Rosa Zimmermann} [06.03.1882 (3 Kinder)]
\item \hyperref[@I1358@]{Valentin Zimmermann} [08.10.1883--02.01.1886]
\item \hyperref[@I14@]{Anna Zimmermann} [07.12.1887--18.06.1969 (6 Kinder)]
\item \hyperref[@I360@]{Thekla Zimmermann} [14.10.1891--24.01.1927 (4 Kinder)]
\item \hyperref[@I968@]{Adolf Zimmermann} [05.03.1894--15.02.1971 (3 Kinder)]
\item \hyperref[@I967@]{Emilie Zimmermann} [02.10.1897--15.09.1983 (4 Kinder)]
\item \hyperref[@I966@]{Elisabeth Zimmermann} [01.01.1904--21.04.1970 (3 Kinder)]
\item \hyperref[@I969@]{Karl Zimmermann} [... (2 Kinder)]
\end{itemize}
\bigbreak
\textsc{{kinder}}
\begin{itemize}
\item Pius Zimmermann [25.01.1916--04.01.1998]
\item Hilde Zimmermann [...]
\item Alfred Zimmermann [...]
\item Paula Zimmermann [...]
\end{itemize}
\medbreak
\textsc{anmerkung}\\
gestorben in Buchen
\medbreak
\textsc{{quellen}}
\begin{enumerate}[label={[\arabic*]}]
\item Sterbebild
\end{enumerate}

\end{person}

\begin{person}[
    surname = {Zimmermann},
    givenname = {Thekla},
    suffix = {1891--1927},
    label = {@I360@},
    filename = {Thekla Zimmermann (1891)}
    ]

\begin{tabular}{cl}
\geboren & 14. Oktober 1891 in Laudenberg\\
\geheiratet & 23. April 1913 in Limbach mit Edmund Engelbert Schölch \\
\gestorben & 24. Januar 1927 in Laudenberg\\
\end{tabular}\\
\medbreak
\textsc{vater}: \hyperref[@I392@]{Valentin Zimmermann} [24.02.1856--16.03.1923 (10 Kinder)]\\
\textsc{mutter}: \hyperref[@I391@]{Thekla Maria Albert} [23.09.1863--24.11.1917 (6 Kinder)]
\medbreak
\textsc{{geschwister}}
\begin{itemize}
\item \hyperref[@I975@]{Maria Zimmermann} [14.11.1879]
\item \hyperref[@I974@]{Rosa Zimmermann} [06.03.1882 (3 Kinder)]
\item \hyperref[@I1358@]{Valentin Zimmermann} [08.10.1883--02.01.1886]
\item \hyperref[@I973@]{Pius Zimmermann} [12.10.1885--15.04.1968 (4 Kinder)]
\item \hyperref[@I14@]{Anna Zimmermann} [07.12.1887--18.06.1969 (6 Kinder)]
\item \hyperref[@I968@]{Adolf Zimmermann} [05.03.1894--15.02.1971 (3 Kinder)]
\item \hyperref[@I967@]{Emilie Zimmermann} [02.10.1897--15.09.1983 (4 Kinder)]
\item \hyperref[@I966@]{Elisabeth Zimmermann} [01.01.1904--21.04.1970 (3 Kinder)]
\item \hyperref[@I969@]{Karl Zimmermann} [... (2 Kinder)]
\end{itemize}
\bigbreak
\textsc{{kinder}}
\begin{itemize}
\item Laura Schölch [05.06.1914--02.02.2000]
\item Rosa Schölch [13.04.1921--01.02.2004]
\item Otto Schölch [25.09.1922--31.03.2018 (5 Kinder)]
\item Erna Schölch [13.01.1927--18.06.2015]
\end{itemize}
\medbreak
\textsc{{quellen}}
\begin{enumerate}[label={[\arabic*]}]
\item \href{http://grabsteine.genealogy.net/tomb.php?cem=3609&tomb=85&b=&lang=de}{genealogy.net Grabstein Projekt, Friedhof Laudenberg}
\item \href{https://www.familysearch.org/tree/person/details/L1TC-B58}{FamilySearch, ID: L1TC-B58}
\end{enumerate}

\end{person}

\begin{person}[
    surname = {Zimmermann},
    givenname = {Adolf},
    suffix = {1894--1971},
    label = {@I968@}
    ]

\begin{tabular}{cl}
\geboren & 05. März 1894 in Laudenberg\\
\geheiratet &  mit Ida Meixner \\
\gestorben & 15. Februar 1971\\
\end{tabular}\\
\medbreak
\textsc{vater}: \hyperref[@I392@]{Valentin Zimmermann} [24.02.1856--16.03.1923 (10 Kinder)]\\
\textsc{mutter}: \hyperref[@I391@]{Thekla Maria Albert} [23.09.1863--24.11.1917 (6 Kinder)]
\medbreak
\textsc{{geschwister}}
\begin{itemize}
\item \hyperref[@I975@]{Maria Zimmermann} [14.11.1879]
\item \hyperref[@I974@]{Rosa Zimmermann} [06.03.1882 (3 Kinder)]
\item \hyperref[@I1358@]{Valentin Zimmermann} [08.10.1883--02.01.1886]
\item \hyperref[@I973@]{Pius Zimmermann} [12.10.1885--15.04.1968 (4 Kinder)]
\item \hyperref[@I14@]{Anna Zimmermann} [07.12.1887--18.06.1969 (6 Kinder)]
\item \hyperref[@I360@]{Thekla Zimmermann} [14.10.1891--24.01.1927 (4 Kinder)]
\item \hyperref[@I967@]{Emilie Zimmermann} [02.10.1897--15.09.1983 (4 Kinder)]
\item \hyperref[@I966@]{Elisabeth Zimmermann} [01.01.1904--21.04.1970 (3 Kinder)]
\item \hyperref[@I969@]{Karl Zimmermann} [... (2 Kinder)]
\end{itemize}
\bigbreak
\textsc{{kinder}}
\begin{itemize}
\item Adolf Zimmermann [07.10.1920--21.11.1988]
\item Maria Zimmermann [07.02.1926--15.12.2000]
\item Robert Zimmermann [13.04.1932--07.07.1995]
\end{itemize}
\medbreak
\textsc{anmerkung}\\
Alfons Bauer in Laudenberg fragen
\medbreak
\end{person}

\begin{person}[
    surname = {Zimmermann},
    givenname = {Emilie},
    suffix = {1897--1983},
    label = {@I967@},
    filename = {Emilie Zimmermann (1897)}
    ]

\begin{tabular}{cl}
\geboren & 02. Oktober 1897 in Laudenberg\\
\geheiratet & 03. Juli 1919 in Langenelz mit Karl Friedel \\
\gestorben & 15. September 1983 in Mudau\\
\end{tabular}\\
\medbreak
\textsc{vater}: \hyperref[@I392@]{Valentin Zimmermann} [24.02.1856--16.03.1923 (10 Kinder)]\\
\textsc{mutter}: \hyperref[@I391@]{Thekla Maria Albert} [23.09.1863--24.11.1917 (6 Kinder)]
\medbreak
\textsc{{geschwister}}
\begin{itemize}
\item \hyperref[@I975@]{Maria Zimmermann} [14.11.1879]
\item \hyperref[@I974@]{Rosa Zimmermann} [06.03.1882 (3 Kinder)]
\item \hyperref[@I1358@]{Valentin Zimmermann} [08.10.1883--02.01.1886]
\item \hyperref[@I973@]{Pius Zimmermann} [12.10.1885--15.04.1968 (4 Kinder)]
\item \hyperref[@I14@]{Anna Zimmermann} [07.12.1887--18.06.1969 (6 Kinder)]
\item \hyperref[@I360@]{Thekla Zimmermann} [14.10.1891--24.01.1927 (4 Kinder)]
\item \hyperref[@I968@]{Adolf Zimmermann} [05.03.1894--15.02.1971 (3 Kinder)]
\item \hyperref[@I966@]{Elisabeth Zimmermann} [01.01.1904--21.04.1970 (3 Kinder)]
\item \hyperref[@I969@]{Karl Zimmermann} [... (2 Kinder)]
\end{itemize}
\bigbreak
\textsc{{kinder}}
\begin{itemize}
\item Irma Friedel [07.12.1926--28.05.2006]
\item Artur Friedel [...]
\item Erwin Friedel [...]
\item Franziska Friedel [um 1931]
\end{itemize}
\medbreak
\textsc{{quellen}}
\begin{enumerate}[label={[\arabic*]}]
\item Langenelz Heiratsbuch 1911–1927, Heiratsregister 1919, Nr. 3
\item Mündliche Überlieferung Franziska Grimm
\item \href{https://www.familysearch.org/tree/person/details/L1TC-1Z4}{FamilySearch, ID: L1TC-1Z4}
\item \href{http://grabsteine.genealogy.net/tomb.php?cem=3902&tomb=122&b=&lang=de}{genealogy.net Grabstein Projekt, Friedhof Mudau}
\end{enumerate}

\end{person}

\begin{person}[
    surname = {Zimmermann},
    givenname = {Elisabeth},
    suffix = {1904--1970},
    label = {@I966@},
    filename = {Elisabeth Mai (1904)}
    ]

\begin{tabular}{cl}
\geboren & 01. Januar 1904 in Laudenberg\\
\geheiratet & 05. Februar 1930 in Mudau mit Wilhelm Mai \\
\gestorben & 21. April 1970 in Langenelz\\
\end{tabular}\\
\medbreak
\textsc{vater}: \hyperref[@I392@]{Valentin Zimmermann} [24.02.1856--16.03.1923 (10 Kinder)]\\
\textsc{mutter}: \hyperref[@I391@]{Thekla Maria Albert} [23.09.1863--24.11.1917 (6 Kinder)]
\medbreak
\textsc{{geschwister}}
\begin{itemize}
\item \hyperref[@I975@]{Maria Zimmermann} [14.11.1879]
\item \hyperref[@I974@]{Rosa Zimmermann} [06.03.1882 (3 Kinder)]
\item \hyperref[@I1358@]{Valentin Zimmermann} [08.10.1883--02.01.1886]
\item \hyperref[@I973@]{Pius Zimmermann} [12.10.1885--15.04.1968 (4 Kinder)]
\item \hyperref[@I14@]{Anna Zimmermann} [07.12.1887--18.06.1969 (6 Kinder)]
\item \hyperref[@I360@]{Thekla Zimmermann} [14.10.1891--24.01.1927 (4 Kinder)]
\item \hyperref[@I968@]{Adolf Zimmermann} [05.03.1894--15.02.1971 (3 Kinder)]
\item \hyperref[@I967@]{Emilie Zimmermann} [02.10.1897--15.09.1983 (4 Kinder)]
\item \hyperref[@I969@]{Karl Zimmermann} [... (2 Kinder)]
\end{itemize}
\bigbreak
\textsc{{kinder}}
\begin{itemize}
\item Helmut Mai [06.09.1932--15.09.2017]
\item Bernhard Mai [... (2 Kinder)]
\item Hyronmia Mai [...]
\end{itemize}
\medbreak
\textsc{anmerkung}\\
Kinder 
* Mudau 1931, Nr. 8
* Mudau 1936, Nr. 27
\medbreak
\textsc{{quellen}}
\begin{enumerate}[label={[\arabic*]}]
\item Geburtenregister Laudenberg 1904, Nr. 1
\item Mudau Heiratsbuch 1928–1937, Heiratsregister 1930, Nr. 2
\end{enumerate}

\end{person}

\begin{person}[
    surname = {Zimmermann},
    givenname = {Karl},
    suffix = {},
    label = {@I969@}
    ]

\begin{tabular}{cl}
\end{tabular}\\
\medbreak
\textsc{vater}: \hyperref[@I392@]{Valentin Zimmermann} [24.02.1856--16.03.1923 (10 Kinder)]\\
\textsc{mutter}: \hyperref[@I391@]{Thekla Maria Albert} [23.09.1863--24.11.1917 (6 Kinder)]
\medbreak
\textsc{{geschwister}}
\begin{itemize}
\item \hyperref[@I975@]{Maria Zimmermann} [14.11.1879]
\item \hyperref[@I974@]{Rosa Zimmermann} [06.03.1882 (3 Kinder)]
\item \hyperref[@I1358@]{Valentin Zimmermann} [08.10.1883--02.01.1886]
\item \hyperref[@I973@]{Pius Zimmermann} [12.10.1885--15.04.1968 (4 Kinder)]
\item \hyperref[@I14@]{Anna Zimmermann} [07.12.1887--18.06.1969 (6 Kinder)]
\item \hyperref[@I360@]{Thekla Zimmermann} [14.10.1891--24.01.1927 (4 Kinder)]
\item \hyperref[@I968@]{Adolf Zimmermann} [05.03.1894--15.02.1971 (3 Kinder)]
\item \hyperref[@I967@]{Emilie Zimmermann} [02.10.1897--15.09.1983 (4 Kinder)]
\item \hyperref[@I966@]{Elisabeth Zimmermann} [01.01.1904--21.04.1970 (3 Kinder)]
\end{itemize}
\bigbreak
\textsc{{kinder}}
\begin{itemize}
\item Helmut Zimmermann [...]
\item Gertrud Zimmermann [...]
\end{itemize}
\medbreak
\textsc{anmerkung}\\
in Ludwigshafen Oppau
hat noch 3. Kind
\medbreak
\end{person}


\addsec{Joseph Michael Röckel  \& Rosa Noe }


\begin{person}[
    surname = {Röckel},
    givenname = {Joseph Michael},
    suffix = {1839--1888},
    label = {@I386@}
    ]

\begin{tabular}{cl}
\geboren & 19. März 1839 in Hollerbach\\
\geheiratet & 20. August 1874 in Langenelz mit Josefa Link \\
 & 17. Juli 1879 in Langenelz mit Rosa Noe \\
\gestorben & 03. Oktober 1888 in Langenelz\\
\end{tabular}\\
\medbreak
\textsc{vater}: \hyperref[@I490@]{Johann Michael Röckel} [1804--02.05.1884 (7 Kinder)]\\
\textsc{mutter}: \hyperref[@I491@]{Anna Theresia Herkert} [um 1801--18.12.1853 (7 Kinder)]
\medbreak
\textsc{{geschwister}}
\begin{itemize}
\item \hyperref[@I496@]{Margaretha Röckel} [10.04.1829--03.02.1886 (11 Kinder)]
\item \hyperref[@I497@]{Anna Theresia Röckel} [01.07.1831--25.12.1865]
\item \hyperref[@I498@]{Johann Theodor Röckel} [01.11.1833--25.09.1918 (1 Kind)]
\item \hyperref[@I499@]{Eva Katharina Röckel} [13.07.1836]
\item \hyperref[@I500@]{Maria Josepha Röckel} [13.02.1842--19.02.1842]
\item \hyperref[@I501@]{Johann Martin Röckel} [12.07.1843--21.12.1906]
\end{itemize}
\bigbreak
\textsc{{kinder}}
\begin{itemize}
\item \hyperref[@I1268@]{Theodor Röckel} [03.06.1875--24.02.1876]
\item \hyperref[@I1269@]{Emma Röckel} [28.06.1876]
\item \hyperref[@I489@]{Michael Röckel} [21.07.1877]
\item \hyperref[@I954@]{Rosa Röckel} [28.02.1879--29.1880]
\item \hyperref[@I1154@]{Ida Röckel} [12.12.1879--09.12.1955 (5 Kinder)]
\item \hyperref[@I955@]{Anna Röckel} [05.05.1881--12.05.1882]
\item \hyperref[@I15@]{Otto Röckel} [20.11.1882--14.06.1965 (6 Kinder)]
\item \hyperref[@I956@]{Josef Röckel} [07.09.1884--21.10.1884]
\item \hyperref[@I472@]{Wilhelm Röckel} [07.01.1887--25.11.1968 (7 Kinder)]
\end{itemize}
\medbreak
\textsc{anmerkung}\\
in späteren Dokumenten ist der Name nur Michel
\medbreak
\textsc{{quellen}}
\begin{enumerate}[label={[\arabic*]}]
\item \href{http://www.landesarchiv-bw.de/plink/?f=4-1119414-240}{GLA Karlsruhe, Hollerbach, katholische Gemeinde: Standesbuch 1810–1842, Geburtenregister 1839, Nr. 2 (Bild 240)}
\item Langenelz Standesbuch 1870–1875, Heiratsregister 1874, Nr. 2
\item Langenelz Standesbuch 1876–1879, Heiratsregister 1879, Nr. 1
\item Langenelz Standesbuch 1885–1890, Sterberegister 1888, Nr. 4
\item \href{https://www.familysearch.org/tree/person/details/LKSD-TQG}{FamilySearch, ID: LKSD-TQG}
\item \href{http://gedbas.genealogy.net/person/show/1172958280}{genealogy.net}
\end{enumerate}

\end{person}

\begin{person}[
    surname = {Noe},
    givenname = {Rosa},
    suffix = {1857--1920},
    label = {@I387@},
    filename = {Rosa Noe (1857)}
    ]

\begin{tabular}{cl}
\geboren & 15. Juni 1857 in Balsbach\\
\taufe & 16. Juni 1857 in Limbach\\
\geheiratet & 17. Juli 1879 in Langenelz mit Joseph Michael Röckel \\
 & 08. Mai 1890 in Langenelz mit Sebastian Müller \\
\gestorben & 28. August 1920 in Langenelz\\
\end{tabular}\\
\medbreak
\textsc{vater}: \hyperref[@I504@]{Franz Noe} [09.12.1824--13.08.1897 (11 Kinder)]\\
\textsc{mutter}: \hyperref[@I496@]{Margaretha Röckel} [10.04.1829--03.02.1886 (11 Kinder)]
\medbreak
\textsc{{geschwister}}
\begin{itemize}
\item \hyperref[@I505@]{Margaretha Noe} [10.12.1855--17.01.1924 (7 Kinder)]
\item \hyperref[@I506@]{Philippina Noe} [21.04.1859]
\item \hyperref[@I507@]{Karolina Noe} [12.03.1861--23.03.1861]
\item \hyperref[@I508@]{Katharina Noe} [10.02.1862]
\item \hyperref[@I509@]{Wilhelm Noe} [25.03.1864]
\item \hyperref[@I510@]{Theresia Noe} [10.04.1865]
\item \hyperref[@I511@]{Anna Noe} [26.07.1867--14.09.1867]
\item \hyperref[@I1747@]{Maria Anna Noe} [11.04.1870--25.09.1870]
\item \hyperref[@I1748@]{Franz Karl Noe} [08.09.1871]
\item \hyperref[@I1749@]{Emma Noe} [10.02.1874]
\end{itemize}
\bigbreak
\textsc{{kinder}}
\begin{itemize}
\item \hyperref[@I1154@]{Ida Röckel} [12.12.1879--09.12.1955 (5 Kinder)]
\item \hyperref[@I955@]{Anna Röckel} [05.05.1881--12.05.1882]
\item \hyperref[@I15@]{Otto Röckel} [20.11.1882--14.06.1965 (6 Kinder)]
\item \hyperref[@I956@]{Josef Röckel} [07.09.1884--21.10.1884]
\item \hyperref[@I472@]{Wilhelm Röckel} [07.01.1887--25.11.1968 (7 Kinder)]
\item \hyperref[@I960@]{Franz Karl Müller} [21.10.1891--08.1918]
\item \hyperref[@I961@]{Sebastian Müller} [19.01.1893--12.11.1915]
\item \hyperref[@I481@]{Maria Müller} [07.09.1895--27.11.1972 (4 Kinder)]
\item \hyperref[@I962@]{Rosa Müller} [02.04.1897--12.07.1979]
\item \hyperref[@I963@]{Pius Müller} [28.01.1899]
\end{itemize}
\medbreak
\textsc{{quellen}}
\begin{enumerate}[label={[\arabic*]}]
\item \href{http://www.landesarchiv-bw.de/plink/?f=4-1120207-256}{GLA Karlsruhe, Balsbach, evangelische und katholische Gemeinde: Standesbuch 1810–1866, Geburtenregister 1857, Nr. 11 (Bild 256)}
\item Langenelz Standesbuch 1876–1879, Heiratsregister 1879, Nr. 1
\item Langenelz Standesbuch 1885–1890, Heiratsregister 1890, Nr. 1
\item Langenelz Sterbebuch 1911–1927, Sterberegister 1920, Nr. 5
\item \href{https://www.familysearch.org/tree/person/details/LVPW-DW3}{FamilySearch, ID: LVPW-DW3}
\item \href{http://gedbas.genealogy.net/person/show/1172957274}{genealogy.net}
\end{enumerate}

\end{person}

\begin{person}[
    surname = {Röckel},
    givenname = {Theodor},
    suffix = {1875--1876},
    label = {@I1268@}
    ]

\begin{tabular}{cl}
\geboren & 03. Juni 1875 in Langenelz\\
\gestorben & 24. Februar 1876 in Langenelz\\
\end{tabular}\\
\medbreak
\textsc{vater}: \hyperref[@I386@]{Joseph Michael Röckel} [19.03.1839--03.10.1888 (9 Kinder)]\\
\textsc{mutter}: \hyperref[@I488@]{Josefa Link} [28.03.1846--04.03.1879 (4 Kinder)]
\medbreak
\textsc{{geschwister}}
\begin{itemize}
\item \hyperref[@I1269@]{Emma Röckel} [28.06.1876]
\item \hyperref[@I489@]{Michael Röckel} [21.07.1877]
\item \hyperref[@I1154@]{Ida Röckel} [12.12.1879--09.12.1955 (5 Kinder)]
\item \hyperref[@I954@]{Rosa Röckel} [28.02.1879--29.1880]
\item \hyperref[@I955@]{Anna Röckel} [05.05.1881--12.05.1882]
\item \hyperref[@I15@]{Otto Röckel} [20.11.1882--14.06.1965 (6 Kinder)]
\item \hyperref[@I956@]{Josef Röckel} [07.09.1884--21.10.1884]
\item \hyperref[@I472@]{Wilhelm Röckel} [07.01.1887--25.11.1968 (7 Kinder)]
\end{itemize}
\bigbreak
\textsc{{quellen}}
\begin{enumerate}[label={[\arabic*]}]
\item Langenelz Standesbuch 1870–1875, Geburtenregister 1875, Nr. 6
\item Langenelz Standesbuch 1876–1879, Sterberegister 1876, Nr. 2
\item \href{https://www.familysearch.org/tree/person/details/GMQK-ZFH}{FamilySearch, ID: GMQK-ZFH}
\end{enumerate}

\end{person}

\begin{person}[
    surname = {Röckel},
    givenname = {Emma},
    suffix = {1876},
    label = {@I1269@}
    ]

\begin{tabular}{cl}
\geboren & 28. Juni 1876 in Langenelz\\
\taufe & 28. Juni 1876 in Mudau\\
\end{tabular}\\
\medbreak
\textsc{vater}: \hyperref[@I386@]{Joseph Michael Röckel} [19.03.1839--03.10.1888 (9 Kinder)]\\
\textsc{mutter}: \hyperref[@I488@]{Josefa Link} [28.03.1846--04.03.1879 (4 Kinder)]
\medbreak
\textsc{{geschwister}}
\begin{itemize}
\item \hyperref[@I1268@]{Theodor Röckel} [03.06.1875--24.02.1876]
\item \hyperref[@I489@]{Michael Röckel} [21.07.1877]
\item \hyperref[@I1154@]{Ida Röckel} [12.12.1879--09.12.1955 (5 Kinder)]
\item \hyperref[@I954@]{Rosa Röckel} [28.02.1879--29.1880]
\item \hyperref[@I955@]{Anna Röckel} [05.05.1881--12.05.1882]
\item \hyperref[@I15@]{Otto Röckel} [20.11.1882--14.06.1965 (6 Kinder)]
\item \hyperref[@I956@]{Josef Röckel} [07.09.1884--21.10.1884]
\item \hyperref[@I472@]{Wilhelm Röckel} [07.01.1887--25.11.1968 (7 Kinder)]
\end{itemize}
\bigbreak
\textsc{{quellen}}
\begin{enumerate}[label={[\arabic*]}]
\item Langenelz Standesbuch 1876–1879, Geburtenregister 1876, Nr. 3
\item \href{https://www.familysearch.org/tree/person/details/GMQK-8V4}{FamilySearch, ID: GMQK-8V4}
\end{enumerate}

\end{person}

\begin{person}[
    surname = {Röckel},
    givenname = {Michael},
    suffix = {1877},
    label = {@I489@}
    ]

\begin{tabular}{cl}
\geboren & 21. Juli 1877 in Langenelz\\
\taufe & 21. Juli 1877 in Mudau\\
\end{tabular}\\
\medbreak
\textsc{vater}: \hyperref[@I386@]{Joseph Michael Röckel} [19.03.1839--03.10.1888 (9 Kinder)]\\
\textsc{mutter}: \hyperref[@I488@]{Josefa Link} [28.03.1846--04.03.1879 (4 Kinder)]
\medbreak
\textsc{{geschwister}}
\begin{itemize}
\item \hyperref[@I1268@]{Theodor Röckel} [03.06.1875--24.02.1876]
\item \hyperref[@I1269@]{Emma Röckel} [28.06.1876]
\item \hyperref[@I1154@]{Ida Röckel} [12.12.1879--09.12.1955 (5 Kinder)]
\item \hyperref[@I954@]{Rosa Röckel} [28.02.1879--29.1880]
\item \hyperref[@I955@]{Anna Röckel} [05.05.1881--12.05.1882]
\item \hyperref[@I15@]{Otto Röckel} [20.11.1882--14.06.1965 (6 Kinder)]
\item \hyperref[@I956@]{Josef Röckel} [07.09.1884--21.10.1884]
\item \hyperref[@I472@]{Wilhelm Röckel} [07.01.1887--25.11.1968 (7 Kinder)]
\end{itemize}
\bigbreak
\textsc{{quellen}}
\begin{enumerate}[label={[\arabic*]}]
\item Langenelz Standesbuch 1876–1879, Geburtenregister 1877, Nr. 8
\item \href{https://www.familysearch.org/tree/person/details/GMQK-ZFJ}{FamilySearch, ID: GMQK-ZFJ}
\end{enumerate}

\end{person}

\begin{person}[
    surname = {Röckel},
    givenname = {Rosa},
    suffix = {1879--1880},
    label = {@I954@}
    ]

\begin{tabular}{cl}
\geboren & 28. Februar 1879 in Langenelz\\
\taufe & 28. Februar 1879 in Mudau\\
\gestorben & 29.1880\\
\end{tabular}\\
\medbreak
\textsc{vater}: \hyperref[@I386@]{Joseph Michael Röckel} [19.03.1839--03.10.1888 (9 Kinder)]\\
\textsc{mutter}: \hyperref[@I488@]{Josefa Link} [28.03.1846--04.03.1879 (4 Kinder)]
\medbreak
\textsc{{geschwister}}
\begin{itemize}
\item \hyperref[@I1268@]{Theodor Röckel} [03.06.1875--24.02.1876]
\item \hyperref[@I1269@]{Emma Röckel} [28.06.1876]
\item \hyperref[@I489@]{Michael Röckel} [21.07.1877]
\item \hyperref[@I1154@]{Ida Röckel} [12.12.1879--09.12.1955 (5 Kinder)]
\item \hyperref[@I955@]{Anna Röckel} [05.05.1881--12.05.1882]
\item \hyperref[@I15@]{Otto Röckel} [20.11.1882--14.06.1965 (6 Kinder)]
\item \hyperref[@I956@]{Josef Röckel} [07.09.1884--21.10.1884]
\item \hyperref[@I472@]{Wilhelm Röckel} [07.01.1887--25.11.1968 (7 Kinder)]
\end{itemize}
\bigbreak
\textsc{anmerkung}\\
Sterbedatum unsicher
\medbreak
\textsc{{quellen}}
\begin{enumerate}[label={[\arabic*]}]
\item \href{https://www.familysearch.org/tree/person/details/GMQK-8VJ}{FamilySearch, ID: GMQK-8VJ}
\end{enumerate}

\end{person}

\begin{person}[
    surname = {Röckel},
    givenname = {Ida},
    suffix = {1879--1955},
    label = {@I1154@},
    filename = {Ida Röckel (1879)}
    ]

\begin{tabular}{cl}
\geboren & 12. Dezember 1879 in Langenelz\\
\geheiratet & 14. Februar 1906 in Mudau mit Ludwig Zimmermann \\
\gestorben & 09. Dezember 1955 in Mudau\\
\end{tabular}\\
\medbreak
\textsc{vater}: \hyperref[@I386@]{Joseph Michael Röckel} [19.03.1839--03.10.1888 (9 Kinder)]\\
\textsc{mutter}: \hyperref[@I387@]{Rosa Noe} [15.06.1857--28.08.1920 (10 Kinder)]
\medbreak
\textsc{{geschwister}}
\begin{itemize}
\item \hyperref[@I1268@]{Theodor Röckel} [03.06.1875--24.02.1876]
\item \hyperref[@I1269@]{Emma Röckel} [28.06.1876]
\item \hyperref[@I489@]{Michael Röckel} [21.07.1877]
\item \hyperref[@I954@]{Rosa Röckel} [28.02.1879--29.1880]
\item \hyperref[@I955@]{Anna Röckel} [05.05.1881--12.05.1882]
\item \hyperref[@I15@]{Otto Röckel} [20.11.1882--14.06.1965 (6 Kinder)]
\item \hyperref[@I956@]{Josef Röckel} [07.09.1884--21.10.1884]
\item \hyperref[@I472@]{Wilhelm Röckel} [07.01.1887--25.11.1968 (7 Kinder)]
\item \hyperref[@I960@]{Franz Karl Müller} [21.10.1891--08.1918]
\item \hyperref[@I961@]{Sebastian Müller} [19.01.1893--12.11.1915]
\item \hyperref[@I481@]{Maria Müller} [07.09.1895--27.11.1972 (4 Kinder)]
\item \hyperref[@I962@]{Rosa Müller} [02.04.1897--12.07.1979]
\item \hyperref[@I963@]{Pius Müller} [28.01.1899]
\end{itemize}
\bigbreak
\textsc{{kinder}}
\begin{itemize}
\item August Zimmermann [17.08.1908--31.08.1945 (2 Kinder)]
\item Josef Zimmermann [13.03.1910--2004]
\item Otto Zimmermann [20.04.1912--23.01.1996]
\item Paula Rosa Zimmermann [24.06.1922--26.08.1997]
\item Albert Zimmermann [...]
\end{itemize}
\medbreak
\textsc{{quellen}}
\begin{enumerate}[label={[\arabic*]}]
\item Mudau Geburts-, Heirats- und Sterbebuch 1906–1910, Heiratsregister 1906, Nr. 4
\item \href{https://www.familysearch.org/tree/person/details/L1TQ-VLK}{FamilySearch, ID: L1TQ-VLK}
\end{enumerate}

\end{person}

\begin{person}[
    surname = {Röckel},
    givenname = {Anna},
    suffix = {1881--1882},
    label = {@I955@}
    ]

\begin{tabular}{cl}
\geboren & 05. Mai 1881 in Langenelz\\
\gestorben & 12. Mai 1882 in Langenelz\\
\end{tabular}\\
\medbreak
\textsc{vater}: \hyperref[@I386@]{Joseph Michael Röckel} [19.03.1839--03.10.1888 (9 Kinder)]\\
\textsc{mutter}: \hyperref[@I387@]{Rosa Noe} [15.06.1857--28.08.1920 (10 Kinder)]
\medbreak
\textsc{{geschwister}}
\begin{itemize}
\item \hyperref[@I1268@]{Theodor Röckel} [03.06.1875--24.02.1876]
\item \hyperref[@I1269@]{Emma Röckel} [28.06.1876]
\item \hyperref[@I489@]{Michael Röckel} [21.07.1877]
\item \hyperref[@I1154@]{Ida Röckel} [12.12.1879--09.12.1955 (5 Kinder)]
\item \hyperref[@I954@]{Rosa Röckel} [28.02.1879--29.1880]
\item \hyperref[@I15@]{Otto Röckel} [20.11.1882--14.06.1965 (6 Kinder)]
\item \hyperref[@I956@]{Josef Röckel} [07.09.1884--21.10.1884]
\item \hyperref[@I472@]{Wilhelm Röckel} [07.01.1887--25.11.1968 (7 Kinder)]
\item \hyperref[@I960@]{Franz Karl Müller} [21.10.1891--08.1918]
\item \hyperref[@I961@]{Sebastian Müller} [19.01.1893--12.11.1915]
\item \hyperref[@I481@]{Maria Müller} [07.09.1895--27.11.1972 (4 Kinder)]
\item \hyperref[@I962@]{Rosa Müller} [02.04.1897--12.07.1979]
\item \hyperref[@I963@]{Pius Müller} [28.01.1899]
\end{itemize}
\bigbreak
\textsc{anmerkung}\\
Sterbemonat koennte Maerz sein
\medbreak
\textsc{{quellen}}
\begin{enumerate}[label={[\arabic*]}]
\item Langenelz Standesbuch 1880–1884, Geburtenregister 1881, Nr. 5
\item \href{https://www.familysearch.org/tree/person/details/GMQK-DNQ}{FamilySearch, ID: GMQK-DNQ}
\end{enumerate}

\end{person}

\begin{person}[
    surname = {Röckel},
    givenname = {Josef},
    suffix = {1884--1884},
    label = {@I956@}
    ]

\begin{tabular}{cl}
\geboren & 07. September 1884 in Langenelz\\
\gestorben & 21. Oktober 1884 in Langenelz\\
\end{tabular}\\
\medbreak
\textsc{vater}: \hyperref[@I386@]{Joseph Michael Röckel} [19.03.1839--03.10.1888 (9 Kinder)]\\
\textsc{mutter}: \hyperref[@I387@]{Rosa Noe} [15.06.1857--28.08.1920 (10 Kinder)]
\medbreak
\textsc{{geschwister}}
\begin{itemize}
\item \hyperref[@I1268@]{Theodor Röckel} [03.06.1875--24.02.1876]
\item \hyperref[@I1269@]{Emma Röckel} [28.06.1876]
\item \hyperref[@I489@]{Michael Röckel} [21.07.1877]
\item \hyperref[@I1154@]{Ida Röckel} [12.12.1879--09.12.1955 (5 Kinder)]
\item \hyperref[@I954@]{Rosa Röckel} [28.02.1879--29.1880]
\item \hyperref[@I955@]{Anna Röckel} [05.05.1881--12.05.1882]
\item \hyperref[@I15@]{Otto Röckel} [20.11.1882--14.06.1965 (6 Kinder)]
\item \hyperref[@I472@]{Wilhelm Röckel} [07.01.1887--25.11.1968 (7 Kinder)]
\item \hyperref[@I960@]{Franz Karl Müller} [21.10.1891--08.1918]
\item \hyperref[@I961@]{Sebastian Müller} [19.01.1893--12.11.1915]
\item \hyperref[@I481@]{Maria Müller} [07.09.1895--27.11.1972 (4 Kinder)]
\item \hyperref[@I962@]{Rosa Müller} [02.04.1897--12.07.1979]
\item \hyperref[@I963@]{Pius Müller} [28.01.1899]
\end{itemize}
\bigbreak
\textsc{{quellen}}
\begin{enumerate}[label={[\arabic*]}]
\item Langenelz Standesbuch 1880–1884, Geburtenregister 1884, Nr. 10
\item \href{https://www.familysearch.org/tree/person/details/GMQK-X32}{FamilySearch, ID: GMQK-X32}
\end{enumerate}

\end{person}

\begin{person}[
    surname = {Röckel},
    givenname = {Wilhelm},
    suffix = {1887--1968},
    label = {@I472@},
    filename = {Wilhelm Röckel (1887)}
    ]

\begin{tabular}{cl}
\geboren & 07. Januar 1887 in Langenelz\\
\geheiratet & 14. Februar 1920 in Mudau mit Ida Maria Franziska Berberich \\
\gestorben & 25. November 1968 in Mudau\\
\end{tabular}\\
\medbreak
\textsc{vater}: \hyperref[@I386@]{Joseph Michael Röckel} [19.03.1839--03.10.1888 (9 Kinder)]\\
\textsc{mutter}: \hyperref[@I387@]{Rosa Noe} [15.06.1857--28.08.1920 (10 Kinder)]
\medbreak
\textsc{{geschwister}}
\begin{itemize}
\item \hyperref[@I1268@]{Theodor Röckel} [03.06.1875--24.02.1876]
\item \hyperref[@I1269@]{Emma Röckel} [28.06.1876]
\item \hyperref[@I489@]{Michael Röckel} [21.07.1877]
\item \hyperref[@I1154@]{Ida Röckel} [12.12.1879--09.12.1955 (5 Kinder)]
\item \hyperref[@I954@]{Rosa Röckel} [28.02.1879--29.1880]
\item \hyperref[@I955@]{Anna Röckel} [05.05.1881--12.05.1882]
\item \hyperref[@I15@]{Otto Röckel} [20.11.1882--14.06.1965 (6 Kinder)]
\item \hyperref[@I956@]{Josef Röckel} [07.09.1884--21.10.1884]
\item \hyperref[@I960@]{Franz Karl Müller} [21.10.1891--08.1918]
\item \hyperref[@I961@]{Sebastian Müller} [19.01.1893--12.11.1915]
\item \hyperref[@I481@]{Maria Müller} [07.09.1895--27.11.1972 (4 Kinder)]
\item \hyperref[@I962@]{Rosa Müller} [02.04.1897--12.07.1979]
\item \hyperref[@I963@]{Pius Müller} [28.01.1899]
\end{itemize}
\bigbreak
\textsc{{kinder}}
\begin{itemize}
\item Mathilde Röckel [14.03.1921--22.10.2014]
\item Gisela Röckel [17.07.1923--17.03.1989]
\item Agnes Röckel [05.08.1925--13.04.2017]
\item Camilla Röckel [20.07.1927--25.06.1995]
\item Edeltrud Röckel [08.02.1931]
\item Hubert Röckel [23.12.1933--16.11.2016]
\item Alban Röckel [...]
\end{itemize}
\medbreak
\textsc{{quellen}}
\begin{enumerate}[label={[\arabic*]}]
\item Langenelz Standesbuch 1885–1890, Geburtenregister 1887, Nr. 1
\item Mudau Heiratsbuch 1920–1927, Heiratsregister 1920, Nr. 6
\item Mudau Sterbebuch 1968–1975, Sterberegister 1968, Nr. 11
\item \href{https://www.familysearch.org/tree/person/details/L1TQ-VQK}{FamilySearch, ID: L1TQ-VQK}
\item \href{http://grabsteine.genealogy.net/tomb.php?cem=3902&tomb=372&b=&lang=de}{genealogy.net Grabstein Projekt, Friedhof Mudau}
\end{enumerate}

\end{person}

\begin{person}[
    surname = {Müller},
    givenname = {Franz Karl},
    suffix = {1891--1918},
    label = {@I960@}
    ]

\begin{tabular}{cl}
\geboren & 21. Oktober 1891 in Langenelz\\
\gestorben & August 1918\\
\end{tabular}\\
\medbreak
\textsc{vater}: Sebastian Müller [10.01.1852--20.10.1926 (5 Kinder)]\\
\textsc{mutter}: \hyperref[@I387@]{Rosa Noe} [15.06.1857--28.08.1920 (10 Kinder)]
\medbreak
\textsc{{geschwister}}
\begin{itemize}
\item \hyperref[@I1154@]{Ida Röckel} [12.12.1879--09.12.1955 (5 Kinder)]
\item \hyperref[@I955@]{Anna Röckel} [05.05.1881--12.05.1882]
\item \hyperref[@I15@]{Otto Röckel} [20.11.1882--14.06.1965 (6 Kinder)]
\item \hyperref[@I956@]{Josef Röckel} [07.09.1884--21.10.1884]
\item \hyperref[@I472@]{Wilhelm Röckel} [07.01.1887--25.11.1968 (7 Kinder)]
\item \hyperref[@I961@]{Sebastian Müller} [19.01.1893--12.11.1915]
\item \hyperref[@I481@]{Maria Müller} [07.09.1895--27.11.1972 (4 Kinder)]
\item \hyperref[@I962@]{Rosa Müller} [02.04.1897--12.07.1979]
\item \hyperref[@I963@]{Pius Müller} [28.01.1899]
\end{itemize}
\bigbreak
\textsc{{quellen}}
\begin{enumerate}[label={[\arabic*]}]
\item Langenelz Standesbuch 1891–1899, Geburtenregister 1891, Nr. 14
\item \href{https://www.familysearch.org/tree/person/details/GMQK-6BC}{FamilySearch, ID: GMQK-6BC}
\end{enumerate}

\end{person}

\begin{person}[
    surname = {Müller},
    givenname = {Sebastian},
    suffix = {1893--1915},
    label = {@I961@}
    ]

\begin{tabular}{cl}
\geboren & 19. Januar 1893 in Langenelz\\
\gestorben & 12. November 1915 in Bazancourt\\
\end{tabular}\\
\medbreak
\textsc{vater}: Sebastian Müller [10.01.1852--20.10.1926 (5 Kinder)]\\
\textsc{mutter}: \hyperref[@I387@]{Rosa Noe} [15.06.1857--28.08.1920 (10 Kinder)]
\medbreak
\textsc{{geschwister}}
\begin{itemize}
\item \hyperref[@I1154@]{Ida Röckel} [12.12.1879--09.12.1955 (5 Kinder)]
\item \hyperref[@I955@]{Anna Röckel} [05.05.1881--12.05.1882]
\item \hyperref[@I15@]{Otto Röckel} [20.11.1882--14.06.1965 (6 Kinder)]
\item \hyperref[@I956@]{Josef Röckel} [07.09.1884--21.10.1884]
\item \hyperref[@I472@]{Wilhelm Röckel} [07.01.1887--25.11.1968 (7 Kinder)]
\item \hyperref[@I960@]{Franz Karl Müller} [21.10.1891--08.1918]
\item \hyperref[@I481@]{Maria Müller} [07.09.1895--27.11.1972 (4 Kinder)]
\item \hyperref[@I962@]{Rosa Müller} [02.04.1897--12.07.1979]
\item \hyperref[@I963@]{Pius Müller} [28.01.1899]
\end{itemize}
\bigbreak
\textsc{anmerkung}\\
im Krieg gefallen (siehe Sterbeurkunde)
\medbreak
\textsc{{quellen}}
\begin{enumerate}[label={[\arabic*]}]
\item Langenelz Standesbuch 1891–1899, Geburtenregister 1893, Nr. 2
\item Langenelz Sterbebuch 1911–1927, Sterberegister 1916, Nr. 1
\item \href{https://www.familysearch.org/tree/person/details/GMQK-6D4}{FamilySearch, ID: GMQK-6D4}
\end{enumerate}

\end{person}

\begin{person}[
    surname = {Müller},
    givenname = {Maria},
    suffix = {1895--1972},
    label = {@I481@},
    filename = {Maria Müller (1895)}
    ]

\begin{tabular}{cl}
\geboren & 07. September 1895 in Langenelz\\
\geheiratet & 05. August 1919 in Mudau mit Karl Mechler \\
\gestorben & 27. November 1972 in Mudau\\
\end{tabular}\\
\medbreak
\textsc{vater}: Sebastian Müller [10.01.1852--20.10.1926 (5 Kinder)]\\
\textsc{mutter}: \hyperref[@I387@]{Rosa Noe} [15.06.1857--28.08.1920 (10 Kinder)]
\medbreak
\textsc{{geschwister}}
\begin{itemize}
\item \hyperref[@I1154@]{Ida Röckel} [12.12.1879--09.12.1955 (5 Kinder)]
\item \hyperref[@I955@]{Anna Röckel} [05.05.1881--12.05.1882]
\item \hyperref[@I15@]{Otto Röckel} [20.11.1882--14.06.1965 (6 Kinder)]
\item \hyperref[@I956@]{Josef Röckel} [07.09.1884--21.10.1884]
\item \hyperref[@I472@]{Wilhelm Röckel} [07.01.1887--25.11.1968 (7 Kinder)]
\item \hyperref[@I960@]{Franz Karl Müller} [21.10.1891--08.1918]
\item \hyperref[@I961@]{Sebastian Müller} [19.01.1893--12.11.1915]
\item \hyperref[@I962@]{Rosa Müller} [02.04.1897--12.07.1979]
\item \hyperref[@I963@]{Pius Müller} [28.01.1899]
\end{itemize}
\bigbreak
\textsc{{kinder}}
\begin{itemize}
\item Elsa Mechler [30.08.1920--24.08.2012]
\item Johanna Mechler [25.02.1923--01.09.2007]
\item Alise Mechler [29.05.1925--17.07.2013]
\item Willi Mechler [um 1927]
\end{itemize}
\medbreak
\textsc{anmerkung}\\
Sterbetag nicht sicher
\medbreak
\textsc{{quellen}}
\begin{enumerate}[label={[\arabic*]}]
\item Langenelz Standesbuch 1891–1899, Geburtenregister 1895, Nr. 9
\item Mudau Heiratsbuch 1911–1919, Heiratsregister 1919, Nr. 10
\item Mudau Sterbebuch 1968–1975, Sterberegister 1972, Nr. 10
\item \href{https://www.familysearch.org/tree/person/details/L1TQ-LCT}{FamilySearch, ID: L1TQ-LCT}
\item \href{http://grabsteine.genealogy.net/tomb.php?cem=3902&tomb=413&b=&lang=de}{genealogy.net Grabstein Projekt, Friedhof Mudau}
\end{enumerate}

\end{person}

\begin{person}[
    surname = {Müller},
    givenname = {Rosa},
    suffix = {1897--1979},
    label = {@I962@}
    ]

\begin{tabular}{cl}
\geboren & 02. April 1897 in Langenelz\\
\gestorben & 12. Juli 1979 in Münstertal\\
\end{tabular}\\
\medbreak
\textsc{vater}: Sebastian Müller [10.01.1852--20.10.1926 (5 Kinder)]\\
\textsc{mutter}: \hyperref[@I387@]{Rosa Noe} [15.06.1857--28.08.1920 (10 Kinder)]
\medbreak
\textsc{{geschwister}}
\begin{itemize}
\item \hyperref[@I1154@]{Ida Röckel} [12.12.1879--09.12.1955 (5 Kinder)]
\item \hyperref[@I955@]{Anna Röckel} [05.05.1881--12.05.1882]
\item \hyperref[@I15@]{Otto Röckel} [20.11.1882--14.06.1965 (6 Kinder)]
\item \hyperref[@I956@]{Josef Röckel} [07.09.1884--21.10.1884]
\item \hyperref[@I472@]{Wilhelm Röckel} [07.01.1887--25.11.1968 (7 Kinder)]
\item \hyperref[@I960@]{Franz Karl Müller} [21.10.1891--08.1918]
\item \hyperref[@I961@]{Sebastian Müller} [19.01.1893--12.11.1915]
\item \hyperref[@I481@]{Maria Müller} [07.09.1895--27.11.1972 (4 Kinder)]
\item \hyperref[@I963@]{Pius Müller} [28.01.1899]
\end{itemize}
\bigbreak
\textsc{anmerkung}\\
Schwestertante
\medbreak
\textsc{{quellen}}
\begin{enumerate}[label={[\arabic*]}]
\item Langenelz Standesbuch 1891–1899, Geburtenregister 1897, Nr. 2
\item Sterberegister Münstertal Schwarzwald 1979, Nr. 19
\item \href{https://www.familysearch.org/tree/person/details/GMQK-6BL}{FamilySearch, ID: GMQK-6BL}
\end{enumerate}

\end{person}

\begin{person}[
    surname = {Müller},
    givenname = {Pius},
    suffix = {1899},
    label = {@I963@}
    ]

\begin{tabular}{cl}
\geboren & 28. Januar 1899 in Langenelz\\
\end{tabular}\\
\medbreak
\textsc{vater}: Sebastian Müller [10.01.1852--20.10.1926 (5 Kinder)]\\
\textsc{mutter}: \hyperref[@I387@]{Rosa Noe} [15.06.1857--28.08.1920 (10 Kinder)]
\medbreak
\textsc{{geschwister}}
\begin{itemize}
\item \hyperref[@I1154@]{Ida Röckel} [12.12.1879--09.12.1955 (5 Kinder)]
\item \hyperref[@I955@]{Anna Röckel} [05.05.1881--12.05.1882]
\item \hyperref[@I15@]{Otto Röckel} [20.11.1882--14.06.1965 (6 Kinder)]
\item \hyperref[@I956@]{Josef Röckel} [07.09.1884--21.10.1884]
\item \hyperref[@I472@]{Wilhelm Röckel} [07.01.1887--25.11.1968 (7 Kinder)]
\item \hyperref[@I960@]{Franz Karl Müller} [21.10.1891--08.1918]
\item \hyperref[@I961@]{Sebastian Müller} [19.01.1893--12.11.1915]
\item \hyperref[@I481@]{Maria Müller} [07.09.1895--27.11.1972 (4 Kinder)]
\item \hyperref[@I962@]{Rosa Müller} [02.04.1897--12.07.1979]
\end{itemize}
\bigbreak
\textsc{{quellen}}
\begin{enumerate}[label={[\arabic*]}]
\item Langenelz Standesbuch 1891–1899, Geburtenregister 1899, Nr. 1
\item \href{https://www.familysearch.org/tree/person/details/GMQK-D8K}{FamilySearch, ID: GMQK-D8K}
\end{enumerate}

\end{person}


\addsec{Valentin Mechler  \& Eva Katharina Schäfer }


\begin{person}[
    surname = {Mechler},
    givenname = {Valentin},
    suffix = {1855--1928},
    label = {@I426@},
    filename = {Valentin Mechler (1855)}
    ]

\begin{tabular}{cl}
\geboren & 26. Mai 1855 in Mudau\\
\geheiratet & 22. Oktober 1885 in Schlossau mit Eva Katharina Schäfer \\
\gestorben & 04. Januar 1928 in Mudau\\
\end{tabular}\\
\medbreak
\textsc{vater}: \hyperref[@I946@]{Johann Valentin Mechler} [08.03.1815--15.12.1886 (5 Kinder)]\\
\textsc{mutter}: \hyperref[@I947@]{Katharina Scholl} [10.11.1813--11.05.1890 (5 Kinder)]
\medbreak
\textsc{{geschwister}}
\begin{itemize}
\item \hyperref[@I1750@]{Theresa Mechler} [16.08.1840]
\item \hyperref[@I1751@]{Katharina Mechler} [17.08.1843]
\item \hyperref[@I1752@]{Karl Mechler} [13.05.1847--18.12.1870]
\item \hyperref[@I1753@]{Wilhelm Mechler} [25.05.1851]
\end{itemize}
\bigbreak
\textsc{{kinder}}
\begin{itemize}
\item \hyperref[@I1261@]{Wilhelm Mechler} [17.02.1887--06.04.1956 (1 Kind)]
\item \hyperref[@I16@]{Maria Anna Mechler} [15.08.1889--25.10.1966 (6 Kinder)]
\item \hyperref[@I480@]{Karl Mechler} [15.08.1889--30.12.1968 (4 Kinder)]
\item \hyperref[@I1267@]{Rosa Theresia Mechler} [19.08.1894--06.02.1982]
\end{itemize}
\medbreak
\textsc{{quellen}}
\begin{enumerate}[label={[\arabic*]}]
\item \href{http://www.landesarchiv-bw.de/plink/?f=4-1119480-185}{GLA Karlsruhe, Mudau, katholische Gemeinde: Standesbuch 1845–1855, Geburtenregister 1855, Nr. 12 (Bild 185)}
\item Schloßau Heiratsbuch 1870–1899, Heiratsregister 1885, Nr. 4
\item Mudau Sterbebuch 1920–1928, Sterberegister 1928, Nr. 1
\item \href{https://www.familysearch.org/tree/person/details/LT2X-249}{FamilySearch, ID: LT2X-249}
\end{enumerate}

\end{person}

\begin{person}[
    surname = {Schäfer},
    givenname = {Eva Katharina},
    suffix = {1855--1942},
    label = {@I388@},
    filename = {Eva Katharina Schäfer (1855)}
    ]

\begin{tabular}{cl}
\geboren & 06. Oktober 1855 in Schlossau\\
\geheiratet & 22. Oktober 1885 in Schlossau mit Valentin Mechler \\
\gestorben & 21. April 1942 in Mudau\\
\end{tabular}\\
\medbreak
\textsc{vater}: \hyperref[@I948@]{Johann Josef Schäfer} [23.11.1811--21.01.1883 (12 Kinder)]\\
\textsc{mutter}: \hyperref[@I949@]{Maria Anna Farrenkopf} [29.08.1819--01.06.1905 (8 Kinder)]
\medbreak
\textsc{{geschwister}}
\begin{itemize}
\item \hyperref[@I1866@]{Johann Valentin Schäfer} [19.04.1835]
\item \hyperref[@I1867@]{Rosina Schäfer} [30.09.1837]
\item \hyperref[@I1871@]{Anna Christina Schäfer} [28.07.1839]
\item \hyperref[@I1870@]{Margaretha Schäfer} [28.04.1842]
\item \hyperref[@I1396@]{Karl Schäfer} [14.05.1844--02.02.1933]
\item \hyperref[@I1397@]{Linus Schäfer} [23.09.1846--nach 1885]
\item \hyperref[@I1398@]{Eva Clara Schäfer} [10.02.1850--08.07.1881]
\item \hyperref[@I1399@]{Maria Anna Schäfer} [05.10.1852]
\item \hyperref[@I1400@]{Franz Schäfer} [27.03.1858--06.10.1950]
\item \hyperref[@I1401@]{Rosina Schäfer} [14.09.1860]
\item \hyperref[@I1402@]{Ferdinand Schäfer} [28.06.1864]
\end{itemize}
\bigbreak
\textsc{{kinder}}
\begin{itemize}
\item \hyperref[@I1261@]{Wilhelm Mechler} [17.02.1887--06.04.1956 (1 Kind)]
\item \hyperref[@I16@]{Maria Anna Mechler} [15.08.1889--25.10.1966 (6 Kinder)]
\item \hyperref[@I480@]{Karl Mechler} [15.08.1889--30.12.1968 (4 Kinder)]
\item \hyperref[@I1267@]{Rosa Theresia Mechler} [19.08.1894--06.02.1982]
\end{itemize}
\medbreak
\textsc{{quellen}}
\begin{enumerate}[label={[\arabic*]}]
\item \href{http://www.landesarchiv-bw.de/plink/?f=4-1119607-38}{GLA Karlsruhe, Schlossau, katholische Gemeinde: Standesbuch 1851–1866, Geburtenregister 1855, Nr. 16 (Bild 38)}
\item Schloßau Heiratsbuch 1870–1899, Heiratsregister 1885, Nr. 4
\item Schloßau Sterbebuch 1934–1965, Sterberegister 1942, Nr. 10
\item \href{https://www.familysearch.org/tree/person/details/LT2X-249}{FamilySearch, ID: LT2X-249}
\end{enumerate}

\end{person}

\begin{person}[
    surname = {Mechler},
    givenname = {Wilhelm},
    suffix = {1887--1956},
    label = {@I1261@}
    ]

\begin{tabular}{cl}
\geboren & 17. Februar 1887 in Mudau\\
\gestorben & 06. April 1956 in Freiburg\\
\end{tabular}\\
\medbreak
\textsc{vater}: \hyperref[@I426@]{Valentin Mechler} [26.05.1855--04.01.1928 (4 Kinder)]\\
\textsc{mutter}: \hyperref[@I388@]{Eva Katharina Schäfer} [06.10.1855--21.04.1942 (4 Kinder)]
\medbreak
\textsc{{geschwister}}
\begin{itemize}
\item \hyperref[@I480@]{Karl Mechler} [15.08.1889--30.12.1968 (4 Kinder)]
\item \hyperref[@I16@]{Maria Anna Mechler} [15.08.1889--25.10.1966 (6 Kinder)]
\item \hyperref[@I1267@]{Rosa Theresia Mechler} [19.08.1894--06.02.1982]
\end{itemize}
\bigbreak
\textsc{{kinder}}
\begin{itemize}
\item Rita Mechler [...]
\end{itemize}
\medbreak
\textsc{{quellen}}
\begin{enumerate}[label={[\arabic*]}]
\item Mudau Standesbuch 1885–1887, Geburtenregister 1887, Nr. 7
\item Sterberegister Freiburg 1956, Nr. 607
\item Mündliche Überlieferung Erich Zimmermann
\end{enumerate}

\end{person}

\begin{person}[
    surname = {Mechler},
    givenname = {Karl},
    suffix = {1889--1968},
    label = {@I480@},
    filename = {Karl Mechler (1889)}
    ]

\begin{tabular}{cl}
\geboren & 15. August 1889 in Mudau\\
\geheiratet & 05. August 1919 in Mudau mit Maria Müller \\
\gestorben & 30. Dezember 1968 in Mudau\\
\end{tabular}\\
\medbreak
\textsc{vater}: \hyperref[@I426@]{Valentin Mechler} [26.05.1855--04.01.1928 (4 Kinder)]\\
\textsc{mutter}: \hyperref[@I388@]{Eva Katharina Schäfer} [06.10.1855--21.04.1942 (4 Kinder)]
\medbreak
\textsc{{geschwister}}
\begin{itemize}
\item \hyperref[@I1261@]{Wilhelm Mechler} [17.02.1887--06.04.1956 (1 Kind)]
\item \hyperref[@I16@]{Maria Anna Mechler} [15.08.1889--25.10.1966 (6 Kinder)]
\item \hyperref[@I1267@]{Rosa Theresia Mechler} [19.08.1894--06.02.1982]
\end{itemize}
\bigbreak
\textsc{{kinder}}
\begin{itemize}
\item Elsa Mechler [30.08.1920--24.08.2012]
\item Johanna Mechler [25.02.1923--01.09.2007]
\item Alise Mechler [29.05.1925--17.07.2013]
\item Willi Mechler [um 1927]
\end{itemize}
\medbreak
\textsc{{quellen}}
\begin{enumerate}[label={[\arabic*]}]
\item Mudau Standesbuch 1888–1890, Geburtenregister 1889, Nr. 27
\item Mudau Heiratsbuch 1911–1919, Heiratsregister 1919, Nr. 10
\item Mudau Sterbebuch 1968–1975, Sterberegister 1968, Nr. 13
\item \href{http://grabsteine.genealogy.net/tomb.php?cem=3902&tomb=413&b=&lang=de}{genealogy.net Grabstein Projekt, Friedhof Mudau}
\end{enumerate}

\end{person}

\begin{person}[
    surname = {Mechler},
    givenname = {Rosa Theresia},
    suffix = {1894--1982},
    label = {@I1267@},
    filename = {Rosa Theresia Mechler (1894)}
    ]

\begin{tabular}{cl}
\geboren & 19. August 1894 in Mudau\\
\gestorben & 06. Februar 1982 in Mudau\\
\end{tabular}\\
\medbreak
\textsc{vater}: \hyperref[@I426@]{Valentin Mechler} [26.05.1855--04.01.1928 (4 Kinder)]\\
\textsc{mutter}: \hyperref[@I388@]{Eva Katharina Schäfer} [06.10.1855--21.04.1942 (4 Kinder)]
\medbreak
\textsc{{geschwister}}
\begin{itemize}
\item \hyperref[@I1261@]{Wilhelm Mechler} [17.02.1887--06.04.1956 (1 Kind)]
\item \hyperref[@I480@]{Karl Mechler} [15.08.1889--30.12.1968 (4 Kinder)]
\item \hyperref[@I16@]{Maria Anna Mechler} [15.08.1889--25.10.1966 (6 Kinder)]
\end{itemize}
\bigbreak
\textsc{anmerkung}\\
Ledig, war Haushilfe bei Apotheker Hettering (?). Hat dessen Haus beim Birkenweg/Donebacher Strasse geerbt (Jetzt Drabinsiki/Münch)
\medbreak
\textsc{{quellen}}
\begin{enumerate}[label={[\arabic*]}]
\item Geburtenregister Mudau 1894, Nr. 39
\end{enumerate}

\end{person}


\addsec{Franz Karl Galm  \& Karolina Schwanninger }


\begin{person}[
    surname = {Galm},
    givenname = {Franz Karl},
    suffix = {1854--1934},
    label = {@I144@},
    filename = {Karl Galm (1854)}
    ]

\begin{tabular}{cl}
\geboren & 05. September 1854 in Langenelz\\
\geheiratet & 02. März 1882 in Langenelz mit Karolina Schwanninger \\
\gestorben & 06. Dezember 1934 in Langenelz\\
\bestattet &  in Mudau\\
\end{tabular}\\
\medbreak
\textsc{vater}: \hyperref[@I146@]{Johann Josef Galm} [04.07.1822--23.04.1887 (7 Kinder)]\\
\textsc{mutter}: \hyperref[@I147@]{Anna Maria Münch} [29.09.1821--03.07.1891 (7 Kinder)]
\medbreak
\textsc{{geschwister}}
\begin{itemize}
\item \hyperref[@I183@]{Maria Karolina Galm} [21.01.1850--31.07.1908]
\item \hyperref[@I197@]{Rosalia Galm} [12.08.1852--21.08.1854]
\item \hyperref[@I180@]{Julius Galm} [21.02.1857--03.01.1929 (4 Kinder)]
\item \hyperref[@I181@]{Johann Josef Galm} [23.04.1859--10.05.1910 (1 Kind)]
\item \hyperref[@I198@]{Anna Maria Galm} [20.07.1862--02.12.1942 (2 Kinder)]
\item \hyperref[@I182@]{Wilhelm Galm} [02.02.1865--15.08.1943 (2 Kinder)]
\end{itemize}
\bigbreak
\textsc{{kinder}}
\begin{itemize}
\item \hyperref[@I7@]{Julius Galm} [28.02.1883--23.05.1929 (7 Kinder)]
\item \hyperref[@I163@]{Anna Galm} [04.05.1884--10.08.1955 (10 Kinder)]
\item \hyperref[@I164@]{Karl Galm} [06.10.1886--13.10.1963]
\item \hyperref[@I165@]{Karoline Galm} [06.06.1888--13.07.1951]
\item \hyperref[@I2031@]{Joseph Galm} [23.12.1890--16.01.1891]
\item \hyperref[@I166@]{Philipp Galm} [28.12.1891--25.02.1953 (3 Kinder)]
\item \hyperref[@I167@]{Anton Galm} [02.12.1893--08.05.1915]
\item \hyperref[@I2032@]{Wilhelm Galm} [20.08.1895--24.09.1895]
\item \hyperref[@I168@]{Ida Galm} [31.07.1898--19.10.1988 (3 Kinder)]
\item \hyperref[@I169@]{Maria Galm} [04.09.1899--06.12.1962 (3 Kinder)]
\end{itemize}
\medbreak
\textsc{{quellen}}
\begin{enumerate}[label={[\arabic*]}]
\item \href{http://www.landesarchiv-bw.de/plink/?f=4-1119438-98}{GLA Karlsruhe, Langenelz, katholische Gemeinde: Standesbuch 1839–1870, Geburtenregister 1854, Nr. 6 (Bild 98)}
\item Langenelz Standesbuch 1880–1884, Heiratsregister 1882, Nr. 1
\item Heimatbuch Philipp Galm, Seite 23
\item Ahnentafel Galm
\item Ahnentafel Helga Schölch
\item Ahnentafel Erich Schnorr
\item \href{https://www.familysearch.org/tree/person/details/LVPW-YHB}{FamilySearch, ID: LVPW-YHB}
\end{enumerate}

\end{person}

\begin{person}[
    surname = {Schwanninger},
    givenname = {Karolina},
    suffix = {1858--1926},
    label = {@I145@}
    ]

\begin{tabular}{cl}
\geboren & 14. September 1858 in Mörschenhardt\\
\geheiratet & 02. März 1882 in Langenelz mit Franz Karl Galm \\
\gestorben & 02. Juli 1926 in Langenelz\\
\bestattet & 05. Juli 1926 in Mudau\\
\end{tabular}\\
\medbreak
\textsc{vater}: \hyperref[@I148@]{Franz, Josef Schwanninger} [11.05.1823--06.04.1905 (8 Kinder)]\\
\textsc{mutter}: \hyperref[@I149@]{Margaretha Rögner} [04.04.1834--20.08.1906 (8 Kinder)]
\medbreak
\textsc{{geschwister}}
\begin{itemize}
\item \hyperref[@I1302@]{Wilhelm Schwanninger} [18.02.1857--nach 1904 (9 Kinder)]
\item \hyperref[@I1172@]{Margaretha Schwanninger} [31.07.1860]
\item \hyperref[@I1303@]{Rosina Schwanninger} [09.11.1861--10.07.1932 (4 Kinder)]
\item \hyperref[@I1304@]{Friedrich Schwanninger} [04.04.1863--04.06.1867]
\item \hyperref[@I1305@]{Josefa Schwanninger} [16.07.1865]
\item \hyperref[@I1873@]{Katharina Schwanninger} [25.11.1873]
\item \hyperref[@I2108@]{Ida Schwanninger} [...]
\end{itemize}
\bigbreak
\textsc{{kinder}}
\begin{itemize}
\item \hyperref[@I7@]{Julius Galm} [28.02.1883--23.05.1929 (7 Kinder)]
\item \hyperref[@I163@]{Anna Galm} [04.05.1884--10.08.1955 (10 Kinder)]
\item \hyperref[@I164@]{Karl Galm} [06.10.1886--13.10.1963]
\item \hyperref[@I165@]{Karoline Galm} [06.06.1888--13.07.1951]
\item \hyperref[@I2031@]{Joseph Galm} [23.12.1890--16.01.1891]
\item \hyperref[@I166@]{Philipp Galm} [28.12.1891--25.02.1953 (3 Kinder)]
\item \hyperref[@I167@]{Anton Galm} [02.12.1893--08.05.1915]
\item \hyperref[@I2032@]{Wilhelm Galm} [20.08.1895--24.09.1895]
\item \hyperref[@I168@]{Ida Galm} [31.07.1898--19.10.1988 (3 Kinder)]
\item \hyperref[@I169@]{Maria Galm} [04.09.1899--06.12.1962 (3 Kinder)]
\end{itemize}
\medbreak
\textsc{anmerkung}\\
Ihre Hauptbeschäftigung im Winter war das Spinnen, das sie besonders liebte. Sie hat in einem Winter oft bis zu 50 m Lain...
\medbreak
\textsc{{quellen}}
\begin{enumerate}[label={[\arabic*]}]
\item \href{http://www.landesarchiv-bw.de/plink/?f=4-1119442-109}{GLA Karlsruhe, Mörschenhardt, katholische Gemeinde: Standesbuch 1839–1870, Geburtenregister 1858, Nr. 5 (Bild 109)}
\item Langenelz Standesbuch 1880–1884, Heiratsregister 1882, Nr. 1
\item Langenelz Sterbebuch 1911–1927, Sterberegister 1926, Nr. 3
\item Heimatbuch Philipp Galm, Seite 27
\item Ahnentafel Alfons Bauer
\item Ahnentafel Galm
\item Ahnentafel Helga Schölch
\item \href{https://www.familysearch.org/tree/person/details/LTC6-8JZ}{FamilySearch, ID: LTC6-8JZ}
\end{enumerate}

\end{person}

\begin{person}[
    surname = {Galm},
    givenname = {Anna},
    suffix = {1884--1955},
    label = {@I163@},
    filename = {Anna Frank (1884)}
    ]

\begin{tabular}{cl}
\geboren & 04. Mai 1884 in Langenelz\\
\geheiratet & 11. Juli 1906 in Rumpfen mit Linus Frank \\
\gestorben & 10. August 1955 in Rumpfen\\
\end{tabular}\\
\medbreak
\textsc{vater}: \hyperref[@I144@]{Franz Karl Galm} [05.09.1854--06.12.1934 (10 Kinder)]\\
\textsc{mutter}: \hyperref[@I145@]{Karolina Schwanninger} [14.09.1858--02.07.1926 (10 Kinder)]
\medbreak
\textsc{{geschwister}}
\begin{itemize}
\item \hyperref[@I7@]{Julius Galm} [28.02.1883--23.05.1929 (7 Kinder)]
\item \hyperref[@I164@]{Karl Galm} [06.10.1886--13.10.1963]
\item \hyperref[@I165@]{Karoline Galm} [06.06.1888--13.07.1951]
\item \hyperref[@I2031@]{Joseph Galm} [23.12.1890--16.01.1891]
\item \hyperref[@I166@]{Philipp Galm} [28.12.1891--25.02.1953 (3 Kinder)]
\item \hyperref[@I167@]{Anton Galm} [02.12.1893--08.05.1915]
\item \hyperref[@I2032@]{Wilhelm Galm} [20.08.1895--24.09.1895]
\item \hyperref[@I168@]{Ida Galm} [31.07.1898--19.10.1988 (3 Kinder)]
\item \hyperref[@I169@]{Maria Galm} [04.09.1899--06.12.1962 (3 Kinder)]
\end{itemize}
\bigbreak
\textsc{{kinder}}
\begin{itemize}
\item Karl Frank [11.07.1909--23.08.1970 (6 Kinder)]
\item Lydia Frank [03.08.1912--30.10.1996]
\item Amalia Frank [08.01.1914--23.01.1998]
\item Alois Frank [28.08.1919--06.09.1998]
\item Leo Frank [09.07.1922--01.09.2002]
\item Bruno Frank [05.08.1925--07.09.1975]
\item Maria Frank [19.09.1928]
\item Rosa Frank [22.05.1930]
\item Linus Frank [...]
\item Wilhelm Frank [...]
\end{itemize}
\medbreak
\textsc{{quellen}}
\begin{enumerate}[label={[\arabic*]}]
\item Heimatbuch Philipp Galm, Seite 33
\item \href{https://www.familysearch.org/tree/person/details/L1TQ-H9C}{FamilySearch, ID: L1TQ-H9C}
\end{enumerate}

\end{person}

\begin{person}[
    surname = {Galm},
    givenname = {Karl},
    suffix = {1886--1963},
    label = {@I164@},
    filename = {Karl Galm (1886)}
    ]

\begin{tabular}{cl}
\geboren & 06. Oktober 1886 in Langenelz\\
\gestorben & 13. Oktober 1963 in Langenelz\\
\end{tabular}\\
\medbreak
\textsc{vater}: \hyperref[@I144@]{Franz Karl Galm} [05.09.1854--06.12.1934 (10 Kinder)]\\
\textsc{mutter}: \hyperref[@I145@]{Karolina Schwanninger} [14.09.1858--02.07.1926 (10 Kinder)]
\medbreak
\textsc{{geschwister}}
\begin{itemize}
\item \hyperref[@I7@]{Julius Galm} [28.02.1883--23.05.1929 (7 Kinder)]
\item \hyperref[@I163@]{Anna Galm} [04.05.1884--10.08.1955 (10 Kinder)]
\item \hyperref[@I165@]{Karoline Galm} [06.06.1888--13.07.1951]
\item \hyperref[@I2031@]{Joseph Galm} [23.12.1890--16.01.1891]
\item \hyperref[@I166@]{Philipp Galm} [28.12.1891--25.02.1953 (3 Kinder)]
\item \hyperref[@I167@]{Anton Galm} [02.12.1893--08.05.1915]
\item \hyperref[@I2032@]{Wilhelm Galm} [20.08.1895--24.09.1895]
\item \hyperref[@I168@]{Ida Galm} [31.07.1898--19.10.1988 (3 Kinder)]
\item \hyperref[@I169@]{Maria Galm} [04.09.1899--06.12.1962 (3 Kinder)]
\end{itemize}
\bigbreak
\textsc{{quellen}}
\begin{enumerate}[label={[\arabic*]}]
\item Heimatbuch Philipp Galm, Seite 35
\item \href{https://www.familysearch.org/tree/person/details/L1TQ-H9V}{FamilySearch, ID: L1TQ-H9V}
\end{enumerate}

\end{person}

\begin{person}[
    surname = {Galm},
    givenname = {Karoline},
    suffix = {1888--1951},
    label = {@I165@},
    filename = {Karoline Galm (1888)}
    ]

\begin{tabular}{cl}
\geboren & 06. Juni 1888 in Langenelz\\
\geheiratet &  mit Valentin Bauer \\
\gestorben & 13. Juli 1951 in Balsbach\\
\end{tabular}\\
\medbreak
\textsc{vater}: \hyperref[@I144@]{Franz Karl Galm} [05.09.1854--06.12.1934 (10 Kinder)]\\
\textsc{mutter}: \hyperref[@I145@]{Karolina Schwanninger} [14.09.1858--02.07.1926 (10 Kinder)]
\medbreak
\textsc{{geschwister}}
\begin{itemize}
\item \hyperref[@I7@]{Julius Galm} [28.02.1883--23.05.1929 (7 Kinder)]
\item \hyperref[@I163@]{Anna Galm} [04.05.1884--10.08.1955 (10 Kinder)]
\item \hyperref[@I164@]{Karl Galm} [06.10.1886--13.10.1963]
\item \hyperref[@I2031@]{Joseph Galm} [23.12.1890--16.01.1891]
\item \hyperref[@I166@]{Philipp Galm} [28.12.1891--25.02.1953 (3 Kinder)]
\item \hyperref[@I167@]{Anton Galm} [02.12.1893--08.05.1915]
\item \hyperref[@I2032@]{Wilhelm Galm} [20.08.1895--24.09.1895]
\item \hyperref[@I168@]{Ida Galm} [31.07.1898--19.10.1988 (3 Kinder)]
\item \hyperref[@I169@]{Maria Galm} [04.09.1899--06.12.1962 (3 Kinder)]
\end{itemize}
\bigbreak
\textsc{anmerkung}\\
Großmutter von Alfons Bauer
nach Tod des ersten Mannes nochmal einen Bruder geheiratet
\medbreak
\textsc{{quellen}}
\begin{enumerate}[label={[\arabic*]}]
\item Heimatbuch Philipp Galm, Seite 41
\item \href{https://www.familysearch.org/tree/person/details/L1TQ-H9B}{FamilySearch, ID: L1TQ-H9B}
\end{enumerate}

\end{person}

\begin{person}[
    surname = {Galm},
    givenname = {Joseph},
    suffix = {1890--1891},
    label = {@I2031@}
    ]

\begin{tabular}{cl}
\geboren & 23. Dezember 1890 in Langenelz\\
\gestorben & 16. Januar 1891 in Langenelz\\
\end{tabular}\\
\medbreak
\textsc{vater}: \hyperref[@I144@]{Franz Karl Galm} [05.09.1854--06.12.1934 (10 Kinder)]\\
\textsc{mutter}: \hyperref[@I145@]{Karolina Schwanninger} [14.09.1858--02.07.1926 (10 Kinder)]
\medbreak
\textsc{{geschwister}}
\begin{itemize}
\item \hyperref[@I7@]{Julius Galm} [28.02.1883--23.05.1929 (7 Kinder)]
\item \hyperref[@I163@]{Anna Galm} [04.05.1884--10.08.1955 (10 Kinder)]
\item \hyperref[@I164@]{Karl Galm} [06.10.1886--13.10.1963]
\item \hyperref[@I165@]{Karoline Galm} [06.06.1888--13.07.1951]
\item \hyperref[@I166@]{Philipp Galm} [28.12.1891--25.02.1953 (3 Kinder)]
\item \hyperref[@I167@]{Anton Galm} [02.12.1893--08.05.1915]
\item \hyperref[@I2032@]{Wilhelm Galm} [20.08.1895--24.09.1895]
\item \hyperref[@I168@]{Ida Galm} [31.07.1898--19.10.1988 (3 Kinder)]
\item \hyperref[@I169@]{Maria Galm} [04.09.1899--06.12.1962 (3 Kinder)]
\end{itemize}
\bigbreak
\textsc{anmerkung}\\
Geburt koennte auch 22. dec sein
mit 24 Tagen gestorben
\medbreak
\textsc{{quellen}}
\begin{enumerate}[label={[\arabic*]}]
\item Langenelz Standesbuch 1885–1890, Geburtenregister 1890, Nr. 8
\item Langenelz Standesbuch 1891–1899, Sterberegister 1891, Nr. 1
\item \href{https://www.familysearch.org/tree/person/details/G9NG-PXW}{FamilySearch, ID: G9NG-PXW}
\end{enumerate}

\end{person}

\begin{person}[
    surname = {Galm},
    givenname = {Philipp},
    suffix = {1891--1953},
    label = {@I166@},
    filename = {Philipp Galm (1891)}
    ]

\begin{tabular}{cl}
\geboren & 28. Dezember 1891 in Langenelz\\
\geheiratet & 10. April 1921 in Limbach mit Juliana Münch \\
\gestorben & 25. Februar 1953\\
\end{tabular}\\
\medbreak
\textsc{vater}: \hyperref[@I144@]{Franz Karl Galm} [05.09.1854--06.12.1934 (10 Kinder)]\\
\textsc{mutter}: \hyperref[@I145@]{Karolina Schwanninger} [14.09.1858--02.07.1926 (10 Kinder)]
\medbreak
\textsc{{geschwister}}
\begin{itemize}
\item \hyperref[@I7@]{Julius Galm} [28.02.1883--23.05.1929 (7 Kinder)]
\item \hyperref[@I163@]{Anna Galm} [04.05.1884--10.08.1955 (10 Kinder)]
\item \hyperref[@I164@]{Karl Galm} [06.10.1886--13.10.1963]
\item \hyperref[@I165@]{Karoline Galm} [06.06.1888--13.07.1951]
\item \hyperref[@I2031@]{Joseph Galm} [23.12.1890--16.01.1891]
\item \hyperref[@I167@]{Anton Galm} [02.12.1893--08.05.1915]
\item \hyperref[@I2032@]{Wilhelm Galm} [20.08.1895--24.09.1895]
\item \hyperref[@I168@]{Ida Galm} [31.07.1898--19.10.1988 (3 Kinder)]
\item \hyperref[@I169@]{Maria Galm} [04.09.1899--06.12.1962 (3 Kinder)]
\end{itemize}
\bigbreak
\textsc{{kinder}}
\begin{itemize}
\item Hermann Galm [1924--2001]
\item Willi Galm [...]
\item Rita Galm [...]
\end{itemize}
\medbreak
\textsc{anmerkung}\\
pruefe Todesdatum
hatte 5 Kinder
\medbreak
\textsc{{quellen}}
\begin{enumerate}[label={[\arabic*]}]
\item Langenelz Standesbuch 1891–1899, Geburtenregister 1891, Nr. 15
\item Heimatbuch Philipp Galm, Seite 41
\item \href{https://www.familysearch.org/tree/person/details/L1TQ-H16}{FamilySearch, ID: L1TQ-H16}
\end{enumerate}

\end{person}

\begin{person}[
    surname = {Galm},
    givenname = {Anton},
    suffix = {1893--1915},
    label = {@I167@},
    filename = {Anton Galm (1893)}
    ]

\begin{tabular}{cl}
\geboren & 02. Dezember 1893 in Langenelz\\
\gestorben & 08. Mai 1915 in Loretto-Höhe\\
\end{tabular}\\
\medbreak
\textsc{vater}: \hyperref[@I144@]{Franz Karl Galm} [05.09.1854--06.12.1934 (10 Kinder)]\\
\textsc{mutter}: \hyperref[@I145@]{Karolina Schwanninger} [14.09.1858--02.07.1926 (10 Kinder)]
\medbreak
\textsc{{geschwister}}
\begin{itemize}
\item \hyperref[@I7@]{Julius Galm} [28.02.1883--23.05.1929 (7 Kinder)]
\item \hyperref[@I163@]{Anna Galm} [04.05.1884--10.08.1955 (10 Kinder)]
\item \hyperref[@I164@]{Karl Galm} [06.10.1886--13.10.1963]
\item \hyperref[@I165@]{Karoline Galm} [06.06.1888--13.07.1951]
\item \hyperref[@I2031@]{Joseph Galm} [23.12.1890--16.01.1891]
\item \hyperref[@I166@]{Philipp Galm} [28.12.1891--25.02.1953 (3 Kinder)]
\item \hyperref[@I2032@]{Wilhelm Galm} [20.08.1895--24.09.1895]
\item \hyperref[@I168@]{Ida Galm} [31.07.1898--19.10.1988 (3 Kinder)]
\item \hyperref[@I169@]{Maria Galm} [04.09.1899--06.12.1962 (3 Kinder)]
\end{itemize}
\bigbreak
\textsc{{quellen}}
\begin{enumerate}[label={[\arabic*]}]
\item Langenelz Standesbuch 1891–1899, Geburtenregister 1893, Nr. 12
\item Heimatbuch Philipp Galm, Seite 43
\item \href{https://www.familysearch.org/tree/person/details/L1TQ-4F6}{FamilySearch, ID: L1TQ-4F6}
\end{enumerate}

\end{person}

\begin{person}[
    surname = {Galm},
    givenname = {Wilhelm},
    suffix = {1895--1895},
    label = {@I2032@}
    ]

\begin{tabular}{cl}
\geboren & 20. August 1895 in Langenelz\\
\gestorben & 24. September 1895 in Langenelz\\
\end{tabular}\\
\medbreak
\textsc{vater}: \hyperref[@I144@]{Franz Karl Galm} [05.09.1854--06.12.1934 (10 Kinder)]\\
\textsc{mutter}: \hyperref[@I145@]{Karolina Schwanninger} [14.09.1858--02.07.1926 (10 Kinder)]
\medbreak
\textsc{{geschwister}}
\begin{itemize}
\item \hyperref[@I7@]{Julius Galm} [28.02.1883--23.05.1929 (7 Kinder)]
\item \hyperref[@I163@]{Anna Galm} [04.05.1884--10.08.1955 (10 Kinder)]
\item \hyperref[@I164@]{Karl Galm} [06.10.1886--13.10.1963]
\item \hyperref[@I165@]{Karoline Galm} [06.06.1888--13.07.1951]
\item \hyperref[@I2031@]{Joseph Galm} [23.12.1890--16.01.1891]
\item \hyperref[@I166@]{Philipp Galm} [28.12.1891--25.02.1953 (3 Kinder)]
\item \hyperref[@I167@]{Anton Galm} [02.12.1893--08.05.1915]
\item \hyperref[@I168@]{Ida Galm} [31.07.1898--19.10.1988 (3 Kinder)]
\item \hyperref[@I169@]{Maria Galm} [04.09.1899--06.12.1962 (3 Kinder)]
\end{itemize}
\bigbreak
\textsc{anmerkung}\\
mit 5 Wochen gestorben
\medbreak
\textsc{{quellen}}
\begin{enumerate}[label={[\arabic*]}]
\item Langenelz Standesbuch 1891–1899, Geburtenregister 1895, Nr. 8
\item Langenelz Standesbuch 1891–1899, Sterberegister 1895, Nr. 5
\item \href{https://www.familysearch.org/tree/person/details/G9NP-MVC}{FamilySearch, ID: G9NP-MVC}
\end{enumerate}

\end{person}

\begin{person}[
    surname = {Galm},
    givenname = {Ida},
    suffix = {1898--1988},
    label = {@I168@},
    filename = {Ida Galm (1898)}
    ]

\begin{tabular}{cl}
\geboren & 31. Juli 1898 in Langenelz\\
\geheiratet &  mit Michael Grim \\
\gestorben & 19. Oktober 1988 in Hesselbach \\
\end{tabular}\\
\medbreak
\textsc{vater}: \hyperref[@I144@]{Franz Karl Galm} [05.09.1854--06.12.1934 (10 Kinder)]\\
\textsc{mutter}: \hyperref[@I145@]{Karolina Schwanninger} [14.09.1858--02.07.1926 (10 Kinder)]
\medbreak
\textsc{{geschwister}}
\begin{itemize}
\item \hyperref[@I7@]{Julius Galm} [28.02.1883--23.05.1929 (7 Kinder)]
\item \hyperref[@I163@]{Anna Galm} [04.05.1884--10.08.1955 (10 Kinder)]
\item \hyperref[@I164@]{Karl Galm} [06.10.1886--13.10.1963]
\item \hyperref[@I165@]{Karoline Galm} [06.06.1888--13.07.1951]
\item \hyperref[@I2031@]{Joseph Galm} [23.12.1890--16.01.1891]
\item \hyperref[@I166@]{Philipp Galm} [28.12.1891--25.02.1953 (3 Kinder)]
\item \hyperref[@I167@]{Anton Galm} [02.12.1893--08.05.1915]
\item \hyperref[@I2032@]{Wilhelm Galm} [20.08.1895--24.09.1895]
\item \hyperref[@I169@]{Maria Galm} [04.09.1899--06.12.1962 (3 Kinder)]
\end{itemize}
\bigbreak
\textsc{{kinder}}
\begin{itemize}
\item Michael Grim [15.08.1929--22.12.2003]
\item Anna Grim [22.10.1932--21.03.2015]
\item Hildegard Grim [...]
\end{itemize}
\medbreak
\textsc{{quellen}}
\begin{enumerate}[label={[\arabic*]}]
\item Langenelz Standesbuch 1891–1899, Geburtenregister 1898, Nr. 7
\item Sterberegister Hesseneck 1988, Nr. 5
\item Heimatbuch Philipp Galm, Seite 43
\item \href{https://www.familysearch.org/tree/person/details/L1TQ-H3G}{FamilySearch, ID: L1TQ-H3G}
\item \href{http://grabsteine.genealogy.net/tomb.php?cem=1353&tomb=34&b=&lang=de}{genealogy.net Grabstein Projekt, Friedhof Hesselbach}
\end{enumerate}

\end{person}

\begin{person}[
    surname = {Galm},
    givenname = {Maria},
    suffix = {1899--1962},
    label = {@I169@},
    filename = {Maria Büchler (1899)}
    ]

\begin{tabular}{cl}
\geboren & 04. September 1899 in Langenelz\\
\geheiratet & 09. August 1926 in Schlossau mit Emil Büchler \\
\gestorben & 06. Dezember 1962 in Schlossau\\
\end{tabular}\\
\medbreak
\textsc{vater}: \hyperref[@I144@]{Franz Karl Galm} [05.09.1854--06.12.1934 (10 Kinder)]\\
\textsc{mutter}: \hyperref[@I145@]{Karolina Schwanninger} [14.09.1858--02.07.1926 (10 Kinder)]
\medbreak
\textsc{{geschwister}}
\begin{itemize}
\item \hyperref[@I7@]{Julius Galm} [28.02.1883--23.05.1929 (7 Kinder)]
\item \hyperref[@I163@]{Anna Galm} [04.05.1884--10.08.1955 (10 Kinder)]
\item \hyperref[@I164@]{Karl Galm} [06.10.1886--13.10.1963]
\item \hyperref[@I165@]{Karoline Galm} [06.06.1888--13.07.1951]
\item \hyperref[@I2031@]{Joseph Galm} [23.12.1890--16.01.1891]
\item \hyperref[@I166@]{Philipp Galm} [28.12.1891--25.02.1953 (3 Kinder)]
\item \hyperref[@I167@]{Anton Galm} [02.12.1893--08.05.1915]
\item \hyperref[@I2032@]{Wilhelm Galm} [20.08.1895--24.09.1895]
\item \hyperref[@I168@]{Ida Galm} [31.07.1898--19.10.1988 (3 Kinder)]
\end{itemize}
\bigbreak
\textsc{{kinder}}
\begin{itemize}
\item Leo Büchler [1926]
\item Lucia Büchler [1928]
\item Erich Büchler [1932]
\end{itemize}
\medbreak
\textsc{{quellen}}
\begin{enumerate}[label={[\arabic*]}]
\item Langenelz Standesbuch 1891–1899, Geburtenregister 1899, Nr. 7
\item Schloßau Sterbebuch 1934–1965, Sterberegister 1962, Nr. 13
\item Heimatbuch Philipp Galm, Seite 44
\item \href{https://www.familysearch.org/tree/person/details/L1TQ-HWR}{FamilySearch, ID: L1TQ-HWR}
\end{enumerate}

\end{person}


\addsec{Johann Georg Schüßler  \& Helena Gramlich }


\begin{person}[
    surname = {Schüßler},
    givenname = {Johann Georg},
    suffix = {1858--1937},
    label = {@I150@},
    filename = {Johann Schuessler (1858)}
    ]

\begin{tabular}{cl}
\geboren & 26. August 1858 in Mörschenhardt\\
\taufe & 26. August 1858 in Mudau\\
\geheiratet & 11. Februar 1892 in Mudau mit Helena Gramlich \\
\gestorben & 06. Dezember 1937 in Mörschenhardt\\
\bestattet &  in Donebach\\
\end{tabular}\\
\medbreak
\textsc{vater}: \hyperref[@I152@]{Franz Josef Schüßler} [02.01.1831--13.06.1902 (4 Kinder)]\\
\textsc{mutter}: \hyperref[@I153@]{Margareta Eberschwein} [12.11.1831--25.02.1912 (4 Kinder)]
\medbreak
\textsc{{geschwister}}
\begin{itemize}
\item \hyperref[@I1345@]{Andreas Schüßler} [29.03.1860 (1 Kind)]
\item \hyperref[@I1346@]{Karolina Schüßler} [19.11.1861--29.12.1861]
\item \hyperref[@I1738@]{Wilhelmina Schüßler} [11.11.1868--30.04.1940 (6 Kinder)]
\end{itemize}
\bigbreak
\textsc{{kinder}}
\begin{itemize}
\item \hyperref[@I170@]{Franz Schüssler} [24.11.1892--28.06.1915]
\item \hyperref[@I8@]{Margaretha Schüßler} [11.04.1894--26.07.1978 (7 Kinder)]
\item \hyperref[@I171@]{Konstantin Schüßler} [08.10.1895--27.07.1979 (5 Kinder)]
\item \hyperref[@I176@]{Helena Schüßler} [28.02.1897--07.04.1987 (6 Kinder)]
\item \hyperref[@I172@]{Johann Georg Schüßler} [25.08.1898--29.05.1984 (7 Kinder)]
\item \hyperref[@I174@]{Wilhelm Schüßler} [29.08.1900--16.08.1918]
\item \hyperref[@I1776@]{Maria Schüßler} [02.03.1902--02.03.1902]
\item \hyperref[@I177@]{Emma Wilhelmina Schüßler} [26.03.1903--25.08.1990 (4 Kinder)]
\item \hyperref[@I175@]{Anton Schüßler} [13.12.1904--04.06.1988 (3 Kinder)]
\item \hyperref[@I179@]{Juliana Regina Schüßler} [13.02.1907--25.05.1955 (3 Kinder)]
\end{itemize}
\medbreak
\textsc{anmerkung}\\
Im Winter war seine Hauptbeschäftigung das Leinenweben. In seinen früheren Jahren hatte er sogar farbige Tischdecken und dergleichen gewebt. Er war der letzte Leinenweber des Ortes.
\medbreak
\textsc{{quellen}}
\begin{enumerate}[label={[\arabic*]}]
\item \href{http://www.landesarchiv-bw.de/plink/?f=4-1119442-108}{GLA Karlsruhe, Mörschenhardt, katholische Gemeinde: Standesbuch 1839–1870, Geburtenregister 1858, Nr. 4 (Bild 108)}
\item Heimatbuch Philipp Galm, Seite 29
\item \href{https://www.familysearch.org/tree/person/details/L5GK-WZR}{FamilySearch, ID: L5GK-WZR}
\end{enumerate}

\end{person}

\begin{person}[
    surname = {Gramlich},
    givenname = {Helena},
    suffix = {1864--1943},
    label = {@I151@},
    filename = {Helena Gramlich (1864)}
    ]

\begin{tabular}{cl}
\geboren & 28. Januar 1864 in Mörschenhardt\\
\geheiratet & 11. Februar 1892 in Mudau mit Johann Georg Schüßler \\
\gestorben & 21. November 1943 in Mörschenhardt\\
\bestattet &  in Donebach\\
\end{tabular}\\
\medbreak
\textsc{vater}: \hyperref[@I154@]{Johann Michael Gramlich} [26.09.1823--20.06.1891 (9 Kinder)]\\
\textsc{mutter}: \hyperref[@I155@]{Martha Schwing} [10.06.1835--06.02.1902 (9 Kinder)]
\medbreak
\textsc{{geschwister}}
\begin{itemize}
\item \hyperref[@I736@]{Maria Martha Gramlich} [08.09.1860]
\item \hyperref[@I737@]{Wilhelm Gramlich} [03.11.1861]
\item \hyperref[@I1885@]{Ferdinand Gramlich} [22.09.1866]
\item \hyperref[@I738@]{Maria Klara Gramlich} [24.02.1869]
\item \hyperref[@I1886@]{Michael Gramlich} [31.08.1871--1951]
\item \hyperref[@I739@]{Gottfried Gramlich} [19.02.1874--25.02.1875]
\item \hyperref[@I740@]{Isidor Gramlich} [11.04.1876--27.01.1958]
\item \hyperref[@I1887@]{Emma Gramlich} [05.11.1877--1948]
\end{itemize}
\bigbreak
\textsc{{kinder}}
\begin{itemize}
\item \hyperref[@I170@]{Franz Schüssler} [24.11.1892--28.06.1915]
\item \hyperref[@I8@]{Margaretha Schüßler} [11.04.1894--26.07.1978 (7 Kinder)]
\item \hyperref[@I171@]{Konstantin Schüßler} [08.10.1895--27.07.1979 (5 Kinder)]
\item \hyperref[@I176@]{Helena Schüßler} [28.02.1897--07.04.1987 (6 Kinder)]
\item \hyperref[@I172@]{Johann Georg Schüßler} [25.08.1898--29.05.1984 (7 Kinder)]
\item \hyperref[@I174@]{Wilhelm Schüßler} [29.08.1900--16.08.1918]
\item \hyperref[@I1776@]{Maria Schüßler} [02.03.1902--02.03.1902]
\item \hyperref[@I177@]{Emma Wilhelmina Schüßler} [26.03.1903--25.08.1990 (4 Kinder)]
\item \hyperref[@I175@]{Anton Schüßler} [13.12.1904--04.06.1988 (3 Kinder)]
\item \hyperref[@I179@]{Juliana Regina Schüßler} [13.02.1907--25.05.1955 (3 Kinder)]
\end{itemize}
\medbreak
\textsc{anmerkung}\\
Mit großer Vorliebe Sponn sie im Winter
\medbreak
\textsc{{quellen}}
\begin{enumerate}[label={[\arabic*]}]
\item \href{http://www.landesarchiv-bw.de/plink/?f=4-1119442-142}{GLA Karlsruhe, Mörschenhardt, katholische Gemeinde: Standesbuch 1839–1870, Geburtenregister 1864, Nr. 3 (Bild 142)}
\item Heimatbuch Philipp Galm, Seite 31
\item \href{https://www.familysearch.org/tree/person/details/L5GK-QLR}{FamilySearch, ID: L5GK-QLR}
\end{enumerate}

\end{person}

\begin{person}[
    surname = {Schüssler},
    givenname = {Franz},
    suffix = {1892--1915},
    label = {@I170@},
    filename = {Franz Josef Schuessler (1892)}
    ]

\begin{tabular}{cl}
\geboren & 24. November 1892 in Mörschenhardt\\
\gestorben & 28. Juni 1915\\
\end{tabular}\\
\medbreak
\textsc{vater}: \hyperref[@I150@]{Johann Georg Schüßler} [26.08.1858--06.12.1937 (10 Kinder)]\\
\textsc{mutter}: \hyperref[@I151@]{Helena Gramlich} [28.01.1864--21.11.1943 (10 Kinder)]
\medbreak
\textsc{{geschwister}}
\begin{itemize}
\item \hyperref[@I8@]{Margaretha Schüßler} [11.04.1894--26.07.1978 (7 Kinder)]
\item \hyperref[@I171@]{Konstantin Schüßler} [08.10.1895--27.07.1979 (5 Kinder)]
\item \hyperref[@I176@]{Helena Schüßler} [28.02.1897--07.04.1987 (6 Kinder)]
\item \hyperref[@I172@]{Johann Georg Schüßler} [25.08.1898--29.05.1984 (7 Kinder)]
\item \hyperref[@I174@]{Wilhelm Schüßler} [29.08.1900--16.08.1918]
\item \hyperref[@I1776@]{Maria Schüßler} [02.03.1902--02.03.1902]
\item \hyperref[@I177@]{Emma Wilhelmina Schüßler} [26.03.1903--25.08.1990 (4 Kinder)]
\item \hyperref[@I175@]{Anton Schüßler} [13.12.1904--04.06.1988 (3 Kinder)]
\item \hyperref[@I179@]{Juliana Regina Schüßler} [13.02.1907--25.05.1955 (3 Kinder)]
\end{itemize}
\bigbreak
\textsc{{quellen}}
\begin{enumerate}[label={[\arabic*]}]
\item Mörschenhardt Geburts-, Heirats- und Sterberegister 1890–1899, Geburtenregister 1892, Nr. 6
\item Heimatbuch Philipp Galm, Seite 46
\item \href{https://www.familysearch.org/tree/person/details/L139-7GS}{FamilySearch, ID: L139-7GS}
\end{enumerate}

\end{person}

\begin{person}[
    surname = {Schüßler},
    givenname = {Konstantin},
    suffix = {1895--1979},
    label = {@I171@},
    filename = {Konstantin Schuessler (1895)}
    ]

\begin{tabular}{cl}
\geboren & 08. Oktober 1895 in Mörschenhardt\\
\geheiratet & 29. September 1926 in Mörschenhardt mit Luise Pfeiffenberger \\
\gestorben & 27. Juli 1979 in Buchen\\
\bestattet &  in Donebach\\
\end{tabular}\\
\medbreak
\textsc{vater}: \hyperref[@I150@]{Johann Georg Schüßler} [26.08.1858--06.12.1937 (10 Kinder)]\\
\textsc{mutter}: \hyperref[@I151@]{Helena Gramlich} [28.01.1864--21.11.1943 (10 Kinder)]
\medbreak
\textsc{{geschwister}}
\begin{itemize}
\item \hyperref[@I170@]{Franz Schüssler} [24.11.1892--28.06.1915]
\item \hyperref[@I8@]{Margaretha Schüßler} [11.04.1894--26.07.1978 (7 Kinder)]
\item \hyperref[@I176@]{Helena Schüßler} [28.02.1897--07.04.1987 (6 Kinder)]
\item \hyperref[@I172@]{Johann Georg Schüßler} [25.08.1898--29.05.1984 (7 Kinder)]
\item \hyperref[@I174@]{Wilhelm Schüßler} [29.08.1900--16.08.1918]
\item \hyperref[@I1776@]{Maria Schüßler} [02.03.1902--02.03.1902]
\item \hyperref[@I177@]{Emma Wilhelmina Schüßler} [26.03.1903--25.08.1990 (4 Kinder)]
\item \hyperref[@I175@]{Anton Schüßler} [13.12.1904--04.06.1988 (3 Kinder)]
\item \hyperref[@I179@]{Juliana Regina Schüßler} [13.02.1907--25.05.1955 (3 Kinder)]
\end{itemize}
\bigbreak
\textsc{{kinder}}
\begin{itemize}
\item Gertrud Schüßler [15.06.1928--23.01.2002 (1 Kind)]
\item Willi Schüßler [09.09.1930--09.05.1991]
\item Paul Schüßler [17.04.1935--13.11.2000]
\item Hedwig Schüßler [...]
\item Mechthild Schüßler [...]
\end{itemize}
\medbreak
\textsc{anmerkung}\\
kauft Hof von Müller in Mörschenhardt
\medbreak
\textsc{{quellen}}
\begin{enumerate}[label={[\arabic*]}]
\item Mörschenhardt Geburts-, Heirats- und Sterberegister 1890–1899, Geburtenregister 1895, Nr. 4
\item Mörschenhardt Heiratsbuch 1905–1935, Heiratsregister 1926, Nr. 2
\item Sterberegister Buchen 1979, Nr. 170
\item Heimatbuch Philipp Galm, Seite 48
\item \href{https://www.familysearch.org/tree/person/details/LYCJ-Y7P}{FamilySearch, ID: LYCJ-Y7P}
\item \href{http://grabsteine.genealogy.net/tomb.php?cem=3816&tomb=100&b=&lang=de}{genealogy.net Grabstein Projekt, Friedhof Donebach}
\end{enumerate}

\end{person}

\begin{person}[
    surname = {Schüßler},
    givenname = {Helena},
    suffix = {1897--1987},
    label = {@I176@},
    filename = {Helena Schuessler (1897)}
    ]

\begin{tabular}{cl}
\geboren & 28. Februar 1897 in Mörschenhardt\\
\taufe &  in Mudau\\
\geheiratet & 22. Februar 1922 in Unterscheidental mit Robert Brenneis \\
\gestorben & 07. April 1987 in Scheidental\\
\bestattet &  in Scheidental\\
\end{tabular}\\
\medbreak
\textsc{vater}: \hyperref[@I150@]{Johann Georg Schüßler} [26.08.1858--06.12.1937 (10 Kinder)]\\
\textsc{mutter}: \hyperref[@I151@]{Helena Gramlich} [28.01.1864--21.11.1943 (10 Kinder)]
\medbreak
\textsc{{geschwister}}
\begin{itemize}
\item \hyperref[@I170@]{Franz Schüssler} [24.11.1892--28.06.1915]
\item \hyperref[@I8@]{Margaretha Schüßler} [11.04.1894--26.07.1978 (7 Kinder)]
\item \hyperref[@I171@]{Konstantin Schüßler} [08.10.1895--27.07.1979 (5 Kinder)]
\item \hyperref[@I172@]{Johann Georg Schüßler} [25.08.1898--29.05.1984 (7 Kinder)]
\item \hyperref[@I174@]{Wilhelm Schüßler} [29.08.1900--16.08.1918]
\item \hyperref[@I1776@]{Maria Schüßler} [02.03.1902--02.03.1902]
\item \hyperref[@I177@]{Emma Wilhelmina Schüßler} [26.03.1903--25.08.1990 (4 Kinder)]
\item \hyperref[@I175@]{Anton Schüßler} [13.12.1904--04.06.1988 (3 Kinder)]
\item \hyperref[@I179@]{Juliana Regina Schüßler} [13.02.1907--25.05.1955 (3 Kinder)]
\end{itemize}
\bigbreak
\textsc{{kinder}}
\begin{itemize}
\item Karl Brenneis [07.02.1923--12.03.2008]
\item Helmut Brenneis [24.02.1927--17.09.2013]
\item Gertrud Anna Brenneis [15.05.1931 (3 Kinder)]
\item Alois Brenneis [1933--1990]
\item Erich Brenneis [27.11.1937--25.01.1995]
\item Bernhard Brenneis [...]
\end{itemize}
\medbreak
\textsc{{quellen}}
\begin{enumerate}[label={[\arabic*]}]
\item Mörschenhardt Geburts-, Heirats- und Sterberegister 1890–1899, Geburtenregister 1897, Nr. 1
\item Unterscheidental Heiratsregister 1870–1935, Heiratsregister 1922, Nr. 1
\item Mudau Sterbebuch 1981–1990, Sterberegister 1987, Nr. 2
\item Heimatbuch Philipp Galm, Seite 55
\item \href{https://www.familysearch.org/tree/person/details/L5GK-H1R}{FamilySearch, ID: L5GK-H1R}
\item \href{http://grabsteine.genealogy.net/tomb.php?cem=3815&tomb=98&b=&lang=de}{genealogy.net Grabstein Projekt, Friedhof Scheidental}
\end{enumerate}

\end{person}

\begin{person}[
    surname = {Schüßler},
    givenname = {Johann Georg},
    suffix = {1898--1984},
    label = {@I172@},
    filename = {Johann Georg Schüßler (1898)}
    ]

\begin{tabular}{cl}
\geboren & 25. August 1898 in Mörschenhardt\\
\geheiratet & 10. Februar 1925 in Mörschenhardt mit Maria Anna Grimm \\
\gestorben & 29. Mai 1984 in Schweinfurt\\
\end{tabular}\\
\medbreak
\textsc{vater}: \hyperref[@I150@]{Johann Georg Schüßler} [26.08.1858--06.12.1937 (10 Kinder)]\\
\textsc{mutter}: \hyperref[@I151@]{Helena Gramlich} [28.01.1864--21.11.1943 (10 Kinder)]
\medbreak
\textsc{{geschwister}}
\begin{itemize}
\item \hyperref[@I170@]{Franz Schüssler} [24.11.1892--28.06.1915]
\item \hyperref[@I8@]{Margaretha Schüßler} [11.04.1894--26.07.1978 (7 Kinder)]
\item \hyperref[@I171@]{Konstantin Schüßler} [08.10.1895--27.07.1979 (5 Kinder)]
\item \hyperref[@I176@]{Helena Schüßler} [28.02.1897--07.04.1987 (6 Kinder)]
\item \hyperref[@I174@]{Wilhelm Schüßler} [29.08.1900--16.08.1918]
\item \hyperref[@I1776@]{Maria Schüßler} [02.03.1902--02.03.1902]
\item \hyperref[@I177@]{Emma Wilhelmina Schüßler} [26.03.1903--25.08.1990 (4 Kinder)]
\item \hyperref[@I175@]{Anton Schüßler} [13.12.1904--04.06.1988 (3 Kinder)]
\item \hyperref[@I179@]{Juliana Regina Schüßler} [13.02.1907--25.05.1955 (3 Kinder)]
\end{itemize}
\bigbreak
\textsc{{kinder}}
\begin{itemize}
\item Luzia Schüßler [27.05.1926--12.2018]
\item Maria Schüßler [11.08.1928 (1 Kind)]
\item Magdalena Schüßler [19.08.1930]
\item Franziska Schüßler [29.03.1932--23.03.2019]
\item Sophie Schüßler [14.05.1934]
\item Ottilie Schüßler [01.03.1936 (3 Kinder)]
\item Lore Schüßler [12.02.1941 (5 Kinder)]
\end{itemize}
\medbreak
\textsc{anmerkung}\\
Trauzeuge Konstantin Schüßler (29 Jahre) und Karl Grimm (24 Jahre)
\medbreak
\textsc{{quellen}}
\begin{enumerate}[label={[\arabic*]}]
\item Mörschenhardt Geburts-, Heirats- und Sterberegister 1890–1899, Geburtenregister 1898, Nr. 3
\item Mörschenhardt Heiratsbuch 1905–1935, Heiratsregister 1925, Nr. 2
\item Sterberegister Schwinfurt 1984, Nr. 446
\item Heimatbuch Philipp Galm, Seite 50
\item \href{https://www.familysearch.org/tree/person/details/L139-C9N}{FamilySearch, ID: L139-C9N}
\end{enumerate}

\end{person}

\begin{person}[
    surname = {Schüßler},
    givenname = {Wilhelm},
    suffix = {1900--1918},
    label = {@I174@},
    filename = {Wilhelm Schüßler (1900)}
    ]

\begin{tabular}{cl}
\geboren & 29. August 1900 in Mörschenhardt\\
\gestorben & 16. August 1918 in Offenburg\\
\bestattet &  in Mudau\\
\end{tabular}\\
\medbreak
\textsc{vater}: \hyperref[@I150@]{Johann Georg Schüßler} [26.08.1858--06.12.1937 (10 Kinder)]\\
\textsc{mutter}: \hyperref[@I151@]{Helena Gramlich} [28.01.1864--21.11.1943 (10 Kinder)]
\medbreak
\textsc{{geschwister}}
\begin{itemize}
\item \hyperref[@I170@]{Franz Schüssler} [24.11.1892--28.06.1915]
\item \hyperref[@I8@]{Margaretha Schüßler} [11.04.1894--26.07.1978 (7 Kinder)]
\item \hyperref[@I171@]{Konstantin Schüßler} [08.10.1895--27.07.1979 (5 Kinder)]
\item \hyperref[@I176@]{Helena Schüßler} [28.02.1897--07.04.1987 (6 Kinder)]
\item \hyperref[@I172@]{Johann Georg Schüßler} [25.08.1898--29.05.1984 (7 Kinder)]
\item \hyperref[@I1776@]{Maria Schüßler} [02.03.1902--02.03.1902]
\item \hyperref[@I177@]{Emma Wilhelmina Schüßler} [26.03.1903--25.08.1990 (4 Kinder)]
\item \hyperref[@I175@]{Anton Schüßler} [13.12.1904--04.06.1988 (3 Kinder)]
\item \hyperref[@I179@]{Juliana Regina Schüßler} [13.02.1907--25.05.1955 (3 Kinder)]
\end{itemize}
\bigbreak
\textsc{anmerkung}\\
Kaum 18 Jahre alt, kam er zum Militär. Er war 8 Wochen zur Ausbildung in Offenburg. Er erkrankte dabei und starb wenige Tage danach an Herzlähmung. Seine Eltern ließen die Leiche in die Heimat überführen. Im Mudauer Friedhof liegt er beerdigt.
\medbreak
\textsc{{quellen}}
\begin{enumerate}[label={[\arabic*]}]
\item Mörschenhardt Geburts-, Heirats- und Sterberegister 1900–1904, Geburtenregister 1900, Nr. 6
\item Heimatbuch Philipp Galm, Seite 52
\item \href{https://www.familysearch.org/tree/person/details/GMWC-QWQ}{FamilySearch, ID: GMWC-QWQ}
\end{enumerate}

\end{person}

\begin{person}[
    surname = {Schüßler},
    givenname = {Maria},
    suffix = {1902--1902},
    label = {@I1776@}
    ]

\begin{tabular}{cl}
\geboren & 02. März 1902 in Mörschenhardt\\
\gestorben & 02. März 1902 in Mörschenhardt\\
\end{tabular}\\
\medbreak
\textsc{vater}: \hyperref[@I150@]{Johann Georg Schüßler} [26.08.1858--06.12.1937 (10 Kinder)]\\
\textsc{mutter}: \hyperref[@I151@]{Helena Gramlich} [28.01.1864--21.11.1943 (10 Kinder)]
\medbreak
\textsc{{geschwister}}
\begin{itemize}
\item \hyperref[@I170@]{Franz Schüssler} [24.11.1892--28.06.1915]
\item \hyperref[@I8@]{Margaretha Schüßler} [11.04.1894--26.07.1978 (7 Kinder)]
\item \hyperref[@I171@]{Konstantin Schüßler} [08.10.1895--27.07.1979 (5 Kinder)]
\item \hyperref[@I176@]{Helena Schüßler} [28.02.1897--07.04.1987 (6 Kinder)]
\item \hyperref[@I172@]{Johann Georg Schüßler} [25.08.1898--29.05.1984 (7 Kinder)]
\item \hyperref[@I174@]{Wilhelm Schüßler} [29.08.1900--16.08.1918]
\item \hyperref[@I177@]{Emma Wilhelmina Schüßler} [26.03.1903--25.08.1990 (4 Kinder)]
\item \hyperref[@I175@]{Anton Schüßler} [13.12.1904--04.06.1988 (3 Kinder)]
\item \hyperref[@I179@]{Juliana Regina Schüßler} [13.02.1907--25.05.1955 (3 Kinder)]
\end{itemize}
\bigbreak
\textsc{{quellen}}
\begin{enumerate}[label={[\arabic*]}]
\item Mörschenhardt Geburts-, Heirats- und Sterberegister 1900–1904, Geburtenregister 1902, Nr. 1
\item Mörschenhardt Geburts-, Heirats- und Sterberegister 1900–1904, Sterberegister 1902, Nr. 2
\item \href{https://www.familysearch.org/tree/person/details/G9L2-CFH}{FamilySearch, ID: G9L2-CFH}
\end{enumerate}

\end{person}

\begin{person}[
    surname = {Schüßler},
    givenname = {Emma Wilhelmina},
    suffix = {1903--1990},
    label = {@I177@},
    filename = {Emma Schmitt (1903)}
    ]

\begin{tabular}{cl}
\geboren & 26. März 1903 in Mörschenhardt\\
\geheiratet & um 1926 mit Gregor Schmitt \\
\gestorben & 25. August 1990 in Limbach\\
\bestattet &  in Balsbach\\
\end{tabular}\\
\medbreak
\textsc{vater}: \hyperref[@I150@]{Johann Georg Schüßler} [26.08.1858--06.12.1937 (10 Kinder)]\\
\textsc{mutter}: \hyperref[@I151@]{Helena Gramlich} [28.01.1864--21.11.1943 (10 Kinder)]
\medbreak
\textsc{{geschwister}}
\begin{itemize}
\item \hyperref[@I170@]{Franz Schüssler} [24.11.1892--28.06.1915]
\item \hyperref[@I8@]{Margaretha Schüßler} [11.04.1894--26.07.1978 (7 Kinder)]
\item \hyperref[@I171@]{Konstantin Schüßler} [08.10.1895--27.07.1979 (5 Kinder)]
\item \hyperref[@I176@]{Helena Schüßler} [28.02.1897--07.04.1987 (6 Kinder)]
\item \hyperref[@I172@]{Johann Georg Schüßler} [25.08.1898--29.05.1984 (7 Kinder)]
\item \hyperref[@I174@]{Wilhelm Schüßler} [29.08.1900--16.08.1918]
\item \hyperref[@I1776@]{Maria Schüßler} [02.03.1902--02.03.1902]
\item \hyperref[@I175@]{Anton Schüßler} [13.12.1904--04.06.1988 (3 Kinder)]
\item \hyperref[@I179@]{Juliana Regina Schüßler} [13.02.1907--25.05.1955 (3 Kinder)]
\end{itemize}
\bigbreak
\textsc{{kinder}}
\begin{itemize}
\item Rosa Schmitt [...]
\item Artur Schmitt [...]
\item Gerhard Schmitt [...]
\item Erich Schmitt [...]
\end{itemize}
\medbreak
\textsc{{quellen}}
\begin{enumerate}[label={[\arabic*]}]
\item Mörschenhardt Geburts-, Heirats- und Sterberegister 1900–1904, Geburtenregister 1903, Nr. 3
\item Sterberegister Limbach 1990, Nr. 18
\item Heimatbuch Philipp Galm, Seite 56
\item \href{https://www.familysearch.org/tree/person/details/GMWC-4J9}{FamilySearch, ID: GMWC-4J9}
\end{enumerate}

\end{person}

\begin{person}[
    surname = {Schüßler},
    givenname = {Anton},
    suffix = {1904--1988},
    label = {@I175@},
    filename = {Anton Schuessler (1904)}
    ]

\begin{tabular}{cl}
\geboren & 13. Dezember 1904 in Mörschenhardt\\
\geheiratet & 30. September 1933 in Pforzheim mit Hildegard Galm \\
\gestorben & 04. Juni 1988 in Buchen\\
\end{tabular}\\
\medbreak
\textsc{vater}: \hyperref[@I150@]{Johann Georg Schüßler} [26.08.1858--06.12.1937 (10 Kinder)]\\
\textsc{mutter}: \hyperref[@I151@]{Helena Gramlich} [28.01.1864--21.11.1943 (10 Kinder)]
\medbreak
\textsc{{geschwister}}
\begin{itemize}
\item \hyperref[@I170@]{Franz Schüssler} [24.11.1892--28.06.1915]
\item \hyperref[@I8@]{Margaretha Schüßler} [11.04.1894--26.07.1978 (7 Kinder)]
\item \hyperref[@I171@]{Konstantin Schüßler} [08.10.1895--27.07.1979 (5 Kinder)]
\item \hyperref[@I176@]{Helena Schüßler} [28.02.1897--07.04.1987 (6 Kinder)]
\item \hyperref[@I172@]{Johann Georg Schüßler} [25.08.1898--29.05.1984 (7 Kinder)]
\item \hyperref[@I174@]{Wilhelm Schüßler} [29.08.1900--16.08.1918]
\item \hyperref[@I1776@]{Maria Schüßler} [02.03.1902--02.03.1902]
\item \hyperref[@I177@]{Emma Wilhelmina Schüßler} [26.03.1903--25.08.1990 (4 Kinder)]
\item \hyperref[@I179@]{Juliana Regina Schüßler} [13.02.1907--25.05.1955 (3 Kinder)]
\end{itemize}
\bigbreak
\textsc{{kinder}}
\begin{itemize}
\item Bruno Schüßler [23.07.1939--02.08.1952]
\item Renate Schüßler [...]
\item Helga Schüßler [...]
\end{itemize}
\medbreak
\textsc{{quellen}}
\begin{enumerate}[label={[\arabic*]}]
\item Mörschenhardt Geburts-, Heirats- und Sterberegister 1900–1904, Geburtenregister 1904, Nr. 7
\item Sterberegister Buchen 1988, Nr. 101
\item Heimatbuch Philipp Galm, Seite 54
\item \href{https://www.familysearch.org/tree/person/details/L5TM-B2L}{FamilySearch, ID: L5TM-B2L}
\end{enumerate}

\end{person}

\begin{person}[
    surname = {Schüßler},
    givenname = {Juliana Regina},
    suffix = {1907--1955},
    label = {@I179@},
    filename = {Juliana Spaeth (1907)}
    ]

\begin{tabular}{cl}
\geboren & 13. Februar 1907 in Mörschenhardt\\
\geheiratet & 23. April 1937 in Mudau mit Johann Valentin Späth \\
\gestorben & 25. Mai 1955 in Buchen\\
\end{tabular}\\
\medbreak
\textsc{vater}: \hyperref[@I150@]{Johann Georg Schüßler} [26.08.1858--06.12.1937 (10 Kinder)]\\
\textsc{mutter}: \hyperref[@I151@]{Helena Gramlich} [28.01.1864--21.11.1943 (10 Kinder)]
\medbreak
\textsc{{geschwister}}
\begin{itemize}
\item \hyperref[@I170@]{Franz Schüssler} [24.11.1892--28.06.1915]
\item \hyperref[@I8@]{Margaretha Schüßler} [11.04.1894--26.07.1978 (7 Kinder)]
\item \hyperref[@I171@]{Konstantin Schüßler} [08.10.1895--27.07.1979 (5 Kinder)]
\item \hyperref[@I176@]{Helena Schüßler} [28.02.1897--07.04.1987 (6 Kinder)]
\item \hyperref[@I172@]{Johann Georg Schüßler} [25.08.1898--29.05.1984 (7 Kinder)]
\item \hyperref[@I174@]{Wilhelm Schüßler} [29.08.1900--16.08.1918]
\item \hyperref[@I1776@]{Maria Schüßler} [02.03.1902--02.03.1902]
\item \hyperref[@I177@]{Emma Wilhelmina Schüßler} [26.03.1903--25.08.1990 (4 Kinder)]
\item \hyperref[@I175@]{Anton Schüßler} [13.12.1904--04.06.1988 (3 Kinder)]
\end{itemize}
\bigbreak
\textsc{{kinder}}
\begin{itemize}
\item Helena Späth [01.06.1938 (1 Kind)]
\item Willi Späth [30.06.1939--11.05.1984 (1 Kind)]
\item Siegfried Späth [05.11.1941--24.11.2000 (3 Kinder)]
\end{itemize}
\medbreak
\textsc{anmerkung}\\
Seit 1929 in Langenelz als Hilfe für Margaretha
\medbreak
\textsc{{quellen}}
\begin{enumerate}[label={[\arabic*]}]
\item Mörschenhardt Geburtenbuch 1905–1935, Geburtenregister 1907, Nr. 1
\item Mudau Heiratsbuch 1928–1937, Heiratsregister 1937, Nr. 3
\item Sterberegister Buchen 1955, Nr. 45
\item Heimatbuch Philipp Galm, Seite 58
\item \href{https://www.familysearch.org/tree/person/details/LYZC-TV4}{FamilySearch, ID: LYZC-TV4}
\end{enumerate}

\end{person}


\addsec{Johann Josef Schölch  \& Karoline Mechler }


\begin{person}[
    surname = {Schölch},
    givenname = {Johann Josef},
    suffix = {1865--1939},
    label = {@I156@},
    filename = {Johann Schölch (1865)}
    ]

\begin{tabular}{cl}
\geboren & 29. August 1865 in Laudenberg\\
\taufe & 29. August 1865 in Limbach\\
\geheiratet & 07. Februar 1893 in Langenelz mit Karoline Mechler \\
\gestorben & 04. November 1939 in Langenelz\\
\end{tabular}\\
\medbreak
\textsc{vater}: \hyperref[@I158@]{Johann Philipp Schölch} [30.04.1820--21.05.1898 (10 Kinder)]\\
\textsc{mutter}: \hyperref[@I210@]{Anna Maria Ott} [15.01.1829--09.05.1896 (10 Kinder)]
\medbreak
\textsc{{geschwister}}
\begin{itemize}
\item \hyperref[@I225@]{Ludwig Schölch} [30.06.1849 (12 Kinder)]
\item \hyperref[@I228@]{Margaretha Schölch} [05.10.1851]
\item \hyperref[@I229@]{Lina Schölch} [27.12.1853--28.01.1858]
\item \hyperref[@I230@]{Katharina Schölch} [10.06.1856]
\item \hyperref[@I231@]{Maria Anna Schölch} [06.11.1859]
\item \hyperref[@I232@]{Theresia Schölch} [28.04.1862--16.10.1863]
\item \hyperref[@I233@]{Helena Schölch} [29.09.1864--06.10.1864]
\item \hyperref[@I234@]{Rosalia Schölch} [15.12.1866]
\item \hyperref[@I235@]{Anna Schölch} [08.10.1869]
\end{itemize}
\bigbreak
\textsc{{kinder}}
\begin{itemize}
\item \hyperref[@I429@]{Anna Schölch} [30.10.1893--22.03.1946 (4 Kinder)]
\item \hyperref[@I9@]{Alois Adolf Schölch} [12.03.1895--07.08.1963 (10 Kinder)]
\item \hyperref[@I366@]{Maria Schölch} [30.07.1896--30.08.1969 (2 Kinder)]
\item \hyperref[@I430@]{Karolina Schölch} [07.02.1900--23.09.1971]
\end{itemize}
\medbreak
\textsc{{quellen}}
\begin{enumerate}[label={[\arabic*]}]
\item Langenelz Standesbuch 1891–1899, Heiratsregister 1893, Nr. 1
\item Mudau Sterbebuch 1938–1944, Sterberegister 1939, Nr. 18
\item \href{https://www.familysearch.org/tree/person/details/LV6Z-KSJ}{FamilySearch, ID: LV6Z-KSJ}
\item \href{http://gedbas.genealogy.net/person/show/1172960807}{genealogy.net}
\end{enumerate}

\end{person}

\begin{person}[
    surname = {Mechler},
    givenname = {Karoline},
    suffix = {1870--1933},
    label = {@I157@},
    filename = {Karolina Mechler (1870)}
    ]

\begin{tabular}{cl}
\geboren & 31. März 1870 in Langenelz\\
\geheiratet & 07. Februar 1893 in Langenelz mit Johann Josef Schölch \\
\gestorben & 04. Dezember 1933 in Langenelz\\
\end{tabular}\\
\medbreak
\textsc{vater}: \hyperref[@I159@]{Joseph Mechler} [27.06.1843--16.05.1924 (6 Kinder)]\\
\textsc{mutter}: \hyperref[@I160@]{Maria Josepha Seubert} [27.07.1846--20.07.1906 (6 Kinder)]
\medbreak
\textsc{{geschwister}}
\begin{itemize}
\item \hyperref[@I1430@]{Anna Mechler} [09.08.1872]
\item \hyperref[@I1431@]{Josef Mechler} [29.04.1875]
\item \hyperref[@I1703@]{Wilhelm Mechler} [08.01.1879]
\item \hyperref[@I2085@]{Maria Mechler} [26.03.1882--27.04.1961]
\item \hyperref[@I1704@]{Franz Karl Mechler} [02.09.1884--27.04.1885]
\end{itemize}
\bigbreak
\textsc{{kinder}}
\begin{itemize}
\item \hyperref[@I429@]{Anna Schölch} [30.10.1893--22.03.1946 (4 Kinder)]
\item \hyperref[@I9@]{Alois Adolf Schölch} [12.03.1895--07.08.1963 (10 Kinder)]
\item \hyperref[@I366@]{Maria Schölch} [30.07.1896--30.08.1969 (2 Kinder)]
\item \hyperref[@I430@]{Karolina Schölch} [07.02.1900--23.09.1971]
\end{itemize}
\medbreak
\textsc{{quellen}}
\begin{enumerate}[label={[\arabic*]}]
\item Langenelz Standesbuch 1870–1875, Geburtenregister 1870, Nr. 2
\item Langenelz Standesbuch 1870–1875, Geburtenregister 1871, Nr. 10 (Anerkung von Joseph Mechler)
\item Langenelz Standesbuch 1891–1899, Heiratsregister 1893, Nr. 1
\item Ahnentafel Galm
\item Ahnentafel Helga Schölch
\item \href{https://www.familysearch.org/tree/person/details/L5GK-XPT}{FamilySearch, ID: L5GK-XPT}
\end{enumerate}

\end{person}

\begin{person}[
    surname = {Schölch},
    givenname = {Anna},
    suffix = {1893--1946},
    label = {@I429@},
    filename = {Anna Schoelch (1893)}
    ]

\begin{tabular}{cl}
\geboren & 30. Oktober 1893 in Langenelz\\
\geheiratet & 17. Januar 1921 in Mudau mit Josef Weimer \\
\gestorben & 22. März 1946 in Mudau\\
\end{tabular}\\
\medbreak
\textsc{vater}: \hyperref[@I156@]{Johann Josef Schölch} [29.08.1865--04.11.1939 (4 Kinder)]\\
\textsc{mutter}: \hyperref[@I157@]{Karoline Mechler} [31.03.1870--04.12.1933 (4 Kinder)]
\medbreak
\textsc{{geschwister}}
\begin{itemize}
\item \hyperref[@I9@]{Alois Adolf Schölch} [12.03.1895--07.08.1963 (10 Kinder)]
\item \hyperref[@I366@]{Maria Schölch} [30.07.1896--30.08.1969 (2 Kinder)]
\item \hyperref[@I430@]{Karolina Schölch} [07.02.1900--23.09.1971]
\end{itemize}
\bigbreak
\textsc{{kinder}}
\begin{itemize}
\item Karl Weimer [29.03.1922--05.12.1942]
\item Artur Weimer [26.02.1925--17.07.2000]
\item Anna Weimer [...]
\item Irmgard Weimer [...]
\end{itemize}
\medbreak
\textsc{anmerkung}\\
erste Ehe Hauk (?)
\medbreak
\textsc{{quellen}}
\begin{enumerate}[label={[\arabic*]}]
\item Mudau Standesbuch 1891–1895, Geburtenregister 1893, Nr. 11
\item Mudau Heiratsbuch 1920–1927, Heiratsregister 1921, Nr. 1
\item Heimatbuch Alise Galm
\item \href{https://www.familysearch.org/tree/person/details/LRCN-TQN}{FamilySearch, ID: LRCN-TQN}
\end{enumerate}

\end{person}

\begin{person}[
    surname = {Schölch},
    givenname = {Maria},
    suffix = {1896--1969},
    label = {@I366@},
    filename = {Maria Schoelch (1896)}
    ]

\begin{tabular}{cl}
\geboren & 30. Juli 1896 in Langenelz\\
\geheiratet & 29. Januar 1921 in Mudau mit Karl Müssig \\
\gestorben & 30. August 1969 in Mudau\\
\end{tabular}\\
\medbreak
\textsc{vater}: \hyperref[@I156@]{Johann Josef Schölch} [29.08.1865--04.11.1939 (4 Kinder)]\\
\textsc{mutter}: \hyperref[@I157@]{Karoline Mechler} [31.03.1870--04.12.1933 (4 Kinder)]
\medbreak
\textsc{{geschwister}}
\begin{itemize}
\item \hyperref[@I429@]{Anna Schölch} [30.10.1893--22.03.1946 (4 Kinder)]
\item \hyperref[@I9@]{Alois Adolf Schölch} [12.03.1895--07.08.1963 (10 Kinder)]
\item \hyperref[@I430@]{Karolina Schölch} [07.02.1900--23.09.1971]
\end{itemize}
\bigbreak
\textsc{{kinder}}
\begin{itemize}
\item Josef Müssig [13.03.1922--15.06.1967]
\item Kurt Müssig [...]
\end{itemize}
\medbreak
\textsc{{quellen}}
\begin{enumerate}[label={[\arabic*]}]
\item Mudau Standesbuch 1896–1899, Geburtenregister 1896, Nr. 6
\item Mudau Heiratsbuch 1920–1927, Heiratsregister 1921, Nr. 3
\item Mudau Sterbebuch 1968–1975, Sterberegister 1969, Nr. 14
\item Heimatbuch Alise Galm
\item \href{https://www.familysearch.org/tree/person/details/LRCN-P1S}{FamilySearch, ID: LRCN-P1S}
\end{enumerate}

\end{person}

\begin{person}[
    surname = {Schölch},
    givenname = {Karolina},
    suffix = {1900--1971},
    label = {@I430@},
    filename = {Lina Schoelch (1900)}
    ]

\begin{tabular}{cl}
\geboren & 07. Februar 1900 in Langenelz\\
\gestorben & 23. September 1971 in Karlsruhe\\
\end{tabular}\\
\medbreak
\textsc{vater}: \hyperref[@I156@]{Johann Josef Schölch} [29.08.1865--04.11.1939 (4 Kinder)]\\
\textsc{mutter}: \hyperref[@I157@]{Karoline Mechler} [31.03.1870--04.12.1933 (4 Kinder)]
\medbreak
\textsc{{geschwister}}
\begin{itemize}
\item \hyperref[@I429@]{Anna Schölch} [30.10.1893--22.03.1946 (4 Kinder)]
\item \hyperref[@I9@]{Alois Adolf Schölch} [12.03.1895--07.08.1963 (10 Kinder)]
\item \hyperref[@I366@]{Maria Schölch} [30.07.1896--30.08.1969 (2 Kinder)]
\end{itemize}
\bigbreak
\textsc{{quellen}}
\begin{enumerate}[label={[\arabic*]}]
\item Mudau Standesbuch 1900–1905, Geburtenregister 1900, Nr. 1
\item Sterberegister Weisbach 1971, Nr. 333 
\item \href{https://www.familysearch.org/tree/person/details/LRCN-P1Q}{FamilySearch, ID: LRCN-P1Q}
\end{enumerate}

\end{person}


\addsec{Josef Eduard Schäfer  \& Josefa Karolina Zeller }


\begin{person}[
    surname = {Schäfer},
    givenname = {Josef Eduard},
    suffix = {1861--1923},
    label = {@I161@}
    ]

\begin{tabular}{cl}
\geboren & 13. Oktober 1861 in Beuchen\\
\geheiratet & 10. September 1891 in Beuchen mit Josefa Karolina Zeller \\
\gestorben & 26. September 1923 in Mudau\\
\end{tabular}\\
\medbreak
\textsc{vater}: \hyperref[@I378@]{Franz Matthäus Schäfer} [07.11.1812--20.09.1879 (8 Kinder)]\\
\textsc{mutter}: \hyperref[@I379@]{Anna Maria Trunk} [05.06.1825--27.09.1874 (4 Kinder)]
\medbreak
\textsc{{geschwister}}
\begin{itemize}
\item \hyperref[@I2140@]{Eva Rosina Schäfer} [07.03.1843]
\item \hyperref[@I2141@]{Johann Michael Schäfer} [25.06.1846]
\item \hyperref[@I2142@]{Franz Matthäus Schäfer} [19.08.1849--16.10.1855]
\item \hyperref[@I2143@]{Franz Anton Schäfer} [11.08.1854--25.08.1854]
\item \hyperref[@I564@]{Franz Karl Anton Schäfer} [12.06.1856--01.01.1863]
\item \hyperref[@I565@]{Johann Martin Schäfer} [20.04.1859--17.01.1863]
\item \hyperref[@I566@]{Heinrich August Schäfer} [12.05.1864]
\end{itemize}
\bigbreak
\textsc{{kinder}}
\begin{itemize}
\item \hyperref[@I431@]{Josef Georg Schäfer} [22.06.1889--21.02.1961 (7 Kinder)]
\item \hyperref[@I432@]{August Schäfer} [16.06.1892]
\item \hyperref[@I433@]{Ludwig Eduard Schäfer} [25.02.1894--18.01.1963]
\item \hyperref[@I434@]{Karl Ernst Schäfer} [14.12.1895--09.06.1918]
\item \hyperref[@I10@]{Adelheid Anna Schäfer} [17.08.1897--28.07.1959 (10 Kinder)]
\item \hyperref[@I436@]{Maria Karolina Schäfer} [03.05.1899--18.04.1911]
\item \hyperref[@I435@]{Wilhelm Edmund Schäfer} [08.11.1900]
\end{itemize}
\medbreak
\textsc{{quellen}}
\begin{enumerate}[label={[\arabic*]}]
\item Mudau Sterbebuch 1920–1928, Sterberegister 1923, Nr. 15
\item Ahnentafel Helga Schölch
\item \href{https://www.familysearch.org/tree/person/details/L5GV-19Z?1=1&spouse=L5GV-5RZ}{FamilySearch, ID: L5GV-19Z}
\end{enumerate}

\end{person}

\begin{person}[
    surname = {Zeller},
    givenname = {Josefa Karolina},
    suffix = {1866--1931},
    label = {@I162@}
    ]

\begin{tabular}{cl}
\geboren & 25. Januar 1866 in Beuchen\\
\geheiratet & 10. September 1891 in Beuchen mit Josef Eduard Schäfer \\
\gestorben & 28. November 1931 in Mudau\\
\end{tabular}\\
\medbreak
\textsc{vater}: \hyperref[@I380@]{Johann Valentin Zeller} [26.06.1815--16.04.1879 (5 Kinder)]\\
\textsc{mutter}: \hyperref[@I381@]{Maria Josefa Mehl} [27.06.1832--20.01.1890 (5 Kinder)]
\medbreak
\textsc{{geschwister}}
\begin{itemize}
\item \hyperref[@I597@]{Maria Regina Zeller} [05.09.1859--22.12.1862]
\item \hyperref[@I598@]{Theresia Helena Zeller} [04.06.1862--01.07.1862]
\item \hyperref[@I599@]{Valentin Augustin Zeller} [1863]
\item \hyperref[@I600@]{Anna Adelheid Zeller} [11.11.1868--12.09.1898]
\end{itemize}
\bigbreak
\textsc{{kinder}}
\begin{itemize}
\item \hyperref[@I431@]{Josef Georg Schäfer} [22.06.1889--21.02.1961 (7 Kinder)]
\item \hyperref[@I432@]{August Schäfer} [16.06.1892]
\item \hyperref[@I433@]{Ludwig Eduard Schäfer} [25.02.1894--18.01.1963]
\item \hyperref[@I434@]{Karl Ernst Schäfer} [14.12.1895--09.06.1918]
\item \hyperref[@I10@]{Adelheid Anna Schäfer} [17.08.1897--28.07.1959 (10 Kinder)]
\item \hyperref[@I436@]{Maria Karolina Schäfer} [03.05.1899--18.04.1911]
\item \hyperref[@I435@]{Wilhelm Edmund Schäfer} [08.11.1900]
\end{itemize}
\medbreak
\textsc{anmerkung}\\
Conflict: vlt Karoline
\medbreak
\textsc{{quellen}}
\begin{enumerate}[label={[\arabic*]}]
\item Ahnentafel Galm
\item Ahnentafel Helga Schölch
\item \href{https://www.familysearch.org/tree/person/details/L5GV-5RZ}{FamilySearch, ID: L5GV-5RZ}
\end{enumerate}

\end{person}

\begin{person}[
    surname = {Schäfer},
    givenname = {Josef Georg},
    suffix = {1889--1961},
    label = {@I431@},
    filename = {Josef Georg Schaefer (1899)}
    ]

\begin{tabular}{cl}
\geboren & 22. Juni 1889 in Beuchen\\
\geheiratet & 18. November 1921 in Mudau mit Auguste Scheuermann \\
\gestorben & 21. Februar 1961 in Mudau\\
\end{tabular}\\
\medbreak
\textsc{vater}: \hyperref[@I161@]{Josef Eduard Schäfer} [13.10.1861--26.09.1923 (7 Kinder)]\\
\textsc{mutter}: \hyperref[@I162@]{Josefa Karolina Zeller} [25.01.1866--28.11.1931 (7 Kinder)]
\medbreak
\textsc{{geschwister}}
\begin{itemize}
\item \hyperref[@I432@]{August Schäfer} [16.06.1892]
\item \hyperref[@I433@]{Ludwig Eduard Schäfer} [25.02.1894--18.01.1963]
\item \hyperref[@I434@]{Karl Ernst Schäfer} [14.12.1895--09.06.1918]
\item \hyperref[@I10@]{Adelheid Anna Schäfer} [17.08.1897--28.07.1959 (10 Kinder)]
\item \hyperref[@I436@]{Maria Karolina Schäfer} [03.05.1899--18.04.1911]
\item \hyperref[@I435@]{Wilhelm Edmund Schäfer} [08.11.1900]
\end{itemize}
\bigbreak
\textsc{{kinder}}
\begin{itemize}
\item Eduard Schäfer [12.10.1923--01.09.1998]
\item Gerhard Schäfer [28.01.1927--09.11.1990]
\item Alois Schäfer [1934--1994]
\item Juliana Schäfer [...]
\item Richard Schäfer [...]
\item Maria Schäfer [...]
\item Paul Schäfer [...]
\end{itemize}
\medbreak
\textsc{{quellen}}
\begin{enumerate}[label={[\arabic*]}]
\item Mudau Standesbuch 1888–1890, Geburtenregister 1889, Nr. 5
\item Mudau Heiratsbuch 1920–1927, Heiratsregister 1921, Nr. 17
\item Mudau Sterbebuch 1956–1967, Sterberegister 1961, Nr. 4
\item \href{https://www.familysearch.org/tree/person/details/LTQD-81L}{FamilySearch, ID: LTQD-81L}
\end{enumerate}

\end{person}

\begin{person}[
    surname = {Schäfer},
    givenname = {August},
    suffix = {1892},
    label = {@I432@},
    filename = {August Schaefer (1892)}
    ]

\begin{tabular}{cl}
\geboren & 16. Juni 1892 in Mudau\\
\geheiratet & November 1921 mit Maria ... \\
\end{tabular}\\
\medbreak
\textsc{vater}: \hyperref[@I161@]{Josef Eduard Schäfer} [13.10.1861--26.09.1923 (7 Kinder)]\\
\textsc{mutter}: \hyperref[@I162@]{Josefa Karolina Zeller} [25.01.1866--28.11.1931 (7 Kinder)]
\medbreak
\textsc{{geschwister}}
\begin{itemize}
\item \hyperref[@I431@]{Josef Georg Schäfer} [22.06.1889--21.02.1961 (7 Kinder)]
\item \hyperref[@I433@]{Ludwig Eduard Schäfer} [25.02.1894--18.01.1963]
\item \hyperref[@I434@]{Karl Ernst Schäfer} [14.12.1895--09.06.1918]
\item \hyperref[@I10@]{Adelheid Anna Schäfer} [17.08.1897--28.07.1959 (10 Kinder)]
\item \hyperref[@I436@]{Maria Karolina Schäfer} [03.05.1899--18.04.1911]
\item \hyperref[@I435@]{Wilhelm Edmund Schäfer} [08.11.1900]
\end{itemize}
\bigbreak
\textsc{anmerkung}\\
wohnhaft in Mannheim
\medbreak
\textsc{{quellen}}
\begin{enumerate}[label={[\arabic*]}]
\item Mudau Standesbuch 1891–1895, Geburtenregister 1892, Nr. 19
\item \href{https://www.familysearch.org/tree/person/details/LTQD-89S}{FamilySearch, ID: LTQD-89S}
\end{enumerate}

\end{person}

\begin{person}[
    surname = {Schäfer},
    givenname = {Ludwig Eduard},
    suffix = {1894--1963},
    label = {@I433@}
    ]

\begin{tabular}{cl}
\geboren & 25. Februar 1894 in Mudau\\
\taufe & 25. Februar 1894 in Mudau\\
\gestorben & 18. Januar 1963 in Mudau\\
\end{tabular}\\
\medbreak
\textsc{vater}: \hyperref[@I161@]{Josef Eduard Schäfer} [13.10.1861--26.09.1923 (7 Kinder)]\\
\textsc{mutter}: \hyperref[@I162@]{Josefa Karolina Zeller} [25.01.1866--28.11.1931 (7 Kinder)]
\medbreak
\textsc{{geschwister}}
\begin{itemize}
\item \hyperref[@I431@]{Josef Georg Schäfer} [22.06.1889--21.02.1961 (7 Kinder)]
\item \hyperref[@I432@]{August Schäfer} [16.06.1892]
\item \hyperref[@I434@]{Karl Ernst Schäfer} [14.12.1895--09.06.1918]
\item \hyperref[@I10@]{Adelheid Anna Schäfer} [17.08.1897--28.07.1959 (10 Kinder)]
\item \hyperref[@I436@]{Maria Karolina Schäfer} [03.05.1899--18.04.1911]
\item \hyperref[@I435@]{Wilhelm Edmund Schäfer} [08.11.1900]
\end{itemize}
\bigbreak
\textsc{{quellen}}
\begin{enumerate}[label={[\arabic*]}]
\item Mudau Standesbuch 1891–1895, Geburtenregister 1894, Nr. 10
\item Heimatbuch Alise Galm
\item \href{https://www.familysearch.org/tree/person/details/L11W-3PL}{FamilySearch, ID: L11W-3PL}
\end{enumerate}

\end{person}

\begin{person}[
    surname = {Schäfer},
    givenname = {Karl Ernst},
    suffix = {1895--1918},
    label = {@I434@},
    filename = {Karl Ernst Schäfer (1895)}
    ]

\begin{tabular}{cl}
\geboren & 14. Dezember 1895 in Mudau\\
\gestorben & 09. Juni 1918 in Frankreich\\
\end{tabular}\\
\medbreak
\textsc{vater}: \hyperref[@I161@]{Josef Eduard Schäfer} [13.10.1861--26.09.1923 (7 Kinder)]\\
\textsc{mutter}: \hyperref[@I162@]{Josefa Karolina Zeller} [25.01.1866--28.11.1931 (7 Kinder)]
\medbreak
\textsc{{geschwister}}
\begin{itemize}
\item \hyperref[@I431@]{Josef Georg Schäfer} [22.06.1889--21.02.1961 (7 Kinder)]
\item \hyperref[@I432@]{August Schäfer} [16.06.1892]
\item \hyperref[@I433@]{Ludwig Eduard Schäfer} [25.02.1894--18.01.1963]
\item \hyperref[@I10@]{Adelheid Anna Schäfer} [17.08.1897--28.07.1959 (10 Kinder)]
\item \hyperref[@I436@]{Maria Karolina Schäfer} [03.05.1899--18.04.1911]
\item \hyperref[@I435@]{Wilhelm Edmund Schäfer} [08.11.1900]
\end{itemize}
\bigbreak
\textsc{{quellen}}
\begin{enumerate}[label={[\arabic*]}]
\item Mudau Standesbuch 1891–1895, Geburtenregister 1895, Nr. 38
\item \href{https://www.familysearch.org/tree/person/details/L11W-SGH}{FamilySearch, ID: L11W-SGH}
\end{enumerate}

\end{person}

\begin{person}[
    surname = {Schäfer},
    givenname = {Maria Karolina},
    suffix = {1899--1911},
    label = {@I436@}
    ]

\begin{tabular}{cl}
\geboren & 03. Mai 1899 in Mudau\\
\gestorben & 18. April 1911\\
\end{tabular}\\
\medbreak
\textsc{vater}: \hyperref[@I161@]{Josef Eduard Schäfer} [13.10.1861--26.09.1923 (7 Kinder)]\\
\textsc{mutter}: \hyperref[@I162@]{Josefa Karolina Zeller} [25.01.1866--28.11.1931 (7 Kinder)]
\medbreak
\textsc{{geschwister}}
\begin{itemize}
\item \hyperref[@I431@]{Josef Georg Schäfer} [22.06.1889--21.02.1961 (7 Kinder)]
\item \hyperref[@I432@]{August Schäfer} [16.06.1892]
\item \hyperref[@I433@]{Ludwig Eduard Schäfer} [25.02.1894--18.01.1963]
\item \hyperref[@I434@]{Karl Ernst Schäfer} [14.12.1895--09.06.1918]
\item \hyperref[@I10@]{Adelheid Anna Schäfer} [17.08.1897--28.07.1959 (10 Kinder)]
\item \hyperref[@I435@]{Wilhelm Edmund Schäfer} [08.11.1900]
\end{itemize}
\bigbreak
\textsc{{quellen}}
\begin{enumerate}[label={[\arabic*]}]
\item Mudau Standesbuch 1896–1899, Geburtenregister 1899, Nr. 19
\item \href{https://www.familysearch.org/tree/person/details/L11W-4ZP}{FamilySearch, ID: L11W-4ZP}
\end{enumerate}

\end{person}

\begin{person}[
    surname = {Schäfer},
    givenname = {Wilhelm Edmund},
    suffix = {1900},
    label = {@I435@},
    filename = {Wilhelm Edmund Schäfer (1900)}
    ]

\begin{tabular}{cl}
\geboren & 08. November 1900 in Mudau\\
\end{tabular}\\
\medbreak
\textsc{vater}: \hyperref[@I161@]{Josef Eduard Schäfer} [13.10.1861--26.09.1923 (7 Kinder)]\\
\textsc{mutter}: \hyperref[@I162@]{Josefa Karolina Zeller} [25.01.1866--28.11.1931 (7 Kinder)]
\medbreak
\textsc{{geschwister}}
\begin{itemize}
\item \hyperref[@I431@]{Josef Georg Schäfer} [22.06.1889--21.02.1961 (7 Kinder)]
\item \hyperref[@I432@]{August Schäfer} [16.06.1892]
\item \hyperref[@I433@]{Ludwig Eduard Schäfer} [25.02.1894--18.01.1963]
\item \hyperref[@I434@]{Karl Ernst Schäfer} [14.12.1895--09.06.1918]
\item \hyperref[@I10@]{Adelheid Anna Schäfer} [17.08.1897--28.07.1959 (10 Kinder)]
\item \hyperref[@I436@]{Maria Karolina Schäfer} [03.05.1899--18.04.1911]
\end{itemize}
\bigbreak
\textsc{anmerkung}\\
wohnhaft in Mannheim
geh. 1933.11.11
\medbreak
\textsc{{quellen}}
\begin{enumerate}[label={[\arabic*]}]
\item Mudau Standesbuch 1900–1905, Geburtenregister 1900, Nr. 38
\item Heimatbuch Alise Galm
\item \href{https://www.familysearch.org/tree/person/details/L11W-WTC}{FamilySearch, ID: L11W-WTC}
\end{enumerate}

\end{person}


\addchap{Ur-Ur-Ur-Gro"seltern}



\addsec{Franz Anton Scheuermann  \& Eva Barbara Zeller }


\begin{person}[
    surname = {Scheuermann},
    givenname = {Franz Anton},
    suffix = {1799--1869},
    label = {@I950@}
    ]

\begin{tabular}{cl}
\geboren & 29. Mai 1799 in Ottorfszell\\
\geheiratet & 31. Oktober 1824 in Ottorfszell mit Josefa Henn \\
 & 03. Juli 1839 in Ottorfszell mit Eva Barbara Zeller \\
\gestorben & 13. Dezember 1869 in Hambrunn\\
\end{tabular}\\
\medbreak
\textsc{vater}: Benedikt Scheuermann [18.03.1757--22.03.1832 (4 Kinder)]\\
\textsc{mutter}: \hyperref[@I1176@]{Maria Katharina Galm} [24.11.1764--05.03.1837 (4 Kinder)]
\medbreak
\textsc{{geschwister}}
\begin{itemize}
\item Maria Anna Scheuermann [1796]
\item Eva Barbara Scheuermann [09.03.1802]
\item Anna Eva Scheuermann [20.02.1803]
\end{itemize}
\bigbreak
\textsc{{kinder}}
\begin{itemize}
\item \hyperref[@I1287@]{Anna Maria Scheuermann} [04.10.1826]
\item \hyperref[@I1288@]{Maria Josepha Scheuermann} [10.03.1829]
\item \hyperref[@I1289@]{Eva Barbara Scheuermann} [02.12.1831]
\item \hyperref[@I1290@]{Johann Michael Scheuermann} [20.12.1833--05.06.1888]
\item \hyperref[@I1291@]{Franz Anton Scheuermann} [17.06.1836--04.01.1904]
\item \hyperref[@I389@]{Johann Valentin Scheuermann} [07.12.1842--12.12.1910 (6 Kinder)]
\item \hyperref[@I1292@]{Johann Josef Scheuermann} [1849]
\end{itemize}
\medbreak
\textsc{anmerkung}\\
nach Eheurkunde gestorben vor 1875
2. Ehe nach anderer Quelle am 02. Juli
In erster Ehe verheiratet mit Josepha Henn aus Watterbach
\medbreak
\textsc{{quellen}}
\begin{enumerate}[label={[\arabic*]}]
\item \href{https://www.familysearch.org/tree/person/details/L1TZ-MQ5}{FamilySearch, ID: L1TZ-MQ5}
\item \href{https://gw.geneanet.org/wzipp?n=scheuermann&oc=&p=franz+anton+1}{Geneanet.org}
\end{enumerate}

\end{person}

\begin{person}[
    surname = {Zeller},
    givenname = {Eva Barbara},
    suffix = {1811--1886},
    label = {@I951@}
    ]

\begin{tabular}{cl}
\geboren & 03. Dezember 1811 in Kirchzell\\
\geheiratet & 03. Juli 1839 in Ottorfszell mit Franz Anton Scheuermann \\
\gestorben & 30. Januar 1886 in Ottorfszell\\
\end{tabular}\\
\medbreak
\textsc{{kinder}}
\begin{itemize}
\item \hyperref[@I389@]{Johann Valentin Scheuermann} [07.12.1842--12.12.1910 (6 Kinder)]
\item \hyperref[@I1292@]{Johann Josef Scheuermann} [1849]
\end{itemize}
\medbreak
\textsc{anmerkung}\\
zweiter Vorname unleserlich. Geburtsname auf Eheurkunde des Johann Valentin könnte auch Zöller sein
\medbreak
\textsc{{quellen}}
\begin{enumerate}[label={[\arabic*]}]
\item Eheurkunde Valentin Scheuermann
\item \href{https://www.familysearch.org/tree/person/details/L1TZ-9LW}{FamilySearch, ID: L1TZ-9LW}
\item \href{https://gw.geneanet.org/wzipp?lang=de&p=eva+barbara&n=zeller}{Geneanet.org}
\end{enumerate}

\end{person}

\begin{person}[
    surname = {Scheuermann},
    givenname = {Anna Maria},
    suffix = {1826},
    label = {@I1287@}
    ]

\begin{tabular}{cl}
\geboren & 04. Oktober 1826 in Ottorfszell\\
\geheiratet & 12. Februar 1857 in Breitenbuch mit Josef Ernst Farrenkopf \\
\end{tabular}\\
\medbreak
\textsc{vater}: \hyperref[@I950@]{Franz Anton Scheuermann} [29.05.1799--13.12.1869 (7 Kinder)]\\
\textsc{mutter}: \hyperref[@I1286@]{Josefa Henn} [15.01.1806--12.06.1838 (5 Kinder)]
\medbreak
\textsc{{geschwister}}
\begin{itemize}
\item \hyperref[@I1288@]{Maria Josepha Scheuermann} [10.03.1829]
\item \hyperref[@I1289@]{Eva Barbara Scheuermann} [02.12.1831]
\item \hyperref[@I1290@]{Johann Michael Scheuermann} [20.12.1833--05.06.1888]
\item \hyperref[@I1291@]{Franz Anton Scheuermann} [17.06.1836--04.01.1904]
\item \hyperref[@I389@]{Johann Valentin Scheuermann} [07.12.1842--12.12.1910 (6 Kinder)]
\item \hyperref[@I1292@]{Johann Josef Scheuermann} [1849]
\end{itemize}
\bigbreak
\end{person}

\begin{person}[
    surname = {Scheuermann},
    givenname = {Maria Josepha},
    suffix = {1829},
    label = {@I1288@}
    ]

\begin{tabular}{cl}
\geboren & 10. März 1829 in Ottorfszell\\
\end{tabular}\\
\medbreak
\textsc{vater}: \hyperref[@I950@]{Franz Anton Scheuermann} [29.05.1799--13.12.1869 (7 Kinder)]\\
\textsc{mutter}: \hyperref[@I1286@]{Josefa Henn} [15.01.1806--12.06.1838 (5 Kinder)]
\medbreak
\textsc{{geschwister}}
\begin{itemize}
\item \hyperref[@I1287@]{Anna Maria Scheuermann} [04.10.1826]
\item \hyperref[@I1289@]{Eva Barbara Scheuermann} [02.12.1831]
\item \hyperref[@I1290@]{Johann Michael Scheuermann} [20.12.1833--05.06.1888]
\item \hyperref[@I1291@]{Franz Anton Scheuermann} [17.06.1836--04.01.1904]
\item \hyperref[@I389@]{Johann Valentin Scheuermann} [07.12.1842--12.12.1910 (6 Kinder)]
\item \hyperref[@I1292@]{Johann Josef Scheuermann} [1849]
\end{itemize}
\bigbreak
\end{person}

\begin{person}[
    surname = {Scheuermann},
    givenname = {Eva Barbara},
    suffix = {1831},
    label = {@I1289@}
    ]

\begin{tabular}{cl}
\geboren & 02. Dezember 1831 in Ottorfszell\\
\end{tabular}\\
\medbreak
\textsc{vater}: \hyperref[@I950@]{Franz Anton Scheuermann} [29.05.1799--13.12.1869 (7 Kinder)]\\
\textsc{mutter}: \hyperref[@I1286@]{Josefa Henn} [15.01.1806--12.06.1838 (5 Kinder)]
\medbreak
\textsc{{geschwister}}
\begin{itemize}
\item \hyperref[@I1287@]{Anna Maria Scheuermann} [04.10.1826]
\item \hyperref[@I1288@]{Maria Josepha Scheuermann} [10.03.1829]
\item \hyperref[@I1290@]{Johann Michael Scheuermann} [20.12.1833--05.06.1888]
\item \hyperref[@I1291@]{Franz Anton Scheuermann} [17.06.1836--04.01.1904]
\item \hyperref[@I389@]{Johann Valentin Scheuermann} [07.12.1842--12.12.1910 (6 Kinder)]
\item \hyperref[@I1292@]{Johann Josef Scheuermann} [1849]
\end{itemize}
\bigbreak
\end{person}

\begin{person}[
    surname = {Scheuermann},
    givenname = {Johann Michael},
    suffix = {1833--1888},
    label = {@I1290@}
    ]

\begin{tabular}{cl}
\geboren & 20. Dezember 1833 in Ottorfszell\\
\geheiratet & 24. Februar 1859 in Ottorfszell mit Maria Josepha Farrenkopf \\
\gestorben & 05. Juni 1888 in Ottorfszell\\
\end{tabular}\\
\medbreak
\textsc{vater}: \hyperref[@I950@]{Franz Anton Scheuermann} [29.05.1799--13.12.1869 (7 Kinder)]\\
\textsc{mutter}: \hyperref[@I1286@]{Josefa Henn} [15.01.1806--12.06.1838 (5 Kinder)]
\medbreak
\textsc{{geschwister}}
\begin{itemize}
\item \hyperref[@I1287@]{Anna Maria Scheuermann} [04.10.1826]
\item \hyperref[@I1288@]{Maria Josepha Scheuermann} [10.03.1829]
\item \hyperref[@I1289@]{Eva Barbara Scheuermann} [02.12.1831]
\item \hyperref[@I1291@]{Franz Anton Scheuermann} [17.06.1836--04.01.1904]
\item \hyperref[@I389@]{Johann Valentin Scheuermann} [07.12.1842--12.12.1910 (6 Kinder)]
\item \hyperref[@I1292@]{Johann Josef Scheuermann} [1849]
\end{itemize}
\bigbreak
\end{person}

\begin{person}[
    surname = {Scheuermann},
    givenname = {Franz Anton},
    suffix = {1836--1904},
    label = {@I1291@}
    ]

\begin{tabular}{cl}
\geboren & 17. Juni 1836 in Ottorfszell\\
\gestorben & 04. Januar 1904 in Ottorfszell\\
\end{tabular}\\
\medbreak
\textsc{vater}: \hyperref[@I950@]{Franz Anton Scheuermann} [29.05.1799--13.12.1869 (7 Kinder)]\\
\textsc{mutter}: \hyperref[@I1286@]{Josefa Henn} [15.01.1806--12.06.1838 (5 Kinder)]
\medbreak
\textsc{{geschwister}}
\begin{itemize}
\item \hyperref[@I1287@]{Anna Maria Scheuermann} [04.10.1826]
\item \hyperref[@I1288@]{Maria Josepha Scheuermann} [10.03.1829]
\item \hyperref[@I1289@]{Eva Barbara Scheuermann} [02.12.1831]
\item \hyperref[@I1290@]{Johann Michael Scheuermann} [20.12.1833--05.06.1888]
\item \hyperref[@I389@]{Johann Valentin Scheuermann} [07.12.1842--12.12.1910 (6 Kinder)]
\item \hyperref[@I1292@]{Johann Josef Scheuermann} [1849]
\end{itemize}
\bigbreak
\end{person}

\begin{person}[
    surname = {Scheuermann},
    givenname = {Johann Josef},
    suffix = {1849},
    label = {@I1292@}
    ]

\begin{tabular}{cl}
\geboren & 1849 in Ottorfszell\\
\end{tabular}\\
\medbreak
\textsc{vater}: \hyperref[@I950@]{Franz Anton Scheuermann} [29.05.1799--13.12.1869 (7 Kinder)]\\
\textsc{mutter}: \hyperref[@I951@]{Eva Barbara Zeller} [03.12.1811--30.01.1886 (2 Kinder)]
\medbreak
\textsc{{geschwister}}
\begin{itemize}
\item \hyperref[@I1287@]{Anna Maria Scheuermann} [04.10.1826]
\item \hyperref[@I1288@]{Maria Josepha Scheuermann} [10.03.1829]
\item \hyperref[@I1289@]{Eva Barbara Scheuermann} [02.12.1831]
\item \hyperref[@I1290@]{Johann Michael Scheuermann} [20.12.1833--05.06.1888]
\item \hyperref[@I1291@]{Franz Anton Scheuermann} [17.06.1836--04.01.1904]
\item \hyperref[@I389@]{Johann Valentin Scheuermann} [07.12.1842--12.12.1910 (6 Kinder)]
\end{itemize}
\bigbreak
\end{person}


\addsec{Johann Josef Schäfer  \& Margaretha Baumann }


\begin{person}[
    surname = {Schäfer},
    givenname = {Johann Josef},
    suffix = {1822--1866},
    label = {@I952@}
    ]

\begin{tabular}{cl}
\geboren & 07. Oktober 1822 in Scheidental\\
\taufe & 07. Oktober 1822\\
\geheiratet & 15. Dezember 1846 in Mudau mit Margaretha Baumann \\
\gestorben & 04. Dezember 1866 in Mudau\\
\bestattet & 06. Dezember 1866 in Mudau\\
\end{tabular}\\
\medbreak
\textsc{vater}: Andreas Schäfer [um 1781--05.04.1853 (6 Kinder)]\\
\textsc{mutter}: \hyperref[@I1203@]{Theresia Bücheler} [um 1786--18.07.1823 (2 Kinder)]
\medbreak
\textsc{{geschwister}}
\begin{itemize}
\item Josepha Schäfer [1805--08.02.1839]
\item Franciscus Andreas Schäfer [29.04.1809]
\item Johann Valentin Schäfer [16.06.1810]
\item Franz Anton Schäfer [13.12.1811--11.09.1854]
\item Franz Barthel Schäfer [08.07.1820]
\end{itemize}
\bigbreak
\textsc{{kinder}}
\begin{itemize}
\item \hyperref[@I1204@]{Michael Schäfer} [29.09.1847--26.12.1924 (4 Kinder)]
\item \hyperref[@I1853@]{Franz Karl Schäfer} [1848]
\item \hyperref[@I1205@]{Josefa Schäfer} [03.07.1850]
\item \hyperref[@I390@]{Margareta Schäfer} [05.10.1852--11.11.1929 (6 Kinder)]
\item \hyperref[@I1206@]{Joseph John Schäfer} [19.03.1856--12.03.1914 (8 Kinder)]
\item \hyperref[@I1207@]{Johann Schäfer} [23.06.1858]
\end{itemize}
\medbreak
\textsc{{quellen}}
\begin{enumerate}[label={[\arabic*]}]
\item \href{http://www.landesarchiv-bw.de/plink/?f=4-1119480-26}{GLA Karlsruhe, Mudau, katholische Gemeinde: Standesbuch 1845–1855, Heiratsregister Mudau 1846, Nr. 14 (Bild 26)}
\item \href{http://www.landesarchiv-bw.de/plink/?f=4-1119481-223}{GLA Karlsruhe, Mudau, katholische Gemeinde: Standesbuch 1856–1866, Sterberegister Mudau 1866, Nr. 51 (Bild 223) (wahrscheinlich)}
\item Eheurkunde Valentin Scheuermann und Margareta Schäfer
\item Ahnentafel USA
\item Ahnentafel Erich Schnorr
\item \href{https://www.familysearch.org/tree/person/details/L1TZ-M4D}{FamilySearch, ID: L1TZ-M4D}
\end{enumerate}

\end{person}

\begin{person}[
    surname = {Baumann},
    givenname = {Margaretha},
    suffix = {1819--1878},
    label = {@I953@}
    ]

\begin{tabular}{cl}
\geboren & 28. Mai 1819 in Mudau\\
\geheiratet & 15. Dezember 1846 in Mudau mit Johann Josef Schäfer \\
\gestorben & 25. Juni 1878 in Mudau\\
\end{tabular}\\
\medbreak
\textsc{vater}: Johann Baumann [um 1775 (11 Kinder)]\\
\textsc{mutter}: \hyperref[@I1171@]{Eva Barbara Grünwald} [um 1793 (6 Kinder)]
\medbreak
\textsc{{geschwister}}
\begin{itemize}
\item Eva Barbara Baumann [24.01.1811]
\item Eva Baumann [22.02.1813--22.02.1813]
\item Johann Joseph Baumann [16.10.1814]
\item Eva Katharina Baumann [13.08.1816--19.08.1836]
\item Franz Joseph Baumann [15.09.1821]
\item Maria Josepha Baumann [10.06.1824]
\item Rosina Baumann [26.02.1827]
\item Maria Anna Baumann [25.03.1830]
\item Anton Baumann [03.01.1834]
\item Maria Theresia Baumann [...]
\end{itemize}
\bigbreak
\textsc{{kinder}}
\begin{itemize}
\item \hyperref[@I1204@]{Michael Schäfer} [29.09.1847--26.12.1924 (4 Kinder)]
\item \hyperref[@I1853@]{Franz Karl Schäfer} [1848]
\item \hyperref[@I1205@]{Josefa Schäfer} [03.07.1850]
\item \hyperref[@I390@]{Margareta Schäfer} [05.10.1852--11.11.1929 (6 Kinder)]
\item \hyperref[@I1206@]{Joseph John Schäfer} [19.03.1856--12.03.1914 (8 Kinder)]
\item \hyperref[@I1207@]{Johann Schäfer} [23.06.1858]
\end{itemize}
\medbreak
\textsc{{quellen}}
\begin{enumerate}[label={[\arabic*]}]
\item \href{http://www.landesarchiv-bw.de/plink/?f=4-1119480-26}{GLA Karlsruhe, Mudau, katholische Gemeinde: Standesbuch 1845–1855, Heiratsregister Mudau 1846, Nr. 14 (Bild 26)}
\item Mudau Geburts-, Heirats- und Sterberegister 1876–1879, Sterberegister 1878, Nr. 16
\item Ahnentafel USA
\item Eheurkunde Valentin Scheuermann und Margareta Schaefer
\item \href{https://www.familysearch.org/tree/person/details/L5PX-62H}{FamilySearch, ID: L5PX-62H}
\end{enumerate}

\end{person}

\begin{person}[
    surname = {Schäfer},
    givenname = {Michael},
    suffix = {1847--1924},
    label = {@I1204@}
    ]

\begin{tabular}{cl}
\geboren & 29. September 1847 in Mudau\\
\geheiratet &  mit Euphrosina Burgmeier \\
\gestorben & 26. Dezember 1924 in Dayton, Ohio, USA\\
\bestattet &  in Dayton, Ohio, USA\\
\end{tabular}\\
\medbreak
\textsc{vater}: \hyperref[@I952@]{Johann Josef Schäfer} [07.10.1822--04.12.1866 (6 Kinder)]\\
\textsc{mutter}: \hyperref[@I953@]{Margaretha Baumann} [28.05.1819--25.06.1878 (6 Kinder)]
\medbreak
\textsc{{geschwister}}
\begin{itemize}
\item \hyperref[@I1853@]{Franz Karl Schäfer} [1848]
\item \hyperref[@I1205@]{Josefa Schäfer} [03.07.1850]
\item \hyperref[@I390@]{Margareta Schäfer} [05.10.1852--11.11.1929 (6 Kinder)]
\item \hyperref[@I1206@]{Joseph John Schäfer} [19.03.1856--12.03.1914 (8 Kinder)]
\item \hyperref[@I1207@]{Johann Schäfer} [23.06.1858]
\end{itemize}
\bigbreak
\textsc{{kinder}}
\begin{itemize}
\item Herman Michael Schaefer [1882--1892]
\item Aloysius Michael Schaefer [28.10.1888--22.02.1931]
\item Alice Schäfer [um 1889]
\item Ollie M Schäfer [um 1889]
\end{itemize}
\medbreak
\textsc{anmerkung}\\
ausgewandert
Todesdatum unsicher
\medbreak
\textsc{{quellen}}
\begin{enumerate}[label={[\arabic*]}]
\item Ahnentafel USA
\item \href{https://www.familysearch.org/tree/person/details/GMQK-VF7}{FamilySearch, ID: GMQK-VF7}
\item https://www.geni.com/people/Michael-Schaefer/6000000000638906114
\end{enumerate}

\end{person}

\begin{person}[
    surname = {Schäfer},
    givenname = {Franz Karl},
    suffix = {1848},
    label = {@I1853@}
    ]

\begin{tabular}{cl}
\geboren & 1848 in Mudau\\
\taufe & 04. November 1848 in Mudau\\
\end{tabular}\\
\medbreak
\textsc{vater}: \hyperref[@I952@]{Johann Josef Schäfer} [07.10.1822--04.12.1866 (6 Kinder)]\\
\textsc{mutter}: \hyperref[@I953@]{Margaretha Baumann} [28.05.1819--25.06.1878 (6 Kinder)]
\medbreak
\textsc{{geschwister}}
\begin{itemize}
\item \hyperref[@I1204@]{Michael Schäfer} [29.09.1847--26.12.1924 (4 Kinder)]
\item \hyperref[@I1205@]{Josefa Schäfer} [03.07.1850]
\item \hyperref[@I390@]{Margareta Schäfer} [05.10.1852--11.11.1929 (6 Kinder)]
\item \hyperref[@I1206@]{Joseph John Schäfer} [19.03.1856--12.03.1914 (8 Kinder)]
\item \hyperref[@I1207@]{Johann Schäfer} [23.06.1858]
\end{itemize}
\bigbreak
\textsc{{quellen}}
\begin{enumerate}[label={[\arabic*]}]
\item \href{https://www.familysearch.org/tree/person/details/G9GY-3HS}{FamilySearch, ID: G9GY-3HS}
\end{enumerate}

\end{person}

\begin{person}[
    surname = {Schäfer},
    givenname = {Josefa},
    suffix = {1850},
    label = {@I1205@}
    ]

\begin{tabular}{cl}
\geboren & 03. Juli 1850 in Mudau\\
\end{tabular}\\
\medbreak
\textsc{vater}: \hyperref[@I952@]{Johann Josef Schäfer} [07.10.1822--04.12.1866 (6 Kinder)]\\
\textsc{mutter}: \hyperref[@I953@]{Margaretha Baumann} [28.05.1819--25.06.1878 (6 Kinder)]
\medbreak
\textsc{{geschwister}}
\begin{itemize}
\item \hyperref[@I1204@]{Michael Schäfer} [29.09.1847--26.12.1924 (4 Kinder)]
\item \hyperref[@I1853@]{Franz Karl Schäfer} [1848]
\item \hyperref[@I390@]{Margareta Schäfer} [05.10.1852--11.11.1929 (6 Kinder)]
\item \hyperref[@I1206@]{Joseph John Schäfer} [19.03.1856--12.03.1914 (8 Kinder)]
\item \hyperref[@I1207@]{Johann Schäfer} [23.06.1858]
\end{itemize}
\bigbreak
\textsc{anmerkung}\\
weiterer Vorname mit M.
\medbreak
\textsc{{quellen}}
\begin{enumerate}[label={[\arabic*]}]
\item Ahnentafel USA
\item \href{https://www.familysearch.org/tree/person/details/GMQK-NZN}{FamilySearch, ID: GMQK-NZN}
\end{enumerate}

\end{person}

\begin{person}[
    surname = {Schäfer},
    givenname = {Joseph John},
    suffix = {1856--1914},
    label = {@I1206@},
    filename = {Joseph John Schäfer (1856)}
    ]

\begin{tabular}{cl}
\geboren & 19. März 1856 in Mudau\\
\taufe & 20. März 1856\\
\geheiratet &  mit Mary Anne Bueker \\
\gestorben & 12. März 1914 in Dayton, Ohio, USA\\
\end{tabular}\\
\medbreak
\textsc{vater}: \hyperref[@I952@]{Johann Josef Schäfer} [07.10.1822--04.12.1866 (6 Kinder)]\\
\textsc{mutter}: \hyperref[@I953@]{Margaretha Baumann} [28.05.1819--25.06.1878 (6 Kinder)]
\medbreak
\textsc{{geschwister}}
\begin{itemize}
\item \hyperref[@I1204@]{Michael Schäfer} [29.09.1847--26.12.1924 (4 Kinder)]
\item \hyperref[@I1853@]{Franz Karl Schäfer} [1848]
\item \hyperref[@I1205@]{Josefa Schäfer} [03.07.1850]
\item \hyperref[@I390@]{Margareta Schäfer} [05.10.1852--11.11.1929 (6 Kinder)]
\item \hyperref[@I1207@]{Johann Schäfer} [23.06.1858]
\end{itemize}
\bigbreak
\textsc{{kinder}}
\begin{itemize}
\item Alma Schäfer [11.1881]
\item Ambrose Schäfer [1884--21.03.1884]
\item Margaret Schäfer [29.08.1885--27.05.1923]
\item Eugene Joseph Schäfer [03.03.1887--26.06.1965 (2 Kinder)]
\item Charles J. Schaefer [08.06.1889--12.06.1962 (3 Kinder)]
\item Agnes Mary Caroline Schäfer [1892--12.12.1895]
\item Mary Agnes Schäfer [1897--10.05.1897]
\item Joseph J. Schäfer [19.12.1898--11.04.1958]
\end{itemize}
\medbreak
\textsc{anmerkung}\\
1872 ausgewandert nach USA
Eingebürgert 1877.10.08
verh. mit Bueker Mary 1880.05.25 in Dayton OH. USA
Geburtsdatum vlt 19. März (20 nach Kirchenbuch)
\medbreak
\textsc{{quellen}}
\begin{enumerate}[label={[\arabic*]}]
\item \href{http://www.landesarchiv-bw.de/plink/?f=4-1119481-5}{GLA Karlsruhe, Mudau, katholische Gemeinde: Standesbuch 1856–1866, Geburtenregister 1856, Nr. 5 (Bild 5)}
\item Ahnentafel USA
\item \href{https://www.familysearch.org/tree/person/details/GMQK-F5N}{FamilySearch, ID: GMQK-F5N}
\item https://www.geni.com/people/Joseph-Schaefer/6000000000635578144
\end{enumerate}

\end{person}

\begin{person}[
    surname = {Schäfer},
    givenname = {Johann},
    suffix = {1858},
    label = {@I1207@}
    ]

\begin{tabular}{cl}
\geboren & 23. Juni 1858 in Mudau\\
\end{tabular}\\
\medbreak
\textsc{vater}: \hyperref[@I952@]{Johann Josef Schäfer} [07.10.1822--04.12.1866 (6 Kinder)]\\
\textsc{mutter}: \hyperref[@I953@]{Margaretha Baumann} [28.05.1819--25.06.1878 (6 Kinder)]
\medbreak
\textsc{{geschwister}}
\begin{itemize}
\item \hyperref[@I1204@]{Michael Schäfer} [29.09.1847--26.12.1924 (4 Kinder)]
\item \hyperref[@I1853@]{Franz Karl Schäfer} [1848]
\item \hyperref[@I1205@]{Josefa Schäfer} [03.07.1850]
\item \hyperref[@I390@]{Margareta Schäfer} [05.10.1852--11.11.1929 (6 Kinder)]
\item \hyperref[@I1206@]{Joseph John Schäfer} [19.03.1856--12.03.1914 (8 Kinder)]
\end{itemize}
\bigbreak
\textsc{{quellen}}
\begin{enumerate}[label={[\arabic*]}]
\item \href{http://www.landesarchiv-bw.de/plink/?f=4-1119481-39}{GLA Karlsruhe, Mudau, katholische Gemeinde: Standesbuch 1856–1866, Geburtenregister 1858, Nr. 15 (Bild 39)}
\item Ahnentafel USA
\end{enumerate}

\end{person}


\addsec{Johann Valentin Zimmermann  \& Josepha Zimmermann }


\begin{person}[
    surname = {Zimmermann},
    givenname = {Johann Valentin},
    suffix = {1830--1861},
    label = {@I396@}
    ]

\begin{tabular}{cl}
\geboren & 31. Juli 1830 in Laudenberg\\
\taufe & 31. Juli 1830 in Limbach\\
\geheiratet & 28. Juni 1851 in Laudenberg mit Josepha Zimmermann \\
\gestorben & 11. Dezember 1861 in Laudenberg\\
\bestattet & 13. Dezember 1861 in Limbach\\
\end{tabular}\\
\medbreak
\textsc{vater}: Johann Valentin Zimmermann [01.06.1793--04.01.1862 (6 Kinder)]\\
\textsc{mutter}: \hyperref[@I398@]{Maria Anna Nörpel} [um 1796 (6 Kinder)]
\medbreak
\textsc{{geschwister}}
\begin{itemize}
\item Franz Joseph Zimmermann [25.08.1817--28.08.1817]
\item Franz Valentin Zimmermann [07.10.1818--29.10.1818]
\item Franz Michael Zimmermann [24.05.1820]
\item Johann Valentin Zimmermann [13.02.1824--15.02.1824]
\item Franz Joseph Zimmermann [01.02.1826--01.10.1878]
\end{itemize}
\bigbreak
\textsc{{kinder}}
\begin{itemize}
\item \hyperref[@I1348@]{Josepha Zimmermann} [05.05.1852--08.05.1852]
\item \hyperref[@I1349@]{Katharina Zimmermann} [07.01.1854]
\item \hyperref[@I392@]{Valentin Zimmermann} [24.02.1856--16.03.1923 (10 Kinder)]
\item \hyperref[@I1350@]{Rosa Zimmermann} [02.04.1859--21.03.1862]
\end{itemize}
\medbreak
\textsc{anmerkung}\\
1854 erweitert Hof durch Kauf von Valentin Link
1856 erweitert Hof durch Kauf von Peter Volk
\medbreak
\textsc{{quellen}}
\begin{enumerate}[label={[\arabic*]}]
\item \href{http://www.landesarchiv-bw.de/plink/?f=4-1119439-107}{GLA Karlsruhe, Laudenberg, katholische Gemeinde: Standesbuch 1810–1870, Geburtenregister 1830, Nr. 9 (Bild 107)}
\item \href{http://www.landesarchiv-bw.de/plink/?f=4-1119439-236}{GLA Karlsruhe, Laudenberg, katholische Gemeinde: Standesbuch 1810–1870, Heiratsregister 1851, Nr. 4 (Bild 236)}
\item \href{http://www.landesarchiv-bw.de/plink/?f=4-1119439-296}{GLA Karlsruhe, Laudenberg, katholische Gemeinde: Standesbuch 1810–1870, Sterberegister 1861, Nr. 4 (Bild 236)}
\item 650 Jahre Laudenberg - Ein altes Dorf im Odenwald, Seite 164
\item \href{https://www.familysearch.org/tree/person/details/LV6Y-TC7}{FamilySearch, ID: LV6Y-TC7}
\item \href{http://gedbas.genealogy.net/person/show/1172964848}{genealogy.net}
\end{enumerate}

\end{person}

\begin{person}[
    surname = {Zimmermann},
    givenname = {Josepha},
    suffix = {1831--1898},
    label = {@I393@}
    ]

\begin{tabular}{cl}
\geboren & 19. März 1831 in Laudenberg\\
\taufe & 19. März 1831 in Limbach\\
\geheiratet & 28. Juni 1851 in Laudenberg mit Johann Valentin Zimmermann \\
 & 26. März 1863 in Limbach mit Joseph Anton Albert \\
\gestorben & 20. November 1898 in Laudenberg\\
\end{tabular}\\
\medbreak
\textsc{vater}: Franz Joseph Zimmermann [um 1797--03.01.1868 (1 Kind)]\\
\textsc{mutter}: \hyperref[@I407@]{Maria Katharina Werner} [um 1801 (1 Kind)]
\medbreak
\textsc{{kinder}}
\begin{itemize}
\item \hyperref[@I1348@]{Josepha Zimmermann} [05.05.1852--08.05.1852]
\item \hyperref[@I1349@]{Katharina Zimmermann} [07.01.1854]
\item \hyperref[@I392@]{Valentin Zimmermann} [24.02.1856--16.03.1923 (10 Kinder)]
\item \hyperref[@I1350@]{Rosa Zimmermann} [02.04.1859--21.03.1862]
\item \hyperref[@I1373@]{Wilhelm Albert} [02.05.1864--18.08.1919 (5 Kinder)]
\item \hyperref[@I1374@]{Johann Adam Albert} [22.06.1866--30.11.1937]
\item \hyperref[@I1375@]{Rosa Albert} [14.04.1869]
\end{itemize}
\medbreak
\textsc{anmerkung}\\
übergibt 1878 den Hof an ihren Sohn aus erster Ehe Valentin Zimmermann
\medbreak
\textsc{{quellen}}
\begin{enumerate}[label={[\arabic*]}]
\item \href{http://www.landesarchiv-bw.de/plink/?f=4-1119439-112}{GLA Karlsruhe, Laudenberg, katholische Gemeinde: Standesbuch 1810–1870, Geburtenregister 1831, Nr. 3 (Bild 112)}
\item \href{http://www.landesarchiv-bw.de/plink/?f=4-1119439-236}{GLA Karlsruhe, Laudenberg, katholische Gemeinde: Standesbuch 1810–1870, Heiratsregister 1851, Nr. 4 (Bild 236)}
\item \href{http://www.landesarchiv-bw.de/plink/?f=4-1119439-309}{GLA Karlsruhe, Laudenberg, katholische Gemeinde: Standesbuch 1810–1870, Heiratsregister 1863, Nr. 1 (Bild 309)}
\item 650 Jahre Laudenberg - Ein altes Dorf im Odenwald, Seite 164
\item \href{https://www.familysearch.org/tree/person/details/LV6B-35K}{FamilySearch, ID: LV6B-35K}
\item \href{http://gedbas.genealogy.net/person/show/1172964864}{genealogy.net}
\end{enumerate}

\end{person}

\begin{person}[
    surname = {Zimmermann},
    givenname = {Josepha},
    suffix = {1852--1852},
    label = {@I1348@}
    ]

\begin{tabular}{cl}
\geboren & 05. Mai 1852 in Laudenberg\\
\taufe & 05. Mai 1852 in Limbach\\
\gestorben & 08. Mai 1852 in Laudenberg\\
\bestattet & 09. Mai 1852 in Limbach\\
\end{tabular}\\
\medbreak
\textsc{vater}: \hyperref[@I396@]{Johann Valentin Zimmermann} [31.07.1830--11.12.1861 (4 Kinder)]\\
\textsc{mutter}: \hyperref[@I393@]{Josepha Zimmermann} [19.03.1831--20.11.1898 (7 Kinder)]
\medbreak
\textsc{{geschwister}}
\begin{itemize}
\item \hyperref[@I1349@]{Katharina Zimmermann} [07.01.1854]
\item \hyperref[@I392@]{Valentin Zimmermann} [24.02.1856--16.03.1923 (10 Kinder)]
\item \hyperref[@I1350@]{Rosa Zimmermann} [02.04.1859--21.03.1862]
\item \hyperref[@I1373@]{Wilhelm Albert} [02.05.1864--18.08.1919 (5 Kinder)]
\item \hyperref[@I1374@]{Johann Adam Albert} [22.06.1866--30.11.1937]
\item \hyperref[@I1375@]{Rosa Albert} [14.04.1869]
\end{itemize}
\bigbreak
\textsc{{quellen}}
\begin{enumerate}[label={[\arabic*]}]
\item GLA Karlsruhe, Laudenberg, katholische Gemeinde: Standesbuch 1810–1870, Geburtenregister 1852, Nr. 3 (Bild 238)
\item GLA Karlsruhe, Laudenberg, katholische Gemeinde: Standesbuch 1810–1870, Sterberegister 1852, Nr. 4 (Bild 241)
\item \href{http://gedbas.genealogy.net/person/show/1172964865}{genealogy.net}
\end{enumerate}

\end{person}

\begin{person}[
    surname = {Zimmermann},
    givenname = {Katharina},
    suffix = {1854},
    label = {@I1349@}
    ]

\begin{tabular}{cl}
\geboren & 07. Januar 1854 in Laudenberg\\
\taufe & 07. Januar 1854 in Limbach\\
\end{tabular}\\
\medbreak
\textsc{vater}: \hyperref[@I396@]{Johann Valentin Zimmermann} [31.07.1830--11.12.1861 (4 Kinder)]\\
\textsc{mutter}: \hyperref[@I393@]{Josepha Zimmermann} [19.03.1831--20.11.1898 (7 Kinder)]
\medbreak
\textsc{{geschwister}}
\begin{itemize}
\item \hyperref[@I1348@]{Josepha Zimmermann} [05.05.1852--08.05.1852]
\item \hyperref[@I392@]{Valentin Zimmermann} [24.02.1856--16.03.1923 (10 Kinder)]
\item \hyperref[@I1350@]{Rosa Zimmermann} [02.04.1859--21.03.1862]
\item \hyperref[@I1373@]{Wilhelm Albert} [02.05.1864--18.08.1919 (5 Kinder)]
\item \hyperref[@I1374@]{Johann Adam Albert} [22.06.1866--30.11.1937]
\item \hyperref[@I1375@]{Rosa Albert} [14.04.1869]
\end{itemize}
\bigbreak
\textsc{anmerkung}\\
geh. Edelmann in Limbach
\medbreak
\textsc{{quellen}}
\begin{enumerate}[label={[\arabic*]}]
\item GLA Karlsruhe, Laudenberg, katholische Gemeinde: Standesbuch 1810–1870, Geburtenregister 1854, Nr. 1 (Bild 249)
\item \href{http://gedbas.genealogy.net/person/show/1172964877}{genealogy.net}
\end{enumerate}

\end{person}

\begin{person}[
    surname = {Zimmermann},
    givenname = {Rosa},
    suffix = {1859--1862},
    label = {@I1350@}
    ]

\begin{tabular}{cl}
\geboren & 02. April 1859 in Laudenberg\\
\taufe & 03. April 1859 in Limbach\\
\gestorben & 21. März 1862 in Laudenberg\\
\bestattet & 23. März 1862 in Limbach\\
\end{tabular}\\
\medbreak
\textsc{vater}: \hyperref[@I396@]{Johann Valentin Zimmermann} [31.07.1830--11.12.1861 (4 Kinder)]\\
\textsc{mutter}: \hyperref[@I393@]{Josepha Zimmermann} [19.03.1831--20.11.1898 (7 Kinder)]
\medbreak
\textsc{{geschwister}}
\begin{itemize}
\item \hyperref[@I1348@]{Josepha Zimmermann} [05.05.1852--08.05.1852]
\item \hyperref[@I1349@]{Katharina Zimmermann} [07.01.1854]
\item \hyperref[@I392@]{Valentin Zimmermann} [24.02.1856--16.03.1923 (10 Kinder)]
\item \hyperref[@I1373@]{Wilhelm Albert} [02.05.1864--18.08.1919 (5 Kinder)]
\item \hyperref[@I1374@]{Johann Adam Albert} [22.06.1866--30.11.1937]
\item \hyperref[@I1375@]{Rosa Albert} [14.04.1869]
\end{itemize}
\bigbreak
\textsc{{quellen}}
\begin{enumerate}[label={[\arabic*]}]
\item GLA Karlsruhe, Laudenberg, katholische Gemeinde: Standesbuch 1810–1870, Geburtenregister 1858, Nr. 13 (Bild 281)
\item GLA Karlsruhe, Laudenberg, katholische Gemeinde: Standesbuch 1810–1870, Sterberegister 1862, Nr. 6 ( Bild 303)
\item \href{http://gedbas.genealogy.net/person/show/1172964953}{genealogy.net}
\end{enumerate}

\end{person}

\begin{person}[
    surname = {Albert},
    givenname = {Wilhelm},
    suffix = {1864--1919},
    label = {@I1373@}
    ]

\begin{tabular}{cl}
\geboren & 02. Mai 1864 in Laudenberg\\
\taufe & 03. Mai 1864 in Limbach\\
\geheiratet & 26. Februar 1889 in Scheringen mit Genovefa Gramlich \\
\gestorben & 18. August 1919\\
\end{tabular}\\
\medbreak
\textsc{vater}: Joseph Anton Albert [31.01.1831--28.11.1909 (3 Kinder)]\\
\textsc{mutter}: \hyperref[@I393@]{Josepha Zimmermann} [19.03.1831--20.11.1898 (7 Kinder)]
\medbreak
\textsc{{geschwister}}
\begin{itemize}
\item \hyperref[@I1348@]{Josepha Zimmermann} [05.05.1852--08.05.1852]
\item \hyperref[@I1349@]{Katharina Zimmermann} [07.01.1854]
\item \hyperref[@I392@]{Valentin Zimmermann} [24.02.1856--16.03.1923 (10 Kinder)]
\item \hyperref[@I1350@]{Rosa Zimmermann} [02.04.1859--21.03.1862]
\item \hyperref[@I1374@]{Johann Adam Albert} [22.06.1866--30.11.1937]
\item \hyperref[@I1375@]{Rosa Albert} [14.04.1869]
\end{itemize}
\bigbreak
\textsc{{kinder}}
\begin{itemize}
\item Wilhelm Albert [04.12.1889--11.07.1933 (5 Kinder)]
\item Adolf Albert [10.02.1891--25.03.1940 (1 Kind)]
\item Engelbert Albert [30.10.1893--06.03.1919]
\item Joseph Albert [21.08.1895--21.03.1945]
\item Emil Albert [22.09.1905--21.12.1946]
\end{itemize}
\medbreak
\textsc{{quellen}}
\begin{enumerate}[label={[\arabic*]}]
\item \href{https://www.familysearch.org/tree/person/details/LV2T-RS7}{FamilySearch, ID: LV2T-RS7}
\item \href{http://gedbas.genealogy.net/person/show/1172976082}{genealogy.net}
\item GLA Karlsruhe, Laudenberg, katholische Gemeinde: Standesbuch 1810–1870, Geburtenregister 1864, Nr. 4 (Bild 316)
\end{enumerate}

\end{person}

\begin{person}[
    surname = {Albert},
    givenname = {Johann Adam},
    suffix = {1866--1937},
    label = {@I1374@}
    ]

\begin{tabular}{cl}
\geboren & 22. Juni 1866 in Laudenberg\\
\taufe & 23. Juni 1866\\
\geheiratet & 11. Februar 1892 mit Maria Elisabeth Throm \\
\gestorben & 30. November 1937 in Langenelz\\
\end{tabular}\\
\medbreak
\textsc{vater}: Joseph Anton Albert [31.01.1831--28.11.1909 (3 Kinder)]\\
\textsc{mutter}: \hyperref[@I393@]{Josepha Zimmermann} [19.03.1831--20.11.1898 (7 Kinder)]
\medbreak
\textsc{{geschwister}}
\begin{itemize}
\item \hyperref[@I1348@]{Josepha Zimmermann} [05.05.1852--08.05.1852]
\item \hyperref[@I1349@]{Katharina Zimmermann} [07.01.1854]
\item \hyperref[@I392@]{Valentin Zimmermann} [24.02.1856--16.03.1923 (10 Kinder)]
\item \hyperref[@I1350@]{Rosa Zimmermann} [02.04.1859--21.03.1862]
\item \hyperref[@I1373@]{Wilhelm Albert} [02.05.1864--18.08.1919 (5 Kinder)]
\item \hyperref[@I1375@]{Rosa Albert} [14.04.1869]
\end{itemize}
\bigbreak
\textsc{{quellen}}
\begin{enumerate}[label={[\arabic*]}]
\item GLA Karlsruhe, Laudenberg, katholische Gemeinde: Standesbuch 1810–1870, Geburtenregister 1866, Nr. 22 (Bild 330)
\item \href{https://www.familysearch.org/tree/person/details/LVRN-VXJ}{FamilySearch, ID: LVRN-VXJ}
\item \href{http://gedbas.genealogy.net/person/show/1172962340}{genealogy.net}
\end{enumerate}

\end{person}

\begin{person}[
    surname = {Albert},
    givenname = {Rosa},
    suffix = {1869},
    label = {@I1375@}
    ]

\begin{tabular}{cl}
\geboren & 14. April 1869 in Laudenberg\\
\taufe & 14. April 1869 in Limbach\\
\geheiratet & 29. Juli 1888 in Laudenberg mit Franz Joseph Henn \\
\end{tabular}\\
\medbreak
\textsc{vater}: Joseph Anton Albert [31.01.1831--28.11.1909 (3 Kinder)]\\
\textsc{mutter}: \hyperref[@I393@]{Josepha Zimmermann} [19.03.1831--20.11.1898 (7 Kinder)]
\medbreak
\textsc{{geschwister}}
\begin{itemize}
\item \hyperref[@I1348@]{Josepha Zimmermann} [05.05.1852--08.05.1852]
\item \hyperref[@I1349@]{Katharina Zimmermann} [07.01.1854]
\item \hyperref[@I392@]{Valentin Zimmermann} [24.02.1856--16.03.1923 (10 Kinder)]
\item \hyperref[@I1350@]{Rosa Zimmermann} [02.04.1859--21.03.1862]
\item \hyperref[@I1373@]{Wilhelm Albert} [02.05.1864--18.08.1919 (5 Kinder)]
\item \hyperref[@I1374@]{Johann Adam Albert} [22.06.1866--30.11.1937]
\end{itemize}
\bigbreak
\textsc{{quellen}}
\begin{enumerate}[label={[\arabic*]}]
\item GLA Karlsruhe, Laudenberg, katholische Gemeinde: Standesbuch 1810–1870, Geburtenregister 1869, Nr. 5 (Bild 351)
\item \href{https://www.familysearch.org/tree/person/details/LVRF-ZS4}{FamilySearch, ID: LVRF-ZS4}
\item \href{http://gedbas.genealogy.net/person/show/1172975843}{genealogy.net}
\end{enumerate}

\end{person}


\addsec{Johann Martin Albert  \& Katharina Hess }


\begin{person}[
    surname = {Albert},
    givenname = {Johann Martin},
    suffix = {1825--1899},
    label = {@I394@}
    ]

\begin{tabular}{cl}
\geboren & 25. Dezember 1825\\
\geheiratet & 23. September 1856 mit Katharina Hess \\
\gestorben & 01. Mai 1899\\
\end{tabular}\\
\medbreak
\textsc{vater}: Joseph Michael Albert [17.07.1796--06.10.1867 (7 Kinder)]\\
\textsc{mutter}: \hyperref[@I414@]{Anna Barbara Hilbert} [um 1796 (7 Kinder)]
\medbreak
\textsc{{geschwister}}
\begin{itemize}
\item Franz Valentin Albert [01.08.1828--15.09.1917]
\item Joseph Anton Albert [31.01.1831--28.11.1909 (3 Kinder)]
\item Johann Joseph Albert [04.04.1833--13.05.1900]
\item Barbara Katharina Albert [20.07.1835--13.12.1918]
\item Maria Anna Albert [20.02.1838]
\item Matthäus Albert [10.03.1840--03.03.1841]
\end{itemize}
\bigbreak
\textsc{{kinder}}
\begin{itemize}
\item \hyperref[@I1359@]{Johann Joseph Albert} [13.09.1857]
\item \hyperref[@I1360@]{Katharina Hildegard Albert} [25.04.1860--26.04.1860]
\item \hyperref[@I391@]{Thekla Maria Albert} [23.09.1863--24.11.1917 (6 Kinder)]
\item \hyperref[@I1361@]{Franz Valentin Albert} [01.12.1866--03.07.1930 (5 Kinder)]
\end{itemize}
\medbreak
\textsc{{quellen}}
\begin{enumerate}[label={[\arabic*]}]
\item \href{https://www.familysearch.org/tree/person/details/LV2T-P53}{FamilySearch, ID: LV2T-P53}
\item \href{http://gedbas.genealogy.net/person/show/1172963633}{genealogy.net}
\end{enumerate}

\end{person}

\begin{person}[
    surname = {Hess},
    givenname = {Katharina},
    suffix = {1832--1875},
    label = {@I395@}
    ]

\begin{tabular}{cl}
\geboren & 03. Juli 1832 in Oberneudorf\\
\geheiratet & 23. September 1856 mit Johann Martin Albert \\
\gestorben & 24. Oktober 1875\\
\end{tabular}\\
\medbreak
\textsc{{kinder}}
\begin{itemize}
\item \hyperref[@I1359@]{Johann Joseph Albert} [13.09.1857]
\item \hyperref[@I1360@]{Katharina Hildegard Albert} [25.04.1860--26.04.1860]
\item \hyperref[@I391@]{Thekla Maria Albert} [23.09.1863--24.11.1917 (6 Kinder)]
\item \hyperref[@I1361@]{Franz Valentin Albert} [01.12.1866--03.07.1930 (5 Kinder)]
\end{itemize}
\medbreak
\textsc{{quellen}}
\begin{enumerate}[label={[\arabic*]}]
\item \href{https://www.familysearch.org/tree/person/details/LV2T-5HR}{FamilySearch, LV2T-5HR}
\item \href{http://gedbas.genealogy.net/person/show/1172970041}{genealogy.net}
\end{enumerate}

\end{person}

\begin{person}[
    surname = {Albert},
    givenname = {Johann Joseph},
    suffix = {1857},
    label = {@I1359@}
    ]

\begin{tabular}{cl}
\geboren & 13. September 1857\\
\geheiratet & 21. Februar 1889 mit Karoline Streun \\
\end{tabular}\\
\medbreak
\textsc{vater}: \hyperref[@I394@]{Johann Martin Albert} [25.12.1825--01.05.1899 (4 Kinder)]\\
\textsc{mutter}: \hyperref[@I395@]{Katharina Hess} [03.07.1832--24.10.1875 (4 Kinder)]
\medbreak
\textsc{{geschwister}}
\begin{itemize}
\item \hyperref[@I1360@]{Katharina Hildegard Albert} [25.04.1860--26.04.1860]
\item \hyperref[@I391@]{Thekla Maria Albert} [23.09.1863--24.11.1917 (6 Kinder)]
\item \hyperref[@I1361@]{Franz Valentin Albert} [01.12.1866--03.07.1930 (5 Kinder)]
\end{itemize}
\bigbreak
\textsc{{quellen}}
\begin{enumerate}[label={[\arabic*]}]
\item \href{https://www.familysearch.org/tree/person/details/LV2T-5ZG}{FamilySearch, ID: LV2T-5ZG}
\end{enumerate}

\end{person}

\begin{person}[
    surname = {Albert},
    givenname = {Katharina Hildegard},
    suffix = {1860--1860},
    label = {@I1360@}
    ]

\begin{tabular}{cl}
\geboren & 25. April 1860\\
\gestorben & 26. April 1860\\
\end{tabular}\\
\medbreak
\textsc{vater}: \hyperref[@I394@]{Johann Martin Albert} [25.12.1825--01.05.1899 (4 Kinder)]\\
\textsc{mutter}: \hyperref[@I395@]{Katharina Hess} [03.07.1832--24.10.1875 (4 Kinder)]
\medbreak
\textsc{{geschwister}}
\begin{itemize}
\item \hyperref[@I1359@]{Johann Joseph Albert} [13.09.1857]
\item \hyperref[@I391@]{Thekla Maria Albert} [23.09.1863--24.11.1917 (6 Kinder)]
\item \hyperref[@I1361@]{Franz Valentin Albert} [01.12.1866--03.07.1930 (5 Kinder)]
\end{itemize}
\bigbreak
\textsc{{quellen}}
\begin{enumerate}[label={[\arabic*]}]
\item \href{https://www.familysearch.org/tree/person/details/LV2T-58Y}{FamilySearch, ID: LV2T-58Y}
\end{enumerate}

\end{person}

\begin{person}[
    surname = {Albert},
    givenname = {Franz Valentin},
    suffix = {1866--1930},
    label = {@I1361@}
    ]

\begin{tabular}{cl}
\geboren & 01. Dezember 1866 in Hambrunn\\
\geheiratet & 06. August 1896 in Langenelz mit Elisabeth Link \\
\gestorben & 03. Juli 1930 in Langenelz\\
\end{tabular}\\
\medbreak
\textsc{vater}: \hyperref[@I394@]{Johann Martin Albert} [25.12.1825--01.05.1899 (4 Kinder)]\\
\textsc{mutter}: \hyperref[@I395@]{Katharina Hess} [03.07.1832--24.10.1875 (4 Kinder)]
\medbreak
\textsc{{geschwister}}
\begin{itemize}
\item \hyperref[@I1359@]{Johann Joseph Albert} [13.09.1857]
\item \hyperref[@I1360@]{Katharina Hildegard Albert} [25.04.1860--26.04.1860]
\item \hyperref[@I391@]{Thekla Maria Albert} [23.09.1863--24.11.1917 (6 Kinder)]
\end{itemize}
\bigbreak
\textsc{{kinder}}
\begin{itemize}
\item Valentin Albert [28.09.1897--06.09.1965]
\item Amalie Elisabeth Albert [14.11.1899--21.01.1978]
\item Maria Albert [02.04.1902--24.12.1993]
\item Ida Albert [21.08.1904--02.01.1973]
\item Alfred Albert [20.01.1907--27.01.2005]
\end{itemize}
\medbreak
\textsc{anmerkung}\\
Amorsch (oberhalb Röckel)
\medbreak
\textsc{{quellen}}
\begin{enumerate}[label={[\arabic*]}]
\item \href{https://www.familysearch.org/tree/person/details/LV2T-5X3}{FamilySearch, ID: LV2T-5X3}
\end{enumerate}

\end{person}


\addsec{Johann Michael Röckel  \& Anna Theresia Herkert }


\begin{person}[
    surname = {Röckel},
    givenname = {Johann Michael},
    suffix = {1804--1884},
    label = {@I490@}
    ]

\begin{tabular}{cl}
\geboren & 1804 in Hollerbach\\
\geheiratet & 12. Februar 1828 in Hollerbach mit Anna Theresia Herkert \\
\gestorben & 02. Mai 1884 in Hollerbach\\
\end{tabular}\\
\medbreak
\textsc{vater}: Johann Röckel [um 1770--21.06.1842 (4 Kinder)]\\
\textsc{mutter}: \hyperref[@I493@]{Margaretha Biemer} [um 1770--30.11.1847 (4 Kinder)]
\medbreak
\textsc{{geschwister}}
\begin{itemize}
\item Anna Maria Röckel [um 1793--30.11.1858]
\item Margaretha Röckel [um 1801]
\item Maria Anna Röckel [um 1802]
\end{itemize}
\bigbreak
\textsc{{kinder}}
\begin{itemize}
\item \hyperref[@I496@]{Margaretha Röckel} [10.04.1829--03.02.1886 (11 Kinder)]
\item \hyperref[@I497@]{Anna Theresia Röckel} [01.07.1831--25.12.1865]
\item \hyperref[@I498@]{Johann Theodor Röckel} [01.11.1833--25.09.1918 (1 Kind)]
\item \hyperref[@I499@]{Eva Katharina Röckel} [13.07.1836]
\item \hyperref[@I386@]{Joseph Michael Röckel} [19.03.1839--03.10.1888 (9 Kinder)]
\item \hyperref[@I500@]{Maria Josepha Röckel} [13.02.1842--19.02.1842]
\item \hyperref[@I501@]{Johann Martin Röckel} [12.07.1843--21.12.1906]
\end{itemize}
\medbreak
\textsc{{quellen}}
\begin{enumerate}[label={[\arabic*]}]
\item GLA Karlsruhe, Hollerbach, katholische Gemeinde: Standesbuch 1810–1842, Heiratsregister 1828, Nr. 1 (Bild 132)
\item \href{https://www.familysearch.org/tree/person/details/9JMM-2YJ}{FamilySearch, ID: 9JMM-2YJ}
\item \href{http://gedbas.genealogy.net/person/show/1172958277}{genealogy.net}
\end{enumerate}

\end{person}

\begin{person}[
    surname = {Herkert},
    givenname = {Anna Theresia},
    suffix = {um 1801--1853},
    label = {@I491@}
    ]

\begin{tabular}{cl}
\geboren & um 1801 in Steinbach\\
\geheiratet & 12. Februar 1828 in Hollerbach mit Johann Michael Röckel \\
\gestorben & 18. Dezember 1853 in Hollerbach\\
\bestattet & 20. Dezember 1853 in Hollerbach\\
\end{tabular}\\
\medbreak
\textsc{vater}: Johann Thomas Herkert [um 1760--28.11.1844 (5 Kinder)]\\
\textsc{mutter}: \hyperref[@I495@]{Katharina Grünwald} [um 1761--07.05.1820 (5 Kinder)]
\medbreak
\textsc{{geschwister}}
\begin{itemize}
\item Catharina Herkert [1795--14.04.1838]
\item Johann Michael Herkert [04.08.1799--08.02.1864]
\item Thomas Herkert [um 1785--03.10.1835]
\item Franz Joseph Herkert [um 1791--06.05.1824]
\end{itemize}
\bigbreak
\textsc{{kinder}}
\begin{itemize}
\item \hyperref[@I496@]{Margaretha Röckel} [10.04.1829--03.02.1886 (11 Kinder)]
\item \hyperref[@I497@]{Anna Theresia Röckel} [01.07.1831--25.12.1865]
\item \hyperref[@I498@]{Johann Theodor Röckel} [01.11.1833--25.09.1918 (1 Kind)]
\item \hyperref[@I499@]{Eva Katharina Röckel} [13.07.1836]
\item \hyperref[@I386@]{Joseph Michael Röckel} [19.03.1839--03.10.1888 (9 Kinder)]
\item \hyperref[@I500@]{Maria Josepha Röckel} [13.02.1842--19.02.1842]
\item \hyperref[@I501@]{Johann Martin Röckel} [12.07.1843--21.12.1906]
\end{itemize}
\medbreak
\textsc{anmerkung}\\
Geburtsort unsicher
\medbreak
\textsc{{quellen}}
\begin{enumerate}[label={[\arabic*]}]
\item GLA Karlsruhe, Hollerbach, katholische Gemeinde: Standesbuch 1810–1842, Heiratsregister 1828, Nr. 1 (Bild 132)
\item GLA Karlsruhe, Hollerbach, katholische Gemeinde: Standesbuch 1843–1870, Sterberegister 1853, Nr. 3 (Bild 91)
\item Heiratsurkunde Joseph Michael Röckel
\item \href{http://gedbas.genealogy.net/person/show/1172969852}{genealogy.net}
\end{enumerate}

\end{person}

\begin{person}[
    surname = {Röckel},
    givenname = {Margaretha},
    suffix = {1829--1886},
    label = {@I496@}
    ]

\begin{tabular}{cl}
\geboren & 10. April 1829 in Hollerbach\\
\taufe & 10. April 1829 in Hollerbach\\
\geheiratet & 15. Februar 1855 in Limbach mit Franz Noe \\
\gestorben & 03. Februar 1886 in Balsbach\\
\end{tabular}\\
\medbreak
\textsc{vater}: \hyperref[@I490@]{Johann Michael Röckel} [1804--02.05.1884 (7 Kinder)]\\
\textsc{mutter}: \hyperref[@I491@]{Anna Theresia Herkert} [um 1801--18.12.1853 (7 Kinder)]
\medbreak
\textsc{{geschwister}}
\begin{itemize}
\item \hyperref[@I497@]{Anna Theresia Röckel} [01.07.1831--25.12.1865]
\item \hyperref[@I498@]{Johann Theodor Röckel} [01.11.1833--25.09.1918 (1 Kind)]
\item \hyperref[@I499@]{Eva Katharina Röckel} [13.07.1836]
\item \hyperref[@I386@]{Joseph Michael Röckel} [19.03.1839--03.10.1888 (9 Kinder)]
\item \hyperref[@I500@]{Maria Josepha Röckel} [13.02.1842--19.02.1842]
\item \hyperref[@I501@]{Johann Martin Röckel} [12.07.1843--21.12.1906]
\end{itemize}
\bigbreak
\textsc{{kinder}}
\begin{itemize}
\item \hyperref[@I505@]{Margaretha Noe} [10.12.1855--17.01.1924 (7 Kinder)]
\item \hyperref[@I387@]{Rosa Noe} [15.06.1857--28.08.1920 (10 Kinder)]
\item \hyperref[@I506@]{Philippina Noe} [21.04.1859]
\item \hyperref[@I507@]{Karolina Noe} [12.03.1861--23.03.1861]
\item \hyperref[@I508@]{Katharina Noe} [10.02.1862]
\item \hyperref[@I509@]{Wilhelm Noe} [25.03.1864]
\item \hyperref[@I510@]{Theresia Noe} [10.04.1865]
\item \hyperref[@I511@]{Anna Noe} [26.07.1867--14.09.1867]
\item \hyperref[@I1747@]{Maria Anna Noe} [11.04.1870--25.09.1870]
\item \hyperref[@I1748@]{Franz Karl Noe} [08.09.1871]
\item \hyperref[@I1749@]{Emma Noe} [10.02.1874]
\end{itemize}
\medbreak
\textsc{{quellen}}
\begin{enumerate}[label={[\arabic*]}]
\item \href{http://www.landesarchiv-bw.de/plink/?f=4-1120207-245}{GLA Karlsruhe, Balsbach, evangelische und katholische Gemeinde: Standesbuch 1810–1866, Heiratsregister 1855, Nr. 1 (Bild 245)}
\item \href{https://www.familysearch.org/tree/person/details/9JMM-L6Q}{FamilySearch, ID: 9JMM-L6Q}
\item \href{http://gedbas.genealogy.net/person/show/1172958283}{genealogy.net}
\end{enumerate}

\end{person}

\begin{person}[
    surname = {Röckel},
    givenname = {Anna Theresia},
    suffix = {1831--1865},
    label = {@I497@}
    ]

\begin{tabular}{cl}
\geboren & 01. Juli 1831 in Hollerbach\\
\taufe & 01. Juli 1831 in Hollerbach\\
\geheiratet & 30. Juli 1863 in Buchen mit Johann Stephan Schwing \\
 &  mit Franz Bachmann \\
\gestorben & 25. Dezember 1865 in Unterneudorf\\
\bestattet & 27. Dezember 1865 in Buchen\\
\end{tabular}\\
\medbreak
\textsc{vater}: \hyperref[@I490@]{Johann Michael Röckel} [1804--02.05.1884 (7 Kinder)]\\
\textsc{mutter}: \hyperref[@I491@]{Anna Theresia Herkert} [um 1801--18.12.1853 (7 Kinder)]
\medbreak
\textsc{{geschwister}}
\begin{itemize}
\item \hyperref[@I496@]{Margaretha Röckel} [10.04.1829--03.02.1886 (11 Kinder)]
\item \hyperref[@I498@]{Johann Theodor Röckel} [01.11.1833--25.09.1918 (1 Kind)]
\item \hyperref[@I499@]{Eva Katharina Röckel} [13.07.1836]
\item \hyperref[@I386@]{Joseph Michael Röckel} [19.03.1839--03.10.1888 (9 Kinder)]
\item \hyperref[@I500@]{Maria Josepha Röckel} [13.02.1842--19.02.1842]
\item \hyperref[@I501@]{Johann Martin Röckel} [12.07.1843--21.12.1906]
\end{itemize}
\bigbreak
\textsc{{quellen}}
\begin{enumerate}[label={[\arabic*]}]
\item \href{https://www.familysearch.org/tree/person/details/LKSD-TQC}{FamilySearch, ID: LKSD-TQC}
\item \href{http://gedbas.genealogy.net/person/show/1172958269}{genealogy.net}
\end{enumerate}

\end{person}

\begin{person}[
    surname = {Röckel},
    givenname = {Johann Theodor},
    suffix = {1833--1918},
    label = {@I498@}
    ]

\begin{tabular}{cl}
\geboren & 01. November 1833 in Hollerbach\\
\taufe & 01. November 1833 in Hollerbach\\
\geheiratet & 25. Juni 1863 in Hollerbach mit Crescentia Galm \\
\gestorben & 25. September 1918\\
\end{tabular}\\
\medbreak
\textsc{vater}: \hyperref[@I490@]{Johann Michael Röckel} [1804--02.05.1884 (7 Kinder)]\\
\textsc{mutter}: \hyperref[@I491@]{Anna Theresia Herkert} [um 1801--18.12.1853 (7 Kinder)]
\medbreak
\textsc{{geschwister}}
\begin{itemize}
\item \hyperref[@I496@]{Margaretha Röckel} [10.04.1829--03.02.1886 (11 Kinder)]
\item \hyperref[@I497@]{Anna Theresia Röckel} [01.07.1831--25.12.1865]
\item \hyperref[@I499@]{Eva Katharina Röckel} [13.07.1836]
\item \hyperref[@I386@]{Joseph Michael Röckel} [19.03.1839--03.10.1888 (9 Kinder)]
\item \hyperref[@I500@]{Maria Josepha Röckel} [13.02.1842--19.02.1842]
\item \hyperref[@I501@]{Johann Martin Röckel} [12.07.1843--21.12.1906]
\end{itemize}
\bigbreak
\textsc{{kinder}}
\begin{itemize}
\item Josef Röckel [11.03.1867--1930]
\end{itemize}
\medbreak
\textsc{anmerkung}\\
Er und sein Sohn waren beide Bürgermeister in Hollerbach (vermutlich)
https://nat.museum-digital.de/index.php?t=objekt\&oges=166700\&cachesLoaded=true
\medbreak
\textsc{{quellen}}
\begin{enumerate}[label={[\arabic*]}]
\item \href{http://gedbas.genealogy.net/person/show/1172958278}{genealogy.net}
\item GLA Karlsruhe, Hollerbach, katholische Gemeinde: Standesbuch 1810–1842, Geburtenregister 1833, Nr. 7 (Bild 200)
\item GLA Karlsruhe, Hollerbach, katholische Gemeinde: Standesbuch 1843–1870, Heiratsregister 1863, Nr. 1 (Bild 143)
\end{enumerate}

\end{person}

\begin{person}[
    surname = {Röckel},
    givenname = {Eva Katharina},
    suffix = {1836},
    label = {@I499@}
    ]

\begin{tabular}{cl}
\geboren & 13. Juli 1836 in Hollerbach\\
\end{tabular}\\
\medbreak
\textsc{vater}: \hyperref[@I490@]{Johann Michael Röckel} [1804--02.05.1884 (7 Kinder)]\\
\textsc{mutter}: \hyperref[@I491@]{Anna Theresia Herkert} [um 1801--18.12.1853 (7 Kinder)]
\medbreak
\textsc{{geschwister}}
\begin{itemize}
\item \hyperref[@I496@]{Margaretha Röckel} [10.04.1829--03.02.1886 (11 Kinder)]
\item \hyperref[@I497@]{Anna Theresia Röckel} [01.07.1831--25.12.1865]
\item \hyperref[@I498@]{Johann Theodor Röckel} [01.11.1833--25.09.1918 (1 Kind)]
\item \hyperref[@I386@]{Joseph Michael Röckel} [19.03.1839--03.10.1888 (9 Kinder)]
\item \hyperref[@I500@]{Maria Josepha Röckel} [13.02.1842--19.02.1842]
\item \hyperref[@I501@]{Johann Martin Röckel} [12.07.1843--21.12.1906]
\end{itemize}
\bigbreak
\textsc{{quellen}}
\begin{enumerate}[label={[\arabic*]}]
\item GLA Karlsruhe, Hollerbach, katholische Gemeinde: Standesbuch 1810–1842, Geburtenregister 1836, Nr. 6 (Bild 220)
\item \href{http://gedbas.genealogy.net/person/show/1172958272}{genealogy.net}
\end{enumerate}

\end{person}

\begin{person}[
    surname = {Röckel},
    givenname = {Maria Josepha},
    suffix = {1842--1842},
    label = {@I500@}
    ]

\begin{tabular}{cl}
\geboren & 13. Februar 1842 in Hollerbach\\
\taufe & 13. Februar 1842 in Hollerbach\\
\gestorben & 19. Februar 1842 in Hollerbach\\
\bestattet & 22. Februar 1842 in Hollerbach\\
\end{tabular}\\
\medbreak
\textsc{vater}: \hyperref[@I490@]{Johann Michael Röckel} [1804--02.05.1884 (7 Kinder)]\\
\textsc{mutter}: \hyperref[@I491@]{Anna Theresia Herkert} [um 1801--18.12.1853 (7 Kinder)]
\medbreak
\textsc{{geschwister}}
\begin{itemize}
\item \hyperref[@I496@]{Margaretha Röckel} [10.04.1829--03.02.1886 (11 Kinder)]
\item \hyperref[@I497@]{Anna Theresia Röckel} [01.07.1831--25.12.1865]
\item \hyperref[@I498@]{Johann Theodor Röckel} [01.11.1833--25.09.1918 (1 Kind)]
\item \hyperref[@I499@]{Eva Katharina Röckel} [13.07.1836]
\item \hyperref[@I386@]{Joseph Michael Röckel} [19.03.1839--03.10.1888 (9 Kinder)]
\item \hyperref[@I501@]{Johann Martin Röckel} [12.07.1843--21.12.1906]
\end{itemize}
\bigbreak
\textsc{{quellen}}
\begin{enumerate}[label={[\arabic*]}]
\item GLA Karlsruhe, Hollerbach, katholische Gemeinde: Standesbuch 1810–1842, Geburtenregister 1842, Nr. 3 (Bild 259)
\item GLA Karlsruhe, Hollerbach, katholische Gemeinde: Standesbuch 1810–1842, Sterberegister 1842, Nr. 2 (Bild 263)
\item \href{http://gedbas.genealogy.net/person/show/1172958287}{genealogy.net}
\end{enumerate}

\end{person}

\begin{person}[
    surname = {Röckel},
    givenname = {Johann Martin},
    suffix = {1843--1906},
    label = {@I501@}
    ]

\begin{tabular}{cl}
\geboren & 12. Juli 1843 in Hollerbach\\
\taufe & 12. Juli 1843 in Hollerbach\\
\geheiratet & 05. Juni 1884 in Hollerbach mit Maria Anna Beckert \\
\gestorben & 21. Dezember 1906\\
\end{tabular}\\
\medbreak
\textsc{vater}: \hyperref[@I490@]{Johann Michael Röckel} [1804--02.05.1884 (7 Kinder)]\\
\textsc{mutter}: \hyperref[@I491@]{Anna Theresia Herkert} [um 1801--18.12.1853 (7 Kinder)]
\medbreak
\textsc{{geschwister}}
\begin{itemize}
\item \hyperref[@I496@]{Margaretha Röckel} [10.04.1829--03.02.1886 (11 Kinder)]
\item \hyperref[@I497@]{Anna Theresia Röckel} [01.07.1831--25.12.1865]
\item \hyperref[@I498@]{Johann Theodor Röckel} [01.11.1833--25.09.1918 (1 Kind)]
\item \hyperref[@I499@]{Eva Katharina Röckel} [13.07.1836]
\item \hyperref[@I386@]{Joseph Michael Röckel} [19.03.1839--03.10.1888 (9 Kinder)]
\item \hyperref[@I500@]{Maria Josepha Röckel} [13.02.1842--19.02.1842]
\end{itemize}
\bigbreak
\textsc{{quellen}}
\begin{enumerate}[label={[\arabic*]}]
\item GLA Karlsruhe, Hollerbach, katholische Gemeinde: Standesbuch 1843–1870, Geburtenregister 1843, Nr. 12 (Bild 5)
\item \href{http://gedbas.genealogy.net/person/show/1172958276}{genealogy.net}
\end{enumerate}

\end{person}


\addsec{Franz Noe  \& Margaretha Röckel }


\begin{person}[
    surname = {Noe},
    givenname = {Franz},
    suffix = {1824--1897},
    label = {@I504@}
    ]

\begin{tabular}{cl}
\geboren & 09. Dezember 1824 in Balsbach\\
\taufe & 10. Dezember 1824 in Limbach\\
\geheiratet & 15. Februar 1855 in Limbach mit Margaretha Röckel \\
\gestorben & 13. August 1897 in Balsbach\\
\bestattet & 16. August 1897 in Balsbach\\
\end{tabular}\\
\medbreak
\textsc{vater}: Franz Mathäus Noe [14.05.1795--30.03.1866 (8 Kinder)]\\
\textsc{mutter}: \hyperref[@I513@]{Katharina Barbara Schwab} [um 1796--13.04.1863 (8 Kinder)]
\medbreak
\textsc{{geschwister}}
\begin{itemize}
\item Katharina Noe [27.07.1827--12.08.1832]
\item Johann Joseph Noe [28.03.1830]
\item Johann Valentin Noe [17.11.1832]
\item Johann Mathäus Noe [28.04.1834]
\item Franz Carl Noe [15.12.1837]
\item Johann Michael Noe [22.03.1837]
\item Adam Ludwig Noe [21.08.1842]
\end{itemize}
\bigbreak
\textsc{{kinder}}
\begin{itemize}
\item \hyperref[@I505@]{Margaretha Noe} [10.12.1855--17.01.1924 (7 Kinder)]
\item \hyperref[@I387@]{Rosa Noe} [15.06.1857--28.08.1920 (10 Kinder)]
\item \hyperref[@I506@]{Philippina Noe} [21.04.1859]
\item \hyperref[@I507@]{Karolina Noe} [12.03.1861--23.03.1861]
\item \hyperref[@I508@]{Katharina Noe} [10.02.1862]
\item \hyperref[@I509@]{Wilhelm Noe} [25.03.1864]
\item \hyperref[@I510@]{Theresia Noe} [10.04.1865]
\item \hyperref[@I511@]{Anna Noe} [26.07.1867--14.09.1867]
\item \hyperref[@I1747@]{Maria Anna Noe} [11.04.1870--25.09.1870]
\item \hyperref[@I1748@]{Franz Karl Noe} [08.09.1871]
\item \hyperref[@I1749@]{Emma Noe} [10.02.1874]
\end{itemize}
\medbreak
\textsc{anmerkung}\\
erwähnt auf Heiratsurkunde Rosa Noe und Sterbeurkunde Josef Michael Roeckel
\medbreak
\textsc{{quellen}}
\begin{enumerate}[label={[\arabic*]}]
\item \href{http://www.landesarchiv-bw.de/plink/?f=4-1120207-245}{GLA Karlsruhe, Balsbach, evangelische und katholische Gemeinde: Standesbuch 1810–1866, Heiratsregister 1855, Nr. 1 (Bild 245)}
\item \href{https://www.familysearch.org/tree/person/details/LVXX-NHY}{FamilySearch, ID: LVXX-NHY}
\item \href{http://gedbas.genealogy.net/person/show/1172956702}{genealogy.net}
\end{enumerate}

\end{person}

\begin{person}[
    surname = {Noe},
    givenname = {Margaretha},
    suffix = {1855--1924},
    label = {@I505@},
    filename = {Margaretha Noe (1855)}
    ]

\begin{tabular}{cl}
\geboren & 10. Dezember 1855 in Balsbach\\
\taufe & 12. Dezember 1855 in Limbach\\
\geheiratet &  mit Franz Karl Heck \\
\gestorben & 17. Januar 1924 in Balsbach\\
\end{tabular}\\
\medbreak
\textsc{vater}: \hyperref[@I504@]{Franz Noe} [09.12.1824--13.08.1897 (11 Kinder)]\\
\textsc{mutter}: \hyperref[@I496@]{Margaretha Röckel} [10.04.1829--03.02.1886 (11 Kinder)]
\medbreak
\textsc{{geschwister}}
\begin{itemize}
\item \hyperref[@I387@]{Rosa Noe} [15.06.1857--28.08.1920 (10 Kinder)]
\item \hyperref[@I506@]{Philippina Noe} [21.04.1859]
\item \hyperref[@I507@]{Karolina Noe} [12.03.1861--23.03.1861]
\item \hyperref[@I508@]{Katharina Noe} [10.02.1862]
\item \hyperref[@I509@]{Wilhelm Noe} [25.03.1864]
\item \hyperref[@I510@]{Theresia Noe} [10.04.1865]
\item \hyperref[@I511@]{Anna Noe} [26.07.1867--14.09.1867]
\item \hyperref[@I1747@]{Maria Anna Noe} [11.04.1870--25.09.1870]
\item \hyperref[@I1748@]{Franz Karl Noe} [08.09.1871]
\item \hyperref[@I1749@]{Emma Noe} [10.02.1874]
\end{itemize}
\bigbreak
\textsc{{kinder}}
\begin{itemize}
\item Karoline Heck [29.03.1882--14.12.1974]
\item Philipine Heck [23.05.1883]
\item Karl Leo Heck [25.11.1884--23.10.1961 (1 Kind)]
\item Eduard Heck [25.12.1886--25.04.1974]
\item Otto Heck [...]
\item Valentin Heck [...]
\item Wilhelm Heck [...]
\end{itemize}
\medbreak
\textsc{anmerkung}\\
an Lungenverschleimung gestorben
\medbreak
\textsc{{quellen}}
\begin{enumerate}[label={[\arabic*]}]
\item \href{http://www.landesarchiv-bw.de/plink/?f=4-1120207-244}{GLA Karlsruhe, Balsbach, evangelische und katholische Gemeinde: Standesbuch 1810–1866, Geburtenregister 1855, Nr. 10 (Bild 244)}
\item \href{https://www.familysearch.org/tree/person/details/L5TX-FJT}{FamilySearch, ID: L5TX-FJT}
\item \href{http://gedbas.genealogy.net/person/show/1172957121}{genealogy.net}
\end{enumerate}

\end{person}

\begin{person}[
    surname = {Noe},
    givenname = {Philippina},
    suffix = {1859},
    label = {@I506@}
    ]

\begin{tabular}{cl}
\geboren & 21. April 1859 in Balsbach\\
\taufe & 25. April 1859 in Limbach\\
\end{tabular}\\
\medbreak
\textsc{vater}: \hyperref[@I504@]{Franz Noe} [09.12.1824--13.08.1897 (11 Kinder)]\\
\textsc{mutter}: \hyperref[@I496@]{Margaretha Röckel} [10.04.1829--03.02.1886 (11 Kinder)]
\medbreak
\textsc{{geschwister}}
\begin{itemize}
\item \hyperref[@I505@]{Margaretha Noe} [10.12.1855--17.01.1924 (7 Kinder)]
\item \hyperref[@I387@]{Rosa Noe} [15.06.1857--28.08.1920 (10 Kinder)]
\item \hyperref[@I507@]{Karolina Noe} [12.03.1861--23.03.1861]
\item \hyperref[@I508@]{Katharina Noe} [10.02.1862]
\item \hyperref[@I509@]{Wilhelm Noe} [25.03.1864]
\item \hyperref[@I510@]{Theresia Noe} [10.04.1865]
\item \hyperref[@I511@]{Anna Noe} [26.07.1867--14.09.1867]
\item \hyperref[@I1747@]{Maria Anna Noe} [11.04.1870--25.09.1870]
\item \hyperref[@I1748@]{Franz Karl Noe} [08.09.1871]
\item \hyperref[@I1749@]{Emma Noe} [10.02.1874]
\end{itemize}
\bigbreak
\textsc{{quellen}}
\begin{enumerate}[label={[\arabic*]}]
\item GLA Karlsruhe, Balsbach, evangelische und katholische Gemeinde: Standesbuch 1810–1866, Geburtenregister 1859, Nr. 3 (Bild 267)
\item \href{https://www.familysearch.org/tree/person/details/LVPW-DWN}{FamilySearch, ID: LVPW-DWN}
\item \href{http://gedbas.genealogy.net/person/show/1172957269}{genealogy.net}
\end{enumerate}

\end{person}

\begin{person}[
    surname = {Noe},
    givenname = {Karolina},
    suffix = {1861--1861},
    label = {@I507@}
    ]

\begin{tabular}{cl}
\geboren & 12. März 1861 in Balsbach\\
\taufe & 13. März 1861 in Limbach\\
\gestorben & 23. März 1861 in Balsbach\\
\bestattet & 25. März 1861 in Limbach\\
\end{tabular}\\
\medbreak
\textsc{vater}: \hyperref[@I504@]{Franz Noe} [09.12.1824--13.08.1897 (11 Kinder)]\\
\textsc{mutter}: \hyperref[@I496@]{Margaretha Röckel} [10.04.1829--03.02.1886 (11 Kinder)]
\medbreak
\textsc{{geschwister}}
\begin{itemize}
\item \hyperref[@I505@]{Margaretha Noe} [10.12.1855--17.01.1924 (7 Kinder)]
\item \hyperref[@I387@]{Rosa Noe} [15.06.1857--28.08.1920 (10 Kinder)]
\item \hyperref[@I506@]{Philippina Noe} [21.04.1859]
\item \hyperref[@I508@]{Katharina Noe} [10.02.1862]
\item \hyperref[@I509@]{Wilhelm Noe} [25.03.1864]
\item \hyperref[@I510@]{Theresia Noe} [10.04.1865]
\item \hyperref[@I511@]{Anna Noe} [26.07.1867--14.09.1867]
\item \hyperref[@I1747@]{Maria Anna Noe} [11.04.1870--25.09.1870]
\item \hyperref[@I1748@]{Franz Karl Noe} [08.09.1871]
\item \hyperref[@I1749@]{Emma Noe} [10.02.1874]
\end{itemize}
\bigbreak
\textsc{{quellen}}
\begin{enumerate}[label={[\arabic*]}]
\item \href{https://www.familysearch.org/tree/person/details/LVPW-D4M}{FamilySearch, ID: LVPW-D4M}
\item \href{http://gedbas.genealogy.net/person/show/1172957056}{genealogy.net}
\end{enumerate}

\end{person}

\begin{person}[
    surname = {Noe},
    givenname = {Katharina},
    suffix = {1862},
    label = {@I508@}
    ]

\begin{tabular}{cl}
\geboren & 10. Februar 1862 in Balsbach\\
\taufe & 11. Februar 1862 in Limbach\\
\geheiratet & 29. Februar 1888 in Rittersbach mit Franz Breidinger \\
\end{tabular}\\
\medbreak
\textsc{vater}: \hyperref[@I504@]{Franz Noe} [09.12.1824--13.08.1897 (11 Kinder)]\\
\textsc{mutter}: \hyperref[@I496@]{Margaretha Röckel} [10.04.1829--03.02.1886 (11 Kinder)]
\medbreak
\textsc{{geschwister}}
\begin{itemize}
\item \hyperref[@I505@]{Margaretha Noe} [10.12.1855--17.01.1924 (7 Kinder)]
\item \hyperref[@I387@]{Rosa Noe} [15.06.1857--28.08.1920 (10 Kinder)]
\item \hyperref[@I506@]{Philippina Noe} [21.04.1859]
\item \hyperref[@I507@]{Karolina Noe} [12.03.1861--23.03.1861]
\item \hyperref[@I509@]{Wilhelm Noe} [25.03.1864]
\item \hyperref[@I510@]{Theresia Noe} [10.04.1865]
\item \hyperref[@I511@]{Anna Noe} [26.07.1867--14.09.1867]
\item \hyperref[@I1747@]{Maria Anna Noe} [11.04.1870--25.09.1870]
\item \hyperref[@I1748@]{Franz Karl Noe} [08.09.1871]
\item \hyperref[@I1749@]{Emma Noe} [10.02.1874]
\end{itemize}
\bigbreak
\textsc{{quellen}}
\begin{enumerate}[label={[\arabic*]}]
\item \href{https://www.familysearch.org/tree/person/details/LVPW-D42}{FamilySearch, ID: LVPW-D42}
\item \href{http://gedbas.genealogy.net/person/show/1172957070}{genealogy.net}
\end{enumerate}

\end{person}

\begin{person}[
    surname = {Noe},
    givenname = {Wilhelm},
    suffix = {1864},
    label = {@I509@}
    ]

\begin{tabular}{cl}
\geboren & 25. März 1864 in Balsbach\\
\taufe & 28. März 1864 in Limbach\\
\end{tabular}\\
\medbreak
\textsc{vater}: \hyperref[@I504@]{Franz Noe} [09.12.1824--13.08.1897 (11 Kinder)]\\
\textsc{mutter}: \hyperref[@I496@]{Margaretha Röckel} [10.04.1829--03.02.1886 (11 Kinder)]
\medbreak
\textsc{{geschwister}}
\begin{itemize}
\item \hyperref[@I505@]{Margaretha Noe} [10.12.1855--17.01.1924 (7 Kinder)]
\item \hyperref[@I387@]{Rosa Noe} [15.06.1857--28.08.1920 (10 Kinder)]
\item \hyperref[@I506@]{Philippina Noe} [21.04.1859]
\item \hyperref[@I507@]{Karolina Noe} [12.03.1861--23.03.1861]
\item \hyperref[@I508@]{Katharina Noe} [10.02.1862]
\item \hyperref[@I510@]{Theresia Noe} [10.04.1865]
\item \hyperref[@I511@]{Anna Noe} [26.07.1867--14.09.1867]
\item \hyperref[@I1747@]{Maria Anna Noe} [11.04.1870--25.09.1870]
\item \hyperref[@I1748@]{Franz Karl Noe} [08.09.1871]
\item \hyperref[@I1749@]{Emma Noe} [10.02.1874]
\end{itemize}
\bigbreak
\textsc{{quellen}}
\begin{enumerate}[label={[\arabic*]}]
\item \href{https://www.familysearch.org/tree/person/details/LVPW-DH4}{FamilySearch, ID: LVPW-DH4}
\item \href{http://gedbas.genealogy.net/person/show/1172957343}{genealogy.net}
\end{enumerate}

\end{person}

\begin{person}[
    surname = {Noe},
    givenname = {Theresia},
    suffix = {1865},
    label = {@I510@}
    ]

\begin{tabular}{cl}
\geboren & 10. April 1865 in Balsbach\\
\taufe & 11. April 1865 in Limbach\\
\end{tabular}\\
\medbreak
\textsc{vater}: \hyperref[@I504@]{Franz Noe} [09.12.1824--13.08.1897 (11 Kinder)]\\
\textsc{mutter}: \hyperref[@I496@]{Margaretha Röckel} [10.04.1829--03.02.1886 (11 Kinder)]
\medbreak
\textsc{{geschwister}}
\begin{itemize}
\item \hyperref[@I505@]{Margaretha Noe} [10.12.1855--17.01.1924 (7 Kinder)]
\item \hyperref[@I387@]{Rosa Noe} [15.06.1857--28.08.1920 (10 Kinder)]
\item \hyperref[@I506@]{Philippina Noe} [21.04.1859]
\item \hyperref[@I507@]{Karolina Noe} [12.03.1861--23.03.1861]
\item \hyperref[@I508@]{Katharina Noe} [10.02.1862]
\item \hyperref[@I509@]{Wilhelm Noe} [25.03.1864]
\item \hyperref[@I511@]{Anna Noe} [26.07.1867--14.09.1867]
\item \hyperref[@I1747@]{Maria Anna Noe} [11.04.1870--25.09.1870]
\item \hyperref[@I1748@]{Franz Karl Noe} [08.09.1871]
\item \hyperref[@I1749@]{Emma Noe} [10.02.1874]
\end{itemize}
\bigbreak
\textsc{{quellen}}
\begin{enumerate}[label={[\arabic*]}]
\item \href{https://www.familysearch.org/tree/person/details/LVPW-DC8}{FamilySearch, ID: LVPW-DC8}
\item \href{http://gedbas.genealogy.net/person/show/1172957301}{genealogy.net}
\end{enumerate}

\end{person}

\begin{person}[
    surname = {Noe},
    givenname = {Anna},
    suffix = {1867--1867},
    label = {@I511@}
    ]

\begin{tabular}{cl}
\geboren & 26. Juli 1867 in Balsbach\\
\taufe & 27. Juli 1867 in Limbach\\
\gestorben & 14. September 1867 in Balsbach\\
\bestattet & 16. September 1867 in Limbach\\
\end{tabular}\\
\medbreak
\textsc{vater}: \hyperref[@I504@]{Franz Noe} [09.12.1824--13.08.1897 (11 Kinder)]\\
\textsc{mutter}: \hyperref[@I496@]{Margaretha Röckel} [10.04.1829--03.02.1886 (11 Kinder)]
\medbreak
\textsc{{geschwister}}
\begin{itemize}
\item \hyperref[@I505@]{Margaretha Noe} [10.12.1855--17.01.1924 (7 Kinder)]
\item \hyperref[@I387@]{Rosa Noe} [15.06.1857--28.08.1920 (10 Kinder)]
\item \hyperref[@I506@]{Philippina Noe} [21.04.1859]
\item \hyperref[@I507@]{Karolina Noe} [12.03.1861--23.03.1861]
\item \hyperref[@I508@]{Katharina Noe} [10.02.1862]
\item \hyperref[@I509@]{Wilhelm Noe} [25.03.1864]
\item \hyperref[@I510@]{Theresia Noe} [10.04.1865]
\item \hyperref[@I1747@]{Maria Anna Noe} [11.04.1870--25.09.1870]
\item \hyperref[@I1748@]{Franz Karl Noe} [08.09.1871]
\item \hyperref[@I1749@]{Emma Noe} [10.02.1874]
\end{itemize}
\bigbreak
\textsc{{quellen}}
\begin{enumerate}[label={[\arabic*]}]
\item \href{https://www.familysearch.org/tree/person/details/LVPW-DCY}{FamilySearch, ID: LVPW-DCY}
\item \href{http://gedbas.genealogy.net/person/show/1172956573}{genealogy.net}
\end{enumerate}

\end{person}

\begin{person}[
    surname = {Noe},
    givenname = {Maria Anna},
    suffix = {1870--1870},
    label = {@I1747@}
    ]

\begin{tabular}{cl}
\geboren & 11. April 1870 in Balsbach\\
\gestorben & 25. September 1870 in Balsbach\\
\end{tabular}\\
\medbreak
\textsc{vater}: \hyperref[@I504@]{Franz Noe} [09.12.1824--13.08.1897 (11 Kinder)]\\
\textsc{mutter}: \hyperref[@I496@]{Margaretha Röckel} [10.04.1829--03.02.1886 (11 Kinder)]
\medbreak
\textsc{{geschwister}}
\begin{itemize}
\item \hyperref[@I505@]{Margaretha Noe} [10.12.1855--17.01.1924 (7 Kinder)]
\item \hyperref[@I387@]{Rosa Noe} [15.06.1857--28.08.1920 (10 Kinder)]
\item \hyperref[@I506@]{Philippina Noe} [21.04.1859]
\item \hyperref[@I507@]{Karolina Noe} [12.03.1861--23.03.1861]
\item \hyperref[@I508@]{Katharina Noe} [10.02.1862]
\item \hyperref[@I509@]{Wilhelm Noe} [25.03.1864]
\item \hyperref[@I510@]{Theresia Noe} [10.04.1865]
\item \hyperref[@I511@]{Anna Noe} [26.07.1867--14.09.1867]
\item \hyperref[@I1748@]{Franz Karl Noe} [08.09.1871]
\item \hyperref[@I1749@]{Emma Noe} [10.02.1874]
\end{itemize}
\bigbreak
\textsc{{quellen}}
\begin{enumerate}[label={[\arabic*]}]
\item \href{https://www.familysearch.org/tree/person/details/G9R2-GQ5}{FamilySearch, ID: G9R2-GQ5}
\end{enumerate}

\end{person}

\begin{person}[
    surname = {Noe},
    givenname = {Franz Karl},
    suffix = {1871},
    label = {@I1748@}
    ]

\begin{tabular}{cl}
\geboren & 08. September 1871\\
\end{tabular}\\
\medbreak
\textsc{vater}: \hyperref[@I504@]{Franz Noe} [09.12.1824--13.08.1897 (11 Kinder)]\\
\textsc{mutter}: \hyperref[@I496@]{Margaretha Röckel} [10.04.1829--03.02.1886 (11 Kinder)]
\medbreak
\textsc{{geschwister}}
\begin{itemize}
\item \hyperref[@I505@]{Margaretha Noe} [10.12.1855--17.01.1924 (7 Kinder)]
\item \hyperref[@I387@]{Rosa Noe} [15.06.1857--28.08.1920 (10 Kinder)]
\item \hyperref[@I506@]{Philippina Noe} [21.04.1859]
\item \hyperref[@I507@]{Karolina Noe} [12.03.1861--23.03.1861]
\item \hyperref[@I508@]{Katharina Noe} [10.02.1862]
\item \hyperref[@I509@]{Wilhelm Noe} [25.03.1864]
\item \hyperref[@I510@]{Theresia Noe} [10.04.1865]
\item \hyperref[@I511@]{Anna Noe} [26.07.1867--14.09.1867]
\item \hyperref[@I1747@]{Maria Anna Noe} [11.04.1870--25.09.1870]
\item \hyperref[@I1749@]{Emma Noe} [10.02.1874]
\end{itemize}
\bigbreak
\textsc{{quellen}}
\begin{enumerate}[label={[\arabic*]}]
\item \href{https://www.familysearch.org/tree/person/details/G9R2-GVG}{FamilySearch, ID: G9R2-GVG}
\end{enumerate}

\end{person}

\begin{person}[
    surname = {Noe},
    givenname = {Emma},
    suffix = {1874},
    label = {@I1749@}
    ]

\begin{tabular}{cl}
\geboren & 10. Februar 1874 in Balsbach\\
\end{tabular}\\
\medbreak
\textsc{vater}: \hyperref[@I504@]{Franz Noe} [09.12.1824--13.08.1897 (11 Kinder)]\\
\textsc{mutter}: \hyperref[@I496@]{Margaretha Röckel} [10.04.1829--03.02.1886 (11 Kinder)]
\medbreak
\textsc{{geschwister}}
\begin{itemize}
\item \hyperref[@I505@]{Margaretha Noe} [10.12.1855--17.01.1924 (7 Kinder)]
\item \hyperref[@I387@]{Rosa Noe} [15.06.1857--28.08.1920 (10 Kinder)]
\item \hyperref[@I506@]{Philippina Noe} [21.04.1859]
\item \hyperref[@I507@]{Karolina Noe} [12.03.1861--23.03.1861]
\item \hyperref[@I508@]{Katharina Noe} [10.02.1862]
\item \hyperref[@I509@]{Wilhelm Noe} [25.03.1864]
\item \hyperref[@I510@]{Theresia Noe} [10.04.1865]
\item \hyperref[@I511@]{Anna Noe} [26.07.1867--14.09.1867]
\item \hyperref[@I1747@]{Maria Anna Noe} [11.04.1870--25.09.1870]
\item \hyperref[@I1748@]{Franz Karl Noe} [08.09.1871]
\end{itemize}
\bigbreak
\textsc{{quellen}}
\begin{enumerate}[label={[\arabic*]}]
\item \href{https://www.familysearch.org/tree/person/details/G9R2-TZG}{FamilySearch, ID: G9R2-TZG}
\end{enumerate}

\end{person}


\addsec{Johann Valentin Mechler  \& Katharina Scholl }


\begin{person}[
    surname = {Mechler},
    givenname = {Johann Valentin},
    suffix = {1815--1886},
    label = {@I946@}
    ]

\begin{tabular}{cl}
\geboren & 08. März 1815 in Mudau\\
\taufe & 09. März 1815 in Mudau\\
\geheiratet & 27. August 1850 in Mudau mit Katharina Scholl \\
\gestorben & 15. Dezember 1886 in Mudau\\
\end{tabular}\\
\medbreak
\textsc{vater}: Johann Valentin Mechler [1768--16.04.1845 (6 Kinder)]\\
\textsc{mutter}: \hyperref[@I1165@]{Eva Catharina Reichert} [...--vor 1850 (6 Kinder)]
\medbreak
\textsc{{geschwister}}
\begin{itemize}
\item Anna Maria Mechler [04.01.1810]
\item Franz Joseph Mechler [11.02.1822--07.01.1823]
\item Theresia Mechler [um 1803]
\item Maria Anna Mechler [um 1807]
\item Katharina Mechler [um 1809]
\end{itemize}
\bigbreak
\textsc{{kinder}}
\begin{itemize}
\item \hyperref[@I1750@]{Theresa Mechler} [16.08.1840]
\item \hyperref[@I1751@]{Katharina Mechler} [17.08.1843]
\item \hyperref[@I1752@]{Karl Mechler} [13.05.1847--18.12.1870]
\item \hyperref[@I1753@]{Wilhelm Mechler} [25.05.1851]
\item \hyperref[@I426@]{Valentin Mechler} [26.05.1855--04.01.1928 (4 Kinder)]
\end{itemize}
\medbreak
\textsc{anmerkung}\\
Todesdatum unleserlich
1851 baut Haus in Donebacher Strasse 
erkennt 2 Kinder von Katharina Scholl aus erster Ehe an (?)
\medbreak
\textsc{{quellen}}
\begin{enumerate}[label={[\arabic*]}]
\item \href{http://www.landesarchiv-bw.de/plink/?f=4-1119444-82}{GLA Karlsruhe, Mudau, katholische Gemeinde: Standesbuch 1812–1820, Taufregister 1815, Nr. 23 (Bild 82)}
\item \href{ http://www.landesarchiv-bw.de/plink/?f=4-1119480-109}{GLA Karlsruhe, Mudau, katholische Gemeinde: Standesbuch 1845–1855, Heiratsregister 1850, Nr. 7 (Bild 109)}
\item Mudau Standesbuch 1885–1887, Sterberegister 1886, Nr. 33
\item Heiratsurkunde Valentin Mechler und Eva Katharina Schaefer
\item Ahnentafel Erich Schnorr
\item \href{https://www.familysearch.org/tree/person/details/L1TZ-M87}{FamilySearch, ID: L1TZ-M87}
\end{enumerate}

\end{person}

\begin{person}[
    surname = {Scholl},
    givenname = {Katharina},
    suffix = {1813--1890},
    label = {@I947@}
    ]

\begin{tabular}{cl}
\geboren & 10. November 1813 in Mudau\\
\taufe & 10. November 1813 in Mudau\\
\geheiratet & 27. August 1850 in Mudau mit Johann Valentin Mechler \\
\gestorben & 11. Mai 1890 in Mudau\\
\end{tabular}\\
\medbreak
\textsc{vater}: Michael Scholl [15.09.1766--10.08.1854 (7 Kinder)]\\
\textsc{mutter}: \hyperref[@I1168@]{Agnes Reichert} [um 1769--27.12.1839 (7 Kinder)]
\medbreak
\textsc{{geschwister}}
\begin{itemize}
\item Franz Joseph Scholl [04.01.1811]
\item Eva Katharina Scholl [20.01.1817]
\item Maria Anna Scholl [um 1801]
\item Johann Sebastian Scholl [um 1808]
\item Anna Maria Scholl [um 1810]
\item Maria Agnes Scholl [um 1810]
\end{itemize}
\bigbreak
\textsc{{kinder}}
\begin{itemize}
\item \hyperref[@I1750@]{Theresa Mechler} [16.08.1840]
\item \hyperref[@I1751@]{Katharina Mechler} [17.08.1843]
\item \hyperref[@I1752@]{Karl Mechler} [13.05.1847--18.12.1870]
\item \hyperref[@I1753@]{Wilhelm Mechler} [25.05.1851]
\item \hyperref[@I426@]{Valentin Mechler} [26.05.1855--04.01.1928 (4 Kinder)]
\end{itemize}
\medbreak
\textsc{anmerkung}\\
74 Jahre alt (nach Sterbeurkunde). Widerspricht Geburtsdatum

Karl Mechler in Mudau Geburtenregister 1849 Nr. 36 (Bild 83)
http://www.landesarchiv-bw.de/plink/?f=4-1119480-83
\medbreak
\textsc{{quellen}}
\begin{enumerate}[label={[\arabic*]}]
\item \href{http://www.landesarchiv-bw.de/plink/?f=4-1119444-45}{GLA Karlsruhe, Mudau, katholische Gemeinde: Standesbuch 1812–1820, Taufregister 1813, Nr. 39 (Bild 45)}
\item \href{ http://www.landesarchiv-bw.de/plink/?f=4-1119480-109}{GLA Karlsruhe, Mudau, katholische Gemeinde: Standesbuch 1845–1855, Heiratsregister 1850, Nr. 7 (Bild 109)}
\item Mudau Standesbuch 1888–1890, Sterberegister 1890, Nr. 10
\item Heiratsurkunde Valentin Mechler und Eva Katharina Schäfer
\item Ahnentafel Erich Schnorr
\item \href{https://www.familysearch.org/tree/person/details/L1TZ-SYR}{FamilySearch, ID: L1TZ-SYR}
\end{enumerate}

\end{person}

\begin{person}[
    surname = {Mechler},
    givenname = {Theresa},
    suffix = {1840},
    label = {@I1750@}
    ]

\begin{tabular}{cl}
\geboren & 16. August 1840 in Mudau\\
\end{tabular}\\
\medbreak
\textsc{vater}: \hyperref[@I946@]{Johann Valentin Mechler} [08.03.1815--15.12.1886 (5 Kinder)]\\
\textsc{mutter}: \hyperref[@I947@]{Katharina Scholl} [10.11.1813--11.05.1890 (5 Kinder)]
\medbreak
\textsc{{geschwister}}
\begin{itemize}
\item \hyperref[@I1751@]{Katharina Mechler} [17.08.1843]
\item \hyperref[@I1752@]{Karl Mechler} [13.05.1847--18.12.1870]
\item \hyperref[@I1753@]{Wilhelm Mechler} [25.05.1851]
\item \hyperref[@I426@]{Valentin Mechler} [26.05.1855--04.01.1928 (4 Kinder)]
\end{itemize}
\bigbreak
\textsc{anmerkung}\\
uneheliches Kind. später annerkannt. kein Eintrag in Zweitschrift Kirchenbuch
\medbreak
\textsc{{quellen}}
\begin{enumerate}[label={[\arabic*]}]
\item Ahnentafel Jürgen Harjehusen
\end{enumerate}

\end{person}

\begin{person}[
    surname = {Mechler},
    givenname = {Katharina},
    suffix = {1843},
    label = {@I1751@}
    ]

\begin{tabular}{cl}
\geboren & 17. August 1843 in Mudau\\
\end{tabular}\\
\medbreak
\textsc{vater}: \hyperref[@I946@]{Johann Valentin Mechler} [08.03.1815--15.12.1886 (5 Kinder)]\\
\textsc{mutter}: \hyperref[@I947@]{Katharina Scholl} [10.11.1813--11.05.1890 (5 Kinder)]
\medbreak
\textsc{{geschwister}}
\begin{itemize}
\item \hyperref[@I1750@]{Theresa Mechler} [16.08.1840]
\item \hyperref[@I1752@]{Karl Mechler} [13.05.1847--18.12.1870]
\item \hyperref[@I1753@]{Wilhelm Mechler} [25.05.1851]
\item \hyperref[@I426@]{Valentin Mechler} [26.05.1855--04.01.1928 (4 Kinder)]
\end{itemize}
\bigbreak
\textsc{anmerkung}\\
uneheliches Kind. später annerkannt. kein Eintrag in Zweitschrift Kirchenbuch
\medbreak
\textsc{{quellen}}
\begin{enumerate}[label={[\arabic*]}]
\item Ahnentafel Jürgen Harjehusen
\end{enumerate}

\end{person}

\begin{person}[
    surname = {Mechler},
    givenname = {Karl},
    suffix = {1847--1870},
    label = {@I1752@}
    ]

\begin{tabular}{cl}
\geboren & 13. Mai 1847 in Mudau\\
\gestorben & 18. Dezember 1870 in Dijon\\
\end{tabular}\\
\medbreak
\textsc{vater}: \hyperref[@I946@]{Johann Valentin Mechler} [08.03.1815--15.12.1886 (5 Kinder)]\\
\textsc{mutter}: \hyperref[@I947@]{Katharina Scholl} [10.11.1813--11.05.1890 (5 Kinder)]
\medbreak
\textsc{{geschwister}}
\begin{itemize}
\item \hyperref[@I1750@]{Theresa Mechler} [16.08.1840]
\item \hyperref[@I1751@]{Katharina Mechler} [17.08.1843]
\item \hyperref[@I1753@]{Wilhelm Mechler} [25.05.1851]
\item \hyperref[@I426@]{Valentin Mechler} [26.05.1855--04.01.1928 (4 Kinder)]
\end{itemize}
\bigbreak
\textsc{{quellen}}
\begin{enumerate}[label={[\arabic*]}]
\item Ahnentafel Jürgen Harjehusen
\end{enumerate}

\end{person}

\begin{person}[
    surname = {Mechler},
    givenname = {Wilhelm},
    suffix = {1851},
    label = {@I1753@}
    ]

\begin{tabular}{cl}
\geboren & 25. Mai 1851 in Mudau\\
\end{tabular}\\
\medbreak
\textsc{vater}: \hyperref[@I946@]{Johann Valentin Mechler} [08.03.1815--15.12.1886 (5 Kinder)]\\
\textsc{mutter}: \hyperref[@I947@]{Katharina Scholl} [10.11.1813--11.05.1890 (5 Kinder)]
\medbreak
\textsc{{geschwister}}
\begin{itemize}
\item \hyperref[@I1750@]{Theresa Mechler} [16.08.1840]
\item \hyperref[@I1751@]{Katharina Mechler} [17.08.1843]
\item \hyperref[@I1752@]{Karl Mechler} [13.05.1847--18.12.1870]
\item \hyperref[@I426@]{Valentin Mechler} [26.05.1855--04.01.1928 (4 Kinder)]
\end{itemize}
\bigbreak
\textsc{{quellen}}
\begin{enumerate}[label={[\arabic*]}]
\item \href{http://www.landesarchiv-bw.de/plink/?f=4-1119480-121}{GLA Karlsruhe, Mudau, katholische Gemeinde: Standesbuch 1845–1855, Geburtenregister 1851, Nr. 26 (Bild 121)}
\item Ahnentafel Jürgen Harjehusen
\end{enumerate}

\end{person}


\addsec{Johann Josef Schäfer  \& Maria Anna Farrenkopf }


\begin{person}[
    surname = {Schäfer},
    givenname = {Johann Josef},
    suffix = {1811--1883},
    label = {@I948@}
    ]

\begin{tabular}{cl}
\geboren & 23. November 1811 in Schlossau\\
\geheiratet & 26. Juni 1834 in Schlossau mit Maria Anna Hamm \\
 & 23. November 1843 in Schlossau mit Maria Anna Farrenkopf \\
\gestorben & 21. Januar 1883 in Schlossau\\
\end{tabular}\\
\medbreak
\textsc{mutter}: \hyperref[@I1169@]{Theresia Schäfer} [04.01.1788--26.09.1869 (1 Kind)]
\medbreak
\textsc{{kinder}}
\begin{itemize}
\item \hyperref[@I1866@]{Johann Valentin Schäfer} [19.04.1835]
\item \hyperref[@I1867@]{Rosina Schäfer} [30.09.1837]
\item \hyperref[@I1871@]{Anna Christina Schäfer} [28.07.1839]
\item \hyperref[@I1870@]{Margaretha Schäfer} [28.04.1842]
\item \hyperref[@I1396@]{Karl Schäfer} [14.05.1844--02.02.1933]
\item \hyperref[@I1397@]{Linus Schäfer} [23.09.1846--nach 1885]
\item \hyperref[@I1398@]{Eva Clara Schäfer} [10.02.1850--08.07.1881]
\item \hyperref[@I1399@]{Maria Anna Schäfer} [05.10.1852]
\item \hyperref[@I388@]{Eva Katharina Schäfer} [06.10.1855--21.04.1942 (4 Kinder)]
\item \hyperref[@I1400@]{Franz Schäfer} [27.03.1858--06.10.1950]
\item \hyperref[@I1401@]{Rosina Schäfer} [14.09.1860]
\item \hyperref[@I1402@]{Ferdinand Schäfer} [28.06.1864]
\end{itemize}
\medbreak
\textsc{{quellen}}
\begin{enumerate}[label={[\arabic*]}]
\item \href{http://www.landesarchiv-bw.de/plink/?f=4-1119448-30}{GLA Karlsruhe, Mudau, katholische Gemeinde: Standesbuch 1834–1837, Heiratsregister 1834, Nr. 21 (Bild 30)}
\item \href{http://www.landesarchiv-bw.de/plink/?f=4-1119606-39}{GLA Karlsruhe, Schlossau, katholische Gemeinde: Standesbuch 1839–1850, Heiratsregister 1843, Nr. 2}
\item Schloßau Sterbebuch 1870–1899, Sterberegister 1883, Nr. 2
\item Heiratsurkunde Valentin Mechler und Eva Katharina Schaefer
\item Ahnentafel Erich Schnorr
\item \href{https://www.familysearch.org/tree/person/details/LT2X-KKL}{FamilySearch, ID: LT2X-KKL}
\end{enumerate}

\end{person}

\begin{person}[
    surname = {Farrenkopf},
    givenname = {Maria Anna},
    suffix = {1819--1905},
    label = {@I949@}
    ]

\begin{tabular}{cl}
\geboren & 29. August 1819 in Breitenbach (Ottorfzell)\\
\geheiratet & 23. November 1843 in Schlossau mit Johann Josef Schäfer \\
\gestorben & 01. Juni 1905 in Schlossau\\
\end{tabular}\\
\medbreak
\textsc{vater}: Georg Anton Farrenkopf [um 1792 (3 Kinder)]\\
\textsc{mutter}: \hyperref[@I1141@]{Anastasia Lösch} [um 1792 (3 Kinder)]
\medbreak
\textsc{{geschwister}}
\begin{itemize}
\item Eva Katharina Farrenkopf [22.11.1822--07.11.1883]
\item Theresia Farrenkopf [um 1830]
\end{itemize}
\bigbreak
\textsc{{kinder}}
\begin{itemize}
\item \hyperref[@I1396@]{Karl Schäfer} [14.05.1844--02.02.1933]
\item \hyperref[@I1397@]{Linus Schäfer} [23.09.1846--nach 1885]
\item \hyperref[@I1398@]{Eva Clara Schäfer} [10.02.1850--08.07.1881]
\item \hyperref[@I1399@]{Maria Anna Schäfer} [05.10.1852]
\item \hyperref[@I388@]{Eva Katharina Schäfer} [06.10.1855--21.04.1942 (4 Kinder)]
\item \hyperref[@I1400@]{Franz Schäfer} [27.03.1858--06.10.1950]
\item \hyperref[@I1401@]{Rosina Schäfer} [14.09.1860]
\item \hyperref[@I1402@]{Ferdinand Schäfer} [28.06.1864]
\end{itemize}
\medbreak
\textsc{anmerkung}\\
88 Jahre Alt?
\medbreak
\textsc{{quellen}}
\begin{enumerate}[label={[\arabic*]}]
\item \href{http://www.landesarchiv-bw.de/plink/?f=4-1119606-39}{GLA Karlsruhe, Schlossau, katholische Gemeinde: Standesbuch 1839–1850, Heiratsregister 1843, Nr. 2}
\item Schloßau Geburts-, Heirats- und Sterberegister 1905–1915, Sterberegister 1905, Nr. 4
\item Heiratsurkunde Valentin Mechler und Eva Katharina Schaefer
\item \href{https://www.familysearch.org/tree/person/details/LT2X-V3Z}{FamilySearch, ID: LT2X-V3Z}
\end{enumerate}

\end{person}

\begin{person}[
    surname = {Schäfer},
    givenname = {Johann Valentin},
    suffix = {1835},
    label = {@I1866@}
    ]

\begin{tabular}{cl}
\geboren & 19. April 1835 in Schlossau\\
\end{tabular}\\
\medbreak
\textsc{vater}: \hyperref[@I948@]{Johann Josef Schäfer} [23.11.1811--21.01.1883 (12 Kinder)]\\
\textsc{mutter}: \hyperref[@I1403@]{Maria Anna Hamm} [11.07.1810--vor 1843 (4 Kinder)]
\medbreak
\textsc{{geschwister}}
\begin{itemize}
\item \hyperref[@I1867@]{Rosina Schäfer} [30.09.1837]
\item \hyperref[@I1871@]{Anna Christina Schäfer} [28.07.1839]
\item \hyperref[@I1870@]{Margaretha Schäfer} [28.04.1842]
\item \hyperref[@I1396@]{Karl Schäfer} [14.05.1844--02.02.1933]
\item \hyperref[@I1397@]{Linus Schäfer} [23.09.1846--nach 1885]
\item \hyperref[@I1398@]{Eva Clara Schäfer} [10.02.1850--08.07.1881]
\item \hyperref[@I1399@]{Maria Anna Schäfer} [05.10.1852]
\item \hyperref[@I388@]{Eva Katharina Schäfer} [06.10.1855--21.04.1942 (4 Kinder)]
\item \hyperref[@I1400@]{Franz Schäfer} [27.03.1858--06.10.1950]
\item \hyperref[@I1401@]{Rosina Schäfer} [14.09.1860]
\item \hyperref[@I1402@]{Ferdinand Schäfer} [28.06.1864]
\end{itemize}
\bigbreak
\end{person}

\begin{person}[
    surname = {Schäfer},
    givenname = {Rosina},
    suffix = {1837},
    label = {@I1867@}
    ]

\begin{tabular}{cl}
\geboren & 30. September 1837 in Schlossau\\
\end{tabular}\\
\medbreak
\textsc{vater}: \hyperref[@I948@]{Johann Josef Schäfer} [23.11.1811--21.01.1883 (12 Kinder)]\\
\textsc{mutter}: \hyperref[@I1403@]{Maria Anna Hamm} [11.07.1810--vor 1843 (4 Kinder)]
\medbreak
\textsc{{geschwister}}
\begin{itemize}
\item \hyperref[@I1866@]{Johann Valentin Schäfer} [19.04.1835]
\item \hyperref[@I1871@]{Anna Christina Schäfer} [28.07.1839]
\item \hyperref[@I1870@]{Margaretha Schäfer} [28.04.1842]
\item \hyperref[@I1396@]{Karl Schäfer} [14.05.1844--02.02.1933]
\item \hyperref[@I1397@]{Linus Schäfer} [23.09.1846--nach 1885]
\item \hyperref[@I1398@]{Eva Clara Schäfer} [10.02.1850--08.07.1881]
\item \hyperref[@I1399@]{Maria Anna Schäfer} [05.10.1852]
\item \hyperref[@I388@]{Eva Katharina Schäfer} [06.10.1855--21.04.1942 (4 Kinder)]
\item \hyperref[@I1400@]{Franz Schäfer} [27.03.1858--06.10.1950]
\item \hyperref[@I1401@]{Rosina Schäfer} [14.09.1860]
\item \hyperref[@I1402@]{Ferdinand Schäfer} [28.06.1864]
\end{itemize}
\bigbreak
\end{person}

\begin{person}[
    surname = {Schäfer},
    givenname = {Anna Christina},
    suffix = {1839},
    label = {@I1871@}
    ]

\begin{tabular}{cl}
\geboren & 28. Juli 1839\\
\end{tabular}\\
\medbreak
\textsc{vater}: \hyperref[@I948@]{Johann Josef Schäfer} [23.11.1811--21.01.1883 (12 Kinder)]\\
\textsc{mutter}: \hyperref[@I1403@]{Maria Anna Hamm} [11.07.1810--vor 1843 (4 Kinder)]
\medbreak
\textsc{{geschwister}}
\begin{itemize}
\item \hyperref[@I1866@]{Johann Valentin Schäfer} [19.04.1835]
\item \hyperref[@I1867@]{Rosina Schäfer} [30.09.1837]
\item \hyperref[@I1870@]{Margaretha Schäfer} [28.04.1842]
\item \hyperref[@I1396@]{Karl Schäfer} [14.05.1844--02.02.1933]
\item \hyperref[@I1397@]{Linus Schäfer} [23.09.1846--nach 1885]
\item \hyperref[@I1398@]{Eva Clara Schäfer} [10.02.1850--08.07.1881]
\item \hyperref[@I1399@]{Maria Anna Schäfer} [05.10.1852]
\item \hyperref[@I388@]{Eva Katharina Schäfer} [06.10.1855--21.04.1942 (4 Kinder)]
\item \hyperref[@I1400@]{Franz Schäfer} [27.03.1858--06.10.1950]
\item \hyperref[@I1401@]{Rosina Schäfer} [14.09.1860]
\item \hyperref[@I1402@]{Ferdinand Schäfer} [28.06.1864]
\end{itemize}
\bigbreak
\end{person}

\begin{person}[
    surname = {Schäfer},
    givenname = {Margaretha},
    suffix = {1842},
    label = {@I1870@}
    ]

\begin{tabular}{cl}
\geboren & 28. April 1842\\
\end{tabular}\\
\medbreak
\textsc{vater}: \hyperref[@I948@]{Johann Josef Schäfer} [23.11.1811--21.01.1883 (12 Kinder)]\\
\textsc{mutter}: \hyperref[@I1403@]{Maria Anna Hamm} [11.07.1810--vor 1843 (4 Kinder)]
\medbreak
\textsc{{geschwister}}
\begin{itemize}
\item \hyperref[@I1866@]{Johann Valentin Schäfer} [19.04.1835]
\item \hyperref[@I1867@]{Rosina Schäfer} [30.09.1837]
\item \hyperref[@I1871@]{Anna Christina Schäfer} [28.07.1839]
\item \hyperref[@I1396@]{Karl Schäfer} [14.05.1844--02.02.1933]
\item \hyperref[@I1397@]{Linus Schäfer} [23.09.1846--nach 1885]
\item \hyperref[@I1398@]{Eva Clara Schäfer} [10.02.1850--08.07.1881]
\item \hyperref[@I1399@]{Maria Anna Schäfer} [05.10.1852]
\item \hyperref[@I388@]{Eva Katharina Schäfer} [06.10.1855--21.04.1942 (4 Kinder)]
\item \hyperref[@I1400@]{Franz Schäfer} [27.03.1858--06.10.1950]
\item \hyperref[@I1401@]{Rosina Schäfer} [14.09.1860]
\item \hyperref[@I1402@]{Ferdinand Schäfer} [28.06.1864]
\end{itemize}
\bigbreak
\end{person}

\begin{person}[
    surname = {Schäfer},
    givenname = {Karl},
    suffix = {1844--1933},
    label = {@I1396@}
    ]

\begin{tabular}{cl}
\geboren & 14. Mai 1844 in Schlossau\\
\gestorben & 02. Februar 1933 in Schlossau\\
\end{tabular}\\
\medbreak
\textsc{vater}: \hyperref[@I948@]{Johann Josef Schäfer} [23.11.1811--21.01.1883 (12 Kinder)]\\
\textsc{mutter}: \hyperref[@I949@]{Maria Anna Farrenkopf} [29.08.1819--01.06.1905 (8 Kinder)]
\medbreak
\textsc{{geschwister}}
\begin{itemize}
\item \hyperref[@I1866@]{Johann Valentin Schäfer} [19.04.1835]
\item \hyperref[@I1867@]{Rosina Schäfer} [30.09.1837]
\item \hyperref[@I1871@]{Anna Christina Schäfer} [28.07.1839]
\item \hyperref[@I1870@]{Margaretha Schäfer} [28.04.1842]
\item \hyperref[@I1397@]{Linus Schäfer} [23.09.1846--nach 1885]
\item \hyperref[@I1398@]{Eva Clara Schäfer} [10.02.1850--08.07.1881]
\item \hyperref[@I1399@]{Maria Anna Schäfer} [05.10.1852]
\item \hyperref[@I388@]{Eva Katharina Schäfer} [06.10.1855--21.04.1942 (4 Kinder)]
\item \hyperref[@I1400@]{Franz Schäfer} [27.03.1858--06.10.1950]
\item \hyperref[@I1401@]{Rosina Schäfer} [14.09.1860]
\item \hyperref[@I1402@]{Ferdinand Schäfer} [28.06.1864]
\end{itemize}
\bigbreak
\textsc{anmerkung}\\
Tod angezeigt von Sohn Joseph Schäfer. Frau vor ihm gestorben
\medbreak
\textsc{{quellen}}
\begin{enumerate}[label={[\arabic*]}]
\item \href{http://www.landesarchiv-bw.de/plink/?f=4-1119606-47}{GLA Karlsruhe, Schlossau, katholische Gemeinde: Standesbuch 1839–1850, Geburtenregister 1844, Nr. 13 (Bild 47)}
\item Schloßau Sterbebuch 1911–1933, Sterberegister 1933, Nr. 2
\item \href{https://www.familysearch.org/tree/person/details/LTCK-569}{FamilySearch, ID: LTCK-569}
\end{enumerate}

\end{person}

\begin{person}[
    surname = {Schäfer},
    givenname = {Linus},
    suffix = {1846--nach 1885},
    label = {@I1397@}
    ]

\begin{tabular}{cl}
\geboren & 23. September 1846 in Schlossau\\
\geheiratet & 04. Februar 1874 in Schlossau mit Maria Anna Haas \\
\gestorben & nach 1885\\
\end{tabular}\\
\medbreak
\textsc{vater}: \hyperref[@I948@]{Johann Josef Schäfer} [23.11.1811--21.01.1883 (12 Kinder)]\\
\textsc{mutter}: \hyperref[@I949@]{Maria Anna Farrenkopf} [29.08.1819--01.06.1905 (8 Kinder)]
\medbreak
\textsc{{geschwister}}
\begin{itemize}
\item \hyperref[@I1866@]{Johann Valentin Schäfer} [19.04.1835]
\item \hyperref[@I1867@]{Rosina Schäfer} [30.09.1837]
\item \hyperref[@I1871@]{Anna Christina Schäfer} [28.07.1839]
\item \hyperref[@I1870@]{Margaretha Schäfer} [28.04.1842]
\item \hyperref[@I1396@]{Karl Schäfer} [14.05.1844--02.02.1933]
\item \hyperref[@I1398@]{Eva Clara Schäfer} [10.02.1850--08.07.1881]
\item \hyperref[@I1399@]{Maria Anna Schäfer} [05.10.1852]
\item \hyperref[@I388@]{Eva Katharina Schäfer} [06.10.1855--21.04.1942 (4 Kinder)]
\item \hyperref[@I1400@]{Franz Schäfer} [27.03.1858--06.10.1950]
\item \hyperref[@I1401@]{Rosina Schäfer} [14.09.1860]
\item \hyperref[@I1402@]{Ferdinand Schäfer} [28.06.1864]
\end{itemize}
\bigbreak
\textsc{{quellen}}
\begin{enumerate}[label={[\arabic*]}]
\item \href{http://www.landesarchiv-bw.de/plink/?f=4-1119606-67}{GLA Karlsruhe, Schlossau, katholische Gemeinde: Standesbuch 1839–1850, Geburtenregister 1846, Nr. 22 (Bild 67)}
\item \href{https://www.familysearch.org/tree/person/details/LTCK-BYM}{FamilySearch, ID: LTCK-BYM}
\end{enumerate}

\end{person}

\begin{person}[
    surname = {Schäfer},
    givenname = {Eva Clara},
    suffix = {1850--1881},
    label = {@I1398@}
    ]

\begin{tabular}{cl}
\geboren & 10. Februar 1850 in Schlossau\\
\gestorben & 08. Juli 1881 in Schlossau\\
\end{tabular}\\
\medbreak
\textsc{vater}: \hyperref[@I948@]{Johann Josef Schäfer} [23.11.1811--21.01.1883 (12 Kinder)]\\
\textsc{mutter}: \hyperref[@I949@]{Maria Anna Farrenkopf} [29.08.1819--01.06.1905 (8 Kinder)]
\medbreak
\textsc{{geschwister}}
\begin{itemize}
\item \hyperref[@I1866@]{Johann Valentin Schäfer} [19.04.1835]
\item \hyperref[@I1867@]{Rosina Schäfer} [30.09.1837]
\item \hyperref[@I1871@]{Anna Christina Schäfer} [28.07.1839]
\item \hyperref[@I1870@]{Margaretha Schäfer} [28.04.1842]
\item \hyperref[@I1396@]{Karl Schäfer} [14.05.1844--02.02.1933]
\item \hyperref[@I1397@]{Linus Schäfer} [23.09.1846--nach 1885]
\item \hyperref[@I1399@]{Maria Anna Schäfer} [05.10.1852]
\item \hyperref[@I388@]{Eva Katharina Schäfer} [06.10.1855--21.04.1942 (4 Kinder)]
\item \hyperref[@I1400@]{Franz Schäfer} [27.03.1858--06.10.1950]
\item \hyperref[@I1401@]{Rosina Schäfer} [14.09.1860]
\item \hyperref[@I1402@]{Ferdinand Schäfer} [28.06.1864]
\end{itemize}
\bigbreak
\textsc{{quellen}}
\begin{enumerate}[label={[\arabic*]}]
\item \href{http://www.landesarchiv-bw.de/plink/?f=4-1119606-113}{GLA Karlsruhe, Schlossau, katholische Gemeinde: Standesbuch 1839–1850, Geburtenregister 1850, Nr. 6 (Bild 113)}
\item \href{https://www.familysearch.org/tree/person/details/LT2X-K2V}{FamilySearch, ID: LT2X-K2V}
\end{enumerate}

\end{person}

\begin{person}[
    surname = {Schäfer},
    givenname = {Maria Anna},
    suffix = {1852},
    label = {@I1399@}
    ]

\begin{tabular}{cl}
\geboren & 05. Oktober 1852 in Schlossau\\
\end{tabular}\\
\medbreak
\textsc{vater}: \hyperref[@I948@]{Johann Josef Schäfer} [23.11.1811--21.01.1883 (12 Kinder)]\\
\textsc{mutter}: \hyperref[@I949@]{Maria Anna Farrenkopf} [29.08.1819--01.06.1905 (8 Kinder)]
\medbreak
\textsc{{geschwister}}
\begin{itemize}
\item \hyperref[@I1866@]{Johann Valentin Schäfer} [19.04.1835]
\item \hyperref[@I1867@]{Rosina Schäfer} [30.09.1837]
\item \hyperref[@I1871@]{Anna Christina Schäfer} [28.07.1839]
\item \hyperref[@I1870@]{Margaretha Schäfer} [28.04.1842]
\item \hyperref[@I1396@]{Karl Schäfer} [14.05.1844--02.02.1933]
\item \hyperref[@I1397@]{Linus Schäfer} [23.09.1846--nach 1885]
\item \hyperref[@I1398@]{Eva Clara Schäfer} [10.02.1850--08.07.1881]
\item \hyperref[@I388@]{Eva Katharina Schäfer} [06.10.1855--21.04.1942 (4 Kinder)]
\item \hyperref[@I1400@]{Franz Schäfer} [27.03.1858--06.10.1950]
\item \hyperref[@I1401@]{Rosina Schäfer} [14.09.1860]
\item \hyperref[@I1402@]{Ferdinand Schäfer} [28.06.1864]
\end{itemize}
\bigbreak
\textsc{{quellen}}
\begin{enumerate}[label={[\arabic*]}]
\item \href{https://www.familysearch.org/tree/person/details/L18V-2LW}{FamilySearch, ID: L18V-2LW}
\end{enumerate}

\end{person}

\begin{person}[
    surname = {Schäfer},
    givenname = {Franz},
    suffix = {1858--1950},
    label = {@I1400@}
    ]

\begin{tabular}{cl}
\geboren & 27. März 1858 in Schlossau\\
\gestorben & 06. Oktober 1950 in Schlossau\\
\end{tabular}\\
\medbreak
\textsc{vater}: \hyperref[@I948@]{Johann Josef Schäfer} [23.11.1811--21.01.1883 (12 Kinder)]\\
\textsc{mutter}: \hyperref[@I949@]{Maria Anna Farrenkopf} [29.08.1819--01.06.1905 (8 Kinder)]
\medbreak
\textsc{{geschwister}}
\begin{itemize}
\item \hyperref[@I1866@]{Johann Valentin Schäfer} [19.04.1835]
\item \hyperref[@I1867@]{Rosina Schäfer} [30.09.1837]
\item \hyperref[@I1871@]{Anna Christina Schäfer} [28.07.1839]
\item \hyperref[@I1870@]{Margaretha Schäfer} [28.04.1842]
\item \hyperref[@I1396@]{Karl Schäfer} [14.05.1844--02.02.1933]
\item \hyperref[@I1397@]{Linus Schäfer} [23.09.1846--nach 1885]
\item \hyperref[@I1398@]{Eva Clara Schäfer} [10.02.1850--08.07.1881]
\item \hyperref[@I1399@]{Maria Anna Schäfer} [05.10.1852]
\item \hyperref[@I388@]{Eva Katharina Schäfer} [06.10.1855--21.04.1942 (4 Kinder)]
\item \hyperref[@I1401@]{Rosina Schäfer} [14.09.1860]
\item \hyperref[@I1402@]{Ferdinand Schäfer} [28.06.1864]
\end{itemize}
\bigbreak
\textsc{anmerkung}\\
Zeigt Tod von Anna Maria Farrenkopf an
\medbreak
\textsc{{quellen}}
\begin{enumerate}[label={[\arabic*]}]
\item \href{https://www.familysearch.org/tree/person/details/LT2X-YHQ}{FamilySearch, ID: LT2X-YHQ}
\end{enumerate}

\end{person}

\begin{person}[
    surname = {Schäfer},
    givenname = {Rosina},
    suffix = {1860},
    label = {@I1401@}
    ]

\begin{tabular}{cl}
\geboren & 14. September 1860 in Schlossau\\
\geheiratet & 22. September 1891 in Steinbach mit Franz Joseph Mechler \\
\end{tabular}\\
\medbreak
\textsc{vater}: \hyperref[@I948@]{Johann Josef Schäfer} [23.11.1811--21.01.1883 (12 Kinder)]\\
\textsc{mutter}: \hyperref[@I949@]{Maria Anna Farrenkopf} [29.08.1819--01.06.1905 (8 Kinder)]
\medbreak
\textsc{{geschwister}}
\begin{itemize}
\item \hyperref[@I1866@]{Johann Valentin Schäfer} [19.04.1835]
\item \hyperref[@I1867@]{Rosina Schäfer} [30.09.1837]
\item \hyperref[@I1871@]{Anna Christina Schäfer} [28.07.1839]
\item \hyperref[@I1870@]{Margaretha Schäfer} [28.04.1842]
\item \hyperref[@I1396@]{Karl Schäfer} [14.05.1844--02.02.1933]
\item \hyperref[@I1397@]{Linus Schäfer} [23.09.1846--nach 1885]
\item \hyperref[@I1398@]{Eva Clara Schäfer} [10.02.1850--08.07.1881]
\item \hyperref[@I1399@]{Maria Anna Schäfer} [05.10.1852]
\item \hyperref[@I388@]{Eva Katharina Schäfer} [06.10.1855--21.04.1942 (4 Kinder)]
\item \hyperref[@I1400@]{Franz Schäfer} [27.03.1858--06.10.1950]
\item \hyperref[@I1402@]{Ferdinand Schäfer} [28.06.1864]
\end{itemize}
\bigbreak
\textsc{{quellen}}
\begin{enumerate}[label={[\arabic*]}]
\item \href{http://www.landesarchiv-bw.de/plink/?f=4-1119607-82}{GLA Karlsruhe, Schlossau, katholische Gemeinde: Standesbuch 1851–1866, Geburtenregister 1860, Nr. 10 (Bild 82)}
\item Steinbach Geburts-, Heirats- und Sterberegister 1890–1899, Heiratsregister 1891, Nr. 3
\item \href{https://www.familysearch.org/tree/person/details/L18V-5QX}{FamilySearch, ID: L18V-5QX}
\end{enumerate}

\end{person}

\begin{person}[
    surname = {Schäfer},
    givenname = {Ferdinand},
    suffix = {1864},
    label = {@I1402@}
    ]

\begin{tabular}{cl}
\geboren & 28. Juni 1864 in Schlossau\\
\end{tabular}\\
\medbreak
\textsc{vater}: \hyperref[@I948@]{Johann Josef Schäfer} [23.11.1811--21.01.1883 (12 Kinder)]\\
\textsc{mutter}: \hyperref[@I949@]{Maria Anna Farrenkopf} [29.08.1819--01.06.1905 (8 Kinder)]
\medbreak
\textsc{{geschwister}}
\begin{itemize}
\item \hyperref[@I1866@]{Johann Valentin Schäfer} [19.04.1835]
\item \hyperref[@I1867@]{Rosina Schäfer} [30.09.1837]
\item \hyperref[@I1871@]{Anna Christina Schäfer} [28.07.1839]
\item \hyperref[@I1870@]{Margaretha Schäfer} [28.04.1842]
\item \hyperref[@I1396@]{Karl Schäfer} [14.05.1844--02.02.1933]
\item \hyperref[@I1397@]{Linus Schäfer} [23.09.1846--nach 1885]
\item \hyperref[@I1398@]{Eva Clara Schäfer} [10.02.1850--08.07.1881]
\item \hyperref[@I1399@]{Maria Anna Schäfer} [05.10.1852]
\item \hyperref[@I388@]{Eva Katharina Schäfer} [06.10.1855--21.04.1942 (4 Kinder)]
\item \hyperref[@I1400@]{Franz Schäfer} [27.03.1858--06.10.1950]
\item \hyperref[@I1401@]{Rosina Schäfer} [14.09.1860]
\end{itemize}
\bigbreak
\textsc{{quellen}}
\begin{enumerate}[label={[\arabic*]}]
\item \href{https://www.familysearch.org/tree/person/details/LT2X-5ML}{FamilySearch, ID: LT2X-5ML}
\end{enumerate}

\end{person}


\addsec{Johann Josef Galm  \& Anna Maria Münch }


\begin{person}[
    surname = {Galm},
    givenname = {Johann Josef},
    suffix = {1822--1887},
    label = {@I146@}
    ]

\begin{tabular}{cl}
\geboren & 04. Juli 1822 in Galmbach\\
\taufe & 04. Juli 1822\\
\geheiratet & 25. November 1847 in Mudau mit Anna Maria Münch \\
\gestorben & 23. April 1887 in Langenelz\\
\end{tabular}\\
\medbreak
\textsc{vater}: Johann Joseph Galm [17.03.1789--16.03.1857 (10 Kinder)]\\
\textsc{mutter}: \hyperref[@I185@]{Eva Rosina Link} [18.05.1790--19.04.1851 (10 Kinder)]
\medbreak
\textsc{{geschwister}}
\begin{itemize}
\item Johann Balthasar Galm [06.01.1812]
\item Johann Philipp Galm [01.05.1813]
\item Margaretha Galm [06.02.1815--19.01.1837]
\item Johann Valentin Galm [1818--um 1873]
\item Eva Katharina Galm [05.04.1819--16.02.1851]
\item Eva Rosina Galm [21.11.1824]
\item Josepha Galm [1827]
\item Maria Karolina Galm [25.02.1830--12.06.1897]
\item Franz Karl Galm [30.07.1832 (7 Kinder)]
\end{itemize}
\bigbreak
\textsc{{kinder}}
\begin{itemize}
\item \hyperref[@I183@]{Maria Karolina Galm} [21.01.1850--31.07.1908]
\item \hyperref[@I197@]{Rosalia Galm} [12.08.1852--21.08.1854]
\item \hyperref[@I144@]{Franz Karl Galm} [05.09.1854--06.12.1934 (10 Kinder)]
\item \hyperref[@I180@]{Julius Galm} [21.02.1857--03.01.1929 (4 Kinder)]
\item \hyperref[@I181@]{Johann Josef Galm} [23.04.1859--10.05.1910 (1 Kind)]
\item \hyperref[@I198@]{Anna Maria Galm} [20.07.1862--02.12.1942 (2 Kinder)]
\item \hyperref[@I182@]{Wilhelm Galm} [02.02.1865--15.08.1943 (2 Kinder)]
\end{itemize}
\medbreak
\textsc{{quellen}}
\begin{enumerate}[label={[\arabic*]}]
\item \href{http://www.landesarchiv-bw.de/plink/?f=4-1119445-23}{GLA Karlsruhe, Mudau, katholische Gemeinde: Standesbuch 1821–1826, Geburtenregister 1822, Nr. 67 (Bild 23)}
\item \href{http://www.landesarchiv-bw.de/plink/?f=4-1119438-56}{GLA Karlsruhe, Langenelz, katholische Gemeinde: Standesbuch 1839–1870, Heiratsregister 1847, Nr. 2 (Bild 56)}
\item Ahnentafel Alfons Bauer
\item \href{https://www.familysearch.org/tree/person/details/LVN5-MSZ}{FamilySearch, ID: LVN5-MSZ}
\item \href{http://gedbas.genealogy.net/person/show/1172977598}{genealogy.net}
\end{enumerate}

\end{person}

\begin{person}[
    surname = {Münch},
    givenname = {Anna Maria},
    suffix = {1821--1891},
    label = {@I147@}
    ]

\begin{tabular}{cl}
\geboren & 29. September 1821 in Reisenbach\\
\taufe & 30. September 1821 in Mudau\\
\geheiratet & 25. November 1847 in Mudau mit Johann Josef Galm \\
\gestorben & 03. Juli 1891 in Langenelz\\
\end{tabular}\\
\medbreak
\textsc{vater}: Peter Johann Münch [03.05.1793--05.12.1844 (8 Kinder)]\\
\textsc{mutter}: \hyperref[@I200@]{Agnes Fertig} [29.01.1803--02.11.1872 (8 Kinder)]
\medbreak
\textsc{{geschwister}}
\begin{itemize}
\item Johann Valentin Münch [14.08.1824--09.12.1894]
\item Franz Josef Münch [25.07.1827]
\item Michael Münch [31.01.1831]
\item Johann Peter Münch [16.07.1833]
\item Maria Anna Münch [24.10.1835--21.02.1837]
\item Paulina Münch [03.07.1838 (7 Kinder)]
\item Carl Münch [27.12.1842 (2 Kinder)]
\end{itemize}
\bigbreak
\textsc{{kinder}}
\begin{itemize}
\item \hyperref[@I183@]{Maria Karolina Galm} [21.01.1850--31.07.1908]
\item \hyperref[@I197@]{Rosalia Galm} [12.08.1852--21.08.1854]
\item \hyperref[@I144@]{Franz Karl Galm} [05.09.1854--06.12.1934 (10 Kinder)]
\item \hyperref[@I180@]{Julius Galm} [21.02.1857--03.01.1929 (4 Kinder)]
\item \hyperref[@I181@]{Johann Josef Galm} [23.04.1859--10.05.1910 (1 Kind)]
\item \hyperref[@I198@]{Anna Maria Galm} [20.07.1862--02.12.1942 (2 Kinder)]
\item \hyperref[@I182@]{Wilhelm Galm} [02.02.1865--15.08.1943 (2 Kinder)]
\end{itemize}
\medbreak
\textsc{{quellen}}
\begin{enumerate}[label={[\arabic*]}]
\item \href{http://www.landesarchiv-bw.de/plink/?f=4-1119445-8}{GLA Karlsruhe, Mudau, katholische Gemeinde: Standesbuch 1821–1826, Geburtenregister 1821, Nr. 100 (Bild 8)}
\item \href{http://www.landesarchiv-bw.de/plink/?f=4-1119438-56}{GLA Karlsruhe, Langenelz, katholische Gemeinde: Standesbuch 1839–1870, Heiratsregister 1847, Nr. 2 (Bild 56)}
\item Ahnentafel Alfons Bauer
\item Ahnentafel Helga Schölch
\item \href{https://www.familysearch.org/tree/person/details/LVN5-MJW}{FamilySearch, ID: LVN5-MJW}
\item \href{http://gedbas.genealogy.net/person/show/1172977599}{genealogy.net}
\end{enumerate}

\end{person}

\begin{person}[
    surname = {Galm},
    givenname = {Maria Karolina},
    suffix = {1850--1908},
    label = {@I183@}
    ]

\begin{tabular}{cl}
\geboren & 21. Januar 1850 in Langenelz\\
\taufe & 21. Januar 1850 in Mudau\\
\geheiratet & 26. November 1874 in Waldhausen mit Johann Valentin Hemberger \\
\gestorben & 31. Juli 1908 in Heidersbach\\
\end{tabular}\\
\medbreak
\textsc{vater}: \hyperref[@I146@]{Johann Josef Galm} [04.07.1822--23.04.1887 (7 Kinder)]\\
\textsc{mutter}: \hyperref[@I147@]{Anna Maria Münch} [29.09.1821--03.07.1891 (7 Kinder)]
\medbreak
\textsc{{geschwister}}
\begin{itemize}
\item \hyperref[@I197@]{Rosalia Galm} [12.08.1852--21.08.1854]
\item \hyperref[@I144@]{Franz Karl Galm} [05.09.1854--06.12.1934 (10 Kinder)]
\item \hyperref[@I180@]{Julius Galm} [21.02.1857--03.01.1929 (4 Kinder)]
\item \hyperref[@I181@]{Johann Josef Galm} [23.04.1859--10.05.1910 (1 Kind)]
\item \hyperref[@I198@]{Anna Maria Galm} [20.07.1862--02.12.1942 (2 Kinder)]
\item \hyperref[@I182@]{Wilhelm Galm} [02.02.1865--15.08.1943 (2 Kinder)]
\end{itemize}
\bigbreak
\textsc{anmerkung}\\
Im Jahre 1874 heiratete sie nach Heidersbach
\medbreak
\textsc{{quellen}}
\begin{enumerate}[label={[\arabic*]}]
\item Heimatbuch Philipp Galm, Seite 63
\item \href{https://www.familysearch.org/tree/person/details/LVPW-YQS}{FamilySearch, ID: LVPW-YQS}
\end{enumerate}

\end{person}

\begin{person}[
    surname = {Galm},
    givenname = {Rosalia},
    suffix = {1852--1854},
    label = {@I197@}
    ]

\begin{tabular}{cl}
\geboren & 12. August 1852 in Langenelz\\
\taufe & 12. August 1852 in Mudau\\
\gestorben & 21. August 1854 in Langenelz\\
\bestattet & 23. August 1854 in Mudau\\
\end{tabular}\\
\medbreak
\textsc{vater}: \hyperref[@I146@]{Johann Josef Galm} [04.07.1822--23.04.1887 (7 Kinder)]\\
\textsc{mutter}: \hyperref[@I147@]{Anna Maria Münch} [29.09.1821--03.07.1891 (7 Kinder)]
\medbreak
\textsc{{geschwister}}
\begin{itemize}
\item \hyperref[@I183@]{Maria Karolina Galm} [21.01.1850--31.07.1908]
\item \hyperref[@I144@]{Franz Karl Galm} [05.09.1854--06.12.1934 (10 Kinder)]
\item \hyperref[@I180@]{Julius Galm} [21.02.1857--03.01.1929 (4 Kinder)]
\item \hyperref[@I181@]{Johann Josef Galm} [23.04.1859--10.05.1910 (1 Kind)]
\item \hyperref[@I198@]{Anna Maria Galm} [20.07.1862--02.12.1942 (2 Kinder)]
\item \hyperref[@I182@]{Wilhelm Galm} [02.02.1865--15.08.1943 (2 Kinder)]
\end{itemize}
\bigbreak
\textsc{{quellen}}
\begin{enumerate}[label={[\arabic*]}]
\item \href{https://www.familysearch.org/tree/person/details/LVPW-YWG}{FamilySearch, ID: LVPW-YWG}
\end{enumerate}

\end{person}

\begin{person}[
    surname = {Galm},
    givenname = {Julius},
    suffix = {1857--1929},
    label = {@I180@}
    ]

\begin{tabular}{cl}
\geboren & 21. Februar 1857 in Langenelz\\
\taufe & 21. Februar 1857 in Mudau\\
\geheiratet & 21. August 1888 in Langenelz mit Rosina Schwanninger \\
\gestorben & 03. Januar 1929 in Unterscheidental\\
\bestattet &  in Oberscheidental\\
\end{tabular}\\
\medbreak
\textsc{vater}: \hyperref[@I146@]{Johann Josef Galm} [04.07.1822--23.04.1887 (7 Kinder)]\\
\textsc{mutter}: \hyperref[@I147@]{Anna Maria Münch} [29.09.1821--03.07.1891 (7 Kinder)]
\medbreak
\textsc{{geschwister}}
\begin{itemize}
\item \hyperref[@I183@]{Maria Karolina Galm} [21.01.1850--31.07.1908]
\item \hyperref[@I197@]{Rosalia Galm} [12.08.1852--21.08.1854]
\item \hyperref[@I144@]{Franz Karl Galm} [05.09.1854--06.12.1934 (10 Kinder)]
\item \hyperref[@I181@]{Johann Josef Galm} [23.04.1859--10.05.1910 (1 Kind)]
\item \hyperref[@I198@]{Anna Maria Galm} [20.07.1862--02.12.1942 (2 Kinder)]
\item \hyperref[@I182@]{Wilhelm Galm} [02.02.1865--15.08.1943 (2 Kinder)]
\end{itemize}
\bigbreak
\textsc{{kinder}}
\begin{itemize}
\item Karl Galm [24.05.1889--09.06.1905]
\item Julius Galm [20.08.1890--nach 1932]
\item Albert Galm [12.05.1892]
\item Laura Galm [um 1898--15.05.1905]
\end{itemize}
\medbreak
\textsc{anmerkung}\\
Heiratetete nach Unterscheidental
Tod angezeigt von Sohn Julius Galm
\medbreak
\textsc{{quellen}}
\begin{enumerate}[label={[\arabic*]}]
\item Langenelz Standesbuch 1885–1890, Heiratsregister 1888, Nr. 6
\item Unterscheidental Sterberegister 1870–1935, Sterberegister 1929, Nr. 1
\item Heimatbuch Philipp Galm, Seite 62
\item \href{https://www.familysearch.org/tree/person/details/LVPW-YZJ}{FamilySearch, ID: LVPW-YZJ}
\end{enumerate}

\end{person}

\begin{person}[
    surname = {Galm},
    givenname = {Johann Josef},
    suffix = {1859--1910},
    label = {@I181@}
    ]

\begin{tabular}{cl}
\geboren & 23. April 1859 in Langenelz\\
\taufe & 23. April 1859 in Mudau\\
\geheiratet & 07. Januar 1886 in Oberscheidental mit Karolina Brenneis \\
\gestorben & 10. Mai 1910 in Oberscheidental\\
\end{tabular}\\
\medbreak
\textsc{vater}: \hyperref[@I146@]{Johann Josef Galm} [04.07.1822--23.04.1887 (7 Kinder)]\\
\textsc{mutter}: \hyperref[@I147@]{Anna Maria Münch} [29.09.1821--03.07.1891 (7 Kinder)]
\medbreak
\textsc{{geschwister}}
\begin{itemize}
\item \hyperref[@I183@]{Maria Karolina Galm} [21.01.1850--31.07.1908]
\item \hyperref[@I197@]{Rosalia Galm} [12.08.1852--21.08.1854]
\item \hyperref[@I144@]{Franz Karl Galm} [05.09.1854--06.12.1934 (10 Kinder)]
\item \hyperref[@I180@]{Julius Galm} [21.02.1857--03.01.1929 (4 Kinder)]
\item \hyperref[@I198@]{Anna Maria Galm} [20.07.1862--02.12.1942 (2 Kinder)]
\item \hyperref[@I182@]{Wilhelm Galm} [02.02.1865--15.08.1943 (2 Kinder)]
\end{itemize}
\bigbreak
\textsc{{kinder}}
\begin{itemize}
\item Josef Galm [1886]
\end{itemize}
\medbreak
\textsc{anmerkung}\\
heiratete nach Oberscheidental.
Starb vor dem Weltkrieg
\medbreak
\textsc{{quellen}}
\begin{enumerate}[label={[\arabic*]}]
\item Oberscheidental Heiratsregister 1870–1935 und Scheidental 1936–1938, Heiratsregister 1886, Nr. 1
\item Oberscheidental Sterberegister 1870–1935 und Scheidental 1936–1938, Sterberegister 1910, Nr. 1
\item Heimatbuch Philipp Galm, Seite 62
\item \href{https://www.familysearch.org/tree/person/details/LVPW-YDQ}{FamilySearch, ID: LVPW-YDQ}
\end{enumerate}

\end{person}

\begin{person}[
    surname = {Galm},
    givenname = {Anna Maria},
    suffix = {1862--1942},
    label = {@I198@}
    ]

\begin{tabular}{cl}
\geboren & 20. Juli 1862 in Langenelz\\
\geheiratet & 10. Februar 1887 in Unterscheidental mit Karl Brenneis \\
\gestorben & 02. Dezember 1942 in Scheidental\\
\end{tabular}\\
\medbreak
\textsc{vater}: \hyperref[@I146@]{Johann Josef Galm} [04.07.1822--23.04.1887 (7 Kinder)]\\
\textsc{mutter}: \hyperref[@I147@]{Anna Maria Münch} [29.09.1821--03.07.1891 (7 Kinder)]
\medbreak
\textsc{{geschwister}}
\begin{itemize}
\item \hyperref[@I183@]{Maria Karolina Galm} [21.01.1850--31.07.1908]
\item \hyperref[@I197@]{Rosalia Galm} [12.08.1852--21.08.1854]
\item \hyperref[@I144@]{Franz Karl Galm} [05.09.1854--06.12.1934 (10 Kinder)]
\item \hyperref[@I180@]{Julius Galm} [21.02.1857--03.01.1929 (4 Kinder)]
\item \hyperref[@I181@]{Johann Josef Galm} [23.04.1859--10.05.1910 (1 Kind)]
\item \hyperref[@I182@]{Wilhelm Galm} [02.02.1865--15.08.1943 (2 Kinder)]
\end{itemize}
\bigbreak
\textsc{{kinder}}
\begin{itemize}
\item Robert Brenneis [30.06.1891--22.01.1965 (6 Kinder)]
\item Maria Brenneis [30.05.1907--30.05.1907]
\end{itemize}
\medbreak
\textsc{{quellen}}
\begin{enumerate}[label={[\arabic*]}]
\item Unterscheidental Heiratsregister 1870–1935, Heiratsregister 1887, Nr. 3
\item Scheidental Sterbebuch 1939–1974, Sterberegister 1942, Nr. 5
\item \href{https://www.familysearch.org/tree/person/details/LVPW-Y6T}{FamilySearch, ID: LVPW-Y6T}
\end{enumerate}

\end{person}

\begin{person}[
    surname = {Galm},
    givenname = {Wilhelm},
    suffix = {1865--1943},
    label = {@I182@}
    ]

\begin{tabular}{cl}
\geboren & 02. Februar 1865 in Langenelz\\
\geheiratet & 09. Juli 1896 in Mudau mit Bertha Pfaff \\
\gestorben & 15. August 1943 in Langenelz\\
\end{tabular}\\
\medbreak
\textsc{vater}: \hyperref[@I146@]{Johann Josef Galm} [04.07.1822--23.04.1887 (7 Kinder)]\\
\textsc{mutter}: \hyperref[@I147@]{Anna Maria Münch} [29.09.1821--03.07.1891 (7 Kinder)]
\medbreak
\textsc{{geschwister}}
\begin{itemize}
\item \hyperref[@I183@]{Maria Karolina Galm} [21.01.1850--31.07.1908]
\item \hyperref[@I197@]{Rosalia Galm} [12.08.1852--21.08.1854]
\item \hyperref[@I144@]{Franz Karl Galm} [05.09.1854--06.12.1934 (10 Kinder)]
\item \hyperref[@I180@]{Julius Galm} [21.02.1857--03.01.1929 (4 Kinder)]
\item \hyperref[@I181@]{Johann Josef Galm} [23.04.1859--10.05.1910 (1 Kind)]
\item \hyperref[@I198@]{Anna Maria Galm} [20.07.1862--02.12.1942 (2 Kinder)]
\end{itemize}
\bigbreak
\textsc{{kinder}}
\begin{itemize}
\item Ida Galm [25.09.1898]
\item Wilhelm Galm [15.04.1900--22.11.1985 (1 Kind)]
\end{itemize}
\medbreak
\textsc{anmerkung}\\
nach der Schulzeit diente er einige Jahre als Knecht in Steinbach.
Unseren heutigen (1930er) Nußbaum hatte er gepflanzt.
Im Jahr 1897 heiratete er nach Unterlangenelz
\medbreak
\textsc{{quellen}}
\begin{enumerate}[label={[\arabic*]}]
\item Langenelz Standesbuch 1891–1899, Heiratsregister 1896, Nr. 1 
\item Mudau Sterbebuch 1938–1944, Sterberegister 1943, Nr. 28
\item Heimatbuch Philipp Galm, Seite 63
\end{enumerate}

\end{person}


\addsec{Franz, Josef Schwanninger  \& Margaretha Rögner }


\begin{person}[
    surname = {Schwanninger},
    givenname = {Franz, Josef},
    suffix = {1823--1905},
    label = {@I148@}
    ]

\begin{tabular}{cl}
\geboren & 11. Mai 1823 in Mörschenhardt\\
\taufe & 11. Mai 1823\\
\geheiratet & 20. Mai 1856 mit Margaretha Rögner \\
\gestorben & 06. April 1905 in Mörschenhardt\\
\end{tabular}\\
\medbreak
\textsc{vater}: Franz Josef Schwanninger [05.07.1793--13.04.1837 (5 Kinder)]\\
\textsc{mutter}: \hyperref[@I323@]{Maria Anna Glock} [30.04.1789--17.03.1855 (5 Kinder)]
\medbreak
\textsc{{geschwister}}
\begin{itemize}
\item Maria Anna Schwanninger [29.11.1824]
\item Veronika Schwanninger [02.06.1826]
\item Margarethe Schwanninger [15.05.1828--15.09.1891]
\item Anna Maria Schwanninger [05.02.1830]
\end{itemize}
\bigbreak
\textsc{{kinder}}
\begin{itemize}
\item \hyperref[@I1302@]{Wilhelm Schwanninger} [18.02.1857--nach 1904 (9 Kinder)]
\item \hyperref[@I145@]{Karolina Schwanninger} [14.09.1858--02.07.1926 (10 Kinder)]
\item \hyperref[@I1172@]{Margaretha Schwanninger} [31.07.1860]
\item \hyperref[@I1303@]{Rosina Schwanninger} [09.11.1861--10.07.1932 (4 Kinder)]
\item \hyperref[@I1304@]{Friedrich Schwanninger} [04.04.1863--04.06.1867]
\item \hyperref[@I1305@]{Josefa Schwanninger} [16.07.1865]
\item \hyperref[@I1873@]{Katharina Schwanninger} [25.11.1873]
\item \hyperref[@I2108@]{Ida Schwanninger} [...]
\end{itemize}
\medbreak
\textsc{{quellen}}
\begin{enumerate}[label={[\arabic*]}]
\item Ahnentafel Alfons Bauer
\item Ahnentafel Helga Schölch
\item \href{https://www.familysearch.org/tree/person/details/LR9J-FMS}{FamilySearch, ID: LR9J-FMS}
\end{enumerate}

\end{person}

\begin{person}[
    surname = {Rögner},
    givenname = {Margaretha},
    suffix = {1834--1906},
    label = {@I149@}
    ]

\begin{tabular}{cl}
\geboren & 04. April 1834 in Mörschenhardt\\
\taufe & 04. April 1834 in Mudau\\
\geheiratet & 20. Mai 1856 mit Franz, Josef Schwanninger \\
\gestorben & 20. August 1906 in Mörschenhardt\\
\end{tabular}\\
\medbreak
\textsc{vater}: Franz Josef Rögner [03.02.1807--01.12.1841 (5 Kinder)]\\
\textsc{mutter}: \hyperref[@I325@]{Anna Margaretha Link} [09.09.1802--06.04.1862 (5 Kinder)]
\medbreak
\textsc{{geschwister}}
\begin{itemize}
\item Anna Maria Rögner [06.05.1832]
\item Maria Anna Rögner [02.07.1836]
\item Franz Josef Rögner [07.11.1838--01.05.1839]
\item Friedrich Rögner [25.03.1840]
\end{itemize}
\bigbreak
\textsc{{kinder}}
\begin{itemize}
\item \hyperref[@I1302@]{Wilhelm Schwanninger} [18.02.1857--nach 1904 (9 Kinder)]
\item \hyperref[@I145@]{Karolina Schwanninger} [14.09.1858--02.07.1926 (10 Kinder)]
\item \hyperref[@I1172@]{Margaretha Schwanninger} [31.07.1860]
\item \hyperref[@I1303@]{Rosina Schwanninger} [09.11.1861--10.07.1932 (4 Kinder)]
\item \hyperref[@I1304@]{Friedrich Schwanninger} [04.04.1863--04.06.1867]
\item \hyperref[@I1305@]{Josefa Schwanninger} [16.07.1865]
\item \hyperref[@I1873@]{Katharina Schwanninger} [25.11.1873]
\item \hyperref[@I2108@]{Ida Schwanninger} [...]
\end{itemize}
\medbreak
\textsc{{quellen}}
\begin{enumerate}[label={[\arabic*]}]
\item Ahnentafel Alfons Bauer
\item Ahnentafel Helga Schölch
\item \href{https://www.familysearch.org/tree/person/details/LTC6-N21}{FamilySearch, ID: LTC6-N21}
\end{enumerate}

\end{person}

\begin{person}[
    surname = {Schwanninger},
    givenname = {Wilhelm},
    suffix = {1857--nach 1904},
    label = {@I1302@}
    ]

\begin{tabular}{cl}
\geboren & 18. Februar 1857 in Mörschenhardt\\
\geheiratet & 10. Juli 1888 in Mudau mit Karoline Repp \\
\gestorben & nach 1904\\
\end{tabular}\\
\medbreak
\textsc{vater}: \hyperref[@I148@]{Franz, Josef Schwanninger} [11.05.1823--06.04.1905 (8 Kinder)]\\
\textsc{mutter}: \hyperref[@I149@]{Margaretha Rögner} [04.04.1834--20.08.1906 (8 Kinder)]
\medbreak
\textsc{{geschwister}}
\begin{itemize}
\item \hyperref[@I145@]{Karolina Schwanninger} [14.09.1858--02.07.1926 (10 Kinder)]
\item \hyperref[@I1172@]{Margaretha Schwanninger} [31.07.1860]
\item \hyperref[@I1303@]{Rosina Schwanninger} [09.11.1861--10.07.1932 (4 Kinder)]
\item \hyperref[@I1304@]{Friedrich Schwanninger} [04.04.1863--04.06.1867]
\item \hyperref[@I1305@]{Josefa Schwanninger} [16.07.1865]
\item \hyperref[@I1873@]{Katharina Schwanninger} [25.11.1873]
\item \hyperref[@I2108@]{Ida Schwanninger} [...]
\end{itemize}
\bigbreak
\textsc{{kinder}}
\begin{itemize}
\item Otto Schwanninger [22.04.1889--18.06.1917]
\item Josef Schwanninger [07.09.1890--05.07.1891]
\item Maria Schwanninger [01.07.1892--um 1963]
\item Alfons Schwanninger [15.09.1894]
\item Anna Schwanninger [30.10.1896--26.10.1968]
\item Rosa Schwanninger [27.05.1898--17.02.1988 (6 Kinder)]
\item Anton Schwanninger [13.06.1900--08.02.1901]
\item Frida Schwanninger [08.03.1902--20.04.1987]
\item Alois Schwanninger [14.11.1903--22.11.1903]
\end{itemize}
\medbreak
\textsc{{quellen}}
\begin{enumerate}[label={[\arabic*]}]
\item \href{https://www.familysearch.org/tree/person/details/LTC6-NLF}{FamilySearch, ID: LTC6-NLF}
\end{enumerate}

\end{person}

\begin{person}[
    surname = {Schwanninger},
    givenname = {Margaretha},
    suffix = {1860},
    label = {@I1172@}
    ]

\begin{tabular}{cl}
\geboren & 31. Juli 1860 in Mörschenhardt\\
\geheiratet & 09. Mai 1878 in Rumpfen mit Karl Ludwig Schwing \\
\end{tabular}\\
\medbreak
\textsc{vater}: \hyperref[@I148@]{Franz, Josef Schwanninger} [11.05.1823--06.04.1905 (8 Kinder)]\\
\textsc{mutter}: \hyperref[@I149@]{Margaretha Rögner} [04.04.1834--20.08.1906 (8 Kinder)]
\medbreak
\textsc{{geschwister}}
\begin{itemize}
\item \hyperref[@I1302@]{Wilhelm Schwanninger} [18.02.1857--nach 1904 (9 Kinder)]
\item \hyperref[@I145@]{Karolina Schwanninger} [14.09.1858--02.07.1926 (10 Kinder)]
\item \hyperref[@I1303@]{Rosina Schwanninger} [09.11.1861--10.07.1932 (4 Kinder)]
\item \hyperref[@I1304@]{Friedrich Schwanninger} [04.04.1863--04.06.1867]
\item \hyperref[@I1305@]{Josefa Schwanninger} [16.07.1865]
\item \hyperref[@I1873@]{Katharina Schwanninger} [25.11.1873]
\item \hyperref[@I2108@]{Ida Schwanninger} [...]
\end{itemize}
\bigbreak
\textsc{{quellen}}
\begin{enumerate}[label={[\arabic*]}]
\item \href{http://www.landesarchiv-bw.de/plink/?f=4-1119442-119}{GLA Karlsruhe, Mörschenhardt, katholische Gemeinde: Standesbuch 1839–1870, Geburtenregister 1860, Nr. 8 (Bild 119)}
\item Rumpfen Geburts-, Heirats- und Sterberegister 1870–1879, Heiratsregister 1878, Nr. 1
\item \href{https://www.familysearch.org/tree/person/details/L128-HHM}{FamilySearch, ID: L128-HHM}
\end{enumerate}

\end{person}

\begin{person}[
    surname = {Schwanninger},
    givenname = {Rosina},
    suffix = {1861--1932},
    label = {@I1303@}
    ]

\begin{tabular}{cl}
\geboren & 09. November 1861 in Mörschenhardt\\
\geheiratet & 21. August 1888 in Langenelz mit Julius Galm \\
\gestorben & 10. Juli 1932 in Unterscheidental\\
\end{tabular}\\
\medbreak
\textsc{vater}: \hyperref[@I148@]{Franz, Josef Schwanninger} [11.05.1823--06.04.1905 (8 Kinder)]\\
\textsc{mutter}: \hyperref[@I149@]{Margaretha Rögner} [04.04.1834--20.08.1906 (8 Kinder)]
\medbreak
\textsc{{geschwister}}
\begin{itemize}
\item \hyperref[@I1302@]{Wilhelm Schwanninger} [18.02.1857--nach 1904 (9 Kinder)]
\item \hyperref[@I145@]{Karolina Schwanninger} [14.09.1858--02.07.1926 (10 Kinder)]
\item \hyperref[@I1172@]{Margaretha Schwanninger} [31.07.1860]
\item \hyperref[@I1304@]{Friedrich Schwanninger} [04.04.1863--04.06.1867]
\item \hyperref[@I1305@]{Josefa Schwanninger} [16.07.1865]
\item \hyperref[@I1873@]{Katharina Schwanninger} [25.11.1873]
\item \hyperref[@I2108@]{Ida Schwanninger} [...]
\end{itemize}
\bigbreak
\textsc{{kinder}}
\begin{itemize}
\item Karl Galm [24.05.1889--09.06.1905]
\item Julius Galm [20.08.1890--nach 1932]
\item Albert Galm [12.05.1892]
\item Laura Galm [um 1898--15.05.1905]
\end{itemize}
\medbreak
\textsc{{quellen}}
\begin{enumerate}[label={[\arabic*]}]
\item Langenelz Standesbuch 1885–1890, Heiratsregister 1888, Nr. 6
\item Unterscheidental Sterberegister 1870–1935, Sterberegister 1932, Nr. 3
\item \href{https://www.familysearch.org/tree/person/details/LTC6-NL2}{FamilySearch, ID: LTC6-NL2}
\end{enumerate}

\end{person}

\begin{person}[
    surname = {Schwanninger},
    givenname = {Friedrich},
    suffix = {1863--1867},
    label = {@I1304@}
    ]

\begin{tabular}{cl}
\geboren & 04. April 1863 in Mörschenhardt\\
\gestorben & 04. Juni 1867 in Mörschenhardt\\
\end{tabular}\\
\medbreak
\textsc{vater}: \hyperref[@I148@]{Franz, Josef Schwanninger} [11.05.1823--06.04.1905 (8 Kinder)]\\
\textsc{mutter}: \hyperref[@I149@]{Margaretha Rögner} [04.04.1834--20.08.1906 (8 Kinder)]
\medbreak
\textsc{{geschwister}}
\begin{itemize}
\item \hyperref[@I1302@]{Wilhelm Schwanninger} [18.02.1857--nach 1904 (9 Kinder)]
\item \hyperref[@I145@]{Karolina Schwanninger} [14.09.1858--02.07.1926 (10 Kinder)]
\item \hyperref[@I1172@]{Margaretha Schwanninger} [31.07.1860]
\item \hyperref[@I1303@]{Rosina Schwanninger} [09.11.1861--10.07.1932 (4 Kinder)]
\item \hyperref[@I1305@]{Josefa Schwanninger} [16.07.1865]
\item \hyperref[@I1873@]{Katharina Schwanninger} [25.11.1873]
\item \hyperref[@I2108@]{Ida Schwanninger} [...]
\end{itemize}
\bigbreak
\textsc{{quellen}}
\begin{enumerate}[label={[\arabic*]}]
\item \href{https://www.familysearch.org/tree/person/details/LTC6-N7K}{FamilySearch, ID: LTC6-N7K}
\end{enumerate}

\end{person}

\begin{person}[
    surname = {Schwanninger},
    givenname = {Josefa},
    suffix = {1865},
    label = {@I1305@}
    ]

\begin{tabular}{cl}
\geboren & 16. Juli 1865 in Mörschenhardt\\
\taufe &  in Mudau\\
\geheiratet & 30. Juni 1890 in Karlsruhe mit Peter Wendelin Wachter \\
\gestorben &  in Reilingen\\
\end{tabular}\\
\medbreak
\textsc{vater}: \hyperref[@I148@]{Franz, Josef Schwanninger} [11.05.1823--06.04.1905 (8 Kinder)]\\
\textsc{mutter}: \hyperref[@I149@]{Margaretha Rögner} [04.04.1834--20.08.1906 (8 Kinder)]
\medbreak
\textsc{{geschwister}}
\begin{itemize}
\item \hyperref[@I1302@]{Wilhelm Schwanninger} [18.02.1857--nach 1904 (9 Kinder)]
\item \hyperref[@I145@]{Karolina Schwanninger} [14.09.1858--02.07.1926 (10 Kinder)]
\item \hyperref[@I1172@]{Margaretha Schwanninger} [31.07.1860]
\item \hyperref[@I1303@]{Rosina Schwanninger} [09.11.1861--10.07.1932 (4 Kinder)]
\item \hyperref[@I1304@]{Friedrich Schwanninger} [04.04.1863--04.06.1867]
\item \hyperref[@I1873@]{Katharina Schwanninger} [25.11.1873]
\item \hyperref[@I2108@]{Ida Schwanninger} [...]
\end{itemize}
\bigbreak
\textsc{anmerkung}\\
geh. Peter Wendelin Wachter in Karlsruhe 1890.06.30

War im Kloster. Wurde Heim geschickt da sie krank war. Hat dannach geheiratet und ist nach  Reilingen.
Man ist früh gestorben und sie hat nochmal geheiratet (Krämer). Ist mit ca 95 gestorben.

\medbreak
\textsc{{quellen}}
\begin{enumerate}[label={[\arabic*]}]
\item \href{https://www.familysearch.org/tree/person/details/LTC6-FWN}{FamilySearch, ID: LTC6-FWN}
\end{enumerate}

\end{person}

\begin{person}[
    surname = {Schwanninger},
    givenname = {Katharina},
    suffix = {1873},
    label = {@I1873@}
    ]

\begin{tabular}{cl}
\geboren & 25. November 1873 in Mörschenhardt\\
\taufe &  in Mudau\\
\end{tabular}\\
\medbreak
\textsc{vater}: \hyperref[@I148@]{Franz, Josef Schwanninger} [11.05.1823--06.04.1905 (8 Kinder)]\\
\textsc{mutter}: \hyperref[@I149@]{Margaretha Rögner} [04.04.1834--20.08.1906 (8 Kinder)]
\medbreak
\textsc{{geschwister}}
\begin{itemize}
\item \hyperref[@I1302@]{Wilhelm Schwanninger} [18.02.1857--nach 1904 (9 Kinder)]
\item \hyperref[@I145@]{Karolina Schwanninger} [14.09.1858--02.07.1926 (10 Kinder)]
\item \hyperref[@I1172@]{Margaretha Schwanninger} [31.07.1860]
\item \hyperref[@I1303@]{Rosina Schwanninger} [09.11.1861--10.07.1932 (4 Kinder)]
\item \hyperref[@I1304@]{Friedrich Schwanninger} [04.04.1863--04.06.1867]
\item \hyperref[@I1305@]{Josefa Schwanninger} [16.07.1865]
\item \hyperref[@I2108@]{Ida Schwanninger} [...]
\end{itemize}
\bigbreak
\textsc{{quellen}}
\begin{enumerate}[label={[\arabic*]}]
\item Geburtenregister Mörschenhardt 1873, Nr. 11
\end{enumerate}

\end{person}

\begin{person}[
    surname = {Schwanninger},
    givenname = {Ida},
    suffix = {},
    label = {@I2108@}
    ]

\begin{tabular}{cl}
\bestattet & 31. August 1864 in Mudau\\
\end{tabular}\\
\medbreak
\textsc{vater}: \hyperref[@I148@]{Franz, Josef Schwanninger} [11.05.1823--06.04.1905 (8 Kinder)]\\
\textsc{mutter}: \hyperref[@I149@]{Margaretha Rögner} [04.04.1834--20.08.1906 (8 Kinder)]
\medbreak
\textsc{{geschwister}}
\begin{itemize}
\item \hyperref[@I1302@]{Wilhelm Schwanninger} [18.02.1857--nach 1904 (9 Kinder)]
\item \hyperref[@I145@]{Karolina Schwanninger} [14.09.1858--02.07.1926 (10 Kinder)]
\item \hyperref[@I1172@]{Margaretha Schwanninger} [31.07.1860]
\item \hyperref[@I1303@]{Rosina Schwanninger} [09.11.1861--10.07.1932 (4 Kinder)]
\item \hyperref[@I1304@]{Friedrich Schwanninger} [04.04.1863--04.06.1867]
\item \hyperref[@I1305@]{Josefa Schwanninger} [16.07.1865]
\item \hyperref[@I1873@]{Katharina Schwanninger} [25.11.1873]
\end{itemize}
\bigbreak
\textsc{{quellen}}
\begin{enumerate}[label={[\arabic*]}]
\item \href{https://www.familysearch.org/tree/person/details/G95H-HFD}{FamilySearch, ID: G95H-HFD}
\end{enumerate}

\end{person}


\addsec{Franz Josef Schüßler  \& Margareta Eberschwein }


\begin{person}[
    surname = {Schüßler},
    givenname = {Franz Josef},
    suffix = {1831--1902},
    label = {@I152@}
    ]

\begin{tabular}{cl}
\geboren & 02. Januar 1831 in Mörschenhardt\\
\taufe & 01. Februar 1831 in Mudau\\
\geheiratet & 06. August 1857 in Mudau mit Margareta Eberschwein \\
\gestorben & 13. Juni 1902 in Mörschenhardt\\
\end{tabular}\\
\medbreak
\textsc{vater}: Franz Josef Schüßler [08.11.1795--29.04.1854 (5 Kinder)]\\
\textsc{mutter}: \hyperref[@I909@]{Eva Katharina Lenz} [12.06.1805 (5 Kinder)]
\medbreak
\textsc{{geschwister}}
\begin{itemize}
\item Johann Valentin Schüßler [04.06.1828]
\item Michael Schüßler [21.10.1836--28.10.1836]
\item Andreas Schüßler [12.12.1837]
\item Konstantin Schüßler [...]
\end{itemize}
\bigbreak
\textsc{{kinder}}
\begin{itemize}
\item \hyperref[@I150@]{Johann Georg Schüßler} [26.08.1858--06.12.1937 (10 Kinder)]
\item \hyperref[@I1345@]{Andreas Schüßler} [29.03.1860 (1 Kind)]
\item \hyperref[@I1346@]{Karolina Schüßler} [19.11.1861--29.12.1861]
\item \hyperref[@I1738@]{Wilhelmina Schüßler} [11.11.1868--30.04.1940 (6 Kinder)]
\end{itemize}
\medbreak
\textsc{anmerkung}\\
3 Brueder sind nach Amerika ausgewandert
\medbreak
\textsc{{quellen}}
\begin{enumerate}[label={[\arabic*]}]
\item \href{http://www.landesarchiv-bw.de/plink/?f=4-1119447-49}{GLA Karlsruhe, Mudau, katholische Gemeinde: Standesbuch 1830–1833, Geburtenregister 1831, Nr. 2 (Bild 49)}
\item \href{http://www.landesarchiv-bw.de/plink/?f=4-1119442-105}{GLA Karlsruhe, Mörschenhardt, katholische Gemeinde: Standesbuch 1839–1870, Heiratsregister 1857, Nr. 1 (Bild 105)}
\item Mörschenhardt Geburts-, Heirats- und Sterberegister 1900–1904, Sterberegister 1902, Nr. 5
\item \href{https://www.familysearch.org/tree/person/details/L5GK-4NQ}{FamilySearch, ID: L5GK-4NQ}
\end{enumerate}

\end{person}

\begin{person}[
    surname = {Eberschwein},
    givenname = {Margareta},
    suffix = {1831--1912},
    label = {@I153@}
    ]

\begin{tabular}{cl}
\geboren & 12. November 1831 in Kürnbach\\
\taufe & 20. November 1831 in Kürnbach\\
\geheiratet & 06. August 1857 in Mudau mit Franz Josef Schüßler \\
\gestorben & 25. Februar 1912 in Mörschenhardt\\
\end{tabular}\\
\medbreak
\textsc{vater}: Johann Georg Eberschwein [04.08.1797--26.09.1851 (8 Kinder)]\\
\textsc{mutter}: \hyperref[@I771@]{Friderica Hauser} [09.05.1796--08.06.1878 (8 Kinder)]
\medbreak
\textsc{{geschwister}}
\begin{itemize}
\item Johann Heinrich Eberschwein [13.01.1823--17.11.1823]
\item Johann Heinrich Eberschwein [18.05.1825]
\item Friedrika Susanna Eberschwein [30.07.1826]
\item Carolina Eberschwein [27.08.1829]
\item Samuel Eberschwein [21.06.1833--30.08.1833]
\item Elisabeth Friederica Eberschwein [15.01.1835--06.01.1839]
\item Heinrich Eberschwein [07.11.1836]
\end{itemize}
\bigbreak
\textsc{{kinder}}
\begin{itemize}
\item \hyperref[@I150@]{Johann Georg Schüßler} [26.08.1858--06.12.1937 (10 Kinder)]
\item \hyperref[@I1345@]{Andreas Schüßler} [29.03.1860 (1 Kind)]
\item \hyperref[@I1346@]{Karolina Schüßler} [19.11.1861--29.12.1861]
\item \hyperref[@I1738@]{Wilhelmina Schüßler} [11.11.1868--30.04.1940 (6 Kinder)]
\end{itemize}
\medbreak
\textsc{{quellen}}
\begin{enumerate}[label={[\arabic*]}]
\item \href{http://www.landesarchiv-bw.de/plink/?f=4-1119442-105}{GLA Karlsruhe, Mörschenhardt, katholische Gemeinde: Standesbuch 1839–1870, Heiratsregister 1857, Nr. 1 (Bild 105)}
\item \href{https://www.familysearch.org/tree/person/details/L5GK-ZNY}{FamilySearch, ID: L5GK-ZNY}
\end{enumerate}

\end{person}

\begin{person}[
    surname = {Schüßler},
    givenname = {Andreas},
    suffix = {1860},
    label = {@I1345@}
    ]

\begin{tabular}{cl}
\geboren & 29. März 1860 in Mörschenhardt\\
\end{tabular}\\
\medbreak
\textsc{vater}: \hyperref[@I152@]{Franz Josef Schüßler} [02.01.1831--13.06.1902 (4 Kinder)]\\
\textsc{mutter}: \hyperref[@I153@]{Margareta Eberschwein} [12.11.1831--25.02.1912 (4 Kinder)]
\medbreak
\textsc{{geschwister}}
\begin{itemize}
\item \hyperref[@I150@]{Johann Georg Schüßler} [26.08.1858--06.12.1937 (10 Kinder)]
\item \hyperref[@I1346@]{Karolina Schüßler} [19.11.1861--29.12.1861]
\item \hyperref[@I1738@]{Wilhelmina Schüßler} [11.11.1868--30.04.1940 (6 Kinder)]
\end{itemize}
\bigbreak
\textsc{{kinder}}
\begin{itemize}
\item Arthur Schüßler [...]
\end{itemize}
\medbreak
\textsc{anmerkung}\\
Lehrer erst in Unterscheidental, dann bei Raststadt
Als er Anton Schüssler rumstehen sah lies er ihn nach Pforzheim zur Uhrmacherausbildung schicken
\medbreak
\textsc{{quellen}}
\begin{enumerate}[label={[\arabic*]}]
\item \href{http://www.landesarchiv-bw.de/plink/?f=4-1119442-121}{GLA Karlsruhe, Mörschenhardt, katholische Gemeinde: Standesbuch 1839–1870, Geburtenregister 1860, Nr. 5 (Bild 121)}
\item \href{https://www.familysearch.org/tree/person/details/G9J8-6RC}{FamilySearch, ID: G9J8-6RC}
\end{enumerate}

\end{person}

\begin{person}[
    surname = {Schüßler},
    givenname = {Karolina},
    suffix = {1861--1861},
    label = {@I1346@}
    ]

\begin{tabular}{cl}
\geboren & 19. November 1861 in Mörschenhardt\\
\gestorben & 29. Dezember 1861 in Mörschenhardt\\
\end{tabular}\\
\medbreak
\textsc{vater}: \hyperref[@I152@]{Franz Josef Schüßler} [02.01.1831--13.06.1902 (4 Kinder)]\\
\textsc{mutter}: \hyperref[@I153@]{Margareta Eberschwein} [12.11.1831--25.02.1912 (4 Kinder)]
\medbreak
\textsc{{geschwister}}
\begin{itemize}
\item \hyperref[@I150@]{Johann Georg Schüßler} [26.08.1858--06.12.1937 (10 Kinder)]
\item \hyperref[@I1345@]{Andreas Schüßler} [29.03.1860 (1 Kind)]
\item \hyperref[@I1738@]{Wilhelmina Schüßler} [11.11.1868--30.04.1940 (6 Kinder)]
\end{itemize}
\bigbreak
\textsc{{quellen}}
\begin{enumerate}[label={[\arabic*]}]
\item \href{http://www.landesarchiv-bw.de/plink/?f=4-1119442-126}{GLA Karlsruhe, Mörschenhardt, katholische Gemeinde: Standesbuch 1839–1870, Geburtenregister 1861, Nr. 7 (Bild 126)}
\item \href{http://www.landesarchiv-bw.de/plink/?f=4-1119442-128}{GLA Karlsruhe, Mörschenhardt, katholische Gemeinde: Standesbuch 1839–1870, Sterberegister 1861, Nr. 4 (Bild 128)}
\end{enumerate}

\end{person}

\begin{person}[
    surname = {Schüßler},
    givenname = {Wilhelmina},
    suffix = {1868--1940},
    label = {@I1738@}
    ]

\begin{tabular}{cl}
\geboren & 11. November 1868 in Mörschenhardt\\
\geheiratet & 13. Juli 1893 in Mudau mit Valentin Link \\
\gestorben & 30. April 1940 in Donebach\\
\end{tabular}\\
\medbreak
\textsc{vater}: \hyperref[@I152@]{Franz Josef Schüßler} [02.01.1831--13.06.1902 (4 Kinder)]\\
\textsc{mutter}: \hyperref[@I153@]{Margareta Eberschwein} [12.11.1831--25.02.1912 (4 Kinder)]
\medbreak
\textsc{{geschwister}}
\begin{itemize}
\item \hyperref[@I150@]{Johann Georg Schüßler} [26.08.1858--06.12.1937 (10 Kinder)]
\item \hyperref[@I1345@]{Andreas Schüßler} [29.03.1860 (1 Kind)]
\item \hyperref[@I1346@]{Karolina Schüßler} [19.11.1861--29.12.1861]
\end{itemize}
\bigbreak
\textsc{{kinder}}
\begin{itemize}
\item Anton Link [1894--05.09.1894]
\item Alphons Link [23.09.1895]
\item Valentin Link [15.12.1896]
\item Frieda Link [06.04.1898]
\item Karl Joseph Link [15.11.1899]
\item Alois Link [...]
\end{itemize}
\medbreak
\textsc{anmerkung}\\
Standesamtliche Hochzeit am 13.07 in Donebach
An Weinhnachten kam Familie Schüssler zu ihr und Sie hat einen tollen Nusszopf gebacken
\medbreak
\textsc{{quellen}}
\begin{enumerate}[label={[\arabic*]}]
\item \href{http://www.landesarchiv-bw.de/plink/?f=4-1119442-166}{GLA Karlsruhe, Mörschenhardt, katholische Gemeinde: Standesbuch 1839–1870, Geburtenregister 1868, Nr. 13 (Bild 166)}
\item Donebach Geburts-, Heirats- und Sterberegister 1890–1899, Heiratsregister 1893, Nr. 3
\item Mudau Sterbebuch 1938–1944, Sterberegister 1940, Nr. 12
\item \href{https://www.familysearch.org/tree/person/details/GM2T-1PC}{FamilySearch, ID: GM2T-1PC}
\end{enumerate}

\end{person}


\addsec{Johann Michael Gramlich  \& Martha Schwing }


\begin{person}[
    surname = {Gramlich},
    givenname = {Johann Michael},
    suffix = {1823--1891},
    label = {@I154@}
    ]

\begin{tabular}{cl}
\geboren & 26. September 1823 in Mörschenhardt\\
\taufe & 27. September 1823\\
\geheiratet & 12. September 1861 in Mudau mit Martha Schwing \\
\gestorben & 20. Juni 1891 in Mörschenhardt\\
\bestattet &  in Mudau\\
\end{tabular}\\
\medbreak
\textsc{vater}: Johann Michel Gramlich [23.12.1792--08.03.1855 (8 Kinder)]\\
\textsc{mutter}: \hyperref[@I692@]{Maria Josepha Schäfer} [11.11.1793--12.03.1835 (8 Kinder)]
\medbreak
\textsc{{geschwister}}
\begin{itemize}
\item Josepha Gramlich [04.03.1817--09.04.1817]
\item Franz Joseph Gramlich [10.10.1818]
\item Andreas Gramlich [26.02.1826--17.04.1826]
\item Maria Theresia Gramlich [13.05.1827]
\item Veronika Gramlich [13.02.1831--19.02.1831]
\item Valentin Gramlich [13.02.1831--20.03.1831]
\item Josepha Gramlich [01.08.1832]
\end{itemize}
\bigbreak
\textsc{{kinder}}
\begin{itemize}
\item \hyperref[@I736@]{Maria Martha Gramlich} [08.09.1860]
\item \hyperref[@I737@]{Wilhelm Gramlich} [03.11.1861]
\item \hyperref[@I151@]{Helena Gramlich} [28.01.1864--21.11.1943 (10 Kinder)]
\item \hyperref[@I1885@]{Ferdinand Gramlich} [22.09.1866]
\item \hyperref[@I738@]{Maria Klara Gramlich} [24.02.1869]
\item \hyperref[@I1886@]{Michael Gramlich} [31.08.1871--1951]
\item \hyperref[@I739@]{Gottfried Gramlich} [19.02.1874--25.02.1875]
\item \hyperref[@I740@]{Isidor Gramlich} [11.04.1876--27.01.1958]
\item \hyperref[@I1887@]{Emma Gramlich} [05.11.1877--1948]
\end{itemize}
\medbreak
\textsc{{quellen}}
\begin{enumerate}[label={[\arabic*]}]
\item \href{http://www.landesarchiv-bw.de/plink/?f=4-1119445-38}{GLA Karlsruhe, Mudau, katholische Gemeinde: Standesbuch 1821–1826, Geburtenregister 1823, Nr. 114 (Bild 38)}
\item \href{http://www.landesarchiv-bw.de/plink/?f=4-1119442-127}{GLA Karlsruhe, Mörschenhardt, katholische Gemeinde: Standesbuch 1839–1870, Heiratsregister 1861, Nr. 1 (127)}
\item \href{https://www.familysearch.org/tree/person/details/L5GK-2YR}{FamilySearch, ID: L5GK-2YR}
\item https://www.wikitree.com/wiki/Gramlich-62
\end{enumerate}

\end{person}

\begin{person}[
    surname = {Schwing},
    givenname = {Martha},
    suffix = {1835--1902},
    label = {@I155@}
    ]

\begin{tabular}{cl}
\geboren & 10. Juni 1835 in Donebach\\
\taufe & 10. Juni 1835 in Mudau\\
\geheiratet & 12. September 1861 in Mudau mit Johann Michael Gramlich \\
\gestorben & 06. Februar 1902 in Mörschenhardt\\
\end{tabular}\\
\medbreak
\textsc{vater}: Johann Martin Schwing [19.11.1806--03.02.1884 (5 Kinder)]\\
\textsc{mutter}: \hyperref[@I741@]{Maria Klara Schwab} [13.02.1814--14.03.1859 (5 Kinder)]
\medbreak
\textsc{{geschwister}}
\begin{itemize}
\item Katharina Schwing [13.03.1838--15.03.1838]
\item Klara Schwing [09.08.1839--14.09.1841]
\item Franz Schwing [22.05.1846--26.05.1924]
\item Martin Schwing [21.12.1849--02.01.1850]
\end{itemize}
\bigbreak
\textsc{{kinder}}
\begin{itemize}
\item \hyperref[@I736@]{Maria Martha Gramlich} [08.09.1860]
\item \hyperref[@I737@]{Wilhelm Gramlich} [03.11.1861]
\item \hyperref[@I151@]{Helena Gramlich} [28.01.1864--21.11.1943 (10 Kinder)]
\item \hyperref[@I1885@]{Ferdinand Gramlich} [22.09.1866]
\item \hyperref[@I738@]{Maria Klara Gramlich} [24.02.1869]
\item \hyperref[@I1886@]{Michael Gramlich} [31.08.1871--1951]
\item \hyperref[@I739@]{Gottfried Gramlich} [19.02.1874--25.02.1875]
\item \hyperref[@I740@]{Isidor Gramlich} [11.04.1876--27.01.1958]
\item \hyperref[@I1887@]{Emma Gramlich} [05.11.1877--1948]
\end{itemize}
\medbreak
\textsc{{quellen}}
\begin{enumerate}[label={[\arabic*]}]
\item \href{http://www.landesarchiv-bw.de/plink/?f=4-1119448-57}{GLA Karlsruhe, Mudau, katholische Gemeinde: Standesbuch 1834–1837, Geburtenregister 1835, Nr. 62 (Bild 57)}
\item \href{http://www.landesarchiv-bw.de/plink/?f=4-1119442-127}{GLA Karlsruhe, Mörschenhardt, katholische Gemeinde: Standesbuch 1839–1870, Heiratsregister 1861, Nr. 1 (127)}
\item \href{https://www.familysearch.org/tree/person/details/L5GK-LXS}{FamilySearch, ID: L5GK-LXS}
\item https://www.wikitree.com/wiki/Schwing-23
\end{enumerate}

\end{person}

\begin{person}[
    surname = {Gramlich},
    givenname = {Maria Martha},
    suffix = {1860},
    label = {@I736@}
    ]

\begin{tabular}{cl}
\geboren & 08. September 1860\\
\end{tabular}\\
\medbreak
\textsc{vater}: \hyperref[@I154@]{Johann Michael Gramlich} [26.09.1823--20.06.1891 (9 Kinder)]\\
\textsc{mutter}: \hyperref[@I155@]{Martha Schwing} [10.06.1835--06.02.1902 (9 Kinder)]
\medbreak
\textsc{{geschwister}}
\begin{itemize}
\item \hyperref[@I737@]{Wilhelm Gramlich} [03.11.1861]
\item \hyperref[@I151@]{Helena Gramlich} [28.01.1864--21.11.1943 (10 Kinder)]
\item \hyperref[@I1885@]{Ferdinand Gramlich} [22.09.1866]
\item \hyperref[@I738@]{Maria Klara Gramlich} [24.02.1869]
\item \hyperref[@I1886@]{Michael Gramlich} [31.08.1871--1951]
\item \hyperref[@I739@]{Gottfried Gramlich} [19.02.1874--25.02.1875]
\item \hyperref[@I740@]{Isidor Gramlich} [11.04.1876--27.01.1958]
\item \hyperref[@I1887@]{Emma Gramlich} [05.11.1877--1948]
\end{itemize}
\bigbreak
\textsc{{quellen}}
\begin{enumerate}[label={[\arabic*]}]
\item \href{https://www.familysearch.org/tree/person/details/L139-9YZ}{FamilySearch, ID: L139-9YZ}
\end{enumerate}

\end{person}

\begin{person}[
    surname = {Gramlich},
    givenname = {Wilhelm},
    suffix = {1861},
    label = {@I737@}
    ]

\begin{tabular}{cl}
\geboren & 03. November 1861 in Mörschenhardt\\
\taufe & 03. November 1861 in Mudau\\
\geheiratet & 13. August 1895 in Mudau mit Anna Bertha Berberich \\
\end{tabular}\\
\medbreak
\textsc{vater}: \hyperref[@I154@]{Johann Michael Gramlich} [26.09.1823--20.06.1891 (9 Kinder)]\\
\textsc{mutter}: \hyperref[@I155@]{Martha Schwing} [10.06.1835--06.02.1902 (9 Kinder)]
\medbreak
\textsc{{geschwister}}
\begin{itemize}
\item \hyperref[@I736@]{Maria Martha Gramlich} [08.09.1860]
\item \hyperref[@I151@]{Helena Gramlich} [28.01.1864--21.11.1943 (10 Kinder)]
\item \hyperref[@I1885@]{Ferdinand Gramlich} [22.09.1866]
\item \hyperref[@I738@]{Maria Klara Gramlich} [24.02.1869]
\item \hyperref[@I1886@]{Michael Gramlich} [31.08.1871--1951]
\item \hyperref[@I739@]{Gottfried Gramlich} [19.02.1874--25.02.1875]
\item \hyperref[@I740@]{Isidor Gramlich} [11.04.1876--27.01.1958]
\item \hyperref[@I1887@]{Emma Gramlich} [05.11.1877--1948]
\end{itemize}
\bigbreak
\textsc{{quellen}}
\begin{enumerate}[label={[\arabic*]}]
\item Mörschenhardt Geburts-, Heirats- und Sterberegister 1890–1899, Heiratsregister 1895, Nr. 2
\item \href{https://www.familysearch.org/tree/person/details/LT8H-92F}{FamilySearch, ID: LT8H-92F}
\end{enumerate}

\end{person}

\begin{person}[
    surname = {Gramlich},
    givenname = {Ferdinand},
    suffix = {1866},
    label = {@I1885@}
    ]

\begin{tabular}{cl}
\geboren & 22. September 1866 in Mudau\\
\end{tabular}\\
\medbreak
\textsc{vater}: \hyperref[@I154@]{Johann Michael Gramlich} [26.09.1823--20.06.1891 (9 Kinder)]\\
\textsc{mutter}: \hyperref[@I155@]{Martha Schwing} [10.06.1835--06.02.1902 (9 Kinder)]
\medbreak
\textsc{{geschwister}}
\begin{itemize}
\item \hyperref[@I736@]{Maria Martha Gramlich} [08.09.1860]
\item \hyperref[@I737@]{Wilhelm Gramlich} [03.11.1861]
\item \hyperref[@I151@]{Helena Gramlich} [28.01.1864--21.11.1943 (10 Kinder)]
\item \hyperref[@I738@]{Maria Klara Gramlich} [24.02.1869]
\item \hyperref[@I1886@]{Michael Gramlich} [31.08.1871--1951]
\item \hyperref[@I739@]{Gottfried Gramlich} [19.02.1874--25.02.1875]
\item \hyperref[@I740@]{Isidor Gramlich} [11.04.1876--27.01.1958]
\item \hyperref[@I1887@]{Emma Gramlich} [05.11.1877--1948]
\end{itemize}
\bigbreak
\textsc{anmerkung}\\
ins schwaebische (Rottenburg oder aehnlich)
\medbreak
\textsc{{quellen}}
\begin{enumerate}[label={[\arabic*]}]
\item Mündliche Überlieferung Maria Schäfer
\item \href{https://www.familysearch.org/tree/person/details/G9FJ-1MB}{FamilySearch, ID: G9FJ-1MB}
\end{enumerate}

\end{person}

\begin{person}[
    surname = {Gramlich},
    givenname = {Maria Klara},
    suffix = {1869},
    label = {@I738@}
    ]

\begin{tabular}{cl}
\geboren & 24. Februar 1869 in Mörschenhardt\\
\taufe & 25. Februar 1869 in Mudau\\
\end{tabular}\\
\medbreak
\textsc{vater}: \hyperref[@I154@]{Johann Michael Gramlich} [26.09.1823--20.06.1891 (9 Kinder)]\\
\textsc{mutter}: \hyperref[@I155@]{Martha Schwing} [10.06.1835--06.02.1902 (9 Kinder)]
\medbreak
\textsc{{geschwister}}
\begin{itemize}
\item \hyperref[@I736@]{Maria Martha Gramlich} [08.09.1860]
\item \hyperref[@I737@]{Wilhelm Gramlich} [03.11.1861]
\item \hyperref[@I151@]{Helena Gramlich} [28.01.1864--21.11.1943 (10 Kinder)]
\item \hyperref[@I1885@]{Ferdinand Gramlich} [22.09.1866]
\item \hyperref[@I1886@]{Michael Gramlich} [31.08.1871--1951]
\item \hyperref[@I739@]{Gottfried Gramlich} [19.02.1874--25.02.1875]
\item \hyperref[@I740@]{Isidor Gramlich} [11.04.1876--27.01.1958]
\item \hyperref[@I1887@]{Emma Gramlich} [05.11.1877--1948]
\end{itemize}
\bigbreak
\textsc{{quellen}}
\begin{enumerate}[label={[\arabic*]}]
\item \href{https://www.familysearch.org/tree/person/details/LYPW-VGG}{FamilySearch, ID: LYPW-VGG}
\end{enumerate}

\end{person}

\begin{person}[
    surname = {Gramlich},
    givenname = {Michael},
    suffix = {1871--1951},
    label = {@I1886@}
    ]

\begin{tabular}{cl}
\geboren & 31. August 1871 in Mörschenhardt\\
\gestorben & 1951 in Mörschenhardt\\
\end{tabular}\\
\medbreak
\textsc{vater}: \hyperref[@I154@]{Johann Michael Gramlich} [26.09.1823--20.06.1891 (9 Kinder)]\\
\textsc{mutter}: \hyperref[@I155@]{Martha Schwing} [10.06.1835--06.02.1902 (9 Kinder)]
\medbreak
\textsc{{geschwister}}
\begin{itemize}
\item \hyperref[@I736@]{Maria Martha Gramlich} [08.09.1860]
\item \hyperref[@I737@]{Wilhelm Gramlich} [03.11.1861]
\item \hyperref[@I151@]{Helena Gramlich} [28.01.1864--21.11.1943 (10 Kinder)]
\item \hyperref[@I1885@]{Ferdinand Gramlich} [22.09.1866]
\item \hyperref[@I738@]{Maria Klara Gramlich} [24.02.1869]
\item \hyperref[@I739@]{Gottfried Gramlich} [19.02.1874--25.02.1875]
\item \hyperref[@I740@]{Isidor Gramlich} [11.04.1876--27.01.1958]
\item \hyperref[@I1887@]{Emma Gramlich} [05.11.1877--1948]
\end{itemize}
\bigbreak
\textsc{anmerkung}\\
ledig in Mörschenhardt. Hatte Bienen
\medbreak
\textsc{{quellen}}
\begin{enumerate}[label={[\arabic*]}]
\item Mörschenhardt Geburts-, Heirats- und Sterberegister 1870–1879, Geburtenregister 1871, Nr. 7
\item Mörschenhardt Sterbebuch 1946–1971, Sterberegister 1951, Nr. 3
\item \href{https://www.familysearch.org/tree/person/details/G9FJ-GFS}{FamilySearch, ID: G9FJ-GFS}
\end{enumerate}

\end{person}

\begin{person}[
    surname = {Gramlich},
    givenname = {Gottfried},
    suffix = {1874--1875},
    label = {@I739@}
    ]

\begin{tabular}{cl}
\geboren & 19. Februar 1874 in Mörschenhardt\\
\gestorben & 25. Februar 1875 in Mörschenhardt\\
\end{tabular}\\
\medbreak
\textsc{vater}: \hyperref[@I154@]{Johann Michael Gramlich} [26.09.1823--20.06.1891 (9 Kinder)]\\
\textsc{mutter}: \hyperref[@I155@]{Martha Schwing} [10.06.1835--06.02.1902 (9 Kinder)]
\medbreak
\textsc{{geschwister}}
\begin{itemize}
\item \hyperref[@I736@]{Maria Martha Gramlich} [08.09.1860]
\item \hyperref[@I737@]{Wilhelm Gramlich} [03.11.1861]
\item \hyperref[@I151@]{Helena Gramlich} [28.01.1864--21.11.1943 (10 Kinder)]
\item \hyperref[@I1885@]{Ferdinand Gramlich} [22.09.1866]
\item \hyperref[@I738@]{Maria Klara Gramlich} [24.02.1869]
\item \hyperref[@I1886@]{Michael Gramlich} [31.08.1871--1951]
\item \hyperref[@I740@]{Isidor Gramlich} [11.04.1876--27.01.1958]
\item \hyperref[@I1887@]{Emma Gramlich} [05.11.1877--1948]
\end{itemize}
\bigbreak
\textsc{{quellen}}
\begin{enumerate}[label={[\arabic*]}]
\item Mörschenhardt Geburts-, Heirats- und Sterberegister 1870–1879, Geburtenregister 1874, Nr. 3
\item Mörschenhardt Geburts-, Heirats- und Sterberegister 1870–1879, Sterberegister 1875, Nr. 1
\item \href{https://www.familysearch.org/tree/person/details/L13M-TYK}{FamilySearch, ID: L13M-TYK}
\end{enumerate}

\end{person}

\begin{person}[
    surname = {Gramlich},
    givenname = {Isidor},
    suffix = {1876--1958},
    label = {@I740@}
    ]

\begin{tabular}{cl}
\geboren & 11. April 1876 in Mörschenhardt\\
\geheiratet & 12. Februar 1903 in Mörschenhardt mit Marie Mechler \\
\gestorben & 27. Januar 1958 in Mörschenhardt\\
\end{tabular}\\
\medbreak
\textsc{vater}: \hyperref[@I154@]{Johann Michael Gramlich} [26.09.1823--20.06.1891 (9 Kinder)]\\
\textsc{mutter}: \hyperref[@I155@]{Martha Schwing} [10.06.1835--06.02.1902 (9 Kinder)]
\medbreak
\textsc{{geschwister}}
\begin{itemize}
\item \hyperref[@I736@]{Maria Martha Gramlich} [08.09.1860]
\item \hyperref[@I737@]{Wilhelm Gramlich} [03.11.1861]
\item \hyperref[@I151@]{Helena Gramlich} [28.01.1864--21.11.1943 (10 Kinder)]
\item \hyperref[@I1885@]{Ferdinand Gramlich} [22.09.1866]
\item \hyperref[@I738@]{Maria Klara Gramlich} [24.02.1869]
\item \hyperref[@I1886@]{Michael Gramlich} [31.08.1871--1951]
\item \hyperref[@I739@]{Gottfried Gramlich} [19.02.1874--25.02.1875]
\item \hyperref[@I1887@]{Emma Gramlich} [05.11.1877--1948]
\end{itemize}
\bigbreak
\textsc{anmerkung}\\
auf dem Hof geblieben
geboren am 08. April nach alter Quelle
\medbreak
\textsc{{quellen}}
\begin{enumerate}[label={[\arabic*]}]
\item Mörschenhardt Geburts-, Heirats- und Sterberegister 1870–1879, Geburtenregister 1876, Nr. 3
\item Mörschenhardt Geburts-, Heirats- und Sterberegister 1900–1904, Heiratsregister 1903, Nr. 1
\item Mörschenhardt Sterbebuch 1946–1971, Sterberegister 1958, Nr. 2
\item \href{https://www.familysearch.org/tree/person/details/L13M-R5L}{FamilySearch, ID: L13M-R5L}
\end{enumerate}

\end{person}

\begin{person}[
    surname = {Gramlich},
    givenname = {Emma},
    suffix = {1877--1948},
    label = {@I1887@}
    ]

\begin{tabular}{cl}
\geboren & 05. November 1877 in Mörschenhardt\\
\gestorben & 1948 in Mörschenhardt\\
\end{tabular}\\
\medbreak
\textsc{vater}: \hyperref[@I154@]{Johann Michael Gramlich} [26.09.1823--20.06.1891 (9 Kinder)]\\
\textsc{mutter}: \hyperref[@I155@]{Martha Schwing} [10.06.1835--06.02.1902 (9 Kinder)]
\medbreak
\textsc{{geschwister}}
\begin{itemize}
\item \hyperref[@I736@]{Maria Martha Gramlich} [08.09.1860]
\item \hyperref[@I737@]{Wilhelm Gramlich} [03.11.1861]
\item \hyperref[@I151@]{Helena Gramlich} [28.01.1864--21.11.1943 (10 Kinder)]
\item \hyperref[@I1885@]{Ferdinand Gramlich} [22.09.1866]
\item \hyperref[@I738@]{Maria Klara Gramlich} [24.02.1869]
\item \hyperref[@I1886@]{Michael Gramlich} [31.08.1871--1951]
\item \hyperref[@I739@]{Gottfried Gramlich} [19.02.1874--25.02.1875]
\item \hyperref[@I740@]{Isidor Gramlich} [11.04.1876--27.01.1958]
\end{itemize}
\bigbreak
\textsc{anmerkung}\\
ledig daheim
\medbreak
\textsc{{quellen}}
\begin{enumerate}[label={[\arabic*]}]
\item Mörschenhardt Geburts-, Heirats- und Sterberegister 1870–1879, Geburtenregister 1877, Nr. 7
\item Mörschenhardt Sterbebuch 1946–1971, Sterberegister 1948, Nr. 2
\item \href{https://www.familysearch.org/tree/person/details/G9FV-DPK}{FamilySearch, ID: G9FV-DPK}
\end{enumerate}

\end{person}


\addsec{Johann Philipp Schölch  \& Anna Maria Ott }


\begin{person}[
    surname = {Schölch},
    givenname = {Johann Philipp},
    suffix = {1820--1898},
    label = {@I158@}
    ]

\begin{tabular}{cl}
\geboren & 30. April 1820 in Unterscheidental\\
\taufe & 01. Mai 1820\\
\geheiratet & 04. Mai 1848 in Limbach mit Anna Maria Ott \\
\gestorben & 21. Mai 1898 in Laudenberg\\
\end{tabular}\\
\medbreak
\textsc{vater}: Johann Peter Schelg [03.02.1771--24.11.1845 (7 Kinder)]\\
\textsc{mutter}: \hyperref[@I212@]{Agnes Schäfer} [um 1782--08.12.1843 (7 Kinder)]
\medbreak
\textsc{{geschwister}}
\begin{itemize}
\item Franz Joseph Schelg [13.03.1810]
\item Katharina Schölch [1812]
\item Johann Michael Schölch [28.09.1812]
\item Sebastian Schölch [22.04.1816--11.08.1866]
\item Johann Valentin Schelg [17.09.1823]
\item Franz Andreas Schelg [20.02.1827]
\end{itemize}
\bigbreak
\textsc{{kinder}}
\begin{itemize}
\item \hyperref[@I225@]{Ludwig Schölch} [30.06.1849 (12 Kinder)]
\item \hyperref[@I228@]{Margaretha Schölch} [05.10.1851]
\item \hyperref[@I229@]{Lina Schölch} [27.12.1853--28.01.1858]
\item \hyperref[@I230@]{Katharina Schölch} [10.06.1856]
\item \hyperref[@I231@]{Maria Anna Schölch} [06.11.1859]
\item \hyperref[@I232@]{Theresia Schölch} [28.04.1862--16.10.1863]
\item \hyperref[@I233@]{Helena Schölch} [29.09.1864--06.10.1864]
\item \hyperref[@I156@]{Johann Josef Schölch} [29.08.1865--04.11.1939 (4 Kinder)]
\item \hyperref[@I234@]{Rosalia Schölch} [15.12.1866]
\item \hyperref[@I235@]{Anna Schölch} [08.10.1869]
\end{itemize}
\medbreak
\textsc{{quellen}}
\begin{enumerate}[label={[\arabic*]}]
\item \href{http://www.landesarchiv-bw.de/plink/?f=4-1119444-147}{GLA Karlsruhe, Mudau, katholische Gemeinde: Standesbuch 1812–1820, Geburtenregister 1820, Nr. 34 (Bild 147)}
\item \href{http://www.landesarchiv-bw.de/plink/?f=4-1119439-219}{GLA Karlsruhe, Laudenberg, katholische Gemeinde: Standesbuch 1810–1870, Heiratsregister 1848, Nr. 3 (Bild 219)}
\item \href{https://www.familysearch.org/tree/person/details/LVD3-Z9S}{FamilySearch, ID: LVD3-Z9S}
\item \href{http://gedbas.genealogy.net/person/show/1172960812}{genealogy.net,  }
\item Ahnentafel Erich Schnorr
\end{enumerate}

\end{person}

\begin{person}[
    surname = {Ott},
    givenname = {Anna Maria},
    suffix = {1829--1896},
    label = {@I210@}
    ]

\begin{tabular}{cl}
\geboren & 15. Januar 1829 in Laudenberg\\
\taufe & 15. Januar 1829 in Limbach\\
\geheiratet & 04. Mai 1848 in Limbach mit Johann Philipp Schölch \\
\gestorben & 09. Mai 1896 in Laudenberg\\
\end{tabular}\\
\medbreak
\textsc{vater}: Franz Joseph Ott [um 1801--05.05.1863 (3 Kinder)]\\
\textsc{mutter}: \hyperref[@I237@]{Christina Hess} [um 1805 (3 Kinder)]
\medbreak
\textsc{{geschwister}}
\begin{itemize}
\item Katharina Ott [28.03.1833]
\item Wilhelmine Ott [01.04.1844]
\end{itemize}
\bigbreak
\textsc{{kinder}}
\begin{itemize}
\item \hyperref[@I225@]{Ludwig Schölch} [30.06.1849 (12 Kinder)]
\item \hyperref[@I228@]{Margaretha Schölch} [05.10.1851]
\item \hyperref[@I229@]{Lina Schölch} [27.12.1853--28.01.1858]
\item \hyperref[@I230@]{Katharina Schölch} [10.06.1856]
\item \hyperref[@I231@]{Maria Anna Schölch} [06.11.1859]
\item \hyperref[@I232@]{Theresia Schölch} [28.04.1862--16.10.1863]
\item \hyperref[@I233@]{Helena Schölch} [29.09.1864--06.10.1864]
\item \hyperref[@I156@]{Johann Josef Schölch} [29.08.1865--04.11.1939 (4 Kinder)]
\item \hyperref[@I234@]{Rosalia Schölch} [15.12.1866]
\item \hyperref[@I235@]{Anna Schölch} [08.10.1869]
\end{itemize}
\medbreak
\textsc{anmerkung}\\
Familien Name s'Jocke (von Jacobus / Jakob Ott)
\medbreak
\textsc{{quellen}}
\begin{enumerate}[label={[\arabic*]}]
\item \href{http://www.landesarchiv-bw.de/plink/?f=4-1119439-99}{GLA Karlsruhe, Laudenberg, katholische Gemeinde: Standesbuch 1810–1870, Geburtenregister 1829, Nr. 2 (Bild 99)}
\item \href{http://www.landesarchiv-bw.de/plink/?f=4-1119439-219}{GLA Karlsruhe, Laudenberg, katholische Gemeinde: Standesbuch 1810–1870, Heiratsregister 1848, Nr. 3 (Bild 219)}
\item \href{https://www.familysearch.org/tree/person/details/LV6Z-VLD}{FamilySearch, ID: LV6Z-VLD}
\item \href{http://gedbas.genealogy.net/person/show/1172957478}{genealogy.net}
\end{enumerate}

\end{person}

\begin{person}[
    surname = {Schölch},
    givenname = {Ludwig},
    suffix = {1849},
    label = {@I225@},
    filename = {Ludwig Schölch (1849)}
    ]

\begin{tabular}{cl}
\geboren & 30. Juni 1849 in Laudenberg\\
\taufe & 30. Juni 1849 in Limbach\\
\geheiratet & 30. Januar 1873 in Limbach mit Maria Anna Trunk \\
 & 10. Februar 1885 in Limbach mit Katharina Laura Oeden \\
\end{tabular}\\
\medbreak
\textsc{vater}: \hyperref[@I158@]{Johann Philipp Schölch} [30.04.1820--21.05.1898 (10 Kinder)]\\
\textsc{mutter}: \hyperref[@I210@]{Anna Maria Ott} [15.01.1829--09.05.1896 (10 Kinder)]
\medbreak
\textsc{{geschwister}}
\begin{itemize}
\item \hyperref[@I228@]{Margaretha Schölch} [05.10.1851]
\item \hyperref[@I229@]{Lina Schölch} [27.12.1853--28.01.1858]
\item \hyperref[@I230@]{Katharina Schölch} [10.06.1856]
\item \hyperref[@I231@]{Maria Anna Schölch} [06.11.1859]
\item \hyperref[@I232@]{Theresia Schölch} [28.04.1862--16.10.1863]
\item \hyperref[@I233@]{Helena Schölch} [29.09.1864--06.10.1864]
\item \hyperref[@I156@]{Johann Josef Schölch} [29.08.1865--04.11.1939 (4 Kinder)]
\item \hyperref[@I234@]{Rosalia Schölch} [15.12.1866]
\item \hyperref[@I235@]{Anna Schölch} [08.10.1869]
\end{itemize}
\bigbreak
\textsc{{kinder}}
\begin{itemize}
\item Lina Schölch [24.04.1874--28.01.1875]
\item Wilhelm Schölch [09.12.1875--14.12.1875]
\item Maria Anna Schölch [13.11.1876--24.01.1877]
\item Ida Schölch [24.11.1877]
\item Edmund Engelbert Schölch [11.03.1886--13.04.1959 (5 Kinder)]
\item Alois Schölch [18.01.1887--24.01.1941]
\item Frida Schölch [03.09.1888]
\item Karl Ludwig Schölch [31.12.1890]
\item Berta Schölch [11.02.1893]
\item Rosalia Schölch [23.03.1895]
\item Otto Hieronymus Schölch [06.09.1896]
\item Katharina Laura Schölch [13.09.1898--06.06.1985]
\end{itemize}
\medbreak
\textsc{{quellen}}
\begin{enumerate}[label={[\arabic*]}]
\item GLA Karlsruhe, Laudenberg, katholische Gemeinde: Standesbuch 1810–1870, Geburtenregister 1849, Nr. 7 (Bild 223)
\item \href{https://www.familysearch.org/tree/person/details/LVD3-ZQ5}{FamilySearch, ID: LVD3-ZQ5}
\item \href{http://gedbas.genealogy.net/person/show/1172960830}{genealogy.net}
\end{enumerate}

\end{person}

\begin{person}[
    surname = {Schölch},
    givenname = {Margaretha},
    suffix = {1851},
    label = {@I228@}
    ]

\begin{tabular}{cl}
\geboren & 05. Oktober 1851 in Laudenberg\\
\taufe & 05. Oktober 1851 in Limbach\\
\end{tabular}\\
\medbreak
\textsc{vater}: \hyperref[@I158@]{Johann Philipp Schölch} [30.04.1820--21.05.1898 (10 Kinder)]\\
\textsc{mutter}: \hyperref[@I210@]{Anna Maria Ott} [15.01.1829--09.05.1896 (10 Kinder)]
\medbreak
\textsc{{geschwister}}
\begin{itemize}
\item \hyperref[@I225@]{Ludwig Schölch} [30.06.1849 (12 Kinder)]
\item \hyperref[@I229@]{Lina Schölch} [27.12.1853--28.01.1858]
\item \hyperref[@I230@]{Katharina Schölch} [10.06.1856]
\item \hyperref[@I231@]{Maria Anna Schölch} [06.11.1859]
\item \hyperref[@I232@]{Theresia Schölch} [28.04.1862--16.10.1863]
\item \hyperref[@I233@]{Helena Schölch} [29.09.1864--06.10.1864]
\item \hyperref[@I156@]{Johann Josef Schölch} [29.08.1865--04.11.1939 (4 Kinder)]
\item \hyperref[@I234@]{Rosalia Schölch} [15.12.1866]
\item \hyperref[@I235@]{Anna Schölch} [08.10.1869]
\end{itemize}
\bigbreak
\textsc{{quellen}}
\begin{enumerate}[label={[\arabic*]}]
\item GLA Karlsruhe, Laudenberg, katholische Gemeinde: Standesbuch 1810–1870, Geburtenregister 1851, Nr. 8 (Bild 235)
\item \href{https://www.familysearch.org/tree/person/details/LV6Z-V5Q}{FamilySearch, ID: LV6Z-V5Q}
\item \href{http://gedbas.genealogy.net/person/show/1172960831}{genealogy.net}
\end{enumerate}

\end{person}

\begin{person}[
    surname = {Schölch},
    givenname = {Lina},
    suffix = {1853--1858},
    label = {@I229@}
    ]

\begin{tabular}{cl}
\geboren & 27. Dezember 1853 in Laudenberg\\
\taufe & 27. Dezember 1853 in Limbach\\
\gestorben & 28. Januar 1858 in Laudenberg\\
\bestattet & 30. Januar 1858 in Limbach\\
\end{tabular}\\
\medbreak
\textsc{vater}: \hyperref[@I158@]{Johann Philipp Schölch} [30.04.1820--21.05.1898 (10 Kinder)]\\
\textsc{mutter}: \hyperref[@I210@]{Anna Maria Ott} [15.01.1829--09.05.1896 (10 Kinder)]
\medbreak
\textsc{{geschwister}}
\begin{itemize}
\item \hyperref[@I225@]{Ludwig Schölch} [30.06.1849 (12 Kinder)]
\item \hyperref[@I228@]{Margaretha Schölch} [05.10.1851]
\item \hyperref[@I230@]{Katharina Schölch} [10.06.1856]
\item \hyperref[@I231@]{Maria Anna Schölch} [06.11.1859]
\item \hyperref[@I232@]{Theresia Schölch} [28.04.1862--16.10.1863]
\item \hyperref[@I233@]{Helena Schölch} [29.09.1864--06.10.1864]
\item \hyperref[@I156@]{Johann Josef Schölch} [29.08.1865--04.11.1939 (4 Kinder)]
\item \hyperref[@I234@]{Rosalia Schölch} [15.12.1866]
\item \hyperref[@I235@]{Anna Schölch} [08.10.1869]
\end{itemize}
\bigbreak
\textsc{{quellen}}
\begin{enumerate}[label={[\arabic*]}]
\item GLA Karlsruhe, Laudenberg, katholische Gemeinde: Standesbuch 1810–1870, Geburtenregister 1853, Nr. 14 (Bild 245)
\item GLA Karlsruhe, Laudenberg, katholische Gemeinde: Standesbuch 1810–1870, Sterberegister 1858, Nr. 2 (Bild 277)
\item \href{https://www.familysearch.org/tree/person/details/LV6Z-VR4}{FamilySearch, ID: LV6Z-VR4}
\item \href{http://gedbas.genealogy.net/person/show/1172960828}{genealogy.net}
\end{enumerate}

\end{person}

\begin{person}[
    surname = {Schölch},
    givenname = {Katharina},
    suffix = {1856},
    label = {@I230@}
    ]

\begin{tabular}{cl}
\geboren & 10. Juni 1856 in Laudenberg\\
\taufe & 11. Juni 1856 in Limbach\\
\end{tabular}\\
\medbreak
\textsc{vater}: \hyperref[@I158@]{Johann Philipp Schölch} [30.04.1820--21.05.1898 (10 Kinder)]\\
\textsc{mutter}: \hyperref[@I210@]{Anna Maria Ott} [15.01.1829--09.05.1896 (10 Kinder)]
\medbreak
\textsc{{geschwister}}
\begin{itemize}
\item \hyperref[@I225@]{Ludwig Schölch} [30.06.1849 (12 Kinder)]
\item \hyperref[@I228@]{Margaretha Schölch} [05.10.1851]
\item \hyperref[@I229@]{Lina Schölch} [27.12.1853--28.01.1858]
\item \hyperref[@I231@]{Maria Anna Schölch} [06.11.1859]
\item \hyperref[@I232@]{Theresia Schölch} [28.04.1862--16.10.1863]
\item \hyperref[@I233@]{Helena Schölch} [29.09.1864--06.10.1864]
\item \hyperref[@I156@]{Johann Josef Schölch} [29.08.1865--04.11.1939 (4 Kinder)]
\item \hyperref[@I234@]{Rosalia Schölch} [15.12.1866]
\item \hyperref[@I235@]{Anna Schölch} [08.10.1869]
\end{itemize}
\bigbreak
\textsc{{quellen}}
\begin{enumerate}[label={[\arabic*]}]
\item GLA Karlsruhe, Laudenberg, katholische Gemeinde: Standesbuch 1810–1870, Geburtenregister 1856, Nr. 7 (Bild 262)
\item \href{https://www.familysearch.org/tree/person/details/LV6Z-VT8}{FamilySearch, ID: LV6Z-VT8}
\item \href{http://gedbas.genealogy.net/person/show/1172960826}{genealogy.net}
\end{enumerate}

\end{person}

\begin{person}[
    surname = {Schölch},
    givenname = {Maria Anna},
    suffix = {1859},
    label = {@I231@}
    ]

\begin{tabular}{cl}
\geboren & 06. November 1859 in Laudenberg\\
\end{tabular}\\
\medbreak
\textsc{vater}: \hyperref[@I158@]{Johann Philipp Schölch} [30.04.1820--21.05.1898 (10 Kinder)]\\
\textsc{mutter}: \hyperref[@I210@]{Anna Maria Ott} [15.01.1829--09.05.1896 (10 Kinder)]
\medbreak
\textsc{{geschwister}}
\begin{itemize}
\item \hyperref[@I225@]{Ludwig Schölch} [30.06.1849 (12 Kinder)]
\item \hyperref[@I228@]{Margaretha Schölch} [05.10.1851]
\item \hyperref[@I229@]{Lina Schölch} [27.12.1853--28.01.1858]
\item \hyperref[@I230@]{Katharina Schölch} [10.06.1856]
\item \hyperref[@I232@]{Theresia Schölch} [28.04.1862--16.10.1863]
\item \hyperref[@I233@]{Helena Schölch} [29.09.1864--06.10.1864]
\item \hyperref[@I156@]{Johann Josef Schölch} [29.08.1865--04.11.1939 (4 Kinder)]
\item \hyperref[@I234@]{Rosalia Schölch} [15.12.1866]
\item \hyperref[@I235@]{Anna Schölch} [08.10.1869]
\end{itemize}
\bigbreak
\textsc{{quellen}}
\begin{enumerate}[label={[\arabic*]}]
\item GLA Karlsruhe, Laudenberg, katholische Gemeinde: Standesbuch 1810–1870, Geburtenregister 1859, Nr. 18 (Bild 282)
\item \href{https://www.familysearch.org/tree/person/details/LV6Z-VYB}{FamilySearch, ID: LV6Z-VYB}
\item \href{http://gedbas.genealogy.net/person/show/1172960833}{genealogy.net}
\end{enumerate}

\end{person}

\begin{person}[
    surname = {Schölch},
    givenname = {Theresia},
    suffix = {1862--1863},
    label = {@I232@}
    ]

\begin{tabular}{cl}
\geboren & 28. April 1862 in Laudenberg\\
\taufe & 28. April 1862 in Limbach\\
\gestorben & 16. Oktober 1863 in Laudenberg\\
\bestattet & 18. Oktober 1863 in Limbach\\
\end{tabular}\\
\medbreak
\textsc{vater}: \hyperref[@I158@]{Johann Philipp Schölch} [30.04.1820--21.05.1898 (10 Kinder)]\\
\textsc{mutter}: \hyperref[@I210@]{Anna Maria Ott} [15.01.1829--09.05.1896 (10 Kinder)]
\medbreak
\textsc{{geschwister}}
\begin{itemize}
\item \hyperref[@I225@]{Ludwig Schölch} [30.06.1849 (12 Kinder)]
\item \hyperref[@I228@]{Margaretha Schölch} [05.10.1851]
\item \hyperref[@I229@]{Lina Schölch} [27.12.1853--28.01.1858]
\item \hyperref[@I230@]{Katharina Schölch} [10.06.1856]
\item \hyperref[@I231@]{Maria Anna Schölch} [06.11.1859]
\item \hyperref[@I233@]{Helena Schölch} [29.09.1864--06.10.1864]
\item \hyperref[@I156@]{Johann Josef Schölch} [29.08.1865--04.11.1939 (4 Kinder)]
\item \hyperref[@I234@]{Rosalia Schölch} [15.12.1866]
\item \hyperref[@I235@]{Anna Schölch} [08.10.1869]
\end{itemize}
\bigbreak
\textsc{{quellen}}
\begin{enumerate}[label={[\arabic*]}]
\item GLA Karlsruhe, Laudenberg, katholische Gemeinde: Standesbuch 1810–1870, Geburtenregister 1862, Nr. 3 (Bild 298)
\item GLA Karlsruhe, Laudenberg, katholische Gemeinde: Standesbuch 1810–1870, Sterberegister 1863, Nr. 11 (Bild 313)
\item \href{http://gedbas.genealogy.net/person/show/1172960843}{genealogy.net}
\item \href{https://www.familysearch.org/tree/person/details/LV6Z-V15}{FamilySearch, ID: LV6Z-V15}
\end{enumerate}

\end{person}

\begin{person}[
    surname = {Schölch},
    givenname = {Helena},
    suffix = {1864--1864},
    label = {@I233@}
    ]

\begin{tabular}{cl}
\geboren & 29. September 1864 in Laudenberg\\
\taufe & 29. September 1864 in Limbach\\
\gestorben & 06. Oktober 1864 in Laudenberg\\
\bestattet & 08. Oktober 1864 in Limbach\\
\end{tabular}\\
\medbreak
\textsc{vater}: \hyperref[@I158@]{Johann Philipp Schölch} [30.04.1820--21.05.1898 (10 Kinder)]\\
\textsc{mutter}: \hyperref[@I210@]{Anna Maria Ott} [15.01.1829--09.05.1896 (10 Kinder)]
\medbreak
\textsc{{geschwister}}
\begin{itemize}
\item \hyperref[@I225@]{Ludwig Schölch} [30.06.1849 (12 Kinder)]
\item \hyperref[@I228@]{Margaretha Schölch} [05.10.1851]
\item \hyperref[@I229@]{Lina Schölch} [27.12.1853--28.01.1858]
\item \hyperref[@I230@]{Katharina Schölch} [10.06.1856]
\item \hyperref[@I231@]{Maria Anna Schölch} [06.11.1859]
\item \hyperref[@I232@]{Theresia Schölch} [28.04.1862--16.10.1863]
\item \hyperref[@I156@]{Johann Josef Schölch} [29.08.1865--04.11.1939 (4 Kinder)]
\item \hyperref[@I234@]{Rosalia Schölch} [15.12.1866]
\item \hyperref[@I235@]{Anna Schölch} [08.10.1869]
\end{itemize}
\bigbreak
\textsc{{quellen}}
\begin{enumerate}[label={[\arabic*]}]
\item GLA Karlsruhe, Laudenberg, katholische Gemeinde: Standesbuch 1810–1870, Geburtenregister 1864, Nr. 7 (Bild 316)
\item GLA Karlsruhe, Laudenberg, katholische Gemeinde: Standesbuch 1810–1870, Sterberegister 1864, Nr. 10 (Bild 319)
\item \href{http://gedbas.genealogy.net/person/show/1172960803}{genealogy.net}
\item \href{https://www.familysearch.org/tree/person/details/LV6Z-K9W}{FamilySearch, ID: LV6Z-K9W}
\end{enumerate}

\end{person}

\begin{person}[
    surname = {Schölch},
    givenname = {Rosalia},
    suffix = {1866},
    label = {@I234@}
    ]

\begin{tabular}{cl}
\geboren & 15. Dezember 1866 in Laudenberg\\
\taufe & 15. Dezember 1866 in Limbach\\
\end{tabular}\\
\medbreak
\textsc{vater}: \hyperref[@I158@]{Johann Philipp Schölch} [30.04.1820--21.05.1898 (10 Kinder)]\\
\textsc{mutter}: \hyperref[@I210@]{Anna Maria Ott} [15.01.1829--09.05.1896 (10 Kinder)]
\medbreak
\textsc{{geschwister}}
\begin{itemize}
\item \hyperref[@I225@]{Ludwig Schölch} [30.06.1849 (12 Kinder)]
\item \hyperref[@I228@]{Margaretha Schölch} [05.10.1851]
\item \hyperref[@I229@]{Lina Schölch} [27.12.1853--28.01.1858]
\item \hyperref[@I230@]{Katharina Schölch} [10.06.1856]
\item \hyperref[@I231@]{Maria Anna Schölch} [06.11.1859]
\item \hyperref[@I232@]{Theresia Schölch} [28.04.1862--16.10.1863]
\item \hyperref[@I233@]{Helena Schölch} [29.09.1864--06.10.1864]
\item \hyperref[@I156@]{Johann Josef Schölch} [29.08.1865--04.11.1939 (4 Kinder)]
\item \hyperref[@I235@]{Anna Schölch} [08.10.1869]
\end{itemize}
\bigbreak
\textsc{{quellen}}
\begin{enumerate}[label={[\arabic*]}]
\item GLA Karlsruhe, Laudenberg, katholische Gemeinde: Standesbuch 1810–1870, Geburtenregister 1866, Nr. 13 (Bild 330)
\item \href{https://www.familysearch.org/tree/person/details/LV6Z-KQD}{FamilySearch, ID: LV6Z-KQD}
\item \href{http://gedbas.genealogy.net/person/show/1172960841}{genealogy.net}
\end{enumerate}

\end{person}

\begin{person}[
    surname = {Schölch},
    givenname = {Anna},
    suffix = {1869},
    label = {@I235@}
    ]

\begin{tabular}{cl}
\geboren & 08. Oktober 1869 in Laudenberg\\
\taufe & 08. Oktober 1869 in Limbach\\
\end{tabular}\\
\medbreak
\textsc{vater}: \hyperref[@I158@]{Johann Philipp Schölch} [30.04.1820--21.05.1898 (10 Kinder)]\\
\textsc{mutter}: \hyperref[@I210@]{Anna Maria Ott} [15.01.1829--09.05.1896 (10 Kinder)]
\medbreak
\textsc{{geschwister}}
\begin{itemize}
\item \hyperref[@I225@]{Ludwig Schölch} [30.06.1849 (12 Kinder)]
\item \hyperref[@I228@]{Margaretha Schölch} [05.10.1851]
\item \hyperref[@I229@]{Lina Schölch} [27.12.1853--28.01.1858]
\item \hyperref[@I230@]{Katharina Schölch} [10.06.1856]
\item \hyperref[@I231@]{Maria Anna Schölch} [06.11.1859]
\item \hyperref[@I232@]{Theresia Schölch} [28.04.1862--16.10.1863]
\item \hyperref[@I233@]{Helena Schölch} [29.09.1864--06.10.1864]
\item \hyperref[@I156@]{Johann Josef Schölch} [29.08.1865--04.11.1939 (4 Kinder)]
\item \hyperref[@I234@]{Rosalia Schölch} [15.12.1866]
\end{itemize}
\bigbreak
\textsc{{quellen}}
\begin{enumerate}[label={[\arabic*]}]
\item GLA Karlsruhe, Laudenberg, katholische Gemeinde: Standesbuch 1810–1870, Geburtenregister 1869, Nr. 12 (Bild 351)
\item \href{http://gedbas.genealogy.net/person/show/1172960798}{genealogy.net}
\item \href{https://www.familysearch.org/tree/person/details/LV6Z-K7J}{FamilySearch, ID: LV6Z-K7J}
\end{enumerate}

\end{person}


\addsec{Joseph Mechler  \& Maria Josepha Seubert }


\begin{person}[
    surname = {Mechler},
    givenname = {Joseph},
    suffix = {1843--1924},
    label = {@I159@}
    ]

\begin{tabular}{cl}
\geboren & 27. Juni 1843 in Langenelz\\
\geheiratet & 23. November 1871 in Langenelz mit Maria Josepha Seubert \\
\gestorben & 16. Mai 1924 in Langenelz\\
\end{tabular}\\
\medbreak
\textsc{vater}: Johann Joseph Mechler [um 1813 (6 Kinder)]\\
\textsc{mutter}: \hyperref[@I643@]{Maria Theresia Lenz} [um 1816 (6 Kinder)]
\medbreak
\textsc{{geschwister}}
\begin{itemize}
\item Johan Michael Mechler [07.06.1831]
\item Karl Mechler [27.06.1843]
\item Marie Mechler [08.03.1845--22.12.1908]
\item Katharina Mechler [14.02.1847]
\item Anna Josepha Mechler Mechler [28.07.1850]
\end{itemize}
\bigbreak
\textsc{{kinder}}
\begin{itemize}
\item \hyperref[@I157@]{Karoline Mechler} [31.03.1870--04.12.1933 (4 Kinder)]
\item \hyperref[@I1430@]{Anna Mechler} [09.08.1872]
\item \hyperref[@I1431@]{Josef Mechler} [29.04.1875]
\item \hyperref[@I1703@]{Wilhelm Mechler} [08.01.1879]
\item \hyperref[@I2085@]{Maria Mechler} [26.03.1882--27.04.1961]
\item \hyperref[@I1704@]{Franz Karl Mechler} [02.09.1884--27.04.1885]
\end{itemize}
\medbreak
\textsc{anmerkung}\\
Anmerkung des Amtsgerichts Buchens in Sterbeurkunde vom 1927.02.28
\medbreak
\textsc{{quellen}}
\begin{enumerate}[label={[\arabic*]}]
\item Langenelz Standesbuch 1870–1875, Heiratsregister 1871, Nr. 4
\item Langenelz Sterbebuch 1911–1927, Sterberegister 1924, Nr. 6
\item \href{https://www.familysearch.org/tree/person/details/LVKC-LQX}{FamilySearch, ID: LVKC-LQX}
\end{enumerate}

\end{person}

\begin{person}[
    surname = {Seubert},
    givenname = {Maria Josepha},
    suffix = {1846--1906},
    label = {@I160@}
    ]

\begin{tabular}{cl}
\geboren & 27. Juli 1846 in Langenelz\\
\taufe & 28. Juli 1846 in Mudau\\
\geheiratet & 23. November 1871 in Langenelz mit Joseph Mechler \\
\gestorben & 20. Juli 1906 in Langenelz\\
\end{tabular}\\
\medbreak
\textsc{vater}: Franz Anton Seubert [03.03.1816--11.06.1889 (7 Kinder)]\\
\textsc{mutter}: \hyperref[@I664@]{Maria Josepha Münch} [20.03.1822--15.02.1865 (7 Kinder)]
\medbreak
\textsc{{geschwister}}
\begin{itemize}
\item Franz Valentin Seubert [22.02.1849]
\item Franz Karl Seubert [29.08.1851--20.02.1887]
\item Maria Anna Seubert [26.05.1856]
\item Franz Anton Seubert [11.06.1859]
\item Wilhelm Seubert [05.01.1862]
\item Franziska Seubert [15.02.1865]
\end{itemize}
\bigbreak
\textsc{{kinder}}
\begin{itemize}
\item \hyperref[@I157@]{Karoline Mechler} [31.03.1870--04.12.1933 (4 Kinder)]
\item \hyperref[@I1430@]{Anna Mechler} [09.08.1872]
\item \hyperref[@I1431@]{Josef Mechler} [29.04.1875]
\item \hyperref[@I1703@]{Wilhelm Mechler} [08.01.1879]
\item \hyperref[@I2085@]{Maria Mechler} [26.03.1882--27.04.1961]
\item \hyperref[@I1704@]{Franz Karl Mechler} [02.09.1884--27.04.1885]
\end{itemize}
\medbreak
\textsc{{quellen}}
\begin{enumerate}[label={[\arabic*]}]
\item \href{http://www.landesarchiv-bw.de/plink/?f=4-1119438-49}{GLA Karlsruhe, Langenelz, katholische Gemeinde: Standesbuch 1839–1870, Geburtenregister 1846, Nr. 11 (Bild 49)}
\item Langenelz Standesbuch 1870–1875, Heiratsregister 1871, Nr. 4
\item \href{https://www.familysearch.org/tree/person/details/LVFT-MTY}{FamilySearch, ID: LVFT-MTY}
\end{enumerate}

\end{person}

\begin{person}[
    surname = {Mechler},
    givenname = {Anna},
    suffix = {1872},
    label = {@I1430@}
    ]

\begin{tabular}{cl}
\geboren & 09. August 1872 in Langenelz\\
\taufe & 09. August 1872 in Mudau\\
\end{tabular}\\
\medbreak
\textsc{vater}: \hyperref[@I159@]{Joseph Mechler} [27.06.1843--16.05.1924 (6 Kinder)]\\
\textsc{mutter}: \hyperref[@I160@]{Maria Josepha Seubert} [27.07.1846--20.07.1906 (6 Kinder)]
\medbreak
\textsc{{geschwister}}
\begin{itemize}
\item \hyperref[@I157@]{Karoline Mechler} [31.03.1870--04.12.1933 (4 Kinder)]
\item \hyperref[@I1431@]{Josef Mechler} [29.04.1875]
\item \hyperref[@I1703@]{Wilhelm Mechler} [08.01.1879]
\item \hyperref[@I2085@]{Maria Mechler} [26.03.1882--27.04.1961]
\item \hyperref[@I1704@]{Franz Karl Mechler} [02.09.1884--27.04.1885]
\end{itemize}
\bigbreak
\textsc{{quellen}}
\begin{enumerate}[label={[\arabic*]}]
\item Langenelz Standesbuch 1870–1875, Geburtenregister 1872, Nr. 3
\item \href{https://www.familysearch.org/tree/person/details/GMXR-C2P}{FamilySearch, ID: GMXR-C2P}
\end{enumerate}

\end{person}

\begin{person}[
    surname = {Mechler},
    givenname = {Josef},
    suffix = {1875},
    label = {@I1431@}
    ]

\begin{tabular}{cl}
\geboren & 29. April 1875 in Langenelz\\
\end{tabular}\\
\medbreak
\textsc{vater}: \hyperref[@I159@]{Joseph Mechler} [27.06.1843--16.05.1924 (6 Kinder)]\\
\textsc{mutter}: \hyperref[@I160@]{Maria Josepha Seubert} [27.07.1846--20.07.1906 (6 Kinder)]
\medbreak
\textsc{{geschwister}}
\begin{itemize}
\item \hyperref[@I157@]{Karoline Mechler} [31.03.1870--04.12.1933 (4 Kinder)]
\item \hyperref[@I1430@]{Anna Mechler} [09.08.1872]
\item \hyperref[@I1703@]{Wilhelm Mechler} [08.01.1879]
\item \hyperref[@I2085@]{Maria Mechler} [26.03.1882--27.04.1961]
\item \hyperref[@I1704@]{Franz Karl Mechler} [02.09.1884--27.04.1885]
\end{itemize}
\bigbreak
\textsc{{quellen}}
\begin{enumerate}[label={[\arabic*]}]
\item Langenelz Standesbuch 1870–1875, Geburtenregister 1875, Nr. 4
\item \href{https://www.familysearch.org/tree/person/details/GMXR-ZPD}{FamilySearch, ID: GMXR-ZPD}
\end{enumerate}

\end{person}

\begin{person}[
    surname = {Mechler},
    givenname = {Wilhelm},
    suffix = {1879},
    label = {@I1703@}
    ]

\begin{tabular}{cl}
\geboren & 08. Januar 1879 in Langenelz\\
\taufe & 08. Januar 1879 in Mudau\\
\gestorben &  in USA\\
\end{tabular}\\
\medbreak
\textsc{vater}: \hyperref[@I159@]{Joseph Mechler} [27.06.1843--16.05.1924 (6 Kinder)]\\
\textsc{mutter}: \hyperref[@I160@]{Maria Josepha Seubert} [27.07.1846--20.07.1906 (6 Kinder)]
\medbreak
\textsc{{geschwister}}
\begin{itemize}
\item \hyperref[@I157@]{Karoline Mechler} [31.03.1870--04.12.1933 (4 Kinder)]
\item \hyperref[@I1430@]{Anna Mechler} [09.08.1872]
\item \hyperref[@I1431@]{Josef Mechler} [29.04.1875]
\item \hyperref[@I2085@]{Maria Mechler} [26.03.1882--27.04.1961]
\item \hyperref[@I1704@]{Franz Karl Mechler} [02.09.1884--27.04.1885]
\end{itemize}
\bigbreak
\textsc{anmerkung}\\
1896 nach Amerika ausgewandert
\medbreak
\textsc{{quellen}}
\begin{enumerate}[label={[\arabic*]}]
\item Langenelz Standesbuch 1876–1879, Geburtenregister 1879, Nr. 1
\item \href{https://www.familysearch.org/tree/person/details/L2X2-S5G}{FamilySearch, ID: L2X2-S5G}
\item https://www.auswanderer-bw.de/index.php?externLink=1\&auswandererid=194449
\end{enumerate}

\end{person}

\begin{person}[
    surname = {Mechler},
    givenname = {Maria},
    suffix = {1882--1961},
    label = {@I2085@}
    ]

\begin{tabular}{cl}
\geboren & 26. März 1882 in Langenelz\\
\gestorben & 27. April 1961 in Frankfurt am Main\\
\end{tabular}\\
\medbreak
\textsc{vater}: \hyperref[@I159@]{Joseph Mechler} [27.06.1843--16.05.1924 (6 Kinder)]\\
\textsc{mutter}: \hyperref[@I160@]{Maria Josepha Seubert} [27.07.1846--20.07.1906 (6 Kinder)]
\medbreak
\textsc{{geschwister}}
\begin{itemize}
\item \hyperref[@I157@]{Karoline Mechler} [31.03.1870--04.12.1933 (4 Kinder)]
\item \hyperref[@I1430@]{Anna Mechler} [09.08.1872]
\item \hyperref[@I1431@]{Josef Mechler} [29.04.1875]
\item \hyperref[@I1703@]{Wilhelm Mechler} [08.01.1879]
\item \hyperref[@I1704@]{Franz Karl Mechler} [02.09.1884--27.04.1885]
\end{itemize}
\bigbreak
\textsc{{quellen}}
\begin{enumerate}[label={[\arabic*]}]
\item Geburtenregister Langenelz 1882, Nr. 5
Sterberegister Frankfurt am Main 1961, Nr. 2666
\end{enumerate}

\end{person}

\begin{person}[
    surname = {Mechler},
    givenname = {Franz Karl},
    suffix = {1884--1885},
    label = {@I1704@}
    ]

\begin{tabular}{cl}
\geboren & 02. September 1884 in Langenelz\\
\gestorben & 27. April 1885\\
\end{tabular}\\
\medbreak
\textsc{vater}: \hyperref[@I159@]{Joseph Mechler} [27.06.1843--16.05.1924 (6 Kinder)]\\
\textsc{mutter}: \hyperref[@I160@]{Maria Josepha Seubert} [27.07.1846--20.07.1906 (6 Kinder)]
\medbreak
\textsc{{geschwister}}
\begin{itemize}
\item \hyperref[@I157@]{Karoline Mechler} [31.03.1870--04.12.1933 (4 Kinder)]
\item \hyperref[@I1430@]{Anna Mechler} [09.08.1872]
\item \hyperref[@I1431@]{Josef Mechler} [29.04.1875]
\item \hyperref[@I1703@]{Wilhelm Mechler} [08.01.1879]
\item \hyperref[@I2085@]{Maria Mechler} [26.03.1882--27.04.1961]
\end{itemize}
\bigbreak
\textsc{{quellen}}
\begin{enumerate}[label={[\arabic*]}]
\item Langenelz Standesbuch 1880–1884, Geburtenregister 1884, Nr. 9
\item \href{https://www.familysearch.org/tree/person/details/G9F9-993}{FamilySearch, ID: G9F9-993}
\end{enumerate}

\end{person}


\addsec{Franz Matthäus Schäfer  \& Anna Maria Trunk }


\begin{person}[
    surname = {Schäfer},
    givenname = {Franz Matthäus},
    suffix = {1812--1879},
    label = {@I378@}
    ]

\begin{tabular}{cl}
\geboren & 07. November 1812 in Beuchen\\
\geheiratet & 19. Juni 1855 in Amorbach mit Anna Maria Trunk \\
 & 02. Juni 1842 in Amorbach mit Anna Maria Link \\
\gestorben & 20. September 1879 in Beuchen\\
\end{tabular}\\
\medbreak
\textsc{vater}: Georg Michael Schäfer [20.10.1786--16.03.1855 (7 Kinder)]\\
\textsc{mutter}: \hyperref[@I550@]{Veronika Scheuermann} [1785--25.02.1842 (7 Kinder)]
\medbreak
\textsc{{geschwister}}
\begin{itemize}
\item Maria Anna Schäfer [09.12.1810--29.11.1851]
\item Anna Maria Schäfer [12.12.1814--19.09.1817]
\item Maria Josepha Schäfer [15.03.1817]
\item Anna Maria Schäfer [31.10.1819--09.04.1820]
\item Franz Josef Schäfer [13.02.1821]
\item Maria Magdalena Schäfer [22.07.1824]
\end{itemize}
\bigbreak
\textsc{{kinder}}
\begin{itemize}
\item \hyperref[@I2140@]{Eva Rosina Schäfer} [07.03.1843]
\item \hyperref[@I2141@]{Johann Michael Schäfer} [25.06.1846]
\item \hyperref[@I2142@]{Franz Matthäus Schäfer} [19.08.1849--16.10.1855]
\item \hyperref[@I2143@]{Franz Anton Schäfer} [11.08.1854--25.08.1854]
\item \hyperref[@I564@]{Franz Karl Anton Schäfer} [12.06.1856--01.01.1863]
\item \hyperref[@I565@]{Johann Martin Schäfer} [20.04.1859--17.01.1863]
\item \hyperref[@I161@]{Josef Eduard Schäfer} [13.10.1861--26.09.1923 (7 Kinder)]
\item \hyperref[@I566@]{Heinrich August Schäfer} [12.05.1864]
\end{itemize}
\medbreak
\textsc{{quellen}}
\begin{enumerate}[label={[\arabic*]}]
\item \href{https://www.familysearch.org/tree/person/details/L5GV-T6P}{FamilySearch, ID: L5GV-T6P}
\end{enumerate}

\end{person}

\begin{person}[
    surname = {Trunk},
    givenname = {Anna Maria},
    suffix = {1825--1874},
    label = {@I379@}
    ]

\begin{tabular}{cl}
\geboren & 05. Juni 1825 in Hornbach\\
\taufe &  in Rippberg\\
\geheiratet & 19. Juni 1855 in Amorbach mit Franz Matthäus Schäfer \\
\gestorben & 27. September 1874 in Beuchen\\
\end{tabular}\\
\medbreak
\textsc{vater}: Valentin Trunk [1791--26.04.1868 (4 Kinder)]\\
\textsc{mutter}: \hyperref[@I560@]{Maria Anna Hilbert} [1797--20.11.1856 (4 Kinder)]
\medbreak
\textsc{{geschwister}}
\begin{itemize}
\item Karl Joseph Trunk [02.11.1826]
\item Franz Joseph Trunk [04.10.1828--05.10.1856]
\item Maria Anna Trunk [07.01.1840]
\end{itemize}
\bigbreak
\textsc{{kinder}}
\begin{itemize}
\item \hyperref[@I564@]{Franz Karl Anton Schäfer} [12.06.1856--01.01.1863]
\item \hyperref[@I565@]{Johann Martin Schäfer} [20.04.1859--17.01.1863]
\item \hyperref[@I161@]{Josef Eduard Schäfer} [13.10.1861--26.09.1923 (7 Kinder)]
\item \hyperref[@I566@]{Heinrich August Schäfer} [12.05.1864]
\end{itemize}
\medbreak
\textsc{{quellen}}
\begin{enumerate}[label={[\arabic*]}]
\item \href{https://www.familysearch.org/tree/person/details/L5GV-TBF}{FamilySearch, ID: L5GV-TBF}
\end{enumerate}

\end{person}

\begin{person}[
    surname = {Schäfer},
    givenname = {Eva Rosina},
    suffix = {1843},
    label = {@I2140@}
    ]

\begin{tabular}{cl}
\geboren & 07. März 1843 in Beuchen\\
\end{tabular}\\
\medbreak
\textsc{vater}: \hyperref[@I378@]{Franz Matthäus Schäfer} [07.11.1812--20.09.1879 (8 Kinder)]\\
\textsc{mutter}: \hyperref[@I2139@]{Anna Maria Link} [16.10.1812--30.08.1854 (4 Kinder)]
\medbreak
\textsc{{geschwister}}
\begin{itemize}
\item \hyperref[@I2141@]{Johann Michael Schäfer} [25.06.1846]
\item \hyperref[@I2142@]{Franz Matthäus Schäfer} [19.08.1849--16.10.1855]
\item \hyperref[@I2143@]{Franz Anton Schäfer} [11.08.1854--25.08.1854]
\item \hyperref[@I564@]{Franz Karl Anton Schäfer} [12.06.1856--01.01.1863]
\item \hyperref[@I565@]{Johann Martin Schäfer} [20.04.1859--17.01.1863]
\item \hyperref[@I161@]{Josef Eduard Schäfer} [13.10.1861--26.09.1923 (7 Kinder)]
\item \hyperref[@I566@]{Heinrich August Schäfer} [12.05.1864]
\end{itemize}
\bigbreak
\end{person}

\begin{person}[
    surname = {Schäfer},
    givenname = {Johann Michael},
    suffix = {1846},
    label = {@I2141@}
    ]

\begin{tabular}{cl}
\geboren & 25. Juni 1846 in Beuchen\\
\end{tabular}\\
\medbreak
\textsc{vater}: \hyperref[@I378@]{Franz Matthäus Schäfer} [07.11.1812--20.09.1879 (8 Kinder)]\\
\textsc{mutter}: \hyperref[@I2139@]{Anna Maria Link} [16.10.1812--30.08.1854 (4 Kinder)]
\medbreak
\textsc{{geschwister}}
\begin{itemize}
\item \hyperref[@I2140@]{Eva Rosina Schäfer} [07.03.1843]
\item \hyperref[@I2142@]{Franz Matthäus Schäfer} [19.08.1849--16.10.1855]
\item \hyperref[@I2143@]{Franz Anton Schäfer} [11.08.1854--25.08.1854]
\item \hyperref[@I564@]{Franz Karl Anton Schäfer} [12.06.1856--01.01.1863]
\item \hyperref[@I565@]{Johann Martin Schäfer} [20.04.1859--17.01.1863]
\item \hyperref[@I161@]{Josef Eduard Schäfer} [13.10.1861--26.09.1923 (7 Kinder)]
\item \hyperref[@I566@]{Heinrich August Schäfer} [12.05.1864]
\end{itemize}
\bigbreak
\textsc{{quellen}}
\begin{enumerate}[label={[\arabic*]}]
\item \href{https://www.familysearch.org/tree/person/details/LT53-X1P}{FamilySearch, ID: LT53-X1P}
\end{enumerate}

\end{person}

\begin{person}[
    surname = {Schäfer},
    givenname = {Franz Matthäus},
    suffix = {1849--1855},
    label = {@I2142@}
    ]

\begin{tabular}{cl}
\geboren & 19. August 1849 in Beuchen\\
\gestorben & 16. Oktober 1855 in Beuchen\\
\end{tabular}\\
\medbreak
\textsc{vater}: \hyperref[@I378@]{Franz Matthäus Schäfer} [07.11.1812--20.09.1879 (8 Kinder)]\\
\textsc{mutter}: \hyperref[@I2139@]{Anna Maria Link} [16.10.1812--30.08.1854 (4 Kinder)]
\medbreak
\textsc{{geschwister}}
\begin{itemize}
\item \hyperref[@I2140@]{Eva Rosina Schäfer} [07.03.1843]
\item \hyperref[@I2141@]{Johann Michael Schäfer} [25.06.1846]
\item \hyperref[@I2143@]{Franz Anton Schäfer} [11.08.1854--25.08.1854]
\item \hyperref[@I564@]{Franz Karl Anton Schäfer} [12.06.1856--01.01.1863]
\item \hyperref[@I565@]{Johann Martin Schäfer} [20.04.1859--17.01.1863]
\item \hyperref[@I161@]{Josef Eduard Schäfer} [13.10.1861--26.09.1923 (7 Kinder)]
\item \hyperref[@I566@]{Heinrich August Schäfer} [12.05.1864]
\end{itemize}
\bigbreak
\textsc{{quellen}}
\begin{enumerate}[label={[\arabic*]}]
\item \href{https://www.familysearch.org/tree/person/details/LT53-L78}{FamilySearch, ID: LT53-L78}
\end{enumerate}

\end{person}

\begin{person}[
    surname = {Schäfer},
    givenname = {Franz Anton},
    suffix = {1854--1854},
    label = {@I2143@}
    ]

\begin{tabular}{cl}
\geboren & 11. August 1854 in Beuchen\\
\gestorben & 25. August 1854 in Beuchen\\
\end{tabular}\\
\medbreak
\textsc{vater}: \hyperref[@I378@]{Franz Matthäus Schäfer} [07.11.1812--20.09.1879 (8 Kinder)]\\
\textsc{mutter}: \hyperref[@I2139@]{Anna Maria Link} [16.10.1812--30.08.1854 (4 Kinder)]
\medbreak
\textsc{{geschwister}}
\begin{itemize}
\item \hyperref[@I2140@]{Eva Rosina Schäfer} [07.03.1843]
\item \hyperref[@I2141@]{Johann Michael Schäfer} [25.06.1846]
\item \hyperref[@I2142@]{Franz Matthäus Schäfer} [19.08.1849--16.10.1855]
\item \hyperref[@I564@]{Franz Karl Anton Schäfer} [12.06.1856--01.01.1863]
\item \hyperref[@I565@]{Johann Martin Schäfer} [20.04.1859--17.01.1863]
\item \hyperref[@I161@]{Josef Eduard Schäfer} [13.10.1861--26.09.1923 (7 Kinder)]
\item \hyperref[@I566@]{Heinrich August Schäfer} [12.05.1864]
\end{itemize}
\bigbreak
\textsc{{quellen}}
\begin{enumerate}[label={[\arabic*]}]
\item \href{https://www.familysearch.org/tree/person/details/LT53-2LY}{FamilySearch, ID: LT53-2LY}
\end{enumerate}

\end{person}

\begin{person}[
    surname = {Schäfer},
    givenname = {Franz Karl Anton},
    suffix = {1856--1863},
    label = {@I564@}
    ]

\begin{tabular}{cl}
\geboren & 12. Juni 1856 in Beuchen\\
\gestorben & 01. Januar 1863 in Beuchen\\
\end{tabular}\\
\medbreak
\textsc{vater}: \hyperref[@I378@]{Franz Matthäus Schäfer} [07.11.1812--20.09.1879 (8 Kinder)]\\
\textsc{mutter}: \hyperref[@I379@]{Anna Maria Trunk} [05.06.1825--27.09.1874 (4 Kinder)]
\medbreak
\textsc{{geschwister}}
\begin{itemize}
\item \hyperref[@I2140@]{Eva Rosina Schäfer} [07.03.1843]
\item \hyperref[@I2141@]{Johann Michael Schäfer} [25.06.1846]
\item \hyperref[@I2142@]{Franz Matthäus Schäfer} [19.08.1849--16.10.1855]
\item \hyperref[@I2143@]{Franz Anton Schäfer} [11.08.1854--25.08.1854]
\item \hyperref[@I565@]{Johann Martin Schäfer} [20.04.1859--17.01.1863]
\item \hyperref[@I161@]{Josef Eduard Schäfer} [13.10.1861--26.09.1923 (7 Kinder)]
\item \hyperref[@I566@]{Heinrich August Schäfer} [12.05.1864]
\end{itemize}
\bigbreak
\textsc{{quellen}}
\begin{enumerate}[label={[\arabic*]}]
\item \href{https://www.familysearch.org/tree/person/details/LT53-62X}{FamilySearch, ID: LT53-62X}
\end{enumerate}

\end{person}

\begin{person}[
    surname = {Schäfer},
    givenname = {Johann Martin},
    suffix = {1859--1863},
    label = {@I565@}
    ]

\begin{tabular}{cl}
\geboren & 20. April 1859 in Beuchen\\
\gestorben & 17. Januar 1863 in Beuchen\\
\end{tabular}\\
\medbreak
\textsc{vater}: \hyperref[@I378@]{Franz Matthäus Schäfer} [07.11.1812--20.09.1879 (8 Kinder)]\\
\textsc{mutter}: \hyperref[@I379@]{Anna Maria Trunk} [05.06.1825--27.09.1874 (4 Kinder)]
\medbreak
\textsc{{geschwister}}
\begin{itemize}
\item \hyperref[@I2140@]{Eva Rosina Schäfer} [07.03.1843]
\item \hyperref[@I2141@]{Johann Michael Schäfer} [25.06.1846]
\item \hyperref[@I2142@]{Franz Matthäus Schäfer} [19.08.1849--16.10.1855]
\item \hyperref[@I2143@]{Franz Anton Schäfer} [11.08.1854--25.08.1854]
\item \hyperref[@I564@]{Franz Karl Anton Schäfer} [12.06.1856--01.01.1863]
\item \hyperref[@I161@]{Josef Eduard Schäfer} [13.10.1861--26.09.1923 (7 Kinder)]
\item \hyperref[@I566@]{Heinrich August Schäfer} [12.05.1864]
\end{itemize}
\bigbreak
\textsc{{quellen}}
\begin{enumerate}[label={[\arabic*]}]
\item \href{https://www.familysearch.org/tree/person/details/LT53-63T}{FamilySearch, ID: LT53-63T}
\end{enumerate}

\end{person}

\begin{person}[
    surname = {Schäfer},
    givenname = {Heinrich August},
    suffix = {1864},
    label = {@I566@}
    ]

\begin{tabular}{cl}
\geboren & 12. Mai 1864 in Beuchen\\
\end{tabular}\\
\medbreak
\textsc{vater}: \hyperref[@I378@]{Franz Matthäus Schäfer} [07.11.1812--20.09.1879 (8 Kinder)]\\
\textsc{mutter}: \hyperref[@I379@]{Anna Maria Trunk} [05.06.1825--27.09.1874 (4 Kinder)]
\medbreak
\textsc{{geschwister}}
\begin{itemize}
\item \hyperref[@I2140@]{Eva Rosina Schäfer} [07.03.1843]
\item \hyperref[@I2141@]{Johann Michael Schäfer} [25.06.1846]
\item \hyperref[@I2142@]{Franz Matthäus Schäfer} [19.08.1849--16.10.1855]
\item \hyperref[@I2143@]{Franz Anton Schäfer} [11.08.1854--25.08.1854]
\item \hyperref[@I564@]{Franz Karl Anton Schäfer} [12.06.1856--01.01.1863]
\item \hyperref[@I565@]{Johann Martin Schäfer} [20.04.1859--17.01.1863]
\item \hyperref[@I161@]{Josef Eduard Schäfer} [13.10.1861--26.09.1923 (7 Kinder)]
\end{itemize}
\bigbreak
\textsc{{quellen}}
\begin{enumerate}[label={[\arabic*]}]
\item \href{https://www.familysearch.org/tree/person/details/LT53-62G}{FamilySearch, ID: LT53-62G}
\end{enumerate}

\end{person}


\addsec{Johann Valentin Zeller  \& Maria Josefa Mehl }


\begin{person}[
    surname = {Zeller},
    givenname = {Johann Valentin},
    suffix = {1815--1879},
    label = {@I380@}
    ]

\begin{tabular}{cl}
\geboren & 26. Juni 1815 in Beuchen\\
\geheiratet & 24. August 1858 mit Maria Josefa Mehl \\
\gestorben & 16. April 1879 in Beuchen\\
\end{tabular}\\
\medbreak
\textsc{vater}: Johann Michael Zeller [22.05.1776--03.03.1823 (3 Kinder)]\\
\textsc{mutter}: \hyperref[@I602@]{Maria Anna Herkert} [10.11.1781--12.03.1848 (3 Kinder)]
\medbreak
\textsc{{geschwister}}
\begin{itemize}
\item Franz Michael Zeller [01.12.1807]
\item Johann Josef Zeller [06.03.1810]
\end{itemize}
\bigbreak
\textsc{{kinder}}
\begin{itemize}
\item \hyperref[@I597@]{Maria Regina Zeller} [05.09.1859--22.12.1862]
\item \hyperref[@I598@]{Theresia Helena Zeller} [04.06.1862--01.07.1862]
\item \hyperref[@I599@]{Valentin Augustin Zeller} [1863]
\item \hyperref[@I162@]{Josefa Karolina Zeller} [25.01.1866--28.11.1931 (7 Kinder)]
\item \hyperref[@I600@]{Anna Adelheid Zeller} [11.11.1868--12.09.1898]
\end{itemize}
\medbreak
\textsc{{quellen}}
\begin{enumerate}[label={[\arabic*]}]
\item \href{https://www.familysearch.org/tree/person/details/L5GV-RD6}{FamilySearch, ID: L5GV-RD6}
\end{enumerate}

\end{person}

\begin{person}[
    surname = {Mehl},
    givenname = {Maria Josefa},
    suffix = {1832--1890},
    label = {@I381@}
    ]

\begin{tabular}{cl}
\geboren & 27. Juni 1832 in Beuchen\\
\geheiratet & 24. August 1858 mit Johann Valentin Zeller \\
\gestorben & 20. Januar 1890 in Beuchen\\
\end{tabular}\\
\medbreak
\textsc{vater}: Johann Michael Mehl [03.08.1801--21.11.1864 (8 Kinder)]\\
\textsc{mutter}: \hyperref[@I586@]{Anna Eva Ballweg} [19.04.1795--31.03.1866 (8 Kinder)]
\medbreak
\textsc{{geschwister}}
\begin{itemize}
\item Franz Josef Mehl [20.03.1823--14.04.1823]
\item Johann Michael Mehl [13.03.1824--22.02.1902]
\item Anna Eva Mehl [17.03.1827--12.06.1897]
\item Franz Josef Mehl [09.10.1829--31.05.1895]
\item Franz Karl Mehl [16.04.1835--26.04.1835]
\item Maria Anna Mehl [04.01.1837--23.10.1888]
\item Franz Martin Mehl [08.01.1840--24.01.1844]
\end{itemize}
\bigbreak
\textsc{{kinder}}
\begin{itemize}
\item \hyperref[@I597@]{Maria Regina Zeller} [05.09.1859--22.12.1862]
\item \hyperref[@I598@]{Theresia Helena Zeller} [04.06.1862--01.07.1862]
\item \hyperref[@I599@]{Valentin Augustin Zeller} [1863]
\item \hyperref[@I162@]{Josefa Karolina Zeller} [25.01.1866--28.11.1931 (7 Kinder)]
\item \hyperref[@I600@]{Anna Adelheid Zeller} [11.11.1868--12.09.1898]
\end{itemize}
\medbreak
\textsc{{quellen}}
\begin{enumerate}[label={[\arabic*]}]
\item \href{https://www.familysearch.org/tree/person/details/L5GV-YFQ}{FamilySearch, ID: L5GV-YFQ}
\end{enumerate}

\end{person}

\begin{person}[
    surname = {Zeller},
    givenname = {Maria Regina},
    suffix = {1859--1862},
    label = {@I597@}
    ]

\begin{tabular}{cl}
\geboren & 05. September 1859 in Beuchen\\
\gestorben & 22. Dezember 1862 in Beuchen\\
\end{tabular}\\
\medbreak
\textsc{vater}: \hyperref[@I380@]{Johann Valentin Zeller} [26.06.1815--16.04.1879 (5 Kinder)]\\
\textsc{mutter}: \hyperref[@I381@]{Maria Josefa Mehl} [27.06.1832--20.01.1890 (5 Kinder)]
\medbreak
\textsc{{geschwister}}
\begin{itemize}
\item \hyperref[@I598@]{Theresia Helena Zeller} [04.06.1862--01.07.1862]
\item \hyperref[@I599@]{Valentin Augustin Zeller} [1863]
\item \hyperref[@I162@]{Josefa Karolina Zeller} [25.01.1866--28.11.1931 (7 Kinder)]
\item \hyperref[@I600@]{Anna Adelheid Zeller} [11.11.1868--12.09.1898]
\end{itemize}
\bigbreak
\textsc{{quellen}}
\begin{enumerate}[label={[\arabic*]}]
\item \href{https://www.familysearch.org/tree/person/details/LT57-MNT}{FamilySearch, ID: LT57-MNT}
\end{enumerate}

\end{person}

\begin{person}[
    surname = {Zeller},
    givenname = {Theresia Helena},
    suffix = {1862--1862},
    label = {@I598@}
    ]

\begin{tabular}{cl}
\geboren & 04. Juni 1862 in Beuchen\\
\gestorben & 01. Juli 1862 in Beuchen\\
\end{tabular}\\
\medbreak
\textsc{vater}: \hyperref[@I380@]{Johann Valentin Zeller} [26.06.1815--16.04.1879 (5 Kinder)]\\
\textsc{mutter}: \hyperref[@I381@]{Maria Josefa Mehl} [27.06.1832--20.01.1890 (5 Kinder)]
\medbreak
\textsc{{geschwister}}
\begin{itemize}
\item \hyperref[@I597@]{Maria Regina Zeller} [05.09.1859--22.12.1862]
\item \hyperref[@I599@]{Valentin Augustin Zeller} [1863]
\item \hyperref[@I162@]{Josefa Karolina Zeller} [25.01.1866--28.11.1931 (7 Kinder)]
\item \hyperref[@I600@]{Anna Adelheid Zeller} [11.11.1868--12.09.1898]
\end{itemize}
\bigbreak
\textsc{{quellen}}
\begin{enumerate}[label={[\arabic*]}]
\item \href{https://www.familysearch.org/tree/person/details/LT5Q-6RZ}{FamilySearch, ID: LT5Q-6RZ}
\end{enumerate}

\end{person}

\begin{person}[
    surname = {Zeller},
    givenname = {Valentin Augustin},
    suffix = {1863},
    label = {@I599@}
    ]

\begin{tabular}{cl}
\geboren & 1863 in Beuchen\\
\end{tabular}\\
\medbreak
\textsc{vater}: \hyperref[@I380@]{Johann Valentin Zeller} [26.06.1815--16.04.1879 (5 Kinder)]\\
\textsc{mutter}: \hyperref[@I381@]{Maria Josefa Mehl} [27.06.1832--20.01.1890 (5 Kinder)]
\medbreak
\textsc{{geschwister}}
\begin{itemize}
\item \hyperref[@I597@]{Maria Regina Zeller} [05.09.1859--22.12.1862]
\item \hyperref[@I598@]{Theresia Helena Zeller} [04.06.1862--01.07.1862]
\item \hyperref[@I162@]{Josefa Karolina Zeller} [25.01.1866--28.11.1931 (7 Kinder)]
\item \hyperref[@I600@]{Anna Adelheid Zeller} [11.11.1868--12.09.1898]
\end{itemize}
\bigbreak
\textsc{{quellen}}
\begin{enumerate}[label={[\arabic*]}]
\item \href{https://www.familysearch.org/tree/person/details/LT5Q-TG6}{FamilySearch, ID: LT5Q-TG6}
\end{enumerate}

\end{person}

\begin{person}[
    surname = {Zeller},
    givenname = {Anna Adelheid},
    suffix = {1868--1898},
    label = {@I600@}
    ]

\begin{tabular}{cl}
\geboren & 11. November 1868 in Beuchen\\
\gestorben & 12. September 1898 in Beuchen\\
\end{tabular}\\
\medbreak
\textsc{vater}: \hyperref[@I380@]{Johann Valentin Zeller} [26.06.1815--16.04.1879 (5 Kinder)]\\
\textsc{mutter}: \hyperref[@I381@]{Maria Josefa Mehl} [27.06.1832--20.01.1890 (5 Kinder)]
\medbreak
\textsc{{geschwister}}
\begin{itemize}
\item \hyperref[@I597@]{Maria Regina Zeller} [05.09.1859--22.12.1862]
\item \hyperref[@I598@]{Theresia Helena Zeller} [04.06.1862--01.07.1862]
\item \hyperref[@I599@]{Valentin Augustin Zeller} [1863]
\item \hyperref[@I162@]{Josefa Karolina Zeller} [25.01.1866--28.11.1931 (7 Kinder)]
\end{itemize}
\bigbreak
\textsc{{quellen}}
\begin{enumerate}[label={[\arabic*]}]
\item \href{https://www.familysearch.org/tree/person/details/LT5Q-RK6}{FamilySearch, ID: LT5Q-RK6}
\end{enumerate}

\end{person}

