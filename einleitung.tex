\addchap{Vorwort}
Eine Zusammenstellung

\addchap{Herkunft der Daten}

\addsec{Alte Ahnentafeln}

Die erste Anlaufstelle sind of alte Ahnentafeln. Diese werden oft über die Generationen weitergereicht, abgeschrieben und ergänzt. Während sie einen guten Einstieg bieten sind sie doch mit Vorsicht zu genie"sen da sich über die Zeit oftmals Fehler eingeschlichen haben.

\addsec{Kirchenbücher und Personenstandsregister}

Die erste systematische Erfassung von Personendaten (Geburt/Taufe, Hochzeit, Tod/Bestattung) kam in der Form von Kirchenb"ucher. Diese wurden je nach Region erstmals zwischen dem 16. und 18. Jarhundert gef"uhrt.\footnote{\href{https://www.landesarchiv-bw.de/web/57490}{Landesarchive Baden Württemberg: Ab dem Aufkommen der Kirchenbücher}}. Die alten Bücher wurden jedoch zu deren Schutz von Pfarrgemeinden in Zentrale Archive gebracht. Aufgrund unserer Zugehörigkeit zum Erzbistum Freiburg befinden sich die hiesigen Bücher im \href{http://www.katholische-archive.de/Di\%C3\%B6zesanarchive/Freiburg/tabid/78/Default.aspx}{Erzbischöfliches Archiv Freiburg}.


Das Gro"sherzogtum Baden lies ab 1810 Standesbücher als Zweitschriften der Kirchenbücher anfertigen. Diese sind heute digitalisiert und online unter \href{https://www2.landesarchiv-bw.de/ofs21/olf/startbild.php?bestand=12390}{Generallandesarchiv Karlsruhe Bestand 390 Standesbücher} einsehbar.\\

Im Jahr 1870 wurden in Baden (später auch in ganz Deutschland) schlie"slich die amtlichen Personstandsregister eingeführt\footnote{\href{https://www.landesarchiv-bw.de/web/57479}{Landesarchive Baden Württemberg: Ab Einführung der Personenstandsregister}}. 


\addsec{Online Datenbanken}

Die Genealogical Society of Utah (GSU)
Kirche Jesu Christi der Heiligen der Letzten Tage (Mormonen)
im Jahr 1938 fingen sie damit an Kirchenbücher und andere relevante Dokumente in den USA auf Mikrofilme zu fotografieren. 10 Jahre später schlie"slich auch in Europa Diese lagern 

Diese können in einem von über 5000 Family History Center weltweit eingesehen werden. 1998 wurde begonnen die existierenden Bestände zu digitalisieren. Diese sind gro"steils online einsehbar 
\begin{center}
    \url{https://www.familysearch.org/de/}
\end{center}
und zum Teil auch indiziert.

\footnote{\href{https://en.wikipedia.org/wiki/FamilySearch}{FamilySearch auf \textsc{Wikipedia}}}\\


Daneben gibt es noch eine Vielzahl kleinerer Projekte. Der Verein für Computergenealogie e.V. betreibt auf seiner \href{http://compgen.de/}{Webseite} verschiedene Datenbanken. Zum einen können Ahnenforscher ihre Ergebnisse anderen Interessierten bereitstellen. \\

Ein weiteres nützliches Projekt ist das \href{http://grabsteine.genealogy.net/}{Grabstein Projekt}. Alte Grabsteine werden nach dem Ende der Liegezeit oftmals geschreddert und als Straßenunterbau verwendet. Damit sind wertvolle Informationen für immer verloren. Um dem entgegenzuwirken hat es sich der Verein zur Aufgabe gemacht Friedhöfe in Deutschland abzufotografieren und die Bilder für jeden online zugänglich in eine Datenbank einzupflegen. 



